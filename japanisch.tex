\documentclass[emulatestandardclasses]{scrartcl}
\usepackage{graphicx}
\usepackage{CJKutf8} % japanese
\usepackage{color}
\usepackage[ngerman]{babel}
\usepackage{hyperref}
\usepackage{fullpage}
\usepackage[utf8]{inputenc}
\usepackage{calc} 
\usepackage{enumitem}
\usepackage{titlesec}
\date{\vspace{-3ex}}
\begin{document}

\title{
	\includegraphics*[bb=0 0 720 200, width=0.72\textwidth]{ErstesSem/images/hu_logo.png}\\
	\vspace{25pt}
	Japanisch 1}
\subtitle{\vspace{10pt}
			Jutta Borchert\\
			Sprachkurs WS 17/18\\
          Institut für Asien- und Afrikawissenschaften\\ 
          Humboldt Universit"at zu Berlin}
\author{Lennard Wolf\\
        \small{\href{mailto:lennard.wolf@student.hu-berlin.de}{lennard.wolf@student.hu-berlin.de}}}
\maketitle
\begin{abstract}
Lernintensiver integrierter Anfängerkurs zum Erwerb von Grundkenntnissen und Grundfertigkeiten auf etwa der Stufe A1 des Gemeinsamen Europäischen Referenzrahmens. Gegenstand sind auf der Grundlage des Lehrwerkes \emph{Minna no nihongo I} ca. 850 lexikalische Einheiten, 200 chinesische Zeichen und elementare Wort- und Satzgrammatik.Die Kenntnis der beiden Silbenalphabete Hiragana und Katakana am Beginn wird empfohlen. Anfänger ohne weitere Vorkenntnisse sollten mindestens sechs Stunden/Woche Selbststudienzeit einplanen.

\end{abstract}
\newpage

\tableofcontents
%\listoffigures
\newpage


\section{Woche 1\\(03.11.17)}

\subsection{Sitzungsnotizen}

\subsubsection{Kanji}

\begin{itemize}
  \item Beim Schreiben an der hangeschriebenen Form orientieren, nicht an der Druckform!
  \item On-Lesung ( 音読み「おにょみ」) oder \emph{sinojapanische Lesung} lehnt sich an den Klang des entsprechenden chinesischen Wortes an
  \item Kun-Lesung ( 訓読み「くにょみ」) lehnt sich von der Bedeutung her an den chinesischen Zeichen an, hat aber die altjapanische Lesung
  \item 
\end{itemize}



\end{document}
