\documentclass[emulatestandardclasses]{scrartcl}
\usepackage{graphicx}
\usepackage{CJKutf8} % japanese
\usepackage{color}
\usepackage[ngerman]{babel}
\usepackage{hyperref}
\usepackage{fullpage}
\usepackage[utf8]{inputenc}
\usepackage{setspace}
\usepackage{calc} 
\usepackage{enumitem}
\usepackage{titlesec}
\date{\vspace{-3ex}}
\begin{document}

\title{
	\includegraphics*[bb=0 0 720 200, width=0.72\textwidth]{ErstesSem/images/hu_logo.png}\\
	\vspace{25pt}
	Japan und die architektonische Moderne
	\\
	Der japanische Garten}
\subtitle{\vspace{10pt}
			Prof. Dr. Kai Kappel, Prof. Dr. Salomon, Dr. Tina Zürn\\
			Proseminar SS 2018\\
          Institut für Philosophie\\ 
          Humboldt Universit"at zu Berlin}
\author{Lennard Wolf\\
        \small{\href{mailto:lennard.wolf@student.hu-berlin.de}{lennard.wolf@student.hu-berlin.de}}}
\maketitle
\begin{abstract}

\noindent \textbf{Japan und die architektonische Moderne.\\ Rezeptions- und Transformationsgeschichte}\newline
\indent Gegenstand dieses gemeinsamen Seminars des Instituts für Asien- und Afrikawissenschaften und des Instituts für Kunst- und Bildgeschichte ist die Bau-, Wohn- und Erinnerungskultur Japans. Dabei interessieren uns insbesondere Aspekte der Rezeption und Transformation traditioneller Wohn- und Lebensformen des vormodernen Japan von der Mitte des 19. Jahrhunderts bis in die Gegenwart. Die Frage nach dem kulturell Eigenen und dem Anderen soll besonders vor dem Hintergrund des Modernediskurses, zunehmender Globalisierungseffekte und transkultureller Austauschprozesse gestellt werden. Das Seminar bezieht auch Orte in Berlin (etwa die Mori-Ôgai-Gedenkstätte) ein.\\
\textbf{Der japanische Garten | Geschichte, Sinngehalt, Rezeption}\\
\indent Gegenstand dieses gemeinsamen Seminars des Instituts für Asien- und Afrika­wissenschaften und des Instituts für Kunst- und Bildgeschichte ist die Genese, der Sinngehalt, die Nutzung und die Verbreitung des japanischen Gartens. Dieser verarbeitet seit der Meiji-Zeit auch westliche Vorstellungen von Gartenkunst und ist in Europa und Amerika, wo in der Moderne zahlreiche Anverwandlungen entstehen, Spiegelbild der Vorstellungen ostasiatischer Kultur. Daher ist in besonderer Weise die mediale Vermittlung des japanischen Gartens zu betrachten.

\end{abstract}
\newpage

\tableofcontents
%\listoffigures
\newpage


\section{Architektonische Moderne: Einführungssitzung\\(19.04.18)}

\subsection{Organisation}

\begin{itemize}
  \item Moodle PW: Katsura
  \item Hausarbeit Abgabe 30.09.
  \item Seminararbeit??
  \item "`maximal verstörend"'
  \item Aufbau Gesellschaft und ?
  \item 1 Monat Anmeldezeitraum für MAP
\end{itemize}

\subsection{Einleitendes}

\begin{itemize}
  \item Rezeption und Transformation: Japaner*innen, die die westliche Moderne kennenlernen
  \item Rezeptionsprozess beginnt schon bei der Auswahl der Medien etc. 
  \item Wieviel Europa ist dabei, wenn nach Japan zurückgekehrt wird
  \item Für Regionalstudien Asien Afrika: Was bedeutet das alles für die Geschichte des Übergangs in die Moderne, wird häufig eurozentristisch erzählt, hier die Möglichkeit, aus einer anderen Sicht zu betrachten
  \item Japan wird immer wieder neu entdeckt (Limitierung des Bildes?)
  \item 17.5. Treffen an den Gärten der Welt
  \item Tetsuro Yoshida: Das japanische Wohnhaus
\end{itemize}

\subsection{Vortrag}

\begin{itemize}
  \item \emph{Capsule Tower} 12.7.
  \item Koolhaas, Rem und Hans Ulrich Obrist (2011). Project Japan – Metabolism Talks
  \item Lin, Zhongjie (2010). Kenzo Tange and the Metabolist Movement
  \item Boyd, Robin (1968). New Directions in Japanese Architecture
  \item \url{http://www.spiegel.de/artikel/a-947404.html}
  \item Handout kann auch hinterher abgegeben werden (diskussion aus der stunde mit rein); bis zu 1-2 Wochen später
  \item Zentrale Aussagen, nicht mehr als 2-3 Seiten
  \item Metabolistische und strukturlistische Konzepte - in altjapanischem Maß? (Tanges Plan für die Bucht von Tokio, Projekte Ocean City und Space City, Yamanashi Center in Koofu, Festival Plaza der Weltausstellung in Osaka, Nagakin Capsule Tower in Tokio)
  \item Zum welchem gemeinsamen Fazit kommen wir, was sind die Fragen die wir uns stellen
  \item Transkulturalität ansprechen
  \item Vergleichend anssetzen: Alte vs Neue Fotos
  \item Grundriss,
  \item Kurzvorstellung - Baugeschichte (Geschichte erzählen) - Baubeschreibung (Kontext - Material - Klar def. Beschreibrichtung - Himmelrichtung - Augenmerk auf Erschließung) - ggf.: Bauanalyse, denkmalpflegerischer Umgang
  \item Bildwörterbuch zur Einführung in die japanische Kultur. Architektur und Religion. 
  \item judith.dreiling@hu-berlin.de
\end{itemize}


\section{Japanische Gärten: Einführungssitzung\\(19.04.18)}

\begin{itemize}
  \item Moodle PW: Zen
  \item Jiro Takei, Marc Peter Keane: Sakuteiki oder die Kunst des japanischen Gartens (Handbuch über Gartengestaltung aus der Heian-Zeit)
  \item Sir William Temples \emph{Sharawadgi}-Begriff ("`beauty without order"')
  \item 鳳凰殿 (ほうおうでん) (Chicago 1893) ?
  \item Anfragen ob Namen in Kanji auch in die Präsentationen kommen könnten?
\end{itemize}


\section{Architektonische Moderne: Inhaltliche Einführung\\(26.04.18)}

\subsection{Reisen und die Wahrnehmung des Eigenen/Fremden}

\begin{itemize}
  \item しょうじ sind die rechteckig eingeteilten Fenster
  \item Mitnahme aus Auslandsaufenthalten immer Transformationsprozesse durch schöpferische Aneignung
  \item Reisetraditionen: Bildungsreise "`Grand Tour"' (Italien); heute: "`Italy in 6 days"'; Pilgerreisen, Wallfahrten
  \item Japan als neues Italien?
  \item CIAM internationale Architektenvereinigung
  \item Öffnung Japans 1854, durch amerikanische Flotte erzwungen
  \item These: In langen Reisen kann das einst Fremde zu etwas vertrautem werden!
\end{itemize}

\subsection{Transkulturelle Austauschprozesse - wechselseitige Blicke Japan, USA und Europa in der Moderne}

\begin{itemize}
  \item Heutige Kulturen al Inseln zu bezeichnen wäre falsch. Sie haben eine neue Form angenommen, sie sind in sich "`transkulturell"' (vgl. Wolfang Welsch 1997) | Absage zu kultureller Homogenität
  \item Großer Austausch
\end{itemize}


\section{Japanische Gärten: Christian Century in Japan (1549 - 1650)\\(26.04.18)}

\subsection{Salomon: Historischer Kontext}

\begin{itemize}
  \item Luis Frois
  \item Von den Häusern, Bauten, Gärten und Früchten
  \item Häuser E: Hoch, mehrere Stockwerke, J: Flach, ebenerdig
  \item Fundament
  \item Türangeln, しき
  \item Zimmerteilung: Stein vs. 
  \item Gewebte Tapeten vs. bemaltes Papier
  \item Hoch schlafen vs. auf dem Boden
  \item Betten sind stets ausgebreitet vs. werden eingerollt
  \item Während des Baus wird Holz angefertigt vs. erst alles bearbeitet 
  \item Künstliche Insel: holländische Handelsfaktorei (alles nichtchristliche wurde gehandelt)
  \item Mit der Öffnung wird der Ton gegenüber Japan verächtlicher
  \item Basil Hall Chamberlain: Things Japanese (1890) \url{https://en.wikisource.org/wiki/Things_Japanese}
\end{itemize}

\subsection{Zürn: Gärten}

\begin{itemize}
  \item Gartenkunst auch häufig Diletantismus
  \item Dieses Seminar wird historischer Sein als erstes Seminar, VOträge sind also unter diesem Gesichtspunkt immer zu betrachten
  \item Eigenschaften: Rechtwinklig begrenzt (Zaun, Mauer) | Kontinuierliche Pflege | Ordnung, Gliederung
  \item Teegarten
  \item Edward S. Morse: Japanese Homes and their Surroundings
  \item Marie Luise Gothein: Geschichte der Gartenkunst
  \item 100 gardens of tokyo
  \item Prinz Genji, erster Roman
  \item The Gardens of Japan von Teiji Itoh 
\end{itemize}


\subsubsection{Ästhetisches Grundvokabular für Gartenbeschreibung}

\begin{itemize}
  \item 島 (しま): Garten/Insel "`von einer Wildnis abgetrenntes Land"' (heute nur noch \emph{Insel})
  \item 庭 (にわ): Garten Hof (im 心電図くり Stil der Heian Zeit)
  \item 山水 wörtl. sinojapan. für Landschaft
  \item 仮山水 - かさんすい scheint wie eine Landschaft
  \item 仮 - か - Schein
  \item 自然 - Kreatur die sich selbst geschaffen hat - Natur (erst seit 19. Jhd.)
  \item 生得の山水 (しょうとくのさんすい) lebensechte Landschaft
  \item 天地 - tian die chinesische Bezeichnung für das Natürliche und das Übernatürliche - Himmel und Erde
\end{itemize}



\section{Architektonische Moderne: Villa Katsura\\(03.05.18)}

\begin{itemize}
  \item Bruno Taut: "`Weltwunder"'
  \item Hideyoshi stirb 1594 - Machtvakuum
  \item Dekret: Jede Provinz darf nur noch ein Schloß haben
  \item Samurai siedeln sich um dieses Schloß um damit: weniger Militär und Tokugawas Machposition gesichert ist
  \item Toshihito inspiriert durch Prinz Genji, baut (aus Geldmangel nur) einen Teepavillon am Fluß Katsura - erweitert später um Garten, Teich
  \item Wurde sehr berühmt und sogar der Kaiser ließ sich inspirieren
  \item Bambuszaun um gesamtes Grundstück
  \item Wurde häufig renoviert
  \item Bruno Taut: Der Begriff "`Palast"' vs. "`Villa"'
  \item Taut: Es gibt sowohl extrem dekorierte Häuser, also auch extrem minimalistsiche in Japan - Das Sanssouci der japanischen Kaiser, Akropolis Japans
  \item Wo ist der Übergang zwischen Nachahmen zu Inspiration zu Tradition 
  \item 
\end{itemize}



\end{document}
