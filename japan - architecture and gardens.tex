\documentclass[emulatestandardclasses]{scrartcl}
\usepackage{graphicx}
\usepackage{CJKutf8} % japanese
\usepackage{color}
\usepackage[ngerman]{babel}
\usepackage{hyperref}
\usepackage{fullpage}
\usepackage[utf8]{inputenc}
\usepackage{calc} 
\usepackage{enumitem}
\usepackage{titlesec}
\date{\vspace{-3ex}}
\begin{document}

\title{
	\includegraphics*[bb=0 0 720 200, width=0.72\textwidth]{ErstesSem/images/hu_logo.png}\\
	\vspace{25pt}
	Japan und die architektonische Moderne
	\\
	Der japanische Garten}
\subtitle{\vspace{10pt}
			Prof. Dr. Kai Kappel, Prof. Dr. Salomon, Dr. Tina Zürn\\
			Proseminar SS 2018\\
          Institut für Philosophie\\ 
          Humboldt Universit"at zu Berlin}
\author{Lennard Wolf\\
        \small{\href{mailto:lennard.wolf@student.hu-berlin.de}{lennard.wolf@student.hu-berlin.de}}}
\maketitle
\begin{abstract}

\noindent \textbf{Japan und die architektonische Moderne.\\ Rezeptions- und Transformationsgeschichte}\newline
\indent Gegenstand dieses gemeinsamen Seminars des Instituts für Asien- und Afrikawissenschaften und des Instituts für Kunst- und Bildgeschichte ist die Bau-, Wohn- und Erinnerungskultur Japans. Dabei interessieren uns insbesondere Aspekte der Rezeption und Transformation traditioneller Wohn- und Lebensformen des vormodernen Japan von der Mitte des 19. Jahrhunderts bis in die Gegenwart. Die Frage nach dem kulturell Eigenen und dem Anderen soll besonders vor dem Hintergrund des Modernediskurses, zunehmender Globalisierungseffekte und transkultureller Austauschprozesse gestellt werden. Das Seminar bezieht auch Orte in Berlin (etwa die Mori-Ôgai-Gedenkstätte) ein.\\
\textbf{Der japanische Garten | Geschichte, Sinngehalt, Rezeption}\\
\indent Gegenstand dieses gemeinsamen Seminars des Instituts für Asien- und Afrika­wissenschaften und des Instituts für Kunst- und Bildgeschichte ist die Genese, der Sinngehalt, die Nutzung und die Verbreitung des japanischen Gartens. Dieser verarbeitet seit der Meiji-Zeit auch westliche Vorstellungen von Gartenkunst und ist in Europa und Amerika, wo in der Moderne zahlreiche Anverwandlungen entstehen, Spiegelbild der Vorstellungen ostasiatischer Kultur. Daher ist in besonderer Weise die mediale Vermittlung des japanischen Gartens zu betrachten.

\end{abstract}
\newpage

\tableofcontents
%\listoffigures
\newpage


\section{Architektonische Moderne: Einführungssitzung\\(19.04.18)}

\subsection{Organisation}

\begin{itemize}
  \item Moodle PW: Katsura
  \item Hausarbeit Abgabe 30.09.
  \item Seminararbeit??
  \item "`maximal verstörend"'
  \item Aufbau Gesellschaft und ?
  \item 1 Monat Anmeldezeitraum für MAP
\end{itemize}

\subsection{Einleitendes}

\begin{itemize}
  \item Rezeption und Transformation: Japaner*innen, die die westliche Moderne kennenlernen
  \item Rezeptionsprozess beginnt schon bei der Auswahl der Medien etc. 
  \item Wieviel Europa ist dabei, wenn nach Japan zurückgekehrt wird
  \item Für Regionalstudien Asien Afrika: Was bedeutet das alles für die Geschichte des Übergangs in die Moderne, wird häufig eurozentristisch erzählt, hier die Möglichkeit, aus einer anderen Sicht zu betrachten
  \item Japan wird immer wieder neu entdeckt (Limitierung des Bildes?)
  \item 17.5. Treffen an den Gärten der Welt
  \item Tetsuro Yoshida: Das japanische Wohnhaus
\end{itemize}

\subsection{Vortrag}

\begin{itemize}
  \item \emph{Capsule Tower} mit 12.7. was ist metabolismus strukturalismus; Gruppe: Hannah Glauner, thema: ????; ? thema: ocean city and space; ??
  \item Koolhaas, Rem und Hans Ulrich Obrist (2011). Project Japan – Metabolism Talks
  \item Lin, Zhongjie (2010). Kenzo Tange and the Metabolist Movement
  \item Boyd, Robin (1968). New Directions in Japanese Architecture
  \item \url{http://www.spiegel.de/artikel/a-947404.html}
  \item Handout kann auch hinterher abgegeben werden (diskussion aus der stunde mit rein); bis zu 1-2 Wochen später
  \item Zentrale Aussagen, nicht mehr als 2-3 Seiten
\end{itemize}


\section{Japanische Gärten: Einführungssitzung\\(19.04.18)}

\begin{itemize}
  \item Moodle PW: Zen
  \item Jiro Takei, Marc Peter Keane: Sakuteiki oder die Kunst des japanischen Gartens (Handbuch über Gartengestaltung aus der Heian-Zeit)
  \item Sir William Temples Sharawadgi-Begriff ("`beauty without order"')
  \item 鳳凰殿 hououden (Chicago 1893) ?
  \item Anfragen ob Namen in Kanji auch in die Präsentationen kommen könnten?
\end{itemize}


\end{document}
