\documentclass[a4paper]{article}

\usepackage[ngerman]{babel}
\usepackage{hyperref}
%\usepackage{fullpage}
\usepackage[utf8]{inputenc}
\usepackage[hang]{footmisc}
\setlength{\footnotemargin}{-0.8em}
\usepackage[style=authoryear-ibid,natbib=true]{biblatex}
\addbibresource{wahrnehmung.bib}
\date{\vspace{-2ex}}
\begin{document}

\title{Natur der Wahrnehmung – Kunst der Täuschung\vspace{-0.5ex}}

\author{Lennard Wolf}

\maketitle

In einem Seminar zu Merleau-Pontys "`Phänomenologie der Wahrnehmung"' stieß ich das erste Mal auf das Konzept des Leibes, das mich bis heute nicht mehr los lässt. Dieses knüpft an heutigen Vorstellungen von \emph{embodiment} in den Kognitionswissenschaften an, dass also Bewusstsein immer auch verkörpert ist. Mit der daraus resultierenden Untrennbarkeit von Denken und Wahrnehmen, sowie der neuen Stellung der Wahrnehmung als allerersten Schritt des Seins und der Erkenntnis, setze ich mich derzeit in einem Seminar zur "`Erfahrung der Realität durch Widerstand"' weiter auseinander. Zentral ist bei diesem die Frage, wie die Unterscheidung zwischen "`real"' und "`nicht real"' erkenntnistheoretisch und entwicklungspsychologisch zu erklären ist: Wie kommt es überhaupt zu der Vorstellung, dass etwas gar nicht wirklich \emph{so} ist, wie es uns erscheint? Was \emph{ist} Wahrnehmung überhaupt, wenn es uns sowohl "`Reales"' als auch "`Nichtreales"' liefern kann? Wieso beharrt die modernere Philosophie darauf, dass meine Finger, die die Tischoberfläche streichen und fühlen, den "`Tisch an sich"' gar nicht berühren? Derlei Fragen zum Thema Schein und Wirklichkeit befassen sich mit der Möglichkeit des Wahrnehmens und Vorstellens des Virtuellen, also des Nichtwirklichen, und dadurch auch mit der Möglichkeit der Täuschung.  

Der Imperativ des Gehirns, jede Erfahrung in das Bild einer konsistenten Welt einzufügen, bringt hervor, dass wir zum einen manchmal (unbewussterweise) uns selbst täuschen, indem wir Dinge ignorieren, vergessen oder einbilden, und dass wir zum anderen Täuschung\-en von außen mit ähnlichen Tricks so uminterpretieren, dass sie im Einklang bleiben mit anderen Wahrnehmungen und unserer Erfahrung. Dass der Mensch versucht, aus der Masse an Sinneswahrnehmungen um jeden Preis ein sinnvolles Gesamtbild zusammenzufügen, halte ich für das Thema des \emph{embodiments} in virtuellen Welten als von zentraler Bedeutung. Ein grundlegendes Experiment war hierzu die Gummihand-Illusion\footnote{Vgl. \Citet{botvinick1998rubber}.}, in der durch simultane Stimulation der eigenen (verdeckten), und einer Gummihand letztere als eigene wahrgenommen wird. Auf solche Weise wird heute versucht, die bisher nur audiovisuellen virtuellen Erfahrungen mit \emph{Oculus Rift} und Co. zu vollständigen Leibeserfahrungen weiterzuentwickeln\footnote{Vgl. \Citet{slater2009inducing}.}. Veröffentlichungen\footnote{Eine vollständige Liste ist unter \url{http://publicationslist.org/melslater} zu finden.} des \emph{Experimental Virtual Environments for Neuroscience and Technology Labs} der Universitat de Barcelona beschreiben erfolgreiche Experimente zu virtuellem \emph{embodiment}, in denen der virtuelle Körper zum Beispiel den eines Kleinkindes\footnote{Vgl. hierzu \Citet{Banakou2013}.} oder den einer anderen Hautfarbe\footnote{Vgl. hierzu \Citet{peck2013putting}.} annimmt. Bei ersterem passt sich unter anderem die Größenwahrnehmung von Objekten, sowie das Bewegungsverhalten dem Körpergefühl an, während bei letzterem zur Einfühlung in das Zugehörigsein zu einer ethnischen Minderheit beigetragen werden kann. Forschung zu Ganzkörpererfahrungen in virtuellen Welten bringt aber nicht bloß bessere Technologien für die Unterhaltungsindustrie hervor, sondern ermöglicht beispielsweise auch neue Ansätze für die experimentelle Kognitionswissenschaft, neue Behandlungsweisen in der psychiatrischen Therapie, als auch neue Denkanstöße für die Phänomenologie, und die Wahrnehmungsphilosophie im Allgemeinen. 

Neben der Wahrnehmung des Körpers in der virtuellen Welt interessiert mich weiterhin, ob und wie im Zeitalter der visuellen Medien die Körperlichkeit sowohl im Bezug auf sich selbst, als auch im Bezug auf die Welt und andere Menschen beeinflusst wird. Klassisch verläuft eine Begegnung mit einer anderen Person über das Wahrnehmen des anderen Leibes, heute aber können wir miteinander kommunizieren ohne dass die andere Person anwesend ist. Kommt es dadurch allmählich zu einer "`Entsinnlichung"'\footnote{\Citet[S. 278]{fuchs} beispielsweise vertritt diese Ansicht.}, einer "`Entkörperung"' des Menschen? Hat die zunehmende Zeit, die wir in digitalen Räumen verbringen statt in der sinnlichen Natur Entfremdung von der Wirklichkeit zur Folge?\footnote{Vgl. hierzu zum Beispiel \Citet[S. 273]{fuchs2}.}   

Als Student der Philosophie interessiert mich besonders, was der virtuelle und der virtualisierte Leib für gängige Theorien des \emph{embodiments} und des Leibes bedeutet. Dieses Thema, wie auch die Wahrnehmung des selbst und anderer in Zeiten des Internets, passt, so denke ich, besonders in das Projekt 3 zur Philosophie und Kulturgeschichte der Wahrnehmung, weshalb ich mich hiermit für dieses an erster Stelle bewerbe. Die Projekte 4 und 2 passen aber ebenso sehr gut, weshalb ich diese alternativ vorschlagen möchte. Mit meinem Wissen über phänomenologische Sichtweisen zur Wahrnehmung, als auch meinen Erfahrungen aus dem Studium des IT Systems Engineering an einem Institut, das viel zum Thema \emph{virtual reality} forscht\footnote{Das \emph{Human Computer Interaction Lab} des HPI beschreibt sich beispielsweise folgendermaßen: "`The objective of our work is to unify the virtual world of the computer with the physical world of the user into a single space."' (Siehe \url{hpi.de/baudisch})}, habe ich einen einzigartigen Blick auf das Gebiet der Wahrnehmung in der Gegenwart und der (nahen) Zukunft. Da dieses Thema ohne praktischen Kontext nicht denkbar ist, trotz alledem aber theoretischer Auseinandersetzung bedarf, hätte ich großes Interesse an einem interdisziplinären Projekt, in dem ich sowohl meine Ideen und Gedanken einbringen und diskutieren, als auch durch andere Studierende aus fremden Disziplinen meine Vorstellungen erweitern und hinterfragen kann. 

\newpage
\nocite{*}
\printbibliography

\end{document}