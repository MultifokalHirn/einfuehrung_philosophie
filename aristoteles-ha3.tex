\documentclass[a4paper]{article}
\usepackage{graphicx}
\usepackage{fullpage}
%\usepackage{parskip}
\usepackage{color}
\usepackage[ngerman]{babel}
\usepackage{hyperref}
\usepackage{calc} 
\usepackage{enumitem}
\usepackage{titlesec}
\usepackage{bussproofs}
\usepackage[export]{adjustbox}
%\pagestyle{headings}

\titleformat{name=\section,numberless}
  {\normalfont\Large\bfseries}
  {}
  {0pt}
  {}
\date{\vspace{-3ex}}
\begin{document}

\title{
    \vspace{-30pt}
	\includegraphics*[width=0.1\textwidth,left]{ErstesSem/images/hu_logo2.png}\\
	\vspace{-10pt}
	Aristoteles: Nikomachische Ethik}
\author{Lennard Wolf\\
        \small{\href{mailto:lennard.wolf@student.hu-berlin.de}{lennard.wolf@student.hu-berlin.de}}}
\maketitle
\vspace{-4pt}

\section*{Lekt"urenotiz -- Handlungstheorie I: Verantwortung [III 1-8]}
\large

\textbf{[III 1]} Gewolltes: Lob, Ungewolltes: Mitleid $\rightarrow$ Unterschied muss klar gemacht werden, auch f"ur Rechtsprechung n"otig | Unterscheidung legt Wert auf Zeitpunkt der Handlung | Gewolltes: Wenn Handlung ihren Ursprung in Person selber findet; Ungewolltes: aus Zwang (Beweggrund \emph{au"serhalb}), Unwissenheit; Problematisch: aus Angst (aber eher gewollt) | F"ur Niedriges erh"alt man Lob, wenn es mit teurem Ziel aufgewogen ist $\rightarrow$ Unedel ist der, der Niedriges aufgrund von unedelen Zielen tut.\newline

\noindent \textbf{[III 2]} Handeln aufgrund von Unwissenheit bez"uglich bestimmter Bedingungen (besonders Mittel und Zweck) ist nicht gegen das Wollen, solches Handeln ist \emph{ohne Wollen}, was sogar \emph{gegen das Wollen} sein kann | Pers"onliche Unwissenheit ist der Grund f"ur ungerechtes und schlechtes Handeln \newline

\noindent \textbf{[III 3]} Gewolltes Handeln ist also ein solches, dessen Ursprung in der Person selber liegt, die zudem alle Bedingungen der Situation kennt | Handlung aus Begierde und Erregung ist auch gewollt (sonst k"onnten Tiere nicht wollen)\newline

\noindent \textbf{[III 4]} Vorsatz $\subset$ Gewolltes | Vors"atze sind nicht Teil des Vernunftlosen (Begierde/Erregung etc.) und auch keine W"unsche oder Meinungen | Vors"atze gehen mit "Uberlegen \emph{logos} und Denken einher\newline

\noindent \textbf{[III 5]} "Uberlegen bezieht sich auf Dinge, die in unserer Macht stehen (durch Handeln), daher nicht die Ziele, sonder jenes, das zum Ziel f"uhrt ("`Welche Handlung ist wie die Beste?"') | Grundlage ist die Untersuchung auf M"oglichkeit | Der Vorsatz bezieht sich auf die \emph{"uberlegte Handlung}\newline

\noindent \textbf{[III 6]} W"unsche beziehen sich aber auf die Ziele, welche jeweils ein (subjektives) Gut sind $\rightarrow$ wird durch eine Handlung ein ungewolltes Ziel erreicht, kann dieses nicht gew"unscht gewesen sein \newline

\noindent \textbf{[III 7]} Ziel ist gew"unscht, entsprechende Handlung vors"atzlich $\rightarrow$ Handlung ist \emph{gewollt} | Da wir vors"atzlich (nicht) Handeln, liegt die Tugend bei uns und damit, ob wir gut sind oder schlecht | Niemand ist gl"uckselig per Zufall, jeder ist schlecht aufgrund des eigenen Wollens | Dies ist im Recht repr"asentiert: Strafe f"ur gewolltes, bewusstes Handeln (man kann f"ur Unwissenheit verantwortlich sein) | Wer nur w"unscht, tugendhaft zu sein, wird dadurch nicht tugendhaft | Charakterliche und k"orperliche Schlechtheit (Untugendhaftigkeit, Krankheit) geht aus Gewolltem hervor, da man Vorkehrungen h"atte treffen k"onnen | Wenn die Tugenden auf dem Wollen beruhen, beeinflussen wir die Dispositionen, und kontrollieren damit das Wollen\newline

\noindent \textbf{[III 8]} Zusammenfassung: Tugend ist Mitte der Dispositionen, welche durch das in unserer Kontrolle stehende Handeln ver"andert werden\newline


\noindent \textbf{Frage 1:} Wieso kann Aristoteles immer Lob und Tadel der anderen als Argument f"ur die Kategorisierung von Begriffen wie "`Vosatz"' und "`Tugend"' anf"uhren? \newline

\noindent \textbf{Frage 2:} Warum ist es nach Aristoteles bei einer Handlung aus Angst nicht ganz deutlich, dass sie gewollt ist? Umst"ande zwingen einen doch immer zur Handlung im Leben, was macht die Angst da anders? 
\end{document}
