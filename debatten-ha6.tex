\documentclass[a4paper, 12pt]{article}
%\usepackage{CJKutf8} % japanese
\usepackage{graphicx}
\usepackage{hyperref}
\usepackage{fullpage}
%\usepackage{parskip}
\usepackage{color}
\usepackage[ngerman]{babel}
\usepackage{hyperref}
\usepackage{calc} 
\usepackage{enumitem}
\usepackage[utf8]{inputenc}
\usepackage{titlesec}
%\pagestyle{headings}
\usepackage{setspace} %halbzeilig
\usepackage[style=authoryear-ibid,natbib=true]{biblatex}
\usepackage[hang]{footmisc}
\setlength{\footnotemargin}{-0.8em}
%\bibliographystyle{natdin}
\addbibresource{debatten-ha6.bib}
\DeclareDatamodelEntrytypes{standard}
\DeclareDatamodelEntryfields[standard]{type,number}
\DeclareBibliographyDriver{standard}{%
  \usebibmacro{bibindex}%
  \usebibmacro{begentry}%
  \usebibmacro{author}%
  \setunit{\labelnamepunct}\newblock
  \usebibmacro{title}%
  \newunit\newblock
  \printfield{number}%
  \setunit{\addspace}\newblock
  \printfield[parens]{type}%
  \newunit\newblock
  \usebibmacro{location+date}%
  \newunit\newblock
  \iftoggle{bbx:url}
    {\usebibmacro{url+urldate}}
    {}%
  \newunit\newblock
  \usebibmacro{addendum+pubstate}%
  \setunit{\bibpagerefpunct}\newblock
  \usebibmacro{pageref}%
  \newunit\newblock
  \usebibmacro{related}%
  \usebibmacro{finentry}}

%\titleformat{name=\section,numberless}
%  {\normalfont\Large\bfseries}
%  {}
%  {0pt}
%  {}
\date{\vspace{-3ex}}
\begin{document}

\title{\vspace{5ex}
	\includegraphics*[bb=0 0 720 200, width=0.72\textwidth]{ErstesSem/images/hu_logo.png}\\
	\vspace{30pt}
	\scshape\LARGE{Zusammenfassung VI}\\\Large{Tribalismus und Ethnizität in Afrika}\vspace{20pt}}
	


\author{Regionalwissenschaftliche Debatten\\
	\vspace{7pt}
          Dozent: Prof. Dr. phil. Vincent Houben\\\vspace{4pt}Lennard Wolf\\
        \small{Matrikelnummer: 583052}\\
        \small{E-Mail: lennard.wolf@hu-berlin.de}}

        %\href{mailto:lennard.wolf@student.hu-berlin.de}{lennard.wolf@student.hu-berlin.de}}}      

\maketitle

\vspace{\fill}

\begin{minipage}[]{0.92\textwidth}
    \centering
    \onehalfspacing
    \large   
    06. Februar 2018\\
    Wintersemester 17/18

    \vspace{-20mm} 
\end{minipage}%
\thispagestyle{empty}
\newpage
%\clearpage
%\thispagestyle{empty}
%\tableofcontents
%\newpage
\setcounter{page}{1}

\begin{onehalfspace} 

%\noindent\textbf{Zusammenfassung}

% Überlegen Sie sich Zwischenüberschriften zu den einzelnen Abschnitten des Textes!

% Erarbeiten Sie sich aus jedem Abschnitt zwei Kernaussagen!

% Versuchen Sie, eine Definition des Konzeptes "Orientalismus" zu formulieren!

\noindent 
\emph{Tribalismus und Ethnizität in Afrika - ein Forschungsüberblick} ist ein 1995 erschienener Artikel, in dem Carola Lentz anhand von Fallstudien die Begriffe des \emph{tribes} und der Ethnizität in Afrika beleuchtet, und so einen Überblick über den Stand der Forschung zu dem Thema gibt.

Der Kontroverse Begriff "`Ethnizität"' ist nicht als Überbleibsel von "`falschem Bewusstsein"' wegzutheoretisieren, und muss daher genauer kulturhistorisch eingeordnet werden. Dafür stellt Lentz drei Richtungen der afrikabezogenen Ethnizitätsforschung vor: Studien zu \emph{tribalism}, die Diskussion um politisierte Ethnizität und nationalstaatliche Integration und Forschung zur kolonialen \emph{invention of tradition}. 

Der Begriff Ethnizität soll als problematisch erkannte Begriffe wie \emph{tribe} ersetzen, die einen Beiklang der Primitivität tragen. Ethnizität aber trägt in sich auch immer ein Moment des \emph{otherns}. Zudem versucht er, verschiedene Formen der sozialen Identität über einen Kamm zu scheren. Es gibt eine Kontroverse zwischen zwei verschiedenen Ansätzen für das Verständnis von Ethnizität. Die "`Konstruktivisten"' sagen, Ethnizität sei keine überhistorische Gruppenzugehörigkeit, sondern eine sehr flexible Kategorie. "`Essentialisten"' meinen wiederum, Ethnizität sei ein Zusammenschluss von Leuten mit "`gegebenen"' biologischen und/oder kulturellen Gemeinsamkeiten. Diese Kontroverse ist aber am Ende politischer Natur, und so werden in der Forschung häufig vermischende Positionen eingenommen. Urbaner \emph{tribalism} durch Arbeitsmigration zeigt besonders auf, wie flexibel Ethnizität gehandhabt wird, während Forschung zu ländlichen \emph{tribes} eher den essentialistischen Ansatz stützen.

Viel Forschung besteht zu dem politischen Moment des Ethnizitätsbegriffs, nach der dieser als \emph{politische Ressource} genutzt wird. Dies kam daher zustande, dass Statuskriterien und materielle Ansprüche sich in der Moderne immer mehr einander anglichen und eine neue Form der Differenzierung herangezogen werden musste. Nachdem Ethnizität als Kategorie dann erst einmal eingeführt wurde, perpetuierte sich diese als immer wichtiger werdende Komponente der Herrschaftsverhältnisse. So wurde aus einem Deutungsmuster der komplexen Realität ein realer, handlungsrelevanter Faktor. In Regionen, wo sich die zentralisierte Macht auflöste, existierte die Ethnizitätskategorie weiter, was zeigt, dass sie nicht \emph{nur} als Instrument von Eliten aufgefasst werden kann.

Einer Forschungsrichtung zufolge ist der Begriff der \emph{tribes} in einem Verwaltungsakt von Kolonialmächten \emph{erfunden} worden und dann von Afrikanern übernommen worden. Daraus ergab sich eine sogenannte "`\emph{invention of traditon}"', die dem europäischen, neotraditionalistischen Weltbild angepasst war. Andere Stimmen wollen wiederum lieber von eine "`\emph{imagination of traditon}"' reden, da der Begriff der \emph{invention} dem europäischen Einfluss einen zu großen Stellenwert beimisst. Dass sich das Bild des durch Stämme organisierten Afrikas so durchsetzen konnte ist darin begründet, dass vor der Ankunft der Kolonisten keine wirklich offizielle Ausformulierung der politischen Verhältnisse existierte. 

Es zeigt sich dass der Ethnizitätsbegriff nur lokal und historisch eingegrenzt und kontextualisiert produktiv verstanden werden kann. Weder die Position, dass Ethnizität und \emph{tribes} reine europäische Erfindungen seien, noch dass sie absolut die Wahrheit wiederspiegeln sind nur für sich genommen haltbar. Die Forschung soll aber weiteren Essentialisierungsprozessen entgegenwirken.

%Sie bleibt ja aber offensichtlich vollkommen in der Moderne hängen (was ja nicht schlimm ist)
%Moderne ist ein edlritch das über sich selbst nachdenkt, aber ultimative angst vor seinem tod hat


\end{onehalfspace}
\nocite{*}
%\bibliography{merleau-ponty-essay}
\printbibliography
\end{document}
