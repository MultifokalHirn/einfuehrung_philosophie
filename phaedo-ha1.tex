\documentclass[a4paper]{article}
\usepackage{graphicx}
\usepackage{fullpage}
%\usepackage{parskip}
\usepackage{color}
\usepackage[ngerman]{babel}
\usepackage{hyperref}
\usepackage{calc} 
\usepackage{enumitem}
\usepackage{titlesec}
\usepackage{bussproofs}
\usepackage[export]{adjustbox}
%\pagestyle{headings}

\titleformat{name=\section,numberless}
  {\normalfont\Large\bfseries}
  {}
  {0pt}
  {}
\date{\vspace{-3ex}}
\begin{document}

\title{
    \vspace{-30pt}
	\includegraphics*[width=0.1\textwidth,left]{ErstesSem/images/hu_logo2.png}\\
	\vspace{-10pt}
	Plato's Phaedo}
\author{Lennard Wolf\\
        \small{\href{mailto:lennard.wolf@student.hu-berlin.de}{lennard.wolf@student.hu-berlin.de}}}
\maketitle
\vspace{-4pt}

\section*{Reading Notes - Week 1}
\large

\textbf{Question} I am wondering why Socrates so ardently insists on not knowing anything in his apology, yet defends his feelings towards death by arguing with apparently "`undeniable"' truths, such as the continuing existence of the soul after death.\newline

\noindent \textbf{Thoughts} For example, Socrates seems to have a strong opinion about people \emph{remembering} facts instead of learning them, since he believes to have very good reason to do so. Either he does not actually think his arguments to be \emph{The Truth}, which would be in line with his common attitude, or he really believes in them, in which case I am confused, as a big point in his defense speech is, that it is his godly duty to find truth by constantly questioning everything and everyone, including himself. So my point is, how come Socrates suddenly comes across as being so sure about his beliefs around the eternal soul, whilst in the past not ever leaving any view unquestioned? 

Moreover, his argument at \emph{68d} for example about brave people fearing something more than death appears to be a rather thin one, as brave people might not actually fear something more than death, but instead put more value on the possible outcome of their action, than on their life. Why would he use such an unsatisfactory example, if he really wishes to convince people? Why is he \emph{so} sure about all this? It sort of feels disingenuous. He appears to have a very different mindset than for example at the end of the \emph{Laches} dialogue, in which all participants, \emph{including Socrates}, agree to go back to school. I feel it could have been a very satisfying last dialogue, if he himself was the one led to conclusions through socratic questioning, instead of him making others agree with his ideas.  
\end{document}
