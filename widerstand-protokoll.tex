\documentclass[a4paper, emulatestandardclasses]{scrartcl}
\usepackage{graphicx}
\usepackage{fullpage}
%\usepackage{parskip}
\usepackage{color}
\usepackage[ngerman]{babel}
\usepackage[utf8]{inputenc}
\usepackage{hyperref}
\usepackage{calc} 
\usepackage{enumitem}
\usepackage{titlesec}
\usepackage{bussproofs}
\usepackage[export]{adjustbox}
%\pagestyle{headings}

\titleformat{name=\section,numberless}
  {\normalfont\Large\bfseries}
  {}
  {0pt}
  {}
\date{\vspace{-3ex}}
\begin{document}

\title{
    \vspace{-30pt}
	\includegraphics*[width=0.1\textwidth,right]{ErstesSem/images/hu_logo2.png}\\
	\vspace{-10pt}
	Wahrnehmung bei Scheler I (23.11.17)}%}\\\vspace{10pt}\small{Lennard Wolf}}
	\subtitle{Die Erfahrung der Realität durch Widerstand (PS, WS 17/18)\\
          Dozent: Dr. Matthias Schlo"sberger\\
          Protokollant: Lennard Wolf}
\maketitle
\vspace{-40pt}

\section*{Vorbemerkungen}
\textbf{Korrekturen von Missverständnissen}

\begin{itemize}
  \item Besserer Begriff für was Husserl mit der "`reinen Empfindung"' meint: "`\emph{Erfahrung in ihrer Selbstgegebenheit}"'. Dieser Begriff signalisiert besser, dass wir auf die fundamentalste Ebene der Erfahrung zurück gehen.
  \item Die Phänomenologische Reduktion ist nicht gerade nur eine Methode, sondern vielmehr eine philosophische \emph{Einstellung}, die als Gegenentwurf zur cartesischen Zweifelsbetrachtung (im Modus der Meditation) zu betrachten wäre. In dieser Einstellung klammern wir alles, was wir schon zu wissen meinen (zB was ein Körper ist, die Trennung von Subjekt und Objekt etc.), ein. 
  \item Die Uneinigkeit der Phänomenologen herrschte bezüglich der Reduktion. Husserl nennt seine die \emph{transzendentale} Reduktion, weil er die Seinsgeltung der (zufälligen) Welt einklammert, um zum absoluten Wissen zu gelangen (später "`idealistische Reduktion"').
\end{itemize}

\noindent\textbf{Anmerkungen zur Geschichte der Phänomenologie}\newline

Wichtige Namen in der 1. Generation der Phänomenologie: Edmund Husserl, Max Scheler, Moritz Geiger, Adolf Reinach, Alexander Pfänder, 2. Generation: z.B. Heidegger mit \emph{Sein und Zeit}.\newline

\noindent \emph{Was zeichnet die phänomenologische Bewegung aus, was ist ihr Grundgedanke?}

Sie ist unter anderem eine erkenntnistheoretische Bewegung, die der bisherigen Erkenntnistheorie vorwirft, schon zu sehr von existierenden Begriffen und Kategorien beeinflusst zu sein. Ihre Theorie des Bewusstseins besagt, dass das Bewusstsein immer intentional ist, dass es immer schon auf etwas gerichtet ist. Folglich gibt es kein Bewusstsein, das erst bei sich ist, und dann auf etwas gerichtet wird. Wir kommen also gar nicht zur welt, wir sind immer schon zur Welt | bewusst sein heißt zur Welt sein.\newline

\noindent\textbf{Weitere Anmerkungen}

\begin{itemize}
  \item Intentionalität ist ein schwieriger Begriff. Husserl: "`Mannigfache Stufen der intentionalität"': Es gibt Gefühle, die bestimmte Intentionalität haben (z.B. man liebt \emph{jemanden}), aber auch nichtintentionale Gefühle (z.B. Traurigkeit, wo man nicht weiß woher das kommt). Husserl und Scheler meinen, dass dies nicht bedeutee, dass diese Traurigkeit nicht intentional ist! Intentionalität bedeutet gerade \emph{nicht}, dass das bezogen sein auf etwas in Urteilsformen übersetzt werden kann. Gerichtetheit, bzw. Intentionalität kann auch unbestimmt sein.
  \item Scheler: Reduktion ist nicht transzendental, die Welt kann nicht einfach ausgeklammert werden. Es gibt bestimmte asiatische Seelentechniken, die dies vermörgen, doch da passiert etwas anderes, als was Husserl sich darunter vorstellt. Denn wenn wir die Welt verschwinden lassen, dann verschwinden wir mit ihr. 
  \item Stattdessen vernachlässigt Scheler die Reduktion auf geschickte Weise und zeigt u.A., dass Erkenntnis vorbegrifflich ist, da der menschliche \emph{Geist} die Fähigkeit zur Ideation hat. Ein berühmtes Beispiel hierfür: Der Prinz Buddha wächst von allen Einflüssen der gemeinen Gesellschaft abgeschirmt im elterlichen Palast auf. In diesem abgeschirmten leben wird er nicht konfrontiert mit dem Phänomen der Armut und dem Phänomen der Krankheit. Er kennt also die Begriffe gar nicht und er kennt die Phänomene nicht, weil er sie nie erlebt oder mitbekommen hat. Der Prinz verlässt eines Tages das Elternhaus und sieht am Wegesrand einen armen Menschen und einen kranken Menschen. Scheler meint nun, dass der Prinz sofort das Wesen von Armut und Krankheit vor sich sieht. Ein Beispiel reicht also aus, um das Wesen der Armut und das Wesen der Krankheit zu erfassen. Eine so gedachte phänomenologische Erkenntnis hieraus ist, dass Erkenntnis zum einen nicht ohne Erfahrung möglich ist, zum anderen dass die Begriffe nicht entstehen, indem wir ganz viele Erfahrungen machen und dann anhand deren Durchschnittswerte ein abstraktes Wesen generieren. Vielmehr erkennen wir schon in der ersten Erfahrung das Wesen, und erkennen es bei der nächsten Begegnung wieder. 
  \item Der Prinz ist ja nicht mit mit den Begriffen bekannt und wenn er also nach dem Wesen der Armut gefragt wird, kann er Reduktion machen, dann so tun als ob er das alles noch nicht kennen würde und die Dinge für sich selbst reden lassen. (?)
  \item Jede Erfahrung des Soseins gründet in der Erfahrung des Daseins von etwas. Wirklichkeitserfahrung (und damit Widerstandserfahrung) ist folglich fundamental für das Erfahren von Sosein. Husserl meint es sei umgekehrt: erst wenn ich die Widerständigkeit der Welt aufhebe, kann ich zum Sosein kommen!
\end{itemize}

\noindent\textbf{Literaturempfehlungen}

\begin{itemize}
  \item Herbert Spiegelberg: \emph{The Phenomenological Movement - A Historical Introduction}. (Band 1 ist eher relevant)
  \item Ferdinand Fellman: \emph{Phänomenologie zur Einführung}. (Kürzer als Spiegelberg)
  \item Dan Zahavi: \emph{Phänomenologie für Einsteiger}. (Sehr Husserl-zentriert)
  \item The Open Commons of Phenomenology (\url{ophen.org})
\end{itemize}

\section*{Textbesprechung}

\emph{Warum ist die Frage nach der Empfindung so zentral?}

Weil er die Grundlage bilden muss für die Frage, wie wir Wissen über der Welt haben können.\newline

\noindent \emph{Was versteht Scheler unter dem Begriff "`Empfindung"'?}

Zum einen gibt es für Scheler die \emph{reine} Empfindung, bei der es sich um eine Fiktion handelt. Zum anderen gibt es die Empfindung als \emph{Teil} von Erlebnissen, Erfahrungen, \emph{cogitationes}, sie alle haben einen Empfindungscharakter ("`Wie fühlt es sich denn an?"'). Empfindungen sind für sie immer mitkonstitutiv.

Begründung für die Fiktionalität der reinen Empfindung: Es gibt keine eineindeutige Beziehung zwischen Reiz und Empfindung. Die "`Konstanzannahme"' lässt sich widerlegen an folgendem Beispiel: Mir streichelt jemand über den Kopf und ich denke es ist jemand den ich mag, doch plötzlich sehe ich im Spiegel vor mir dass es eine ganz unangenehme Person ist. Meine Empfindung ändert sich schlagartig, der Reiz bleibt jedoch das selbe.

Siehe hierzu weiter im Text S. 322 ff.
\newline

\noindent \emph{Warum kann es nicht sein, dass wir "`zu Beginn"' noch nicht zur Welt sind, und dann zur Welt kommen?}

Da sich bei solch einer Theorie große Zirkelprobleme auftun.\newline

\noindent \emph{Wie verhält es sich dann alternativ?} 

Zwei Antworten: 

1. Das Auf-etwas-bezogen-sein ist immer schon ein Bezogen-sein auf Gestalten. Jede Form von Wahrnehmung ist Gestaltwahrnehmung. (Scheler)

2. Die phänomenlogische Vorstellung vom Bewusstsein, nach der wir \emph{immer schon} zur Welt sind.\newline

\noindent \emph{Was ist eine Gestalt?}

Sie ist nach der Gestalttheorie das Primat des Phänomenalen. Unsere Lebenswelt ist nicht gefüllt mit einzelnen "`Wahrnehmungspunkten"', die wir erst zusammenführen müssen, sondern mit Dingen, die eine Gestalt haben. Der Gestaltbegriff ist aber sehr weit zu fassen. Gestalten werden auf unterschiedliche Weise wahrgenommen, man kann sie nicht nur sehen sondern zum Beispiel auch fühlen.

Wenn man sich von einem großen, verpixelten Bild wegbewegt, dann dreht sich so zu sagen ein Schalter um, und wir nehmen dann plötzlich die Gestalt des Dargestellten wahr, wir haben einen Perspektivenwechsel vollzogen.

Beispiel: Stimmengewirr wirkt wie "`reine Empfindung"'. Sobald ich mich auf ein Gespräch richte nicht mehr. (Denn: Worte sind Gestalt von Lauten). Doch schon das Stimmengewirr an sich hat eine Gestalt. Die reine Empfindung ist also nur eine Fiktion, sie bleibt ein "`Rechenpfennig"'.\newline


\noindent \emph{Warum nehmen wir nur Gestalten wahr? Ist das anerzogen oder angeboren?}
Bestimmte Arten der Wahrnehmung von Gestalten müssen uns angeboren sein, denn wenn wir nicht immer schon Gestalten wahrnehmen, kann es uns auch nicht beigebracht werden. Konrad Lorenz spricht von den "`angeborenen Formen möglicher Erfahrung"'.

Bei bestimmten Formen von Autismus ist für die betroffene Person Mimik von anderen nicht erkennbar. Dies deutet darauf hin, dass gewisse (normalerweise universale) Strukturen zur Möglichkeit des Erkennens von mimischen Gestalten nicht vorhanden sind. Dass Gestaltwahrnehmung angeboren ist heißt natürlich nicht, dass die partikularen Gestalten schon irgendwo von Anfang an "`abgespeichert"' sind, Babies lernen erst wahrzunehmen.



\section*{Offene Fragen}

\begin{itemize}
  \item Was ist nun die Beziehung zwischen den möglichen Antworten auf die Frage nach dem Zur-Welt-sein? 
  \item Ist die Frage, ob das Primat der Wahrnehmung von Gestalten angeboren ist, wichtig?
  \item Gibt Scheler uns eine Antwort auf sie?
\end{itemize}




%\begin{description}[leftmargin=!,labelwidth=\widthof{\bfseries Unterschied des Bewusstseins}]
%  \item[Bewusstsein] Prozess der Umkehrung/Selbstpr"ufung des Bewusstseins, sowie seines Bewusstseinsgegenstandes, wodurch das Bewusstsein etwas anderes wird. Und diese Ver"anderung ist die Erfahrung. So bestreitet es einen Weg, "uber welchen es zur 	Selbstthematisierung kommt.
%  \item[\emph{An sich f"ur es}] Das Bild das sich das Bewusstsein von etwas macht, sein Wissen. "Uber dieses kommt das Bewusstsein zur Erfahrung. Bild vom Gegenstand entspricht nicht dem, wie er f"ur sich selbst ist (Wahrheit).
%  \item[Wir] Instanz des Autors/Lesers, die dem Bewusstsein "uber die Schulter schaut und "`redupliziert"' die Erfahrung des Bewusstseins im Buch/Lesen.
%  \item[Unterschied des Bewusstseins] Unterschied zwischen dem \emph{an sich f"ur es} (Wissen) und dem, womit das Bewusstsein dieses Wissen vergleicht (Wahrheit).
%  \item[Vollst"andigkeit der Form] Wird erreicht dadurch, dass durch die doppelte bestimmte Negation alles einseitige Wissen "uber den Gegenstand verworfen wird.
%\end{description}


\end{document}
