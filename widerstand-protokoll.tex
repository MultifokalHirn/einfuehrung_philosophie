\documentclass[a4paper, emulatestandardclasses]{scrartcl}
\usepackage{graphicx}
\usepackage{fullpage}
%\usepackage{parskip}
\usepackage{color}
\usepackage[ngerman]{babel}
\usepackage[utf8]{inputenc}
\usepackage{hyperref}
\usepackage{calc} 
\usepackage{enumitem}
\usepackage{titlesec}
\usepackage{bussproofs}
\usepackage[export]{adjustbox}
%\pagestyle{headings}

\titleformat{name=\section,numberless}
  {\normalfont\Large\bfseries}
  {}
  {0pt}
  {}
\date{\vspace{-3ex}}
\begin{document}

\title{
    \vspace{-30pt}
	\includegraphics*[width=0.1\textwidth,right]{ErstesSem/images/hu_logo2.png}\\
	\vspace{-10pt}
	Wahrnehmung bei Scheler I (23.11.17)}%}\\\vspace{10pt}\small{Lennard Wolf}}
	\subtitle{Die Erfahrung der Realität durch Widerstand (PS, WS 17/18)\\
          Dozent: Dr. Matthias Schlo"sberger\\
          Protokollant: Lennard Wolf}
\maketitle
\vspace{-40pt}

\section*{Vorbemerkungen}
\textbf{Korrekturen von Missverständnissen}

\begin{itemize}
  \item Besserer Begriff für was Husserl mit der "`reinen Empfindung"' meint: "`\emph{Erfahrung in ihrer Selbstgegebenheit}"'. Dieser Begriff signalisiert besser, dass wir auf die fundamentalste Ebene der Erfahrung zurück gehen.
  \item Die Phänomenologische Reduktion ist nicht gerade nur eine Methode, sondern vielmehr eine philosophische \emph{Einstellung}, die als Gegenentwurf zur cartesischen Zweifelsbetrachtung (im Modus der Meditation) zu betrachten wäre. In dieser Einstellung klammern wir alles, was wir schon zu wissen meinen (zB was ein Körper ist, die Trennung von Subjekt und Objekt etc.), ein. 
  \item Die Uneinigkeit der Phänomenologen herrschte bezüglich der Reduktion. Husserl nennt seine die \emph{transzendentale} Reduktion, weil er die Seinsgeltung der (zufälligen) Welt einklammert, um zum absoluten Wissen zu gelangen (später "`idealistische Reduktion"').
\end{itemize}

\noindent\textbf{Weitere Anmerkungen zur Geschichte der Phänomenologie}\newline

Wichtige Namen in der 1. Generation der Phänomenologie: Edmund Husserl, Max Scheler, Moritz Geiger, Adolf Reinach, Alexander Pfänder, 2. Generation: z.B. Heidegger mit \emph{Sein und Zeit}.\newline

\noindent \emph{Was zeichnet die Phänomenologie Bewegung aus, was ist der Grundgedanke?}

Theorie des Bewusstseins: Das das Bewusstsein immer intentional. das Bewusstsein ist immer shcon gerichtet auf -> es gibt kein bewusstsein das erst bei sich ist, und dann auf etwas gerichtet ist (nicht wie kommen wir zur welt, wir sind immer schon zur welt)\newline



\begin{itemize}
  \item Hintentionalität schwieriger begriff. usserl: mannigfache stufen der intentionalität: Beispiel: gefühle, die bestimmte intentionalität haben (man liebt \emph{jmd}); aber auch nicht intentionale gefühle (Traurigkeit, wo man nicht weiß woher das kommt) Husserl und Scheler: das bedeutet nicht dass diese traurigkeit nicht intentional ist! Intention bedeutet gerade nicht, dass das bzogen siein auf etwas, das beseelt sein, in urteilformen übersetzbar sien können; gerichtetheit kann auch unbestimmt sein..
  \item Leute die die reduktion machen wollen, leute scheler die an die reduktion glauben aber es nicht machen, sie sei nicht transzendental, asiatische seelentechniken, und da passiert was anderes als husserl sich da vorstellt, denn wir können die welt verschwinden lassen, aber dann verschwindet 
  \item scheler beispiel berühmt: die schwierigkeiten der reduktion werden geschickt vernachlässigt: quelle irgendwas mit kosmos) | geist ist was den menschen zum menschen macht und vom tier unterscheidet. Geist kann die fähigkeit zur ideation: geshcichte prinz buddha wächst von allen einflüssen der gemeinen gesellschaft abgeschirmt im elterlichen palast auf, in diesem abgeschirmten leben wird ern ihct konfrontiert mit dem phänomen der armut und dem phänomen der krankheit. er kennt die begriffe nicht. und er kennt die phäno nciht weil er sie nie erlebt hat. der prinz verlässt das elternhaus und sieht am wegesrand einen armen und einen kranken. Scheler: einbeispiel reicht aus, um das wesen der armut und das wesen der krankheit zu erfassen. -> phänomenlogisch erkenntnos. es geht nicht ohne erfahrung, zweitens die begriffe entstehen nicht indem wir ganz viele rfahrungen machen und durchschnittswerte machen sondern umgekehrt. Scheler: erkenntnis ist vorbegrifflich. der prinz ist ja nicht mit mit dem begriff bekannt. wenn er gefragt wird nahc dem wesen der armut, kann er reduktion machen, dann so tun als ob er das alles noch nicht kennen würde und die dinge für sich selbst reden lassen. Überall wo wir zu wesenschau machen 
  \item wirklich sein bedeutet widerstand erfahren, jede erfahrung des soseins gründet in der erfahrung des daseins von etwas. wirklichkeitserfahrung ist also fundamental für das sosein. husserl: umgekehrt, erst wenn ich die widerständigkeit der welt aufhebe, kann ich zum sosein kommen!
  \item Eine These des Textes: allein die these, dass wir zwischen geist und körper unterscheiden ist auch schon vorurteil einer tradition. wir wollen also diese begriffe einklammern, ebenso den der empfindung, der ist historisch belastet und kontaminiert. und so anfangen
\end{itemize}

\textbf{Literaturempfehlungen}

\begin{itemize}
  \item Herbert Spiegelberg: \emph{The Phenomenological Movement - A Historical Introduction}. (Band 1 ist eher relevant)
  \item Ferdinand Fellman: \emph{Phänomenologie zur Einführung}. (Kürzer als Spiegelberg)
  \item Dan Zahavi: \emph{Phänomenologie für Einsteiger}. (Sehr Husserl-zentriert)
  \item The Open Commons of Phenomenology (\url{ophen.org})
\end{itemize}


Vorwort erster Band. Schön aufbereitet:

  Vorwrot der ersten AUflage zeitschrift zur üphänomen forschung (publikationsorgan der bewegung)

\section*{Textbesprechung}

\emph{Warum ist die Frage nach der Empfindung so zentral?}

Weil er die Grundlage bilden muss für die Frage, wie wir Wissen über der Welt haben können.\newline

\noindent \emph{was heißt denn jetzt was wir mit empfindung meinen?}



\begin{itemize}
  \item 322 unten: ; bedeutet empfindung denn noch etwas das existiert... Ja sie ist ein rechenpfennig
  \item 321 3 reize beispiel: keine klare verbindung zwischen reiz und empfindung
  \item Konstanzannahme: beispiel dass die sich nicht funktioniert: streicheln über den hinterkopf und dann fühlt es sich plötzlich ganz anders an; dabei ist der reiz ja eigentlich der selbe
\end{itemize}


\noindent \emph{Was ist für Scheler die Empfindung?}

je reiner die empfindung wird, desto mehr verschwindet sie. Empfindung 2 ideen: pure empfindung (fiktion), oder Teil von erlebnisse, erfahrung, cogitationes, haben immer einen empfindungscharakter (wie fühlt es sich denn an?); Empfindungen sind mitkonstitutiv..\newline


\noindent \emph{warum kann es nicht stimmen, dass wir erst nicht zur welt sind an x0 und zur welt kommen an einem zeitpunkt x1:?}

wir kommen in zirkelprobleme wenn wir erklären wollen\newline

\noindent \emph{Wie verhält es sich dann alternativ?} 

2 Lösungen: 1. das bezogen sein auf etwas ist immer schon ein bezogen sein auf gestalten! Jede Form von Warhnehmung ist Gestaltwahrnehmung (im Text: 2. Sagen: der phänomenlogische begriff des bewusstseins dass wir immer schon zur welt sind, (vorherige Sitzung)\newline

\noindent \emph{Was ist eine Gestalt?}

Gestaltbegriff ist sehr weit: Gestalten werden wahrgenommen, man kann auch gestalten fühlen; Bei gestlat gibt es nur an aus schalter, perspektivenwechsel. bei fühlen von etwas kann es graduell sein. Was passiert, wenn ich aus farbklexen auf einem bild wenn ich weggehe ein gesamtbildung, eine gestalt auf einmal sehe.\newline

\noindent \emph{Wie ist das Verhältnis von Empfindung und Wahrnehmung?}

Bei der Wahrnehmung dürfen wir nicht so vorgehen, dass wir kleine teile annehmen die wir zu einem größeren zusammensetzen, sondern umgekehrt. Empfindung ist also nur der rechenpfennig.

Stimmengewirr, wirkt wie Empfindung; sobald ich ein Gespräch habe nicht mehr (Denn: Worte sind Gestalt von Lauten);dies gilt auch für Sprachen die wir nicht sprechen (Denn schon etwas (auch fälschlich) $als$ Sprache zu hören ist eine Gestaltwahrnehmung

\noindent \emph{warum ist es so, dass wir immer schon gestalten wahrnehmen? angeboren Oder beigebracht?}
Bestimmte Formen von Gestalten sind dann angeboren, Beispiel Autismus: Mimik nicht erkennbar; Universale Strukturen zur möglichkeit des Erkennens sind angeboren | Natürlich lernen babies erst wahrzunehmen.

Angeboren/Triebhaft, weil ansonsten nie zur gestalt gekommen werden kann. Konrad Lorenz spricht von den angeborenen Formen möglicher Erfahrung (-> Gestalten) | Kindchenschema
\newline



\section*{Offene Fragen}

\begin{itemize}
  \item Ist die Frage, ob das Primat der Wahrnehmung von Gestalten angeboren ist, wichtig?
  \item Gibt Scheler uns eine Antwort auf sie?
\end{itemize}




%\begin{description}[leftmargin=!,labelwidth=\widthof{\bfseries Unterschied des Bewusstseins}]
%  \item[Bewusstsein] Prozess der Umkehrung/Selbstpr"ufung des Bewusstseins, sowie seines Bewusstseinsgegenstandes, wodurch das Bewusstsein etwas anderes wird. Und diese Ver"anderung ist die Erfahrung. So bestreitet es einen Weg, "uber welchen es zur 	Selbstthematisierung kommt.
%  \item[\emph{An sich f"ur es}] Das Bild das sich das Bewusstsein von etwas macht, sein Wissen. "Uber dieses kommt das Bewusstsein zur Erfahrung. Bild vom Gegenstand entspricht nicht dem, wie er f"ur sich selbst ist (Wahrheit).
%  \item[Wir] Instanz des Autors/Lesers, die dem Bewusstsein "uber die Schulter schaut und "`redupliziert"' die Erfahrung des Bewusstseins im Buch/Lesen.
%  \item[Unterschied des Bewusstseins] Unterschied zwischen dem \emph{an sich f"ur es} (Wissen) und dem, womit das Bewusstsein dieses Wissen vergleicht (Wahrheit).
%  \item[Vollst"andigkeit der Form] Wird erreicht dadurch, dass durch die doppelte bestimmte Negation alles einseitige Wissen "uber den Gegenstand verworfen wird.
%\end{description}


\end{document}
