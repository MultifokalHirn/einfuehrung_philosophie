\documentclass[]{scrartcl}
\usepackage{graphicx}
\usepackage{color}
\usepackage[ngerman]{babel}
\usepackage{hyperref}
\usepackage{fullpage}
\usepackage{calc} 
\usepackage{enumitem}
\usepackage{titlesec}
\newcommand{\todo}[1]{\textcolor{red}{TODO: #1}\PackageWarning{TODO:}{#1!}}
\begin{document}

\title{
	\includegraphics*[width=0.75\textwidth]{images/hu_logo.png}\\
	\vspace{24pt}
	Einf"uhrung in die Sprachphilosophie}
\subtitle{VEV WS 16/17\\
          Dr. Jasper Liptow\\
          Philosophisches Institut I \\ 
          Humboldt Universit"at zu Berlin}
\author{Lennard Wolf\\
        \href{mailto:lennard.wolf@student.hu-berlin.de}{lennard.wolf@student.hu-berlin.de}}
\maketitle
\begin{abstract}

Die Vorlesung soll grundlegendes Wissen "uber Probleme, Begriffe und Positionen der Sprachphilosophie des 20. Jahrhunderts vermitteln. Nach einer einf"uhrenden Einheit, in der das Ph"anomen der sprachlichen Bedeutung, das im Mittelpunkt sprachphilosophischer Untersuchungen steht, herausgearbeitet wird, widmen wir uns ausgew"ahlten systematischen Themen (wie etwa den verschiedenen Dimensionen sprachlicher Bedeutung, dem Zusammenhang von Bedeutung und Gebrauch sprachlicher Ausdr"ucke oder dem Verh"altnis von Sprache und Denken). Die Vorlesung wird von Tutorien begleitet, in denen Texte besprochen werden, die f"ur die Vorlesung eine Rolle spielen, und Fragen der Vorlesung vertieft diskutiert werden k"onnen.


\end{abstract}
\newpage

\tableofcontents
\listoffigures
\newpage


\section{Einf"uhrungssitzung\\(24.10.16)}
\subsection{Organisatorisches}

\begin{itemize}
  \item Jede Woche begleitender Text der vorher zu lesen ist (bis 30.11. k"onnen runtergeladen werden)
  \item In Tutorien werden Texte nachbesprochen
  \item 4 Essays (3-4 Standardseiten) f"ur Tutorien zum Bestehen 
  \item Passwort: Platte
\end{itemize}

\subsection{Was ist Sprachphilosophie?}
\subsubsection{Was ist eine philosophische Methode zur Untersuchung der Sprache?}
\begin{itemize}
  \item Ziel: Erkentnisse "uber Sprache zu gewinnen ohne Empirie anzuwenden
  \item Philosophische Fragen sollen \emph{verschwinden}
  \item Unterschiede zu Empirischer Forschung sind umstritten (Zwei M"oglichkeiten):
  \item \textbf{Unterschiede sind prinzipieller Natur:}
  \item Empirische Wissenschaft untersucht die Welt \emph{direkt}, Philosophie durch eine Analyse unserer Begriffe
  \item Art der Rechtfertigung: a priori vs a posteriori
  \item Art der Belege: Erfahrung vs. Intuition
  \item Modaler Status des Wissens: kontingente Wahrheiten (Empirie) vs notwendige Wahrheiten (Philosophie)
  \item \textbf{Unterschiede sind gradueller Natur:}
  \item 
\end{itemize}

\subsubsection{Welche Aspekte der Sprache sind Gegenstand?}

\begin{itemize}
  \item
  \item
\end{itemize}

\newpage



\section{Frege: "Uber Sinn und Bedeutung I\\(31.10.16)}

\subsection{Vornotizen zum Text}
\subsubsection{Inhalt des Textes}
\begin{itemize}
  \item \emph{R"atsel}: Wie kann $a = b$ einen anderen Erkenntniswert als $ a = a$ haben wenn es wahr ist? $\rightarrow$ \emph{Sinn} 
  \item \emph{Antwort}: Liegt nicht an dem unterschiedlichen Bezug (Bedeutung$_{F}$) und auch nicht an den unterschiedlichen Ausdr"ucken
  \item Zeichen (Eigenname) $\rightarrow$ Sinn (das \emph{Gemeinte}) $\rightarrow$ Bedeutung$_{F}$ (\emph{Auf das gedeutet wird})
  \item \textbf{Beispiel} -- \emph{Zeichen}: Abendstern $\rightarrow$ \emph{Sinn}: ein bestimmtes Himmelsgestirn $\rightarrow$ \emph{Bedeutung}: Venus 
  \item Bei dem Zeichen \emph{Morgenstern} w"are die Bedeutung$_{F}$ identisch, aber der Sinn w"are m"oglicherweise ein anderer
  \item Der Sinn eines Eigennamens ist die \emph{Art des Gegebensein}.
  \item Der Sinn eines Satzes ist der \emph{Gedanke} (objektiver Inhalt der nicht ein Prozess in einem Kopf ist sondern von vielen gedacht werden kann) den er ausdr"uckt.
  \item Die Bedeutung$_{F}$ eines Satzes ist sein Wahrheitswert.
  \item \emph{Das Streben nach Wahrheit also ist es, as uns "uberall vom Sinn zur Bedeutung$_{F}$ vorzudringen treibt.}
  \item Das \emph{Urteil} ist der Fortschritt von einem Gedanken zu seinem Wahrheitswert
\end{itemize}
\subsubsection{Fragen}
\begin{itemize}
  \item Ist Bedeutung nur in realer Welt (kontextunabh"angig) oder kann in zB einem literarischen Kontext eine Bedeutung vorhanden sein? (kontextsensitiv, Stichwort Odysseus)
\end{itemize}

\subsection{The Linguistic Turn}

20. Jahrhundert als Jahrhundert der Sprachphilosophie
\begin{itemize}
  \item Sprache als zentraler Gegenstand philosophischer Untersuchung
  \item K"onigsdisziplin in versch. Jhdt. unterschiedlich (z.B. Ontologie, Epistemologie etc.)
\end{itemize}

\subsection{Die Grundfrage der Sprachphilosophie}

\subsection{Bedeutung als Vorstellungen}

\subsection{Bedeutung als objektive Gegenst"ande}

\subsection{Freges R"atsel}

\subsection{Frege "uber Sinn$_{F}$ und Bedeutung$_{F}$ von Eigennamen}

\textbf{Wichtig:} Seite 42, Fu\ss note 2\\
\textbf{Wichtig:} Lesen: \emph{Der Gedanke }von Frege


\section{Frege: "Uber Sinn und Bedeutung II\\(07.11.16)}


\textbf{Bis Frege:} Theorie der Bedeutung sprachlicher Ausdr"ucke als Theorie der Bedeutung von Namen\\
\textbf{Nach Frege:} Theorie der Bedeutung sprachlicher Ausdr"ucke als Theorie des Aufbaus under Abh"angigkeit der Bedeutung der verschiedenen Arten sprachlicher Ausdr"ucke einer Sprache (\emph{Bedeutungstheorie})


\begin{description}[leftmargin=!,labelwidth=\widthof{\bfseries 2}]
  \item[Eigenname$_{F}$] Ausdr"ucke, die die Funktion haben, f"ur einen einzelnen Gegenstand zu stehen (vgl. 41). Dies umfasst \emph{echte Eigennamen} wie `Sokrates', \emph{Pronomen} wie `dies' oder `der', \emph{definite Kennzeichnungen} wie `der Schnittpunkt der Geraden $a$ und $b$' | Heute spricht man von \emph{singul"aren Termen}, wobei keine Einigkeit dar"uber besteht, ob `definite Kennzeichnungen' tats"achlich singul"are Terme sind.
  \item[Bedeutung$_{F}$ eines Eigennamen$_{F}$] Das Bezeichnete | ein bestimmter Gegenstand
  \item[Sinn$_{F}$ eines Eigennamen$_{F}$] Das, worin die Art des Gegebenseins enthalten ist | die Art, in der der Ausdruck seine Bedeutung$_{F}$ pr"asentiert oder herausgreift.
  \item[Beziehung Sinn$_{F}$ - Bedeutung$_{F}$] Keine Bedeutung$_{F}$ ohne Sinn$_{F}$; Jedoch Sinn$_{F}$ geht ohne Bedeutung$_{F}$ | Zwei Eigennamen$_{F}$ mit selbem Sinn$_{F}$, m"ussen dieselbe Bedeutung$_{F}$ haben | Zwei Eigennamen$_{F}$ k"onnen dieselbe Bedeutung$_{F}$ haben, aber unterschiedlichen Sinn$_{F}$.
  \item[Sinn$_{F}$ eines Satzes] Der \emph{Gedanke}, der von einem Satz zum Ausdruck gebracht wird | Anders als bei Eigennamen$_{F}$: Wir k"onnen den Sinn$_{F}$ von S"atzen nicht erlernen | Das Erfassen der Gedanken, die von S"atzen zum Ausdruck gebracht werden, h"angt mit dem Erfassen des Sinns$_{F}$ der Ausdr"ucke zusammen, aus denen die S"atze aufgebaut sind.
  \item[Bedeutung$_{F}$ eines Satzes] Der \emph{Wahrheitswert},  des Satzes, d.h. der `Umstand, da\ss~er wahr oder da\ss~er falsch ist'.
  \item[Gedanke] Kein psychischer Akt, sondern ein objektiver Inhalt, der gemeinsames Eigentum von vielen sein kann | Heute spricht man von \emph{Propositionen (im Fregeschen Sinn)}. 
  \item[Kompositionalit"at des Sinns$_{F}$] Wenn ich in einem Satz einen Ausdruck mit einem anderen, welcher den selben Sinn$_{F}$ hat, dann "andert sich der Sinn$_{F}$ des Satzes nicht. {\color{red}???}
  \item[Kompositionalit"at der Bedeutung$_{F}$] Wenn ich in einem Satz einen Ausdruck mit einem anderen, welcher die selbe Bedeutung$_{F}$ hat, dann "andert sich die Bedeutung$_{F}$ des Satzes nicht.
  \item[Satz] Seit Frege: Satz ist vollst"ndig, wenn er einen \emph{Gedanken} zum Ausdruck bringt, einen \emph{Wahrheitswert} hat und wir mit ihm \emph{sprachliche Handlungen} vollziehen k"onnen.
  \item[Kontextprinzip] Sprachliche Ausdr"ucke unterhalb der Satzebene (`W"orter') haben (in einem strikten Sinn) nur im Kontext vollst"andiger S"atze Bedeutung
\end{description}

\textbf{Freges Entscheidender Gedanke} Der logisch einfache Satz ist nicht aus zwei (oder mehr) gleichartigen Elementen zusammengef"ugt, sondern zerf"allt in Ausdr"ucke, die sich in ihrer sprachlichen Funktion wesentlich voneinander unterscheiden. (Singul"are Terme [Frege: \emph{Eigenname}], die auf etwas Bezug nehmen und Pr"adikate [Frege: \emph{Begriffsw"orter], die auf die Gegenst"ande zutreffen m"ussen) | Quine: Pr"adikate sind offene S"atze (`\emph{$x$ ist ein kl"uger als $y$}')

\section{Russell: "Uber das Kennzeichnen\\(14.11.16)}

\subsection{Vornotizen zum Text}

\begin{itemize}
  \item Extension
  \item Intension
  \item Kommutatitivit"at??
\end{itemize}

\newpage
\section{"Uber den Dozenten}
Dr. Jasper Liptow absolvierte 1996 seinen Magister an der Universit"at Hamburg, promovierte in Gie\ss en mit einer Arbeit zum Thema \emph{Gebrauchstheorien der Bedeutung} bei Prof. Martin Seel und ist Privatdozent.


\begin{figure}[]
	\centering
	\includegraphics[width=0.32\textwidth]{images/liptow.jpg}
	\caption{Dr. Jasper Liptow. Quelle: \url{https://www.uni-frankfurt.de/45457854/liptow.jpg}}
	\label{fig:liptow}
\end{figure}

%\begin{figure}[h]
%	\centering
%	\includegraphics[width=0.5\textwidth]{images/template.png}
%	\caption{Template Bild}
%	\label{fig:template}
%\end{figure}

\end{document}
