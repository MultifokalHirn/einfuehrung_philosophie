\documentclass[]{scrartcl}
\usepackage{graphicx}
\usepackage{color}
\usepackage{german}
\usepackage{hyperref}
%\pagestyle{headings}

\begin{document}

\title{
	\includegraphics*[width=0.75\textwidth]{images/hu_logo.png}\\
	\vspace{24pt}
	Einf"uhrung in die Sprachphilosophie}
\subtitle{VEV WS 16/17\\
          Dr. Jasper Liptow\\
          Philosophisches Institut I \\ 
          Humboldt Universit"at zu Berlin}
\author{Lennard Wolf\\
        \href{mailto:lennard.wolf@student.hu-berlin.de}{lennard.wolf@student.hu-berlin.de}}
\maketitle
\begin{abstract}

Die Vorlesung soll grundlegendes Wissen "uber Probleme, Begriffe und Positionen der Sprachphilosophie des 20. Jahrhunderts vermitteln. Nach einer einf"uhrenden Einheit, in der das Ph"anomen der sprachlichen Bedeutung, das im Mittelpunkt sprachphilosophischer Untersuchungen steht, herausgearbeitet wird, widmen wir uns ausgew"ahlten systematischen Themen (wie etwa den verschiedenen Dimensionen sprachlicher Bedeutung, dem Zusammenhang von Bedeutung und Gebrauch sprachlicher Ausdr"ucke oder dem Verh"altnis von Sprache und Denken). Die Vorlesung wird von Tutorien begleitet, in denen Texte besprochen werden, die f"ur die Vorlesung eine Rolle spielen, und Fragen der Vorlesung vertieft diskutiert werden k"onnen.


\end{abstract}
\newpage

\tableofcontents
\listoffigures
\newpage


\section{Einf"uhrungssitzung\\(24.10.16)}
\subsection{Organisatorisches}

\begin{itemize}
  \item Jede Woche begleitender Text der vorher zu lesen ist (bis 30.11. können runtergeladen werden)
  \item In Tutorien werden Texte nachbesprochen
  \item 4 Essays (3-4 Standardseiten) f"ur Tutorien zum Bestehen 
  \item Passwort: Platte
\end{itemize}

\subsection{Was ist Sprachphilosophie?}
\subsubsection{Was ist eine philosophische Methode zur Untersuchung der Sprache?}
\begin{itemize}
  \item Ziel: Erkentnisse über Sprache zu gewinnen ohne Empirie anzuwenden
  \item Philosophische Fragen sollen \emph{verschwinden}
  \item Aber: Unterschiede zu Empirischer Forschung sind umstritten (Zwei M"oglichkeiten):
  \item \textbf{Unterschiede sind prinzipieller Natur:}
  \item Empirische Wissenschaft untersucht die Welt \emph{direkt}, Philosophie durch eine Analyse unserer Begriffe
  \item Art der Rechtfertigung: a priori vs a posteriori
  \item Art der Belege: Erfahrung vs. Intuition
  \item Modaler Status des Wissens: kontingente Wahrheiten (Empirie) vs notwendige Wahrheiten (Philosophie)
  \item \textbf{Unterschiede sind gradueller Natur:}
  \item 
\end{itemize}

\subsubsection{Welche Aspekte der Sprache sind Gegenstand?}

\begin{itemize}
  \item
  \item
\end{itemize}

\newpage

\section{Platon, Apologie\\(27.10.16)}


%\begin{figure}[h]
%	\centering
%	\includegraphics[width=0.5\textwidth]{images/template.png}
%	\caption{Template Bild}
%	\label{fig:template}
%\end{figure}



\newpage
\section{"Uber die Dozentin}
Dr. Bettina Fr"ohlich ist..


\end{document}
