\documentclass[a4paper, emulatestandardclasses, 12pt]{scrartcl}
\usepackage{graphicx}
\usepackage{fullpage}
%\usepackage{parskip}
\usepackage{color}
\usepackage[ngerman]{babel}
\usepackage{hyperref}
\usepackage{calc} 
\usepackage{enumitem}
\usepackage{titlesec}
%\pagestyle{headings}
\usepackage{setspace} %halbzeilig
\usepackage[authoryear,round]{natbib}
\bibliographystyle{natdin}

%\titleformat{name=\section,numberless}
%  {\normalfont\Large\bfseries}
%  {}
%  {0pt}
%  {}
\date{\vspace{-3ex}}
\begin{document}

\title{\vspace{5ex}
	\includegraphics*[width=0.72\textwidth]{images/hu_logo.png}\\
	\vspace{30pt}
	\scshape\LARGE{Kurzessay zur\\Beschreibungstheorie der Namen}}
	
	\subtitle{\vspace{20pt}Einf"uhrung in die Sprachphilosophie\\
          \vspace{6pt}
          Tutorium Benjamin\\}


\author{\vspace{-4pt}Lennard Wolf\\
        \small{\href{mailto:lennard.wolf@student.hu-berlin.de}{lennard.wolf@student.hu-berlin.de}}}      

\maketitle

\vspace{\fill}

\begin{minipage}[b]{\textwidth}
    \centering
    \onehalfspacing
    \large   
    14. Dezember 2016\\
    Wintersemester 2016/2017

    \vspace{-20mm} 
\end{minipage}%
\thispagestyle{empty}
\newpage
\clearpage
\setcounter{page}{1}

\begin{onehalfspace} 

\noindent\textbf{Aufgabenstellung:}\\
Stellt dar, was die Beschreibungstheorie der Namen besagt. Entwickelt ein \emph{eigenes} Gegenbeispiel gegen diese Theorie. Skizziert dann, wie die Beschreibungstheorie gegen dieses Beispiel verteidigt werden k"onnte.
\begin{center}
\vspace{-9pt}\line(1,0){450}
\end{center}

\vspace{5mm}
\noindent In dieser Arbeit m"ochte ich eine kurze Einf"uhrung in die Beschreibungstheorie der Namen geben, anhand eines Gegenbeispiels Unzul"anglichkeiten daran aufzeigen und eine m"ogliche Verteidigung der Theorie skizzieren.  

Dazu werde ich wie folgt vorgehen. In Abschnitt $(i)$ stelle ich die Beschreibungstheorie der Namen dar. Das Gegenbeispiel stelle ich in $(ii)$ und werde die Probleme beschreiben, welche dadurch f"ur die Theorie entstehen. Es folgt in $(iii)$ eine "Ubertragung des Beispiels auf Kripkes kausale Auffassung der Referenz und eine Beschreibung dessen, wie diese damit umgehen w"urde.
\vspace{5mm}

\noindent\textbf{$(i)$ Die Beschreibungstheorie der Namen}

\noindent Kripke formalisierte in seiner Vortragsreihe \emph{Naming and Necessity} \citep{kripke1972naming}  die Beschreibungstheorie der Namen, welche so oder in sehr "ahnlicher Form unter anderem von \citet{begriffundgegenstand} und \citet{russell1905denoting} vertreten wurde. Diese Formalisierung umfasst sechs Thesen.

Diese Thesen besagen, dass (Eigen-) Namen jeweils einem \emph{B"undel von Eigenschaften} entsprechen, welche den namentragenden, individuellen Gegenstand einzeln herausgreifen. Der Theorie zufolge glaubt also eine Sprecherin, die in einem Satz von "`$X$"' spricht, dass das Bezugsobjekt dieses Namens eine Menge von Eigenschaften $\varphi$ erf"ullt. Und wenn nun  "`die meisten oder eine ausschlaggebende Mende der $\varphi$'s von einem einzigen Gegenstand $y$ erf"ullt werden"',\footnote{These (3), \cite{begriffundgegenstand}} dann ist $y$ der Referent des Namens "`$X$"' . Sollte es solch einen Gegenstand nicht geben, dann hat "`$X$"' kein (real existierendes) Bezugsobjekt. Au"serdem ist die Aussage "`Wenn $X$ existiert, dann hat $X$ die meisten der $\varphi$'s."' zu einem eine \emph{notwendige Wahrheit}, und zum anderen von der Sprecherin \emph{a priori} gewusst.

Dem f"ugt Kripke noch eine Bedingung (B) hinzu, derzufolge die Eigenschaften eines Namens nicht den Namen selber auf solch eine Weise beinhalten sollen, dass die Aufl"osung zu einem Bezugsobjekt unm"oglich wird (\emph{Zirkularit"atsbedingung}).%\newpage

\noindent Im folgenden seien die f"ur diese Arbeit relevanten Thesen aus \citep{begriffundgegenstand} noch einmal originalgetreu aufgef"uhrt:\vspace{5mm}

\begin{addmargin}[.08\linewidth]{.08\linewidth}% indent 0pt left, .5\linewidth right
\footnotesize
\begin{description}[leftmargin=!,labelwidth=\widthof{\bfseries (B)}]
%    \item[(1)] Jedem Namen oder Bezeichnungsausdruck `$X$' entspricht ein B"undel von Eigenschaften, n"amlich die Familie der Eigenschaften $\varphi$, f"ur die gilt: $A$ meint `$\varphi X$'.
    \item[(2)] $A$ meint, dass eine der Eigenschaften oder einige Eigenschaften zusammen einen bestimmten individuellen Gegenstand als einzigen herausgreifen.
    \item[(3)] Wenn die meisten oder eine ausschlaggebende Menge der $\varphi$'s von einem einzigen Gegenstand $y$ erf"ullt werden, dann ist $y$ der Referent von `$X$'.
%    \item[(4)] Wenn die Abstimmung nicht einen einzigen Gegenstand liefert, dann referiert `$X$' nicht.
%    \item[(5)] Die Aussage "`Wenn $X$ existiert, dann hat $X$ die meisten der $\varphi$'s"' wei"s der Sprecher \emph{a priori}.
%    \item[(6)] Die Aussage "`Wenn $X$ existiert, dann hat $X$ die meisten der $\varphi$'s"' dr"uckt eine notwendige Wahrheit aus (im Idiolekt des Sprechers).
    \item[(B)] F"ur jede gelungene Theorie gilt, dass die Erkl"arung nicht zirkul"ar sein darf. Die Eigenschaften, welche bei der Abstimmung verwendet werden, d"urfen nicht selbst den Begriff der Referenz auf eine Weise enthalten, die seine Eliminierung letztlich unm"oglich macht.
\end{description}
\end{addmargin}
\normalsize

\vspace{3mm}
\noindent\textbf{$(ii)$ Gegenbeispiel}	

\noindent Das folgende Gegenbeispiel zu der Beschreibungstheorie wird zwei der Thesen der Beschreibungstheorie der Namen in Frage stellen. Ich ziehe daf"ur Elon Musk heran, den Gr"under und CEO von Tesla und SpaceX. 

Es begibt sich nun, dass eine Person $A$ von einer Freundin "uber Elon Musk erz"ahlt bekommt, dass dieser sich von seiner Frau getrennt habe. Davor hat $A$ aber noch nie von Elon Musk geh"ort und wei"s daher nichts weiter "uber diesen Herrn, als dass er sich von seiner Frau getrennt zu haben scheint. Laut der Beschreibungstheorie m"usste $A$ nun f"ur den Namen "`Elon Musk"' ein B"undel von exakt identifizierenden Eigenschaften herausgreifen k"onnen, um von der Person reden zu k"onnen. Doch auch wenn $A$ sonst keine weiteren Eigenschaften von Elon Musk kennt, kann $A$ trotzdem den Satz "`Elon Musk tut mir Leid."' aussprechen, es so meinen wie es zu verstehen ist und tats"achlich von der Person selbst reden. Dies zeigt, dass These (2) nicht allgemeing"ultig zu sein scheint.
 
Des Weiteren kann man sich folgendes Szenario vorstellen: Elon Musk hat sich vor der Gr"undung von PayPal\footnote{An dieser hatte er signifikante Beteiligung. PayPal kann als Grundstein f"ur alle darauf gefolgten Unternehmungen Musks angesehen werden, da es ihm die n"otigen finanziellen Mittel gab.} durch einen ihm im Verhalten wie im Aussehen exakt gleichenden Roboter ausgetauscht und selbst Suizid begangen.  Entsprechend wurden die Unternehmen Tesla und SpaceX von eben diesem Roboter gegr"undet und auch nur der Roboter war mit Talulah Riley verheiratet, von welcher er sich nun geschieden hat.

Wenn die Information, dass Elon Musk tats"achlich schon lange tot ist, nun zu Tage tr"ate, dann w"urde der Name "`Elon Musk"', mit welchem die "Offentlichkeit bisher immer den erfolgreichen Unternehmer meinte, nach der Beschreibungstheorie einen fraglichen Status erlangen: Was ist nun die ausschlaggebende Menge der Eigenschaften, die f"ur "`Elon Musk"' ein spezifisches Bezugsobjekt herausgreifen? Spr"ache man dann nicht eher von der Person, die Suizid begangen hat, als von dem Roboter, der all das getan hat, was in der realen Welt der Mann getan hat? Die These (3) w"are daher auch nicht universell g"ultig, denn "`Elon Musk"' scheint sowohl die Person, als auch den Roboter zu referieren.

\vspace{5mm}
\noindent\textbf{$(iii)$ Verteidigung}	

\noindent Fraglich ist jedoch, ob das angef"uhrte Gegenbeispiel die Thesen (2) und (3) \emph{wirklich} hinf"allig macht. Ich m"ochte daher nun eine Gegenargumentation skizzieren, die zeigen w"urde, dass in den angef"uhrten Argumenten gegen die Beschreibungstheorie diese nur nicht genau genug angewandt wurde.

Gegen These (2) wurde angef"uhrt, dass $A$ mit "`Elon Musk"' keine bestimmt herausgreifenden Eigenschaften verbindet, diesen Namen aber trotzdem sinnvoll verwenden kann. "Ubersehen wurde jedoch, dass der Name selber noch eine weitere Eigenschaft impliziert, n"amlich das \emph{Elon Musk Hei"sen}. $A$ meint mit "`Elon Musk"' also jene Person, die "'Elon Musk"` hei"st und sich gerade von ihrer Frau getrennt hat.\footnote{Ich habe zur Vereinfachung die implizierten Eigenschaften wie \emph{Mensch sein} weg gelassen.} Es sollte dem aber noch angef"ugt werden, dass hier \emph{keine} Zirkularit"at wie in (B) beschrieben entstanden ist. Dies l"asst sich dadurch begr"unden, dass man bei der Zergliederung der Eigenschaften beim Eliminieren von \emph{Elon Musk Hei"sen} nicht wieder auf die Person selbst st"o"st, was Zirkul"arit"at zur Folge h"atte, sondern vielmehr auf eine Buchstabenkette auf einem Ausweisdokument oder derartiges.

Gegen These (3) wurde angef"uhrt, dass die identifizierenden Eigenschaften zu "`Elon Musk"' dem Roboter zuzuordnen seien, wenn man mit dem Namen aber wahrscheinlich eher die Person meinen w"urde.  

Dem lie"se sich entgegnen, dass Namen von \emph{Menschen} ausgesprochen und vernommen werden, und sie k"onnen nur etwas referieren im Kontext einer Interpretation. Es gibt keine "`objektiv wahre"' Referenz und die Beschreibungstheorie verlangt dies meinem Verst"andnis nach auch nicht. Die Person, die den Namen spricht oder h"ort, hat auch in diesem Beispiel einen bestimmten Bezugsgegenstand im Kopf, sei es der Roboter, oder die Person. Wenn eine Person $B$ nicht wei"s, dass es sich in Wirklichkeit um einen Roboter handelt, ist das trotzdem \emph{f"ur sie} irrelevant. $B$ meint mit "`Elon Musk"' exakt die eine (\emph{ihrer Meinung nach einzige}) Person, die Tesla und SpaceX gegr"undet hat. Dass sie sich darin irrt, dass es sich um einen Roboter handelt, tut f"ur ihr eigenes Verst"andnis der Unterhaltung nichts zur Sache. Dass sie falsche Schlussfolgerungen aus ihrer fehlerhaften Vorstellung ziehen mag, ist auch nicht wichtig. In dem Moment in dem $B$ die Wahrheit erf"ahrt, muss $B$ f"ur \emph{sich selbst} entscheiden, ob \emph{Robotersein} nun eine zu "`Elon Musk"' geh"orende Eigenschaft ist.



\end{onehalfspace}

\bibliography{sprachphilo-ha-2.bib}

\end{document}
