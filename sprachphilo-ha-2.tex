\documentclass[a4paper, emulatestandardclasses, 12pt]{scrartcl}
\usepackage{graphicx}
\usepackage{fullpage}
%\usepackage{parskip}
\usepackage{color}
\usepackage[ngerman]{babel}
\usepackage{hyperref}
\usepackage{calc} 
\usepackage{enumitem}
\usepackage{titlesec}
%\pagestyle{headings}
\usepackage{setspace} %halbzeilig
\usepackage[authoryear,round]{natbib}
\bibliographystyle{natdin}

%\titleformat{name=\section,numberless}
%  {\normalfont\Large\bfseries}
%  {}
%  {0pt}
%  {}
\date{\vspace{-3ex}}
\begin{document}

\title{\vspace{5ex}
	\includegraphics*[width=0.72\textwidth]{images/hu_logo.png}\\
	\vspace{30pt}
	\scshape\LARGE{Kurzessay zur\\Beschreibungstheorie der Namen}}
	
	\subtitle{\vspace{20pt}Einf"uhrung in die Sprachphilosophie\\
          \vspace{6pt}
          Tutorium Benjamin\\}


\author{\vspace{-4pt}Lennard Wolf\\
        \small{\href{mailto:lennard.wolf@student.hu-berlin.de}{lennard.wolf@student.hu-berlin.de}}}      

\maketitle

\vspace{\fill}

\begin{minipage}[b]{\textwidth}
    \centering
    \onehalfspacing
    \large   
    14. Dezember 2016\\
    Wintersemester 2016/2017

    \vspace{-20mm} 
\end{minipage}%
\thispagestyle{empty}
\newpage
\clearpage
\setcounter{page}{1}

\begin{onehalfspace} 

\noindent\textbf{Aufgabenstellung:}\\
Stellt dar, was die Beschreibungstheorie der Namen besagt. Entwickelt ein \emph{eigenes} Gegenbeispiel gegen diese Theorie.\\\indent Skizziert dann, wie ...?
\begin{center}
\vspace{-9pt}\line(1,0){450}
\end{center}


\noindent In dieser Arbeit m"ochte ich eine kurze Einf"uhrung in die Beschreibungstheorie der Namen geben, welche unter anderem von \citet{begriffundgegenstand} und \citet{russell1905denoting} vertreten wurde, und dann anhand eines Gegenbeispiels Unzul"anglichkeiten daran aufzeigen. Daraufhin 

Dazu werde ich wie folgt vorgehen. In Abschnitt $(i)$ stelle ich die Beschreibungstheorie der Namen dar. Das Gegenbeispiel stelle ich in $(ii)$ vor und in $(iii)$ werde ich genauer auf die Probleme eingehen, welche dadurch f"ur die Theorie entstehen. Es folgt in $(iv)$ eine "Ubertragung des Beispiels auf Kripkes kausale Auffassung der Referenz und eine Beschreibung dessen, wie diese damit umgehen w"urde.
\vspace{5mm}

\noindent\textbf{$(i)$ Die Beschreibungstheorie der Namen}

\noindent In \citet{kripke1972naming}
%\newpage

\noindent\textbf{$(ii)$ Sinn$_{F}$ und Bedeutung$_{F}$}	

\noindent Um das...


\end{onehalfspace}

\bibliography{sprachphilo-ha-2.bib}

\end{document}
