\documentclass[a4paper, emulatestandardclasses, 12pt]{scrartcl}
\usepackage{graphicx}
\usepackage{fullpage}
%\usepackage{parskip}
\usepackage{color}
\usepackage[ngerman]{babel}
\usepackage{hyperref}
\usepackage{calc} 
\usepackage{enumitem}
\usepackage{titlesec}
%\pagestyle{headings}
\usepackage{setspace} %halbzeilig
\usepackage[authoryear,round]{natbib}
\bibliographystyle{natdin}

%\titleformat{name=\section,numberless}
%  {\normalfont\Large\bfseries}
%  {}
%  {0pt}
%  {}
\date{\vspace{-3ex}}
\begin{document}

\title{\vspace{5ex}
	\includegraphics*[width=0.72\textwidth]{images/hu_logo.png}\\
	\vspace{30pt}
	\scshape\LARGE{Kurzessay zur\\Beschreibungstheorie der Namen}}
	
	\subtitle{\vspace{20pt}Einf"uhrung in die Sprachphilosophie\\
          \vspace{6pt}
          Tutorium Benjamin\\}


\author{\vspace{-4pt}Lennard Wolf\\
        \small{\href{mailto:lennard.wolf@student.hu-berlin.de}{lennard.wolf@student.hu-berlin.de}}}      

\maketitle

\vspace{\fill}

\begin{minipage}[b]{\textwidth}
    \centering
    \onehalfspacing
    \large   
    14. Dezember 2016\\
    Wintersemester 2016/2017

    \vspace{-20mm} 
\end{minipage}%
\thispagestyle{empty}
\newpage
\clearpage
\setcounter{page}{1}

\begin{onehalfspace} 

\noindent\textbf{Aufgabenstellung:}\\
Stellt dar, was die Beschreibungstheorie der Namen besagt. Entwickelt ein \emph{eigenes} Gegenbeispiel gegen diese Theorie.\\\indent Skizziert dann, wie ...?
\begin{center}
\vspace{-9pt}\line(1,0){450}
\end{center}


\noindent In dieser Arbeit m"ochte ich eine kurze Einf"uhrung in die Beschreibungstheorie der Namen geben und dann anhand eines Gegenbeispiels Unzul"anglichkeiten daran aufzeigen. Daraufhin 

Dazu werde ich wie folgt vorgehen. In Abschnitt $(i)$ stelle ich die Beschreibungstheorie der Namen dar. Das Gegenbeispiel stelle ich in $(ii)$ vor und in $(iii)$ werde ich genauer auf die Probleme eingehen, welche dadurch f"ur die Theorie entstehen. Es folgt in $(iv)$ eine "Ubertragung des Beispiels auf Kripkes kausale Auffassung der Referenz und eine Beschreibung dessen, wie diese damit umgehen w"urde.
\vspace{5mm}

\noindent\textbf{$(i)$ Die Beschreibungstheorie der Namen}

\noindent Kripke formalisierte in seiner Vortragsreihe \emph{Naming and Necessity} \citep{kripke1972naming}  die Beschreibungstheorie der Namen, welche so oder in sehr "ahnlicher Form unter anderem von \citet{begriffundgegenstand} und \citet{russell1905denoting} vertreten wurde. Diese Formalisierung umfasst sechs Thesen.

Diese Thesen besagen, dass (Eigen-) Namen jeweils einem \emph{B"undel von Eigenschaften} entsprechen, welche den namentragenden, individuellen Gegenstand einzeln herausgreifen. Der Theorie zufolge glaubt also eine Sprecherin, die in einem Satz von "`$X$"' spricht, dass das Bezugsobjekt dieses Namens eine Menge von Eigenschaften $\varphi$ erf"ullt. Und wenn nun  "`die meisten oder eine ausschlaggebende Mende der $\varphi$'s von einem einzigen Gegenstand $y$ erf"ullt werden"',\footnote{These (3), \cite{begriffundgegenstand}} dann ist $y$ der Referent des Namens "`$X$"' . Sollte es solch einen Gegenstand nicht geben, dann hat "`$X$"' kein (real existierendes) Bezugsobjekt. Au"serdem ist die Aussage "`Wenn $X$ existiert, dann hat $X$ die meisten der $\varphi$'s."' zu einem eine \emph{notwendige Wahrheit}, und zum anderen von der Sprecherin \emph{a priori} gewusst.

Dem f"ugt Kripke noch eine Bedingung hinzu, derzufolge die Eigenschaften eines Namens nicht den Namen selber auf solch eine Weise beinhalten sollen, dass die Aufl"osung zu einem Bezugsobjekt unm"oglich wird (\emph{Zirkularit"atsbedingung}).
\vspace{5mm}

\noindent\textbf{$(ii)$ Gegenbeispiel}	

\noindent Das folgende Gegenbeispiel zu der Beschreibungstheorie wird zwei der Thesen der Beschreibungstheorie der Namen in Frage stellen.

Man denke an Elon Musk, den Gr"under und CEO von Tesla und SpaceX. Es begibt sich nun, dass  eine Person $A$ von einer Freundin "uber Elon Musk erz"ahlt bekommt, dass dieser sich von seiner Frau getrennt habe. Davor hat $A$ aber noch nie von Elon Musk geh"ort und wei"s daher nichts weiter "uber diesen Herrn, au"ser dass er sich von seiner Frau getrennt zu haben scheint. Laut der Beschreibungstheorie m"usste $A$ nun f"ur den Namen "`Elon Musk"' ein B"undel von Eigenschaften herausgreifen k"onnen, um von der Person reden zu k"onnen. Doch $A$ kann trotzdem den Satz "`Elon Musk tut mir Leid."' sagen, es so meinen wie es zu verstehen ist und tats"achlich von der Person selbst reden, auch wenn $A$ sonst keine weiteren Eigenschaften von Elon Musk kennt. Dies zeigt, dass These (2)\footnote{"`$A$ meint, dass eine der Eigenschaften oder einige Eigenschaften zusammen einen bestimmten individuellen Gegenstand als einzigen herausgreifen."' \citep{begriffundgegenstand}} nicht allgemeing"ultig zu sein scheint.
 
Des Weiteren kann man sich folgendes Szenario vorstellen: Elon Musk hat sich 

\begin{itemize}
  \item er hat sich ersetzt mit einem roboter, der genau so denkt wie elon musk und so handelt wie das original es tun würde
  \item elon musk selbst beging suizid noch vor der gründung von paypal
  \item 
\end{itemize}



\end{onehalfspace}

\bibliography{sprachphilo-ha-2.bib}

\end{document}
