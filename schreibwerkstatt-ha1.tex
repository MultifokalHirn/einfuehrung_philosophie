\documentclass[a4paper]{article}
\usepackage{graphicx}
\usepackage{fullpage}
\usepackage{parskip}
\usepackage{color}
\usepackage[ngerman]{babel}
\usepackage{hyperref}
\usepackage{calc} 
\usepackage{enumitem}
\usepackage{titlesec}
%\pagestyle{headings}

\titleformat{name=\section,numberless}
  {\normalfont\Large\bfseries}
  {}
  {0pt}
  {}

\begin{document}

\title{
	\includegraphics*[width=0.65\textwidth]{images/hu_logo.png}\\
	\vspace{24pt}
	Philosophische Schreibwerkstatt\\Hausaufgabe}
%\subtitle{Vorlesung WS 16/17\\
%          Prof. Template\\
%          Philosophisches Institut I \\ 
%          Humboldt Universit"at zu Berlin}
\author{Lennard Wolf\\
        \href{mailto:lennard.wolf@student.hu-berlin.de}{lennard.wolf@student.hu-berlin.de}}
\maketitle


\section*{Zusammenfassung der Einleitung des Textes \\ \emph{Gibt es einen moralisch relevanten Unterschied zwischen L"ugen und Irref"uhren?} von Holger Baumann}

Die Einleitung des Essays \emph{Gibt es einen moralisch relevanten Unterschied zwischen L"ugen und Irref"uhren?} von Holger Baumann, erschienen in der Zeitschrift f"ur Praktische Philosophie (Band 2, Heft 1, 2015) erl"autert die Begriffe der L"uge und der Irref"uhrung, f"uhrt in das Thema der Diskussion "uber die moralische Signifikanz des Unterschiedes dieser zwei T"auschungsformen ein, gibt einen Einblick in den Standpunkt des Autors und f"asst die Struktur des Gesamttextes zusammen. 

Laut Baumann handelt es sich bei einer Aussage um eine \emph{L"uge}, wenn die redende Person den mitgeteilten Sachverhalt f"ur falsch h"alt und das Ziel verfolgt, dass die andere Person sie f"ur wahr nimmt. Eine Aussage sei wiederum eine \emph{Irref"uhrung}, wenn die redende Person den mitgeteilten Sachverhalt f"ur wahr h"alt, aber das Ziel verfolgt, dass die andere Person einen falschen Schlu\ss~zieht. Die Absicht der T"auschung sei in beiden F"allen gleich, nur die jeweilige Form verschieden. Ziel des Essays sei es nun zu diskutieren, ob dieser Formunterschied eine moralische Bedeutung hat, und ob entsprechend eines dem anderen in dieser Hinsicht vorzuziehen sei. 

Die am h"aufigsten vertretenen, sowie intuitiveren Meinungen seien, dass Irref"uhrungen ''moralischer'' sind, da L"ugen immer verboten sind, jedoch in gewissen Notsituationen T"auschungen n"otig sind, oder dass sie ''weniger schlecht'' sind, das hei\ss t durch das Ziel der T"auschung an sich den selben deontischen Status haben, aber im Vergleich Irref"uhrungen vorzuziehen sind.
Der Autor wird versuchen zu zeigen, dass der moralische Unterschied in der Beziehung zwischen den Personen, welche durch die Form der T"auschung kommuniziert wird, liegt. 

An Beispielen soll die genannte intuitive Meinung dargestellt, analysiert und systematisch diskutiert werden, dass Irref"uhrungen moralisch besser sind, gefolgt von dem selben Prozess f"ur die Gegenthese. Schlie\ss lich wird noch dargelegt, wie der Vorschlag des Autors der \emph{Irrtumstheorie} von Saul\footnote{Saul, Jennifer. 2012. \emph{Lying, misleading \& what is said}. Oxford: Oxford University Press.} "uberlegen ist.


\newpage

%\begin{figure}[h]
%	\centering
%	\includegraphics[width=0.5\textwidth]{images/template.png}
%	\caption{Template Bild}
%	\label{fig:template}
%\end{figure}


\end{document}
