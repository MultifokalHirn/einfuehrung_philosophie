\documentclass[emulatestandardclasses]{scrartcl}
\usepackage{graphicx}
\usepackage{color}
\usepackage[ngerman]{babel}
\usepackage{hyperref}
\usepackage{fullpage}
\usepackage[dvipsnames]{xcolor}
\usepackage{calc} 
\usepackage{enumitem}
\usepackage{titlesec}
\newcommand{\todo}[1]{\textcolor{red}{TODO: #1}\PackageWarning{TODO:}{#1!}}
\date{\vspace{-3ex}}
\begin{document}

\title{
	\includegraphics*[width=0.75\textwidth]{ErstesSem/images/hu_logo.png}\\
	\vspace{24pt}
	Kritische Theorie der Gesellschaft: Horkheimer und Adorno}
\subtitle{Proseminar SS 17\\
          Dr. Arnd Pollmann\\
          Philosophisches Institut I \\ 
          Humboldt Universit"at zu Berlin}
\author{Lennard Wolf\\
        \small{\href{mailto:lennard.wolf@student.hu-berlin.de}{lennard.wolf@student.hu-berlin.de}}}
\maketitle
\begin{abstract}

In den 1930er-Jahren ist rund um das Institut f"ur Sozialforschung in Frankfurt a.M. ein sozialkritischer und zeitdiagnostischer Forschungszusammenhang entstanden, den man "`Kritische Theorie"' oder auch "`Frankfurter Schule"' nennt. Unter der Schirmherrschaft der Philosophie wurden dort interdisziplin"are Forschungsvorhaben durchgef"uhrt, die im Anschluss an Karl Marx und Sigmund Freud vor allem der zentralen Frage nachgehen sollten, warum in Deutschland, trotz enorm wachsender sozialer Probleme, die sozialistische Revolution ausgeblieben war. Als Gr"undungsv"ater der Frankfurter Schule gelten Max Horkheimer und Theodor W. Adorno. Im Seminar werden wir eine Auswahl ihrer wichtigsten Arbeiten diskutieren. Die beiden Denker haben sich als au"serordentlich einflussreich erwiesen und die Studentenbewegung von 1968 ebenso gepr"agt wie das Denken nachfolgender Generationen kritischer Sozialphilosoph\_innen im Frankfurter "`Raum"' (z.B. J"urgen Habermas oder Axel Honneth).

\end{abstract}
\newpage

\tableofcontents
\listoffigures
\newpage


\section{Einf"uhrung / Adorno: Minima Moralia, Nr. 18
\\(19.04.17)}

\subsection{Einf"uhrung}

\begin{itemize}
  \item Entfremdetes Wohnen, maschinisierung des Wohnens $\rightarrow$ Selbes Denken das zu den KZs gef"uhrt hat
  \item Es gibt kein richtiges Leben im Falschen
\end{itemize}

\subsection{Adorno: Minima Moralia, Nr. 18 ("`Asyl f"ur Obdachlose"')}



\section{Horkheimer: "`Zur Kritik der gegenw"artigen Gesellschaft"'
\\(26.04.17)}

\subsection{Lekt"urenotizen}

\begin{description}[leftmargin=!,labelwidth=\widthof{\bfseries Dialektischer Materialismus}]
  \item[Produktivkr"afte] \emph{Alle} Kr"afte des Menschen, denen er sich bedient, um die Natur zu beherrschen (vor allem Wissenschaft, Technik)
  \item[Produktionsverh"altnisse] Beziehungen zwischen den Menschen, bestimmt durch die Produktivkr"afte 
  \item[Dialektik] ?
  \item[Dialektischer Materialismus] Durch Produktionsverh"altnisse bestimmte "Anderung der Produktivkr"afte $\rightarrow$ "Anderung der Produktionsverh"altnisse
\end{description}

\begin{itemize}
  \item Wie ist "`Kritik"' zu verstehen? Von wo kommt sie? Problem einer Kritik: Die Gesellschaft ist so "uberw"altigend f"ur den einzelnen, dass Kritik nichts mehr "andern kann $\rightarrow$ R"uckgang des Einzelsubjekts
  \item Revolution nach Marx: Unterdr"uckte Masse "uberwirft unterdr"uckende Minderheit aufgrund der immer schlechter werdenden Verh"altnisse $\rightarrow$ Heute: Staatliche Reglementierung macht, dass Bedingungen nicht zu schlecht werden | Wirtschaftliche Regulierung also nicht zum Schutze der Menschen, sondern zum Schutze der Verh"altnisse
   \item Elend ist heute anders als bei Marx: Zwar wird nicht mehr verhungert, aber man ist gefangen in Zustand und die Autonomie ist genommen
  \item Zuk"unftiges Problem: automatisierte Gesellschaft ohne h"ohere Ziele $\rightarrow$ Geist, Phantasie, Autonomie sind bedroht

\end{itemize}


\begin{enumerate}
  \item {\color{NavyBlue}Wie fügt sich der vorliegende Text in den thematischen Gesamtzusammenhang des Seminars ein?}\\
{\color{ForestGreen} Text steht am ende der Frankfurter Schule. Konkreter Text da bestimmte Gegebenheit analysiert werden. Ankn"upfungspunkt zu der Frage "`Muss man wenn man kritisiert eine bessere L"osung kennen?"' die aufkam, weil Adorno nur sagt was/dass schlecht ist.}
  \item {\color{NavyBlue}Wie lautet die zentrale philosophische Frage, auf die der Text eine Antwort zu geben versucht?}\\
{\color{ForestGreen} "`Wie m"ussen wir Gesellschaftskritik verstehen"' Kritik: r"ucksichtsloses in dem Sinne dass...}
    \item {\color{NavyBlue} Wie genau lautet die Antwort, die der Text auf die soeben identifizierte Hauptfrage zu geben versucht?}\\
{\color{ForestGreen} Zu kl"aren.}
    \item {\color{NavyBlue} Welches sind die f"ur das Verst"andnis des vorliegenden Textes zentralen Begriffe und Unterscheidungen?}\\
{\color{ForestGreen} Zu kl"aren.}
    \item {\color{NavyBlue} Wie lautet in aller K"urze das im Text pr"asentierte Gesamtargument?}\\
{\color{ForestGreen} Zu kl"aren.}
    \item {\color{NavyBlue} Wie unterscheidet sich die Theorie der Inauguraldissertation
eigentlich von der in der Kritik der reinen Vernunft (1781), wenn doch beide TI beinhalten?}\\
{\color{ForestGreen} Zu kl"aren.}
\end{enumerate}


\newpage
\section{"Uber den Professor}
Prof. Mustermann ist..


%\begin{figure}[h]
%	\centering
%	\includegraphics[width=0.5\textwidth]{images/template.png}
%	\caption{Template Bild}
%	\label{fig:template}
%\end{figure}

\end{document}
