\documentclass[]{scrartcl}
\usepackage{graphicx}
\usepackage{color}
\usepackage{german}
\usepackage{hyperref}
\usepackage{calc} 
\usepackage{enumitem}
%\pagestyle{headings}

% customize dictum format:
\usepackage[T1]{fontenc}
\setkomafont{dictumtext}{\itshape\small}
\setkomafont{dictumauthor}{\normalfont}
\renewcommand*\dictumwidth{\linewidth}
\renewcommand*\dictumauthorformat[1]{--- #1}
\renewcommand*\dictumrule{}
\newcommand{\todo}[1]{\textcolor{red}{TODO: #1}\PackageWarning{TODO:}{#1!}}

\begin{document}

\title{
	\includegraphics*[width=0.75\textwidth]{images/hu_logo.png}\\
	\vspace{24pt}
	Philosophische Schreibwerkstatt}
\subtitle{"Ubung WS 16/17\\
          Emanuel Viebahn\\
          Philosophisches Institut I \\ 
          Humboldt Universit"at zu Berlin}
\author{Lennard Wolf\\
        \href{mailto:lennard.wolf@student.hu-berlin.de}{lennard.wolf@student.hu-berlin.de}}
\maketitle
\begin{abstract}

In der Schreibwerkstatt werden wir anhand von Beispielen herausarbeiten, was einen gelungenen philosophischen Text ausmacht. Au\ss erdem werden wir durch "Ubungen und einen selbst verfassten Essay das philosophische Schreiben trainieren.

\end{abstract}
\newpage

\tableofcontents
\newpage

\listoffigures
\newpage


\section{Ziele des Kurses}
\begin{enumerate}
  \item Wissen
  \item Glück
  \item Freude
\end{enumerate}

\section{Organisatorisches}

\textcolor{red}{Fehlendes Bild}

\subsection{Wichtige Daten etc.}

\begin{itemize}
    \item Thema 1
    \item Thema 2
  \end{itemize}
\newpage
\section{Thematischer "Uberblick}


\begin{figure}[h]
	\centering
	\includegraphics[width=0.5\textwidth]{images/template.png}
	\caption{Template Bild}
	\label{fig:template}
\end{figure}


\textcolor{red}{Fehlender Text}


\newpage
\subsection{"Uber den Professor}
Prof. Mustermann ist..


\end{document}
