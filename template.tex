\documentclass[]{scrartcl}
\usepackage{graphicx}
\usepackage{color}
\usepackage{german}
\usepackage{hyperref}
\usepackage{calc} 
\usepackage{enumitem}
%\pagestyle{headings}

% customize dictum format:
\usepackage[T1]{fontenc}
\setkomafont{dictumtext}{\itshape\small}
\setkomafont{dictumauthor}{\normalfont}
\renewcommand*\dictumwidth{\linewidth}
\renewcommand*\dictumauthorformat[1]{--- #1}
\renewcommand*\dictumrule{}
\newcommand{\todo}[1]{\textcolor{red}{TODO: #1}\PackageWarning{TODO:}{#1!}}

\begin{document}

\title{
	\includegraphics*[width=0.75\textwidth]{images/hu_logo.png}\\
	\vspace{24pt}
	Template Kurs}
\subtitle{Vorlesung WS 16/17\\
          Prof. Template\\
          Philosophisches Institut I \\ 
          Humboldt Universit"at zu Berlin}
\author{Lennard Wolf\\
        \href{mailto:lennard.wolf@student.hu-berlin.de}{lennard.wolf@student.hu-berlin.de}}
\maketitle
\begin{abstract}

Zusammenfassung des Kurses Zusammenfassung des Kurses Zusammenfassung des Kurses Zusammenfassung des Kurses Zusammenfassung des Kurses Zusammenfassung des Kurses Zusammenfassung des Kurses Zusammenfassung des Kurses Zusammenfassung des Kurses Zusammenfassung des Kurses Zusammenfassung des Kurses Zusammenfassung des Kurses Zusammenfassung des Kurses Zusammenfassung des Kurses Zusammenfassung des Kurses Zusammenfassung des Kurses 

\end{abstract}
\newpage

\tableofcontents

\listoffigures
\newpage


\section{Section}
%\begin{enumerate}
%  \item enumerate
%\end{enumerate}



\subsection{Subsection}

%\begin{itemize}
%  \item Thema 1
%  \item Thema 2
%\end{itemize}

%\begin{figure}[h]
%	\centering
%	\includegraphics[width=0.5\textwidth]{images/template.png}
%	\caption{Template Bild}
%	\label{fig:template}
%\end{figure}


\newpage
\section{"Uber den Professor}
Prof. Mustermann ist..


\end{document}
