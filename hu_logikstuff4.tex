
\usepackage{rotating}
\usepackage{qtree}
%\usepackage{KMcalc} %Lennard

%allgemeine Einstellungen:
%\setlength{\parindent}{0em}
\makeatletter 
\renewcommand{\KMline}[3][]
{
\g@addto@macro\KMboxI{\refstepcounter{KMlineNumber}\if#1\empty\relax\else\label{#1}\fi
\hbox to \wd\KMtempboxA{\vphantom{$   #2 $   #3}\hfill\arabic{KMlineNumber}.}\vskip\fboxsep}
\g@addto@macro\KMboxII{\hbox{\vphantom{$   #2 $   #3}~$  #2$  ~}\vskip\fboxsep}
\g@addto@macro\KMboxIII{\hbox to \wd\KMtempboxB{\vphantom{$   #2 $   #3}\hfill #3}\vskip\fboxsep}
}
\renewcommand{\KMprove}[3][]
{
\g@addto@macro\KMboxI{\refstepcounter{KMlineNumber}\if#1\empty\relax\else\label{#1}\fi
\hbox to \wd\KMtempboxA{\vphantom{\KMProveword  $   #2$   #3}\hfill\arabic{KMlineNumber}.}\vskip 2\fboxsep\vbox\bgroup\vskip\fboxrule\vskip\fboxsep
}
\g@addto@macro\KMboxII{
\hbox{~\vphantom{\KMProveword $   #2 $   #3}~\KMProveword~  $   #2 $  ~ }\vskip 2\fboxsep\hbox\bgroup\vbox{\hbox{~}\vskip\fboxsep\vskip\fboxrule}
\vrule width\fboxrule\vbox\bgroup\hrule height\fboxrule\vskip\fboxsep
}
\g@addto@macro\KMboxIII{
\hbox to \wd\KMtempboxB{\vphantom{\KMProveword $   #2 $   #3}\hfill #3}\vskip 2\fboxsep\vbox\bgroup\vskip\fboxrule\vskip\fboxsep
}
}
\makeatother

%KM-Kalkül, geordnetes paar und gather-Umgebung für Formeln:
\newcommand{\UB}{$\und$B}
\newcommand{\UE}{$\und$E}
\newcommand{\OrE}{$\oder$E}
\newcommand{\OrB}{$\oder$B}
\newcommand{\EB}{$\ex$B}
\newcommand{\EE}{$\ex$E}
\newcommand{\paar}[1]{\left \langle #1 \right \rangle}
\newcommand{\gth}[2]{\begin{gather*} #1 \label{#2} \end{gather*}}
\newcommand{\gthd}[2]{\begin{gather} \begin{gathered}#1 \label{#2}\end{gathered}\end{gather}}

%objektsprachliche Symbole:
\newcommand{\und}{\wedge}
\newcommand{\oder}{\vee}
\newcommand{\then}{\rightarrow}
\newcommand{\eq}{\leftrightarrow}
\newcommand{\uu}{\cup}
\newcommand{\C}{\cap}
\newcommand{\TM}{\subseteq}
\newcommand{\all}{\forall}
\newcommand{\ex}{\exists}
% \neg gibt es schon

%Zahlbereichsymbole, Potenzmenge und Sprachnamen
\newcommand{\NN}{\mathbb{N}}
\newcommand{\QQ}{\mathbb{Q}}
\newcommand{\RR}{\mathbb{R}}
\newcommand{\Pp}{\mathcal{P}}
\newcommand{\Ll}{\ensuremath{\mathcal{L}}}
\newcommand{\pair}[1]{\langle #1 \rangle}
\newcommand{\quine}[1]{\ulcorner #1 \urcorner}
\newcommand{\mq}[1]{\mlq #1 \mrq} % steht für "math quotes"
\newcommand{\sse}{\subseteq}
\newcommand{\gesch}{\cap} % steht für "geschnitten"
\newcommand{\verin}{\cup} % Habe hier mit Absicht einen Buchstaben weggelassen, ähnlich wie \infty
\newcommand{\set}[1]{\{#1\}}
\newcommand{\LPL}{\ensuremath{\mathcal{L}_{PL}}}
\newcommand{\LAL}{{\ensuremath{\mathcal{L}_{AL}}}} % zusätzliche „{}“, um es in Subskripts nutzen zu können
\newcommand{\FmLAL}{\ensuremath{\mathcal{F}m_\LAL}} % zur besseren Lesbarkeit der AL-Definitionen
\newcommand{\SKLAL}{\ensuremath{\mathcal{SK}_\LAL}} % zur besseren Lesbarkeit der AL-Definitionen
\newcommand{\MIb}{{\pair{M, I}, \beta}} % zur Benutzung im Math-Mode


%metasprachliche Symbole:
\newcommand{\Land}{
    \raisebox{-0.12em}{ \begin{turn}{90} $\eqslantgtr$ \end{turn} }
}
\newcommand{\Lor}{
     \raisebox{-0.12em}{ \begin{turn}{90} $\eqslantless$ \end{turn} }
}
\newcommand{\Then}{
    \Rightarrow
}
\newcommand{\Gdw}{
    \Leftrightarrow
}
\newcommand{\Neg}{
    \neg\hspace*{-0.5em}\neg
}
\newcommand{\Forall}{
    \raisebox{0.18em}{\scriptsize{\textbackslash}}\hspace*{-0.175em}\forall
}
\newcommand{\Exists}{
    \exists\hspace*{-0.45em}\exists
}

\newcommand{\nehT}{\Leftarrow}

\DeclareMathSymbol{\mlq}{\mathord}{operators}{``}
\DeclareMathSymbol{\mrq}{\mathord}{operators}{`'}
\newcommand{\concat}{\raisebox{0.45em}{\scalebox{0.7}{$\smallfrown$}}}

%models gibt es schon
%\

\newcommand{\deduces}{\vdash}
\newcommand{\sidew}[1]{\begin{sideways} #1 \end{sideways}}
%Anführungszeichen

\newcommand{\qleft}{\ulcorner}
\newcommand{\qright}{\urcorner}
\newcommand{\qc}[1]{\qleft #1 \qright}
\newcommand{\anf}[1]{`#1'}
\newcommand{\manf}[1]{\text{`}#1\text{}}
\newcommand{\tmanf}[1]{\text{`#1'}}
%\renewcommand{\models}{\vDash}
\newcommand{\Anf}[1]{„#1“}
\newcommand{\cn}{\/^{\smallfrown}}
%Kopf

\newcommand{\varsheet}{}
\newcommand{\sheet}[1]{ %Nummer des Aufgabenblatts
    \renewcommand{\varsheet}{\mbox{\textbf{Aufgabenblatt #1}}}
}

\newcommand{\varhandin}{}
\newcommand{\handin}[1]{ %Abgabedatum
    \renewcommand{\varhandin}{\mbox{\textbf{Abgabedatum: #1}}}
}


\newcommand{\varcourse}{}
\newcommand{\course}[1]{ %Kursname
    \renewcommand{\varcourse}{\mbox{\scshape{#1}}}
}

\newcommand{\header}{
    \begin{tabular}{@{}p{\textwidth}@{}}
        \varcourse\\
        \\
        \makebox[\textwidth][s]{\varsheet \phantom{ } \varhandin}\\
        \hline
        %\varsheet \vspace{\stretch{1}} \varcourse \vspace{\stretch{1}} \varhandin\\
    \end{tabular}\vspace{1.5em plus 0.1em minus 0.1em}
}

\newcommand{\varDatum}{}
\newcommand{\Datum}[1]{ %Klausurdatum
    \renewcommand{\varDatum}{\mbox{\textbf{#1}}}
}

\newcommand{\semesterheader}{
    \begin{tabular}{@{}p{\textwidth}@{}}
        \varcourse\\
        \\
        \makebox[\textwidth][s]{\mbox{\textbf{Semesteraufgabe}} \phantom{ } \varhandin}\\
        \hline
        %\varsheet \vspace{\stretch{1}} \varcourse \vspace{\stretch{1}} \varhandin\\
    \end{tabular}\vspace{1.5em plus 0.1em minus 0.1em}
}
\newcommand{\semesterheaderl}{
    \begin{tabular}{@{}p{\textwidth}@{}}
        \varcourse\\
        \\
        \makebox[\textwidth][s]{\mbox{\textbf{Semesteraufgabe: Lösung}} \phantom{ } \varhandin}\\
        \hline
        %\varsheet \vspace{\stretch{1}} \varcourse \vspace{\stretch{1}} \varhandin\\
    \end{tabular}\vspace{1.5em plus 0.1em minus 0.1em}
}

%Klausurheader ohne Gruppe:
\newcommand{\Nachklausurheader}{
    \begin{tabular}{@{}p{\textwidth}@{}}
        \varcourse\\
        \\
        \makebox[\textwidth][s]{\mbox{\textbf{Nachklausur}} \phantom{ } \varDatum}\\
        \hline
        %\varsheet \vspace{\stretch{1}} \varcourse \vspace{\stretch{1}} \varhandin\\
    \end{tabular}\vspace{1.5em plus 0.1em minus 0.1em}
}

%Klausurheader ohne Gruppe:
\newcommand{\Klausurheader}{
    \begin{tabular}{@{}p{\textwidth}@{}}
        \varcourse\\
        \\
        \makebox[\textwidth][s]{\mbox{\textbf{Klausur}} \phantom{ } \varDatum}\\
        \hline
        %\varsheet \vspace{\stretch{1}} \varcourse \vspace{\stretch{1}} \varhandin\\
    \end{tabular}\vspace{1.5em plus 0.1em minus 0.1em}
}
\newcommand{\Wahlpflicht}{
    \begin{tabular}{@{}p{\textwidth}@{}}
        \\
        \makebox[\textwidth][s]{\mbox{\textbf{Wahlpflichtbereich}} \phantom{ }}\\
	\hline
        %\varsheet \vspace{\stretch{1}} \varcourse \vspace{\stretch{1}} \varhandin\\
    \end{tabular}\vspace{1.5em plus 0.1em minus 0.1em}
}

% % %mit Gruppe:
% \newcommand{\Klausurheader}[1]{
%   \begin{tabular}{@{}p{\textwidth}@{}}
%       \varcourse\hfill \framebox{\scshape{Gruppe #1}}\\
%       \\
%       \makebox[\textwidth][s]{\mbox{\textbf{Klausur}} \phantom{ } \varDatum}\\
%       \hline
%       %\varsheet \vspace{\stretch{1}} \varcourse \vspace{\stretch{1}} \varhandin\\
%   \end{tabular}\vspace{1.5em plus 0.1em minus 0.1em}
% }

%Q-Analyse
\newcommand{\qsep}[1]{\end{array}&\vline&\begin{array}[t]{c}} %der Parameter macht nicht, schafft aber Übersicht im eigentlichen Dokument

\newenvironment{qtab}[1]
    {
        \begin{array}[t]{c@{}c@{}c}
            &&\\
            &\scriptsize\fbox{$#1$}&\\
            \begin{array}[t]{c}
    }
    {       \end{array}
        \end{array}
    }

%Aufgaben

\newcounter{tasks}

 \newenvironment{task}[1]{
    \vspace{0.5em plus 0.3em minus 0.3em}
    \addtocounter{tasks}{1}
    \begin{tabular}{@{}p{\textwidth}@{}}
        \textbf{Aufgabe \arabic{tasks}} (#1)            \\
        %\hline
    \end{tabular}\vspace{0.5em plus 0.1em minus 0.1em}

 }
 {\vspace{1.5em plus 0.3em minus 0.3em}}

\newenvironment{Klausurtask}[2]{
    \vspace{0.5em plus 0.3em minus 0.3em}
    \addtocounter{tasks}{1}
    \begin{tabular}{@{}p{\textwidth}@{}}
        \textbf{Aufgabe \arabic{tasks}} (#1)\hfill (#2 P)\\
        %\hline
    \end{tabular}\vspace{0.5em plus 0.1em minus 0.1em}

}
{\vspace{1.5em plus 0.3em minus 0.3em}}


\newenvironment{ltask}[3]{
    \vspace{0.5em plus 0.3em minus 0.3em}
    \addtocounter{tasks}{#1}
    \begin{tabular}{@{}p{\textwidth}@{}}
        \textbf{Aufgabe \arabic{tasks}} (#2)\hfill (#3 P)\\
        %\hline
    \end{tabular}\vspace{0.5em plus 0.1em minus 0.1em}

}
{\vspace{1.5em plus 0.3em minus 0.3em}}







%Counterausgabe:
\newcommand\enumroman[1]{{(\textit{\roman{#1}})}}
\newcommand\enumalph[1]{{\alph{#1})}}
\newcommand\enumgrec[1]{{\engrec{#1})}}

\newcommand{\enumsubtask}{
    \ifthenelse{\value{subtaskdepth}=1}
        {\enumalph{subtaski}}
        {
            \ifthenelse{\value{subtaskdepth}=2}
                {\enumroman{subtaskii}}

                    \ifthenelse{\value{subtaskdepth}=3}
                    {\enumgrec{subtaskiii}}

                }
}

\newcommand{\usesubtaskcounter}{
    \ifthenelse{\value{subtaskdepth}=1}
        {\usecounter{subtaski}}
        {
            \ifthenelse{\value{subtaskdepth}=2}
                {\usecounter{subtaskii}}
                {\usecounter{subtaskiii}}
        }
}



%Listen:

\newcounter{subtaski}
\newcounter{subtaskii}
\newcounter{subtaskiii}
\newcounter{subtaskdepth}

\newcommand{\setsubtaskmargin}{
    \ifthenelse{\value{subtaskdepth}=1}{
        \setlength{\labelwidth}{2em}
        \setlength{\leftmargin}{2em}
    }
    {
        \ifthenelse{\value{subtaskdepth}=2}{
            \setlength{\labelwidth}{3.5em}
            \setlength{\leftmargin}{3.5em}

        }{
            \setlength{\labelwidth}{1em}
            \setlength{\leftmargin}{1em}
        }
    }
}

\newenvironment{subtasks}
   {    \addtocounter{subtaskdepth}{1}
    \begin{list}{\enumsubtask}{\usesubtaskcounter
                      \setlength{\topsep}{0.5ex plus0.1ex minus0.1ex}
                      \setlength{\itemsep}{0.1ex plus0.1ex minus0.1ex}
                      \setlength{\parsep}{0.1ex plus0.1ex minus0.1ex}
                      \setlength{\labelsep}{0em}
                      \setsubtaskmargin}}
   {    \end{list}
    \addtocounter{subtaskdepth}{-1}}




