\documentclass[a4paper, 12pt]{article}
\usepackage{CJKutf8} % japanese
\usepackage{graphicx}
\usepackage{hyperref}
\usepackage{fullpage}
%\usepackage{parskip}
\usepackage{color}
\usepackage[ngerman]{babel}
\usepackage{hyperref}
\usepackage{calc} 
\usepackage{enumitem}
\usepackage[utf8]{inputenc}
\usepackage{titlesec}
\usepackage{stmaryrd} %lightning symbol in formalisierung
%\pagestyle{headings}
\usepackage{setspace} %halbzeilig
\usepackage[style=authoryear-ibid,natbib=true]{biblatex}
\usepackage[hang]{footmisc}
\setlength{\footnotemargin}{-0.8em}
%\bibliographystyle{natdin}
\addbibresource{nakagin-hausarbeit.bib}
\DeclareDatamodelEntrytypes{standard}
\DeclareDatamodelEntryfields[standard]{type,number}
\DeclareBibliographyDriver{standard}{%
  \usebibmacro{bibindex}%
  \usebibmacro{begentry}%
  \usebibmacro{author}%
  \setunit{\labelnamepunct}\newblock
  \usebibmacro{title}%
  \newunit\newblock
  \printfield{number}%
  \setunit{\addspace}\newblock
  \printfield[parens]{type}%
  \newunit\newblock
  \usebibmacro{location+date}%
  \newunit\newblock
  \iftoggle{bbx:url}
    {\usebibmacro{url+urldate}}
    {}%
  \newunit\newblock
  \usebibmacro{addendum+pubstate}%
  \setunit{\bibpagerefpunct}\newblock
  \usebibmacro{pageref}%
  \newunit\newblock
  \usebibmacro{related}%
  \usebibmacro{finentry}}

%\titleformat{name=\section,numberless}
%  {\normalfont\Large\bfseries}
%  {}
%  {0pt}
%  {}
\date{\vspace{-3ex}}


\begin{document}

\title{\vspace{5ex}
	\includegraphics*[bb=0 0 720 200, width=0.72\textwidth]{ErstesSem/images/hu_logo.png}\\
	\vspace{30pt}
	\scshape\LARGE{Scheiterte der Nakagin Capsule Tower an den ihm zugrunde liegenden Ideen?}\\\vspace{20pt}}
\author{Japan und die architektonische Moderne\\eine Rezeptions- und Transformationsgeschichte (Seminar)\\
	\vspace{7pt}
          Dr. Kai Kappel -- Dr. Harald Salomon -- Dr. Tina Zürn\\\vspace{4pt}Lennard Wolf\\
        \small{Matrikelnummer: 583052}\\
        \small{E-Mail: \href{mailto:lennard.wolf@hu-berlin.de}{lennard.wolf@hu-berlin.de}}\\
        \small{Telefonnummer: +49 176 5687 4131}\\
        \small{Studiengang: Regionalstudien Asien/Afrika (Zweitfach)}\\
        \small{Modul 11: Aufbaukurs Gesellschaft/Transformation}}

\maketitle

\vspace{\fill}

\begin{minipage}[]{0.92\textwidth}
    \centering
    \onehalfspacing
    \large   
    15. Oktober 2018\\
    Sommersemester 2018

    \vspace{-20mm} 
\end{minipage}%
\thispagestyle{empty}
\newpage
%\clearpage
%\thispagestyle{empty}
%\tableofcontents
%\newpage
\setcounter{page}{1}

\begin{onehalfspace} 

\noindent\textbf{$(o)$ Einleitung}

\noindent In dieser Arbeit beschäftige ich mich mit dem tōkyōter \emph{Nakagin Capsule Tower}\footnote{Originaler, japanischer Name "`中銀カプセルタワー"' (nakagin kapuseru tawā). Im folgenden werde ich zuweilen die Abkürzung "`NCT"' verwenden.}, abgebildet auf Seite \pageref{fig:tawa}, und seinem Status als Aushängeschild für den Metabolismus. Ich befasse mich mit seinem historischen und kulturellen Kontext und zeige die zwei Ebenen auf, auf denen er als emblematisch für die gesamte Bewegung gedeutet werden kann. Des weiteren beschäftige ich mich mit der von Zhongjie Lin (2011) vorgelegten These, dass viele der dem Nakagin Capsule Tower zugrunde liegenden Ideen nun auch die selben Ideen seien, die nun drohen, ihn zu zerstören. 

 

\vspace{3mm}

Ich werde wie folgt vorgehen. In Abschnitt $(i)$ 

\vspace{5mm}
\noindent\textbf{$(i)$ Die Bewegung des Metabolismus und ihre Ideen} %beschreibung

\noindent Zhongjie Lin schreibt in seinem Essay "`Nakagin Capsule Tower: Revisiting the Future of the Recent Past"', der Kapselturm sei emblematisch für den Metabolismus.\footnote{\Cite[Siehe][S. 243]{gleiter}} Was dies genau bedeutet und was nicht, werde ich im Laufe dieser Arbeit versuchen herauszuarbeiten. Denn wenn er tatsächlich repräsentativ für die gesamte Bewegung wäre, hätte ich alternativ ein provokantes "`Bezeugt das Scheitern des Nakagin Capsule Towers die Unumsetzbarkeit metabolistischer Architektur?"' als Leitfrage wählen können. Weshalb diese Frage aber relativ einfach zu verneinen ist, wird sich im laufe dieses Abschnitts, in dem ich einen Überblick über die architektonische Bewegung des \emph{Metabolismus} gebe, schnell zeigen.

Der von Tange Kenzō mitinitiierte Metabolismus ist die erste architektonische Avantgardebewegung aus Japan.\footnote{\Cite[Vgl.][S. 243]{gleiter}} Ausgangspunkt der Bewegung war das Manifest Metabolism 1960 - Proposals for a New Urbanism, der zeitlich mit dem Schwinden der Leitbildfunktion der städtebaulichen Utopien von Leuten wie Le Corbusier einherging. Die Metabolisten radikalisierten die utopistischen Ideale dieser Leute und drehten das seit Mitte des 19. Jahrhunderts wegweisende Modernisierungsparadigma \emph{wakon - yosai} (japanische Kultur - westliche Technik) in sein Gegenteil um\footnote{\Cite[Vgl.][S. 243]{gleiter}.}, indem sie, entgegen der Forderung "`Richtung New York? Nein, - via Katsura!"'\footnote{\Cite[Vgl.][S. 2]{taut}.} von Bruno Taut an Japan aus dem Jahr 1933, die westliche Moderne mit ihren expansiven \emph{mega cities} und Wolkenkratzern annahmen und die japanische Kulturtechnik des stetigen Restaurierens und Neubauens, wie sie im Ise Schrein, dem "`Schrein der Architektur überhaupt"'\footnote{\Cite[Siehe][S. 19]{taut}.}, bis heute in Vollendung am Leben ist, auf neue Weise umsetzen wollten. In dieser Kulturtechnik offenbart sich der distinkt japanische Lösungsansatz für die Paradoxie zwischen Veränderung und Kontinuität, der die philosophische Grundlage für die urbanen Pläne der Metabolisten bildet\footnote{\Cite[Vgl.][S. 13]{lin2011nakagin}.} und neben dem Nakagin Capsule Tower besonders bezeichnend in Tange Kenzōs \emph{Yamanashi Broadcasting Center} (1967), das immerzu um neue Module erweitert werden kann, seine Umsetzung findet. Kurokawa Kishō schrieb zu diesem Kerngedanken der Bewegung: "`I thought that architecture is not permanent art, something that is completed and fixed, but rather something that grows towards the future, is expanded upon, renovated and developed. This is the concept of metabolism (metabolize, circulate and recycle)"'\footnote{\Cite[Siehe][S. 6]{agekuro}.}.

Die Pläne der Metabolisten nahmen meist immense Dimensionen an: Die große Bucht in Tōkyō sollte mit einem eigenen Stadtteil bebaut werden,\footnote{Siehe Tange Kenzōs \emph{Tokyo Bay Project} aus dem Jahr 1961.} überdimensionale Wohnwaben sollten bei Kurokawa Kishōs \emph{Helix City} (1961) in der Luft hängen, ebenso wie bei Ekuan Kenjis \emph{Dwelling City} (1964). Die Formsprache japanischer Tempelanlagen ist in metabolistischen Plänen häufig zu entdecken, zum Beispiel in den Dächern der Wohnhäuser für das \emph{Tokyo Bay Project}. Desweiteren ist die Metaphorik japanischer Gärten beispielsweise in den an Seerosen erinnernden Inseln der \emph{Marine City} (1963) von Kikutake Kiyonori wiederzufinden, wie auch der in der klassisch japanischen Gestaltung generell häufig anzutreffende Bezug auf die Natur, sowie ihr steter Einbezug. Der Begriff "`Metabolismus"' ist entsprechend auch der Biologie entnommen\footnote{Metabolismus ist ein Fachbegriff für \emph{Stoffwechsel}.} und kann auch als Bezug auf die architektonische Philosophie hinter dem Ise Schrein aufgefasst werden. 

Die anfänglichen Ideale des Naturbezugs und der klassisch japanischen Metaphern sind nach einem Jahrzehnt aber letztendlich einem "`technoiden Formalismus"’\footnote{\Cite[Siehe][S. 246]{gleiter}.} gewichen, der bei der Expo 1970 in Ôsaka zur Schau gestellt wurde. Dort stelle auch der XXX von Kurokawa Kishō vor, der der Ausgangspunkt für das Projekt des Nakagin Capsule Towers darstellt. Der developer (?).... 




\vspace{5mm}
\noindent\textbf{$(ii)$ Der Nakagin Capsule Tower}

\noindent Der 1972 fertiggestellte \emph{Nakagin Capsule Tower} steht in einem der Wirtschaftszentren Tōkyōs, dem Viertel \emph{Ginza} im Bezirk \emph{Chūō}, nicht unweit der geschäftigen \emph{Shinbashi} Kreuzung. Diese Platzierung ist von zentraler Bedeutung für den Zweck des Gebäudes, auf den ich später noch eingehen werde. Der Name "`Nakagin"' ist zusammengesetzt aus den Kanji "`中"'\footnote{Lesung: "`naka"'; Ungefähre Bedeutung: Mitte, Zentrum, innerhalb.} und "`銀"'\footnote{Lesung: "`gin"'; Ungefähre Bedeutung: Silber, silbern.}, wobei "`銀"' für "`銀座"' (Ginza) steht. So lässt sich für ihn in etwa die Bedeutung "`Im Zentrum Ginzas"' ableiten. Der Turm sticht mit seiner Höhe von 54m heutzutage nicht mehr aus seiner Umgebung heraus: Die angrenzenden Gebäude sind im Schnitt ähnlich hoch und die restlichen Bauten in der Gegend sind meistens höher, wenn nicht sogar sehr viel höher. Er misst eine Grundfläche von 429.5m$^2$ und erzeugt eine begehbare Gesamtfläche von ca. 3091m$^2$. Von dem Charme eines futuristischen Gebäudes, den er in den 70er Jahren wohl gehabt haben muss, ist heute kaum noch etwas übrig geblieben. 


\begin{figure}[h]
    \centering
    \includegraphics*[bb=0 0 2236 3742, width=0.50\textwidth]{naka/IMG_5303.jpg}
    \caption{Der Nakagin Capsule Tower (Oktober 2018).}
    \label{fig:tawa}
\end{figure}

Der Turm besteht im Wesentlichen aus drei Teilen: Dem zweistöckigen Servicebereich mit Empfang, gemeinsamen Waschräumen etc., zwei Turmskeletten und 144 austauschbaren Kapseln, die über 9, bzw. 11 Stockwerke in die Skelette eingehängt sind. Der sogenannte Servicebereich ist als permanente Struktur vorgesehen, ebenso wie die beiden Stahlbetontürme, die 25m tief in den Boden reichen.\footnote{Aussage Kurokawa in XXX.} Für die Kapseln, die aus präfabrizierten Teilen zusammengebaut werden, war ursprünglich wiederum eine Lebenserwartung von 25 Jahren eingeplant - keine der Kapseln wurde jedoch bisher ausgetauscht. Kurokawa sah den Prozess des Verfalls der Kapseln nicht als mechanischen, sondern als sozialen an: Je nachdem wie sich die Eigentümer um ihre Kapsel kümmern, muss im Zweifelsfall früher oder später ein Austausch stattfinden. 

Zwischen den Türmen bestehen auf verschiedenen Höhen insgesamt drei Übergänge. Die beiden Turmskelette sind unterschiedlich hoch und reichen nach der höchsten Kapsel noch ein paar Meter weiter nach oben, wo sie beide schräg angeschnitten sind. Auf dem höheren der beiden Türme steht der Name des Gebäudes in Kanji, sowie in lateinischen Zeichen. Die Türme haben in ihrer Mitte jeweils einen Aufzug, um den herum ein Treppenflur verläuft, der alle zwölf, beziehungsweise vierzehn Stockwerke, und somit die Kapseln, zugänglich macht.

Die unteren beiden Servicegeschosse heben sich stilistisch stark vom Rest des Gebäudes ab. Vom Erdgeschoss ist von außen nur die schwarze Fassade samt großer Glasfenster zu sehen, sowie zwölf Betonpfeiler mit quadratischem Querschnitt, die das herausragende, erste Obergeschoss tragen. Dessen Fassade ist mit weißen, rechteckigen Fliesen bekleidet und wird mittig komplett durch eine Fensterreihe durchzogen. Hinter diesen Fenstern befinden sich Dinge, für die in den Kapseln kein Platz ist, also unteranderem gemeinschaftliche Waschräume und Putzutensilien. Im Erdgeschoss ist ein Foyer (Siehe Abb. \ref{fig:eingang}) mit einer  Unter die Servicegeschossen befindet sich zudem noch ein Kellergeschoss.\footnote{Es liegen mir keine Informationen über dieses Kellergeschoss vor.}

\begin{figure}[h]
    \centering
    \includegraphics*[bb=0 0 3395 2248, width=0.65\textwidth]{naka/eingang.jpg}
    \caption{Das Foyer im Erdgeschoss des Servicebereichs (Oktober 2018).}
    \label{fig:eingang}
\end{figure}

Die in einer Schiffscontainerfabrik vorgefertigten Kapseln aus Stahlblech, die mit Asbest bespritzt und betongrau angestrichen sind, wurden mit einem auf dem Turmskelett befestigten Kran an ihren vorgesehenen Platz erhoben und an zwei tragenden Masten mit vier Hochspannungsschrauben befestigt. An den Stahlrahmen befinden sich die Versorgungsleitungen für Luftaustausch sowie Strom- und Wasserversorgung. Eine Kapsel misst, wie typisch für Schiffscontainer, 4m x 2.5m x 2m, und verfügt in seinem Originalzustand über ein Bett, eine kleine Küche, ein Bad, einen ausklappbaren Tisch samt Stuhl und Schränken. Fest eingebaut sind (beziehungsweise waren) zudem ein Rechner, ein Fernseher, ein Tonbandgerät und ein Digitalwecker. Vom Bett aus kann all dies bedient werden. An einem Ende befindet sich die Tür zum Flur, am anderen Ende ein großes, zweischichtiges Fenster, das an ein Bullauge erinnern lässt und an das ursprünglich einen Ventilator befestigt war. Dieses Fenster kann nicht geöffnet werden, weshalb frische Luft durch die Versorgungsleitung zugeführt werden muss. Diese 1972 als ultramodern angesehenen Kapseln wurden zum damaligen Preis eines Mittelklassewagens angeboten und waren innerhalb eines Monats ausverkauft.\footnote{XXX} Es gibt von jeder Kapsel zwei Ausführungen, die aber jeweils nur eine gespiegelte Version der anderen darstellt. Eine einzelne Kapsel sollte als eine Übernachtungsmöglichkeit für eine Person genutzt werden. 

Heute befinden sich die mit Auffangnetzen verhangenen Kapseln in sehr schlechtem Zustand. Man kann durch einige ihrer Fenster erkennen, dass sie als Abstellkammern genutzt werden, die Außenwände der Kapseln sind grau und heruntergekommen, improvisierte Kabelnetze hängen an ihnen herunter. Im Stadtbild fällt der Turm daher leider, wenn er es überhaupt tut, eher negativ auf. Der Skelettbau dürfte weiterhin stabil sein, doch insgesamt entspricht der Turm nicht mehr den heutigen Standards für Erdbebensicherheit.

\begin{figure}[h]
    \centering
    \includegraphics*[bb=0 0 3901 2246, width=0.65\textwidth]{naka/netze.jpg}
    \caption{Die heruntergekommenen Kapseln, behangen mit Auffangnetzen (Oktober 2018).}
    \label{fig:netze}
\end{figure}

Es werden Führungen durch das Innere des Turmes angeboten, die mit oder AirBnB\footnote{Siehe \url{https://www.airbnb.de/experiences/69248}. Zuletzt abgerufen am 10.10.2018.} oder direkt beim Anbieter \emph{Showcase Toyko}\footnote{Siehe \url{http://showcase-tokyo.com/tours/nakagin.html}. Zuletzt abgerufen am 10.10.2018.} gebucht werden können. Ein Teil der Erträge der Führung geht an das \emph{Nakagin Capsule Tower Building Preservation and Restoration Project}, das sich für den Erhalt des Kapselturms einsetzt und im Jahr 2015 das Buch "`中銀カプセルタワービル 銀座の白い箱舟"'\footnote{中銀カプセルタワービル保存再生プロジェクト (2015). \emph{銀カプセルタワービル 銀座の白い箱舟}. 青月社, Tōkyō. ISBN: 9784810912883.} veröffentlicht hat. Die Führungen werden vom Repräsentanten der Organisation, Maeda Tatsuyuki, geleitet.

Das Gebäude wurde von Kurokawa Kishō gestaltet, und sollte durch seinen Grundgedanken des "`Recyclens"' die Kerngedanken des Metabolismus emblematisch realisieren. Die Gesamtstruktur sollte 200 Jahre bestehen können, während die einzelnen Kapseln einen Lebenszyklus von 25 Jahren haben sollten. Kurokawa hatte die Grundidee das erste Mal in der \emph{Takara Baeutilion} für die Expo 1970 in Ôsaka realisiert, was großes Aufsehen erregte. Watanabe Torizo, der Präsident des Immobilienunternehmens \emph{Nakagin Co.}, war beeindruckt und gab Kurokawa den Bauauftrag für ein ähnliches Gebäude, in dem Leute aber auch tatsächlich leben könnten. Im selben Jahr begannen die Pläne und zwei Jahre später war das Gebäude fertig gestellt. 

Im Jahr 2007 wurde beschlossen, dass der Turm zum Abriss bereit sei.\footnote{Siehe \url{https://www.architecturalrecord.com/articles/3635-kurokawa-s-capsule-tower-to-be-razed}. Zuletzt abgerufen: 13.10.2018.} Zhongjie Lin schreibt hierzu: "`This news astonished many readers because the Nakagin building is not only an icon of postwar modern architecture in Japan, but it also represents a rare and arguably the finest built work resulting from the historic Metabolist movement."'\footnote{\Cite[Siehe][S. 14]{lin2011nakagin}.}.



\vspace{5mm}
\noindent\textbf{$(iii)$ Kurokawas Umsetzung metabolistischer Ideale im NCT}

\noindent Kurokawa Kishō, geboren am 8. April 1934 und verstorben am 12. Oktober 2007, war eine der zentralen Figuren in der Bewegung des Metabolismus. Er studierte als \emph{graduate student} an der Universität Tokyo in der Forschungsabteilung von Tange Kenzō, als 1959, dem letzten Jahr des C.I.A.M. (Congrès Internationaux d'Architecture Moderne), Le Corbusier in einem Brief an Tange verkündete, dass sein Zeitalter vorbei sei.\footnote{\Cite[Siehe][S. 2]{agekuro}.} Dieses Zeitalter, so Kurokawa, war das Zeitalter der Maschine,\footnote{Man denke an Le Corbusiers berühmte Aussage, das Haus sei eine Wohn\emph{maschine}.} und so verkündete er noch im selben Jahr in seinem Text "`From the Age of Machine to the Age of Life"', dass das Zeitalter der neuen architektonischen Generation das "`Zeitalter des Lebens"' (\emph{Age of Life}) sei.\footnote{\Cite[Siehe][S. 3]{agekuro}.} 

Japanische Traditionen wie der Genkan (eingesenkter Eingangsbereich in Wohnung oder Haus, in dem Schuhe abgestellt werden) werden für pragmatische Ziele wie Platzersparnis eingetauscht. 

In seiner sogenannten "`Kapselschrift"' schreibt Kurokawa vom \emph{homo movens}, dem Menschen, der stets mobil ist. Sein eigentliches Heim habe dieser womöglich etwas außerhalb von Tōkyō, doch um unter der Woche schnell auf Arbeit sein zu können, würde dieser "`urbane Nomade"' die Kapsel als Zweitwohnort nutzen können. Viel Platz brauche ein solche Nomade nicht, da er nie zu lang an der selben Stelle bleibe. Die Kapseln sollten laut Kurokawa nicht nur äußerlich an Vogelnester erinnern, sondern in ihrer Benutzung diesen auch gleichen: Der geschäftige Mensch kommt dort vorbei um Energie zu tanken, etwas zu essen und sich dann rasch wieder auf den Weg zu machen. Solche Vogelnester brauchen zudem auch nicht für die Ewigkeit sein, denn der Nomade wird irgendwann fortgehen oder sterben, die Technik und Moden werden sich ändern, die Kirschblüten werden fallen. Sie sind Teil des natürlichen Kreislaufs, des metabolischen, stoffwechselnden Zyklus der Natur. Die Idee der Kapseln hat Kurokawa in seinen weiteren Gebäuden immer wieder auftauchen lassen. Das erste Kapselhotel in Ôsaka stammt von ihm, und kurz nach dem Nakagin Tower gestaltete er das \emph{Capsule House K} (1971 - 1973). 


Der zentrale Gedanke des Metabolismus, dass der organische Kreislauf von Geburt, Veränderung, Verfall und Tod in die Architektur aufgenommen werden soll, ist im Nakagin Capsule Tower aber nur theoretisch umgesetzt worden. Keine Kapsel wurde ausgetauscht, die Grundstruktur entspricht schon längst nicht mehr heutigen Sicherheitsstandards für Erdbeben etc. und Kurokawa kämpft gegen den Abriss des Gebäudes, der eigentlich die logische Konsequenz seines Grundgedankens ist. (Zitat) Dass das Gebäude nie so funktioniert hat wie es sollte war sicher kein Zufall.

Daher würde ich mit meiner These gegen den Metabolismus abschließen wollen: Der Nakagin Capsule Tower ist sowohl emblematisch für den metabolistischen Gedanken an sich, als auch für seine letztendliche Undurchführbarkeit. Die Natur probiert und verwirft, alles ist zyklisch, doch der menschliche Geist kann sich an diesen Gedanken niemals vollends gewöhnen. Er brauch die Sicherheit des Permanenten, ansonsten geht er im Chaos zu Grunde. 

\vspace{5mm}
\noindent\textbf{$(iv)$ Das Scheitern des Projekts}


\begin{itemize}
  \item Die Leute scheinen nicht interessiert gewesen zu sein an dem Gebäude sonst wär es nicht gescheitert
  \item das Langzeitgedächtnis ist bei Menschen relativ kurz und auch wenn Teile ausgetauscht werden, der fundamentale Fehler ist der weiterexistierende Kern, der bestimmte Form aufzwingt
  \item Gründe für das Scheiterns
  \item Finanzielle Lage des Landes
  \item Die Ironie des Kampfes um den Erhalt des Gebäudes
  \item lin argumentiert die werte die dem metabolistischen ANsatz unterliegen seien die selben, die ihn nun zerstören,
\end{itemize}

Zhongjie Lin (2011) stellte die These auf, dass jene Werte des konstanten Erneuerns, die dem metabolistischen Ansatz zugrunde liegen, die selben seien, die ihn 


\vspace{5mm}
\noindent\textbf{$(v)$ }



\begin{itemize}
  \item Metabolismus ist mehr als was in dem Turm umgesetzt wurde
  \item Andere Gebäude sind umgesetzt
  \item Im Fall, dass es 
\end{itemize}



\vspace{5mm}
\noindent\textbf{$(vi)$ Konklusion}

\begin{itemize}
  \item Anderes Modell für Tower?
  \item Ob die Ideen vielleicht doch unumsetzbar sind, oder von den Menschen nicht aufgenommen werden können, bleibt unklar
  \item Dass der Nakagin Tower dies eindeutig bezeugt, ist aber aufgrund der wirtschaftlichen Events in Japan schwer zu bejahen
  \item 
\end{itemize}

------


Gedanken


\begin{itemize}
  \item hohe kosten für land
  \item die fläche könnte viel effizienter genutzt werden!
  \item die frage ist, ob das prinzip der transformation in der japanischen architekturgeschichte beibehalten wird indem die kapseln ausgetauscht werden, oder indem das gesamte gebäude ausgetauscht wird. man muss daran denken dass beim neubau von schreinen das alte holz häufig wiederverwendet wurde
  \item Ise: Prototype of Japanese Architecture
\end{itemize}


\end{onehalfspace}
\nocite{*}
\printbibliography

%\newpage
%\noindent\textbf{Anhang}
%\vspace{6pt}
%
%\noindent 

\end{document}
