\documentclass[a4paper, 12pt]{article}
%\usepackage{CJKutf8} % japanese
\usepackage{graphicx}
\usepackage{hyperref}
\usepackage{fullpage}
%\usepackage{parskip}
\usepackage{color}
\usepackage[ngerman]{babel}
\usepackage{hyperref}
\usepackage{calc} 
\usepackage{enumitem}
\usepackage[utf8]{inputenc}
\usepackage{titlesec}
\usepackage{stmaryrd} %lightning symbol in formalisierung
%\pagestyle{headings}
\usepackage{setspace} %halbzeilig
\usepackage[style=authoryear-ibid,natbib=true]{biblatex}
\usepackage[hang]{footmisc}
\setlength{\footnotemargin}{-0.8em}
%\bibliographystyle{natdin}
\addbibresource{scheler-hausarbeit.bib}
\DeclareDatamodelEntrytypes{standard}
\DeclareDatamodelEntryfields[standard]{type,number}
\DeclareBibliographyDriver{standard}{%
  \usebibmacro{bibindex}%
  \usebibmacro{begentry}%
  \usebibmacro{author}%
  \setunit{\labelnamepunct}\newblock
  \usebibmacro{title}%
  \newunit\newblock
  \printfield{number}%
  \setunit{\addspace}\newblock
  \printfield[parens]{type}%
  \newunit\newblock
  \usebibmacro{location+date}%
  \newunit\newblock
  \iftoggle{bbx:url}
    {\usebibmacro{url+urldate}}
    {}%
  \newunit\newblock
  \usebibmacro{addendum+pubstate}%
  \setunit{\bibpagerefpunct}\newblock
  \usebibmacro{pageref}%
  \newunit\newblock
  \usebibmacro{related}%
  \usebibmacro{finentry}}

%\titleformat{name=\section,numberless}
%  {\normalfont\Large\bfseries}
%  {}
%  {0pt}
%  {}
\date{\vspace{-3ex}}


\begin{document}

\title{\vspace{5ex}
	\includegraphics*[bb=0 0 720 200, width=0.72\textwidth]{ErstesSem/images/hu_logo.png}\\
	\vspace{30pt}
	\scshape\LARGE{Bezeugt das Scheitern des Nakagin Capsule Towers die Unumsetzbarkeit metabolistischer Architektur?}\\\vspace{20pt}}
\author{Japan und die architektonische Moderne\\eine Rezeptions- und Transformationsgeschichte (Seminar)\\
	\vspace{7pt}
          Dr. Kai Kappel -- Dr. Harald Salomon -- Dr. Tina Zürn\\\vspace{4pt}Lennard Wolf\\
        \small{Matrikelnummer: 583052}\\
        \small{E-Mail: \href{mailto:lennard.wolf@hu-berlin.de}{lennard.wolf@hu-berlin.de}}\\
        \small{Telefonnummer: +49 176 5687 4131}\\
        \small{Studiengang: Regionalstudien Asien/Afrika (Zweitfach)}\\
        \small{Modul: ?}}

\maketitle

\vspace{\fill}

\begin{minipage}[]{0.92\textwidth}
    \centering
    \onehalfspacing
    \large   
    10. September 2018\\
    Sommersemester 2018

    \vspace{-20mm} 
\end{minipage}%
\thispagestyle{empty}
\newpage
%\clearpage
%\thispagestyle{empty}
%\tableofcontents
%\newpage
\setcounter{page}{1}

\begin{onehalfspace} 

\noindent\textbf{$(o)$ Einleitung}

\noindent In dieser Arbeit 



Diese Frage kann gestellt werden, doch kann sie gleich 

\vspace{3mm}

Ich werde wie folgt vorgehen. In Abschnitt $(i)$ 

\vspace{5mm}
\noindent\textbf{$(i)$ Die Bewegung des Metabolismus und ihre Ideen} %beschreibung

Die Tante XXX schrieb in ihrem Text XXX, der Kapselturm sei emblematisch für den Metabolismus.\footnote{XXX} Was dies genau bedeutet und was nicht, werde ich im Laufe dieser Arbeit versuchen herauszuarbeiten. Denn wenn er tatsächlich repräsentativ für die gesamte Bewegung wäre, hätte ich alternativ ein provokantes "`Bezeugt das Scheitern des Nakagin Capsule Towers die Unumsetzbarkeit metabolistischer Architektur?"' als Leitfrage wählen können. Weshalb diese Frage aber relativ einfach zu verneinen ist, wird sich im laufe dieses Abschnitts, in dem ich einen Überblick über die architektonische Bewegung des \emph{Metabolismus} gebe, schnell zeigen.

Der sogenannte Metabolismus findet seinen Ursprung in dem Manifest "`XXX"', verfasst von XXX im Jahre 19XX. 
\begin{itemize}
  \item Historischer Kontext
  \item Hauptleute
  \item Manifest
  \item Zu Kurokawa etwas mehr
  \item  
\end{itemize}

snippets

\begin{itemize}
  \item hohe kosten für land
  \item die fläche könnte viel effizienter genutzt werden!
  \item die frage ist, ob das prinzip der transformation in der japanischen architekturgeschichte beibehalten wird indem die kapseln ausgetauscht werden, oder indem das gesamte gebäude ausgetauscht wird. man muss daran denken dass beim neubau von schreinen das alte holz häufig wiederverwendet wurde
  \item Ise: Prototype of Japanese Architecture
  \item 
\end{itemize}


\vspace{5mm}
\noindent\textbf{$(ii)$ Der Nakagin Capsule Tower}

%------------------%
%    Einleitung    %
%------------------%

Baubeschreibung

Der 1972 fertiggestellte Nakagin Capsule Tower steht im Wirtschaftszentrum Tokyos, dem Stadtteil Giza, nicht unweit der geschäftigen Shinbashi Station. Dass diese Platzierung dem Zweck des Gebäudes entspricht, wird sich später bei der Besprechung der Gedanken des Architektens Kurokawa noch zeigen. Der Turm sticht durch seine Höhe von 54m nicht heraus, die Gebäude die an ihn grenzen sind im Schnitt ähnlich hoch, in der Nähe stehe zudem noch um einiges höhere Wolkenkratzer. Er misst eine Grundfläche von 429.5m² und hat eine Gesamtfläche von ca. 3091m². Den Charme eines futuristischen Gebäudes hat es für die Anwohner inzwischen verloren und es wird eher als ein altes, marodes Gebäude wahrgenommen. 
Der Turm besteht im wesentlichen aus “drei” Teilen: Die ersten beiden Service Geschosse mit Empfang (Folien), gemeinsamen Waschräumen etc.; zwei Turmskelette; und 144 austauschbare Kapseln, die über 9, bzw. 11 Stockwerke in diese Skelette eingesetzt werden (Folien). Der sogenannte Servicebereich ist als permanente Struktur vorgesehen, ebenso wie die beiden Stahlbetontürme, die 25m tief in den Boden reichen. (Anm: Diese Zahl habe ich einem Interview mit Kurokawa entnommen, ich konnte sie aber nirgendwo nachprüfen). Für die Kapseln, die aus präfabrizierten Teilen zusammengebaut werden, war ursprünglich wiederum eine Lebenserwartung von 25 Jahren eingeplant - keine der Kapseln wurde jedoch bisher ausgetauscht. Kurokawa sah den Prozess des Verfalls der Kapseln nicht als mechanischen, sondern als sozialen an: Je nachdem wie sich die Eigentümer um ihre Kapsel kümmern, muss im Zweifelsfall früher oder später ein Austausch stattfinden. 
Zwischen den Türmen existieren auf verschiedenen Höhen insgesamt drei Übergänge. Die beiden Türme sind unterschiedlich hoch und reichen nach der höchsten Kapsel noch weiter nach oben, wo sie beide schräg angeschnitten sind. Auf dem höheren der beiden Türme steht in Kanji der Name der Firma, die den Bauauftrag gegeben hat, und der auch als Name für den Turm an sich übernommen wurde: Nakagin. (Zentralbank?) Die Türme haben in ihrer jeweiligen Mitte einen Aufzug, um den herum ein Treppenflur verläuft, der alle 12 bzw 14 Stockwerke zugänglich macht.
Die unteren beiden Servicegeschosse heben sich stark vom Rest des Gebäudes ab. Von Erdgeschoss ist von außen nur die schwarze Fassade, samt großer Glasfenster zu sehen, sowie Betonpfähle, die das heraus ragende erste Stockwerk tragen. Dessen Fassade ist in weiß gehalten und wird durch eine Fensterreihe mittig komplett durchzogen. Hinter diesen Fenstern befinden sich, wie schon erwähnt, gemeinschaftliche Waschräume, Putzutensilien etc. Unter diesen Servicegeschossen gefindet sich zudem noch ein Kellergeschoss. 
Die in einer Schiffscontainerfabrik vorgefertigten Kapseln aus Stahlblech, die mit Asbest bespritzt und betongrau angestrichen sind, wurden mit Krähnen an ihren vorgesehenen Platz erhoben und an zwei tragende Masten mit vier Hochspannungsschrauben befestigt. An den Stahlrahmen befinden sich die Versorgungsleitungen für Luftaustausch sowie Strom- und Wasserversorgung. Eine Kapsel misst, wie typische Schiffscontainer, 4m x 2.5m x 2.5m, und verfügt über ein Bett, eine kleine Küche, ein Bad, einen ausklappbaren Tisch samt Stuhl und Schränken. Fest eingebaut sind zudem ein Rechner, ein Fernseher, ein Tonbandgerät und ein Digitalwecker. Vom Bett aus kann all dies bedient werden. An einem Ende befindet sich die Tür zum Flur, am anderen Ende ein großes, zweischichtiges Fenster, das an ein Bullauge erinnern lässt und an das ursprünglich einen Ventiltor befestigt war. Dieses Fenster kann nicht geöffnet werden, frische Luft wird durch die Versorgungsleitung zugeführt. Diese als ultramodern angesehenen Kapseln wurden zum Preis eines Mittelklassewagens angeboten und waren innerhalb eines Monats ausverkauft. Die ‘Wohnwaben’ stellen jeweils ein Appartement für eine Person dar und es gibt von jeder Kapsel zwei Ausführungen, die aber jeweils nur eine gespiegelte Version der anderen darstellt. 


Historischer Hintergrund

Das Gebäude wurde von Kurokawa Kishoo gestaltet, und sollte durch seinen Grundgedanken des “Recyclens” die Kerngedanken des Metabolismus emblematisch realisieren. Die Gesamtruktur sollte 200 Jahre bestehen können, während die einzelnen Kapseln einen Lebenszyklus von 25 Jahren haben sollten. Kurokawa hatte die Grundidee das erste Mal in der Takara Baeutilion für die Oosaka World Fair 1970 realisiert, was großes Aufsehen erregte. Watanabe Torizo, der Präsident des Immobilienunternehmens Nakagin Co., war beeindruckt und gab Kurokawa den Bauauftrag für ein ähnliches Gebäude, in dem Leute aber auch tatsächlich leben könnten. Im selben Jahr begannen die Pläne und zwei Jahre später war das Gebäude fertig gestellt. 


Gedanken Kurokawas

In seiner Kapselschrift schreibt Kurokawa vom homo movens, dem Menschen, der stets mobil ist. Sein eigentliches Heim habe dieser womöglich etwas außerhalb von Tokyo, doch um unter der Woche schnell auf Arbeit sein zu können, würde dieser “urbane Nomade” die Kapsel als Zweitwohnort nutzen können. Viel Platz brauche so ein Nomade nicht, da er normalerweise unterwegs sei. Sie sollten laut Kurokawa nicht nur äußerlich an Vogelnester erinnern, sondern in ihrer Benutzung diesen auch gleichen: Der geschäftige Mensch kommt dort vorbei um Energie zu tanken, etwas zu essen und dann rasch wieder auf den Weg zu machen. Und diese Vogelnester brauchen nicht für die Ewigkeit sein, denn der Nomade wird irgendwann fortgehen oder sterben, die Technik und Moden werden sich ändern, die Kirschblüten werden fallen. Sie sind Teil des natürlichen Kreislaufs, des metabolischen Zyklus. Die Idee der Kapseln hat Kurokawa in seinen weiteren Gebäuden immer wieder auftauchen lassen. Das erste Kapselhotel in Oosaka stammt von ihm, und kurz nach dem Nakagin Tower gestaltete er das Capsule House K (1971 - 1973). 


Der zentrale Gedanke des Metabolismus, dass der organische Kreislauf von Geburt, Veränderung, Verfall und Tod in die Architektur aufgenommen werden soll, ist im Nakagin Capsule Tower nur theoretisch umgesetzt worden. Keine Kapsel wurde ausgetauscht, die Grundstruktur entspricht schon längst nicht mehr heutigen Sicherheitsstandards für Erdbeben etc. und Kurokawa kämpft gegen den Abriss des Gebäudes, der eigentlich die logische Konsequenz seines Grundgedankens ist. (Zitat) Dass das Gebäude nie so funktioniert hat wie es sollte war sicher kein Zufall.
Daher würde ich mit meiner These gegen den Metabolismus abschließen wollen: Der Nakagin Capsule Tower ist sowohl emblematisch für den metabolistischen Gedanken an sich, als auch für seine letztendliche Undurchführbarkeit. Die Natur probiert und verwirft, alles ist zyklisch, doch der menschliche Geist kann sich an diesen Gedanken niemals vollends gewöhnen. Er brauch die Sicherheit des Permanenten, ansonsten geht er im Chaos zu Grunde. 


%------------------%
%    Hauptteil     %
%------------------%

\begin{itemize}
  \item Baubeschreibung
  \item Beschreibung des finanziellen Modells
  \item Beschreibung des heutigen Zustands 
  \item Der Turm als Symbol für die Bewegung (einzig umgesetztes Gebäude)
\end{itemize}


%------------------%
%    Konklusion    %
%------------------%

\vspace{5mm}
\noindent\textbf{$(iii)$ Argumente für die These}

%------------------%
%    Einleitung    %
%------------------%

%------------------%
%    Hauptteil     %
%------------------%

\begin{itemize}
  \item Die Leute scheinen nicht interessiert gewesen zu sein an dem Gebäude sonst wär es nicht gescheitert
  \item das Langzeitgedächtnis ist bei Menschen relativ kurz und auch wenn Teile ausgetauscht werden, der fundamentale Fehler ist der weiterexistierende Kern, der bestimmte Form aufzwingt
  \item 
\end{itemize}


%------------------%
%    Konklusion    %
%------------------%

\vspace{5mm}
\noindent\textbf{$(iv)$ Der Nakagin Capsule Tower im historischen Kontext}

%------------------%
%    Einleitung    %
%------------------%

%------------------%
%    Hauptteil     %
%------------------%

\begin{itemize}
  \item Im Sinne eines Zeitstrahls durchgehen
  \item Finanzielle situation in japan
  \item fin. sit in der welt
  \item 
\end{itemize}


%------------------%
%    Konklusion    %
%------------------%

\vspace{5mm}
\noindent\textbf{$(v)$ Argumente gegen die These}

%------------------%
%    Einleitung    %
%------------------%

%------------------%
%    Hauptteil     %
%------------------%

\begin{itemize}
  \item Metabolismus ist mehr als was in dem Turm umgesetzt wurde
  \item Andere Gebäude sind umgesetzt
  \item Im Fall, dass es 
\end{itemize}


%------------------%
%    Konklusion    %
%------------------%


\vspace{5mm}
\noindent\textbf{$(vi)$ Konklusion}

\begin{itemize}
  \item Anderes Modell für Tower?
  \item Ob die Ideen vielleicht doch unumsetzbar sind, oder von den Menschen nicht aufgenommen werden können, bleibt unklar
  \item Dass der Nakagin Tower dies eindeutig bezeugt, ist aber aufgrund der wirtschaftlichen Events in Japan schwer zu bejahen
  \item 
\end{itemize}


\end{onehalfspace}
%\nocite{*}
\printbibliography

%\newpage
%\noindent\textbf{Anhang}
%\vspace{6pt}
%
%\noindent 

\end{document}
