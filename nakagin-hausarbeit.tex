\documentclass[a4paper, 12pt]{article}
%\usepackage{CJKutf8} % japanese
\usepackage{graphicx}
\usepackage{hyperref}
\usepackage{fullpage}
%\usepackage{parskip}
\usepackage{color}
\usepackage[ngerman]{babel}
\usepackage{hyperref}
\usepackage{calc} 
\usepackage{enumitem}
\usepackage[utf8]{inputenc}
\usepackage{titlesec}
\usepackage{stmaryrd} %lightning symbol in formalisierung
%\pagestyle{headings}
\usepackage{setspace} %halbzeilig
\usepackage[style=authoryear-ibid,natbib=true]{biblatex}
\usepackage[hang]{footmisc}
\setlength{\footnotemargin}{-0.8em}
%\bibliographystyle{natdin}
\addbibresource{scheler-hausarbeit.bib}
\DeclareDatamodelEntrytypes{standard}
\DeclareDatamodelEntryfields[standard]{type,number}
\DeclareBibliographyDriver{standard}{%
  \usebibmacro{bibindex}%
  \usebibmacro{begentry}%
  \usebibmacro{author}%
  \setunit{\labelnamepunct}\newblock
  \usebibmacro{title}%
  \newunit\newblock
  \printfield{number}%
  \setunit{\addspace}\newblock
  \printfield[parens]{type}%
  \newunit\newblock
  \usebibmacro{location+date}%
  \newunit\newblock
  \iftoggle{bbx:url}
    {\usebibmacro{url+urldate}}
    {}%
  \newunit\newblock
  \usebibmacro{addendum+pubstate}%
  \setunit{\bibpagerefpunct}\newblock
  \usebibmacro{pageref}%
  \newunit\newblock
  \usebibmacro{related}%
  \usebibmacro{finentry}}

%\titleformat{name=\section,numberless}
%  {\normalfont\Large\bfseries}
%  {}
%  {0pt}
%  {}
\date{\vspace{-3ex}}


\begin{document}

\title{\vspace{5ex}
	\includegraphics*[bb=0 0 720 200, width=0.72\textwidth]{ErstesSem/images/hu_logo.png}\\
	\vspace{30pt}
	\scshape\LARGE{Bezeugt das Scheitern des Nakagin Capsule Towers die Unumsetzbarkeit metabolistischer Architektur?}\\\vspace{20pt}}
\author{Japan und die architektonische Moderne\\eine Rezeptions- und Transformationsgeschichte (Seminar)\\
	\vspace{7pt}
          Dr. Kai Kappel -- Dr. Harald Salomon -- Dr. Tina Zürn\\\vspace{4pt}Lennard Wolf\\
        \small{Matrikelnummer: 583052}\\
        \small{E-Mail: \href{mailto:lennard.wolf@hu-berlin.de}{lennard.wolf@hu-berlin.de}}\\
        \small{Telefonnummer: +49 176 5687 4131}\\
        \small{Studiengang: Regionalstudien Asien/Afrika (Zweitfach)}\\
        \small{Modul: ?}}

\maketitle

\vspace{\fill}

\begin{minipage}[]{0.92\textwidth}
    \centering
    \onehalfspacing
    \large   
    10. September 2018\\
    Sommersemester 2018

    \vspace{-20mm} 
\end{minipage}%
\thispagestyle{empty}
\newpage
%\clearpage
%\thispagestyle{empty}
%\tableofcontents
%\newpage
\setcounter{page}{1}

\begin{onehalfspace} 

\noindent\textbf{$(o)$ Einleitung}

\noindent In dieser Arbeit 

Bezeugt das Scheitern des Nakagin Capsule Towers die Unumsetzbarkeit metabolistischer Architektur?

\vspace{3mm}

Ich werde wie folgt vorgehen. In Abschnitt $(i)$ 

\vspace{5mm}
\noindent\textbf{$(i)$ Die Bewegung des Metabolismus und ihre Ideen} %beschreibung

\begin{itemize}
  \item Historischer Kontext
  \item Hauptleute
  \item Manifest
  \item Tange Rolle?
  \item Zu Kurokawa etwas mehr
  \item  
\end{itemize}



\vspace{5mm}
\noindent\textbf{$(ii)$ Der Nakagin Capsule Tower}

%------------------%
%    Einleitung    %
%------------------%



%------------------%
%    Hauptteil     %
%------------------%

\begin{itemize}
  \item Baubeschreibung
  \item Beschreibung des finanziellen Modells
  \item Beschreibung des heutigen Zustands 
  \item Der Turm als Symbol für die Bewegung (einzig umgesetztes Gebäude)
\end{itemize}


%------------------%
%    Konklusion    %
%------------------%

\vspace{5mm}
\noindent\textbf{$(iii)$ Argumente für die These}

%------------------%
%    Einleitung    %
%------------------%

%------------------%
%    Hauptteil     %
%------------------%

\begin{itemize}
  \item Die Leute scheinen nicht interessiert gewesen zu sein an dem Gebäude sonst wär es nicht gescheitert
  \item das Langzeitgedächtnis ist bei Menschen relativ kurz und auch wenn Teile ausgetauscht werden, der fundamentale Fehler ist der weiterexistierende Kern, der bestimmte Form aufzwingt
  \item 
\end{itemize}


%------------------%
%    Konklusion    %
%------------------%

\vspace{5mm}
\noindent\textbf{$(iv)$ Der Nakagin Capsule Tower im historischen Kontext}

%------------------%
%    Einleitung    %
%------------------%

%------------------%
%    Hauptteil     %
%------------------%

\begin{itemize}
  \item Im Sinne eines Zeitstrahls durchgehen
  \item 
\end{itemize}


%------------------%
%    Konklusion    %
%------------------%

\vspace{5mm}
\noindent\textbf{$(v)$ Argumente gegen die These}

%------------------%
%    Einleitung    %
%------------------%

%------------------%
%    Hauptteil     %
%------------------%

\begin{itemize}
  \item Metabolismus ist mehr als was in dem Turm umgesetzt wurde
  \item Andere Gebäude sind umgesetzt
  \item Im Fall, dass es 
\end{itemize}


%------------------%
%    Konklusion    %
%------------------%


\vspace{5mm}
\noindent\textbf{$(vi)$ Konklusion}

\begin{itemize}
  \item Anderes Modell für Tower?
  \item Ob die Ideen vielleicht doch unumsetzbar sind, oder von den Menschen nicht aufgenommen werden können, bleibt unklar
  \item Dass der Nakagin Tower dies eindeutig bezeugt, ist aber aufgrund der wirtschaftlichen Events in Japan schwer zu bejahen
  \item 
\end{itemize}


\end{onehalfspace}
%\nocite{*}
\printbibliography

%\newpage
%\noindent\textbf{Anhang}
%\vspace{6pt}
%
%\noindent 

\end{document}
