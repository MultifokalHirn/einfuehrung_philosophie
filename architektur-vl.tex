\documentclass[]{scrartcl}
\usepackage{graphicx}
\usepackage{color}
\usepackage{german}
\usepackage{hyperref}
\usepackage{calc} 
\usepackage{enumitem}
%\pagestyle{headings}

% customize dictum format:
\usepackage[T1]{fontenc}
\setkomafont{dictumtext}{\itshape\small}
\setkomafont{dictumauthor}{\normalfont}
\renewcommand*\dictumwidth{\linewidth}
\renewcommand*\dictumauthorformat[1]{--- #1}
\renewcommand*\dictumrule{}
\newcommand{\todo}[1]{\textcolor{red}{TODO: #1}\PackageWarning{TODO:}{#1!}}

\begin{document}

\title{
	\includegraphics*[width=0.75\textwidth]{images/hu_logo.png}\\
	\vspace{24pt}
	Einf"uhrung in die Geschichte der Architektur und des St"adtebaus}
\subtitle{Vorlesung WS 16/17\\
          Prof. Dr. Kai Kappel\\
          Institut f"ur Kunst- und Bildgeschichte \\ 
          Humboldt Universit"at zu Berlin}
\author{Lennard Wolf\\
        \href{mailto:lennard.wolf@student.hu-berlin.de}{lennard.wolf@student.hu-berlin.de}}
\maketitle
\begin{abstract}
Die Lehrveranstaltung vermittelt Grundlagen der Beschreibung und Analyse von historischer, moderner und zeitgen"ossischer Architektur. Dazu geh"oren u. a. Beschreibsystematiken, Architekturterminologie, charakteristische Bauaufgaben, Materialfragen, konstruktiv-baustatische Aspekte, Bauzier und Ornament, Bau und Ausstattung sowie die funktionale/liturgische Nutzung. Behandelt werden Aspekte der historischen und zeitgen"ossischen medialen Vermittlung von Architektur durch Plan, Modell, Fotografie, Film, CAD und andere virtuelle Pr"asentationsformen. Vorgestellt werden zudem wichtige methodische Zug"ange.
\end{abstract}
\newpage

\tableofcontents

\listoffigures
\newpage


\section{Einf"uhrende Sitzung\\(19.10.16)}



\subsection{Organisatorisches}

\begin{itemize}
    \item \textbf{Ziele:} Terminologien, Analysefertigkeiten, historisches Kontextwissen
\item 14 Vorlesungen: Geschichte, Methoden, mediale Zug"ange, Definitionen (Grenzziehungen), chronologische Bauaufgaben
\item Tutorien sind Einf"uhrungen zum wissenschaftlichen Arbeiten
  \item Handapparat im Grimmzentrum
  \item Kaufen: Christian Freigang: W"orterbuch der Architektur aus der Reihe \emph{Reclams Universal Bibliothek}
    \item Kaufen: Koepf, Hans/Binding, G"unter: Bildw"orterbuch der Architektur
    \item WBG Achitekturgeschichte Empfehlung
    \item Standardwerk: Francis D. K. Ching (Hrsg.): A Global History of Architecture
    

\end{itemize}


\subsection{Was ist Architekturgeschichte?}

\begin{itemize}

    \item Architekturgeschichte ist nicht Kernkanon der Kunstgeschichte
    \item Institut möchte nicht rein positivistisch arbeiten
    \item "Uberblick als Ziel, kein Kanon oder \emph{Wahrheiten}
    \item Adolf Goldschmidt \& Arthur ..? vorreiter der Kunsthistorie
    \item Offenes Problem der Kunstgeschichte: Transkulturelles Modell der globalen Kunsthistorie
    \item Lexika und Kompendien zumeist tendenziell europ"aoznetrisch
    \item Gropius Manifest nach dem 1. Weltkrieg
    \item Haus am Horn: erster Funkitonalit"atsbeweis des Bauhaus
    \item Bauhaus hatte erst 1927 Architekturlehre
    \item Vorher: Pr"agung durch Richard Wagner mit dem \emph{Gesamtkunstwerk} $\rightarrow$ Gr"oßer als die Summe ihrer Einzelteile

\end{itemize}




\subsection{Darstellung der Architektur}

\begin{itemize}
    \item Auftraggeber von Bauprojekten sollten Fachleute sein (\emph{concepteurs}), was historisch nicht immer der Fall war. Beispiel: Theologen für Sakralbauten
    \item Gro\ss e Frage: wie wird das Bauwerk dargestellt dem Auftraggeber gegen"uber?
    \item Kirche von St. Peter: Das erste begehbare Modell, durch Antonio St. ... 
    \item Alternative: Aufklappbares Kleinmodell aus Holz/Kork
    \item Architekturtheoretiker Vitruv und die Darstellung von Bauwerken
    \item Umriss (\emph{Scaenographia}, mit Perspektive)
    \item Aufriss (\emph{Orthographia})
    \item Grundriss (??? 1m H"ohe),
    \item Beispiel Paestum von Piranesi
    \item Grundriss ist ein Lageplan, standard mit Skala, Nordpfeil
    \item Le Corbousiers Zitat: Grundriss muss genau sein, Vorstellungen fixieren und haben, so zu ordnen dass sie vermittelbar umsetzbar sind, gliederndes Inhaltsverzichnis
    \item Stereophotographie mit 2 Kameras und Winkelmessger"aten zur historischen Festhaltung 
\end{itemize}

\section{Medien der Architektur\\(26.10.16)}

%\begin{figure}[h]
%	\centering
%	\includegraphics[width=0.5\textwidth]{images/template.png}
%	\caption{Template Bild}
%	\label{fig:template}
%\end{figure}




\newpage
\section{"Uber den Professor}
Prof. Dr. Kai Kappel


\end{document}
