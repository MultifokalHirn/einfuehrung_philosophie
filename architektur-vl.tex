\documentclass[]{scrartcl}
\usepackage{graphicx}
\usepackage{color}
\usepackage[ngerman]{babel}
\usepackage{hyperref}
\usepackage{fullpage}
\usepackage{calc} 
\usepackage{enumitem}
\usepackage{titlesec}
\newcommand{\todo}[1]{\textcolor{red}{TODO: #1}\PackageWarning{TODO:}{#1!}}

\begin{document}

\title{
	\includegraphics*[width=0.75\textwidth]{images/hu_logo.png}\\
	\vspace{24pt}
	Einf"uhrung in die Geschichte der Architektur und des St"adtebaus}
\subtitle{Vorlesung WS 16/17\\
          Prof. Dr. Kai Kappel\\
          Institut f"ur Kunst- und Bildgeschichte \\ 
          Humboldt Universit"at zu Berlin}
\author{Lennard Wolf\\
        \href{mailto:lennard.wolf@student.hu-berlin.de}{lennard.wolf@student.hu-berlin.de}}
\maketitle
\begin{abstract}
Die Lehrveranstaltung vermittelt Grundlagen der Beschreibung und Analyse von historischer, moderner und zeitgen"ossischer Architektur. Dazu geh"oren u. a. Beschreibsystematiken, Architekturterminologie, charakteristische Bauaufgaben, Materialfragen, konstruktiv-baustatische Aspekte, Bauzier und Ornament, Bau und Ausstattung sowie die funktionale/liturgische Nutzung. Behandelt werden Aspekte der historischen und zeitgen"ossischen medialen Vermittlung von Architektur durch Plan, Modell, Fotografie, Film, CAD und andere virtuelle Pr"asentationsformen. Vorgestellt werden zudem wichtige methodische Zug"ange.
\end{abstract}
\newpage

\tableofcontents

\listoffigures
\newpage


\section{Einf"uhrende Sitzung\\(19.10.16)}



\subsection{Organisatorisches}

\begin{itemize}
    \item \textbf{Ziele:} Terminologien, Analysefertigkeiten, historisches Kontextwissen
\item 14 Vorlesungen: Geschichte, Methoden, mediale Zug"ange, Definitionen (Grenzziehungen), chronologische Bauaufgaben
\item Tutorien sind Einf"uhrungen zum wissenschaftlichen Arbeiten
  \item Handapparat im Grimmzentrum
  \item Kaufen: Christian Freigang: W"orterbuch der Architektur aus der Reihe \emph{Reclams Universal Bibliothek}
    \item Kaufen: Koepf, Hans/Binding, G"unter: Bildw"orterbuch der Architektur
    \item WBG Achitekturgeschichte Empfehlung
    \item Standardwerk: Francis D. K. Ching (Hrsg.): A Global History of Architecture
    

\end{itemize}


\subsection{Was ist Architekturgeschichte?}

\begin{itemize}

    \item Architekturgeschichte ist nicht Kernkanon der Kunstgeschichte
    \item Institut m"ochte nicht rein positivistisch arbeiten
    \item "Uberblick als Ziel, kein Kanon oder \emph{Wahrheiten}
    \item Adolf Goldschmidt \& Arthur ..{\color{red}(???)} vorreiter der Kunsthistorie
    \item Offenes Problem der Kunstgeschichte: Transkulturelles Modell der globalen Kunsthistorie
    \item Lexika und Kompendien zumeist tendenziell europ"aoznetrisch
    \item Gropius Manifest nach dem 1. Weltkrieg
    \item Haus am Horn: erster Funkitonalit"atsbeweis des Bauhaus
    \item Bauhaus hatte erst 1927 Architekturlehre
    \item Vorher: Pr"agung durch Richard Wagner mit dem \emph{Gesamtkunstwerk} $\rightarrow$ Gr"oßer als die Summe ihrer Einzelteile

\end{itemize}




\subsection{Darstellung der Architektur}

\begin{itemize}
    \item Auftraggeber von Bauprojekten sollten Fachleute sein (\emph{concepteurs}), was historisch nicht immer der Fall war. Beispiel: Theologen f"ur Sakralbauten
    \item Gro\ss e Frage: wie wird das Bauwerk dargestellt dem Auftraggeber gegen"uber?
    \item Kirche von St. Peter: Das erste begehbare Modell, durch Antonio St. ... 
    \item Alternative: Aufklappbares Kleinmodell aus Holz/Kork
    \item Architekturtheoretiker Vitruv und die Darstellung von Bauwerken
    \item Umriss (\emph{Scaenographia}, mit Perspektive)
    \item Aufriss (\emph{Orthographia})
    \item Grundriss ({\color{red}(???)} 1m H"ohe),
    \item Beispiel Paestum von Piranesi
    \item Grundriss ist ein Lageplan, standard mit Skala, Nordpfeil
    \item Le Corbousiers Zitat: Grundriss muss genau sein, Vorstellungen fixieren und haben, so zu ordnen dass sie vermittelbar umsetzbar sind, gliederndes Inhaltsverzichnis
    \item Stereophotographie mit 2 Kameras und Winkelmessger"aten zur historischen Festhaltung 
\end{itemize}

\subsection{Tutorium  (26.10.16)}
Abwesend

\section{Medien der Architektur I\\(26.10.16)}

Abwesend

\subsection{Notizen von Alex}

\begin{itemize}
  \item Es bedarf verschiedener Dasrtellungsmodi, da man nicht immer anwesend sein aknn
  \item Vitruv (Marcus Vitruvius Pollio) 1. Jh. v. Chr. $\rightarrow$ 10 B"ucher "uber die Architektur
  \item 19 Jh. Kunstgeschichte bildet ({\color{red}(???)}) sich als Fach heraus
  \item Die Fotografie ersetzte teilweise die Zeichnung $\rightarrow$ \emph{Messbilder} $\rightarrow$ auch als Vorlage zur Zeichnung
  \item 1851 \emph{Mission h\'{e}liographique}: erste Fotografien vo n Architektur
  \item Fotografieren in der Architektur: aus mittlerer H"ohe | keine Weitwinkel | im Rahmen nicht achsial | kein Blitz | Belichtungszeit $\rightarrow$ gleichm"a\ss ige Ausleuchtung | gro\ss e Blendenzahl
  \item Walter Hege, Wilhelm Pinder; Albert Renger-Patzsch
  \item \textbf{Ansichten:} {\color{red}Orthogonalprojektion ..?} 
  \item Ansichten aller vier Seiten + Dachansicht + L"angs-/Querschnitt
  \item Horizontaler Schnitt (Grundriss)
  \item Unterhalb des Schnittes d"unne durchgezogene Linie
  \item Oberhalb des Schnittes d"unne gestrichelte Linie (\emph{Schnittebene})
  \item Treppe mit Anstiegspfeil
  \item \textbf{Betrachterperspektive} {\color{red}(?)} 
  \item Bedarfsgerechte Planung 19 Jh. (rationales Bauen) $\rightarrow$ Frankfurter K"uche (hochgebaut auf minimaler Fl"ache)
  \item \textbf{Datierungsm"oglichkeiten} $\rightarrow$ Zeitraumseinsch"atzung, nicht genaue Datierung
  \item Steinsichtigkeit (Baun"ate? Steinmetzzeichenwechsel etc.) 
  \item \textbf{Jahresringdatierung} {\color{red}(?)} Dentrochronologie
  \item Radiocarbon C14 {\color{red}(?)}
  \item Planismetrische Darstellung {\color{red}(?)}
\end{itemize}


\subsection{Tutorium zu Primus (02.11.16)}

Immer sch"on Sternchen beim suchen benutzen.
Kubikat benutzen!


\section{Medien der Architektur II\\(02.11.16)}

\begin{itemize}
  \item Es gibt viele verschiedene \emph{Isometrien} (zB auch von unten)
  \item Isometrien haben didaktischen Charakter $\rightarrow$ Je besser sie sind, desto mehr Informationen geben sie dem Betrachter
  \item Adolf Goldschmidt Pionier der medialen Architekturvermittlung an der Friedrich Wilhelms Universit"at ehem. HU (1912)
  \item Buchempfehlung: \emph{Das Kollosseum - Bewundert, bewohnt, ramponiert} von Erik Wegerhoff
  \item Wichtiger Autor: Palladio, Gruft
\end{itemize}

\subsection{Baubeschreibung}

Zu Beginn: Gesamterscheinung und -wirkung im Kontext der Umgebung (Topographie)


\textbf{Reihenfolge:} Grundriss | Au\ss enbau | Innenraum | Von Osten nach Westen (Bei Kirchen mit dem Altar beginnen) | Von unten nach oben | Vom Allgemeinen ins Spezielle (Vom Gro\ss en ins Kleine)

\textbf{Fragen zu Beginn:} Wieviele Geschosse | Anzahl Fensterachsen | allg. Eindruck | Funktion | Kategorien (Zentralbau, L"angsbau) 

\textbf{Raumtypen} $\rightarrow$ z.B. \emph{Saal} [hat keine Untergliederungen, einschiffiger Bau], \emph{Basilika} (mehrschiffig, hohes Hauptschiff mit Fensterreihe), \emph{Halle} (drei ann"ahrend gleich hohe Schiffe), selten: Staffelhalle (Basilika ohne Fensterreihe)

L"asst sich an den folgenden Fragen festmachen: wieviele Schiffe hat ein Raum? [Hauptschiff, Seitenschiffe] und wie sind sie proportioniert? Wie ist die Untergliederung geleistet? $\rightarrow$ \emph{S"aule} [Kreis], \emph{Pfeiler} [Mehreckig], sind die mit B"ogen verbunden (Bogenstellungen $\rightarrow$ \emph{Arkade}) | Wandgliederung? ((hochgelegene?) Reihe von Fenstern?) | Gew"olbt? | Scheidb"ogen? 


\subsubsection{S"aulenordnungen}

\textbf{Grundmerkmale}: untersch. Durchmesser, Proportionierungen (Modul $\rightarrow$ Verhaltnis Durchmesser zu H"ohe), Basen (Fu\ss teil), Kapitelle (Kopfteil) | Namen der S"aulenordnungen: GriechIonisch, Dorisch, Korinthisch, .. {\color{red}(?)} | Wichtig zu wissen: Klassische \emph{Superposition} der S"aulenordnungen war in der Antike \emph{nicht} kanonisch, es war \emph{nicht immer so}

\emph{Grundbegriffe der S"aule}: {\color{red}(???)} Geb"alk besteht aus Architraf, Fries und Gesims (von unten nach oben)

\emph{Griechisch-Dorisch}: Genannt \emph{Dorika} Kannelierte S"aule, Kapitell: Halsring, tellerf"ormiger Echinus {\color{red}(?)}, Abakus (Deckplatte), Geb"alk besteht aus glattem Architraf, geschm"uckter Fries und Gesims (von unten nach oben) 

\emph{R"omisch-Dorisch}: kr"aftigerer Halsring, schwacher Echinus, sonst wie 

Toskanisch: ungeschm"uckter Fries

\subsection{Architekturbeschreibung der Geschichte}

Architekturbeschreibungen gab es schon immer: In der Bibel, alten Manuskripten etc. Und aus solchen alten Beschreibungen werden neue Nachbauten angefertigt!
Architekturbeschreibungen sprechen von (pers"ohnlicher, zeitabh"angiger) Wahrnehmung, Atmosph"are, Stimmung, Zeitgeist etc.. Die Geschichte der Geb"aude erz"ahlt die Geschichte des Ortes.



\subsection{Vortrag}

\textbf{Ziel:} Pers"onliche Ann"aherung an das Bauwerk, kein Copy Paste aus B"uchern; 
Kompetente und dichte Vorstellung vom Bau vermitteln. \\
\textbf{Thema:} Das Alte Museum (Schinkel)
Instruktiver Bildgebrauch (Nicht hin und her schwenken) $\rightarrow$ vergleichend (2 Bilder gleichzeitig) |
Schr"ag zum Bauwerk stellen, nicht davor! (\emph{kommunikatives Dreieck} $\rightarrow$ Gruppe soll Geb"aude und Vortragenden sehen) | 
Schinkel war fasziniert aber auch angewiedert von Rohheit der antiken Tempel. |
Es gibt selten B"ucher nur "uber ein bestimmtes Thema, daher muss man die Quellen 
Fachterminologie ist wichtig, Details | Besprechen, ob bestimmte Bauelemente auf die Zeit hinweisen | Was ist das wichtigste in dem Bild? $\rightarrow$ damit anfangen!
Zum original gehen
Zeit: 6 Minuten\\
\textbf{Struktur:} Einleitung (kurze Einf"hrung) | Hauptteil: Kerndaten (K"unstlerin, Zeit, Provinienz, Erhaltungszustand, Entstehungskontext) -- Beschreibung (in eigenen Worten, kein Selbstzweck: in den roten Faden eingegliedert, Fokus auf Kernthemen) -- Deutung (Intention (des Auftraggebers), Stilgeschichtliche Einordnung) | Fazit (Zusammenfassung, Kernthese wiederholen und damit abschlie\ss en)\\
\textbf{Hausarbeit:} Es liegt nahe, "uber das Vortragsthema zu schreiben

\subsection{Tutorium Praxis des Studierens I (09.11.16)}

\subsubsection{Vortrag Madonna in der Kirche}

Madonna in der Kirche (Jan van Eyck), 1415-1435, 31x14cm | 
"Ol auf Eichenholz, Gem"aldegalerie |
Zu sehen: Mutter Maria mit Kind, Jesus, auf \emph{ihrem} rechten Arm |
Blick des Betrachters wird nach rechts in die Tiefe geleitet |
Im Hintergrund ist M"onch mit Engel zu sehen |
Langhaus-/Mittelschiffwand mit sechsteiliger Triphoriengalerie mit Laufgang |
Durch das Kirchenportal f"allt unnat"urliches Licht (aus Norden) |
Maria f"allt "ubernat"urlich gro\ss~aus und sie hebt sich farblich stark von dem farblich eher monotonen Hintergund der Kirche ab. | 
Maria hat den B"ogen der Schiffwand entlang geneigten Kopf mit jugendlich, zartem Gesicht und tr"agt eine Krone. Sie steht zentral. | Bauelement weisen auf vorreformatiorische Zeit hin

\subsubsection{Vortrag Paul Cezanne}

Paul Cezanne (1839-1906) Aix en Provence, Jung Romantiker, sp"ater Impressionist
Stilleben mit BLumen und Fr"uchten alte Nationalgalerie |
Referentin teilt Bild auf in Mittelsenkrechte und Mittel..., sowie positive und negative Diagonale |
Leichte Aufsicht auf schmalen Tisch der im Zentrum positioniert ist, bildet zweite Basis "uber dem Rand | 
Birne und Pflanze: Verk"orperung von Bildr"aumen | 
Bild ist harmonisch aufgebaut, viel parallel aufgebaut, gr"un und wei\ss~dominieren
Grober Farbauftrag, trotzdem detailliert. Ruhige, harmonische Stimmung, ganzheitlich Geschlossen $\rightarrow$ typisch f"ur Impressionismus

%\subsubsection{Das wissenschaftliche Arbeiten}


\section{??\\(09.11.16)}




\newpage
\section{"Uber den Professor}
Prof. Dr. Kai Kappel

%\begin{figure}[h]
%	\centering
%	\includegraphics[width=0.5\textwidth]{images/template.png}
%	\caption{Template Bild}
%	\label{fig:template}
%\end{figure}
\end{document}
