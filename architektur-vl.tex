\documentclass[emulatestandardclasses]{scrartcl}
\usepackage{graphicx}
\usepackage{color}
\usepackage[ngerman]{babel}
\usepackage{hyperref}
\usepackage{fullpage}
\usepackage{calc} 
\usepackage{enumitem}
\usepackage{titlesec}
\newcommand{\todo}[1]{\textcolor{red}{TODO: #1}\PackageWarning{TODO:}{#1!}}

\begin{document}

\title{
	\includegraphics*[width=0.75\textwidth]{images/hu_logo.png}\\
	\vspace{24pt}
	Einf"uhrung in die Geschichte der\\Architektur und des St"adtebaus}
\subtitle{Vorlesung WS 16/17\\
          Prof. Dr. Kai Kappel\\
          Institut f"ur Kunst- und Bildgeschichte \\ 
          Humboldt Universit"at zu Berlin}
\author{Lennard Wolf\\
        \href{mailto:lennard.wolf@student.hu-berlin.de}{lennard.wolf@student.hu-berlin.de}}
\maketitle
\begin{abstract}
Die Lehrveranstaltung vermittelt Grundlagen der Beschreibung und Analyse von historischer, moderner und zeitgen"ossischer Architektur. Dazu geh"oren u. a. Beschreibsystematiken, Architekturterminologie, charakteristische Bauaufgaben, Materialfragen, konstruktiv-baustatische Aspekte, Bauzier und Ornament, Bau und Ausstattung sowie die funktionale/liturgische Nutzung. Behandelt werden Aspekte der historischen und zeitgen"ossischen medialen Vermittlung von Architektur durch Plan, Modell, Fotografie, Film, CAD und andere virtuelle Pr"asentationsformen. Vorgestellt werden zudem wichtige methodische Zug"ange.
\end{abstract}
\newpage

\tableofcontents

\listoffigures
\newpage


\section{Einf"uhrende Sitzung\\(19.10.16)}



\subsection{Organisatorisches}

\begin{itemize}
    \item \textbf{Ziele:} Terminologien, Analysefertigkeiten, historisches Kontextwissen
\item 14 Vorlesungen: Geschichte, Methoden, mediale Zug"ange, Definitionen (Grenzziehungen), chronologische Bauaufgaben
\item Tutorien sind Einf"uhrungen zum wissenschaftlichen Arbeiten
  \item Handapparat im Grimmzentrum
  \item Kaufen: Christian Freigang: W"orterbuch der Architektur aus der Reihe \emph{Reclams Universal Bibliothek}
    \item Kaufen: Koepf, Hans/Binding, G"unter: Bildw"orterbuch der Architektur
    \item WBG Achitekturgeschichte Empfehlung
    \item Standardwerk: Francis D. K. Ching (Hrsg.): A Global History of Architecture
    

\end{itemize}


\subsection{Was ist Architekturgeschichte?}

\begin{itemize}

    \item Architekturgeschichte ist nicht Kernkanon der Kunstgeschichte
    \item Institut m"ochte nicht rein positivistisch arbeiten
    \item "Uberblick als Ziel, kein Kanon oder \emph{Wahrheiten}
    \item Adolf Goldschmidt \& Arthur ..{\color{red}(???)} vorreiter der Kunsthistorie
    \item Offenes Problem der Kunstgeschichte: Transkulturelles Modell der globalen Kunsthistorie
    \item Lexika und Kompendien zumeist tendenziell europ"aoznetrisch
    \item Gropius Manifest nach dem 1. Weltkrieg
    \item Haus am Horn: erster Funkitonalit"atsbeweis des Bauhaus
    \item Bauhaus hatte erst 1927 Architekturlehre
    \item Vorher: Pr"agung durch Richard Wagner mit dem \emph{Gesamtkunstwerk} $\rightarrow$ Gr"oßer als die Summe ihrer Einzelteile

\end{itemize}




\subsection{Darstellung der Architektur}

\begin{itemize}
    \item Auftraggeber von Bauprojekten sollten Fachleute sein (\emph{concepteurs}), was historisch nicht immer der Fall war. Beispiel: Theologen f"ur Sakralbauten
    \item Gro\ss e Frage: wie wird das Bauwerk dargestellt dem Auftraggeber gegen"uber?
    \item Kirche von St. Peter: Das erste begehbare Modell, durch Antonio St. ... 
    \item Alternative: Aufklappbares Kleinmodell aus Holz/Kork
    \item Architekturtheoretiker Vitruv und die Darstellung von Bauwerken
    \item Umriss (\emph{Scaenographia}, mit Perspektive)
    \item Aufriss (\emph{Orthographia})
    \item Grundriss ({\color{red}(???)} 1m H"ohe),
    \item Beispiel Paestum von Piranesi
    \item Grundriss ist ein Lageplan, standard mit Skala, Nordpfeil
    \item Le Corbusiers Zitat: Grundriss muss genau sein, Vorstellungen fixieren und haben, so zu ordnen dass sie vermittelbar umsetzbar sind, gliederndes Inhaltsverzichnis
    \item Stereophotographie mit 2 Kameras und Winkelmessger"aten zur historischen Festhaltung 
\end{itemize}

\subsection{Tutorium  (26.10.16)}
Abwesend

\section{Medien der Architektur I\\(26.10.16)}

Abwesend

\subsection{Notizen von Alex}

\begin{itemize}
  \item Es bedarf verschiedener Dasrtellungsmodi, da man nicht immer anwesend sein aknn
  \item Vitruv (Marcus Vitruvius Pollio) 1. Jh. v. Chr. $\rightarrow$ 10 B"ucher "uber die Architektur
  \item 19 Jh. Kunstgeschichte bildet ({\color{red}(???)}) sich als Fach heraus
  \item Die Fotografie ersetzte teilweise die Zeichnung $\rightarrow$ \emph{Messbilder} $\rightarrow$ auch als Vorlage zur Zeichnung
  \item 1851 \emph{Mission h\'{e}liographique}: erste Fotografien vo n Architektur
  \item Fotografieren in der Architektur: aus mittlerer H"ohe | keine Weitwinkel | im Rahmen nicht achsial | kein Blitz | Belichtungszeit $\rightarrow$ gleichm"a\ss ige Ausleuchtung | gro\ss e Blendenzahl
  \item Walter Hege, Wilhelm Pinder; Albert Renger-Patzsch
  \item \textbf{Ansichten:} {\color{red}Orthogonalprojektion ..?} 
  \item Ansichten aller vier Seiten + Dachansicht + L"angs-/Querschnitt
  \item Horizontaler Schnitt (Grundriss)
  \item Unterhalb des Schnittes d"unne durchgezogene Linie
  \item Oberhalb des Schnittes d"unne gestrichelte Linie (\emph{Schnittebene})
  \item Treppe mit Anstiegspfeil
  \item \textbf{Betrachterperspektive} {\color{red}(?)} 
  \item Bedarfsgerechte Planung 19 Jh. (rationales Bauen) $\rightarrow$ Frankfurter K"uche (hochgebaut auf minimaler Fl"ache)
  \item \textbf{Datierungsm"oglichkeiten} $\rightarrow$ Zeitraumseinsch"atzung, nicht genaue Datierung
  \item Steinsichtigkeit (Baun"ate? Steinmetzzeichenwechsel etc.) 
  \item \textbf{Jahresringdatierung} {\color{red}(?)} Dentrochronologie
  \item Radiocarbon C14 {\color{red}(?)}
  \item Planismetrische Darstellung {\color{red}(?)}
\end{itemize}


\subsection{Tutorium zu Primus I (02.11.16)}

Immer sch"on Sternchen beim suchen benutzen.
Kubikat benutzen!


\section{Medien der Architektur II\\(02.11.16)}

\begin{itemize}
  \item Es gibt viele verschiedene \emph{Isometrien} (zB auch von unten)
  \item Isometrien haben didaktischen Charakter $\rightarrow$ Je besser sie sind, desto mehr Informationen geben sie dem Betrachter
  \item Adolf Goldschmidt Pionier der medialen Architekturvermittlung an der Friedrich Wilhelms Universit"at ehem. HU (1912)
  \item Buchempfehlung: \emph{Das Kollosseum - Bewundert, bewohnt, ramponiert} von Erik Wegerhoff
  \item Wichtiger Autor: Palladio, Gruft
\end{itemize}

\subsection{Baubeschreibung I}

Zu Beginn: Gesamterscheinung und -wirkung im Kontext der Umgebung (Topographie)


\textbf{Reihenfolge:} Grundriss | Au\ss enbau | Innenraum | Von Osten nach Westen (Bei Kirchen mit dem Altar beginnen) | Von unten nach oben | Vom Allgemeinen ins Spezielle (Vom Gro\ss en ins Kleine)

\textbf{Fragen zu Beginn:} Wieviele Geschosse | Anzahl Fensterachsen | allg. Eindruck | Funktion | Kategorien (Zentralbau, L"angsbau) 

\textbf{Raumtypen} $\rightarrow$ z.B. \emph{Saal} [hat keine Untergliederungen, einschiffiger Bau], \emph{Basilika} (mehrschiffig, hohes Hauptschiff mit Fensterreihe), \emph{Halle} (drei ann"ahrend gleich hohe Schiffe), selten: Staffelhalle (Basilika ohne Fensterreihe)

L"asst sich an den folgenden Fragen festmachen: wieviele Schiffe hat ein Raum? [Hauptschiff, Seitenschiffe] und wie sind sie proportioniert? Wie ist die Untergliederung geleistet? $\rightarrow$ \emph{S"aule} [Kreis], \emph{Pfeiler} [Mehreckig], sind die mit B"ogen verbunden (Bogenstellungen $\rightarrow$ \emph{Arkade}) | Wandgliederung? ((hochgelegene?) Reihe von Fenstern?) | Gew"olbt? | Scheidb"ogen? 


\subsection{Architekturbeschreibung der Geschichte}

Architekturbeschreibungen gab es schon immer: In der Bibel, alten Manuskripten etc. Und aus solchen alten Beschreibungen werden neue Nachbauten angefertigt!
Architekturbeschreibungen sprechen von (pers"ohnlicher, zeitabh"angiger) Wahrnehmung, Atmosph"are, Stimmung, Zeitgeist etc.. Die Geschichte der Geb"aude erz"ahlt die Geschichte des Ortes.


\subsection{Vortrag}

\textbf{Ziel:} Pers"onliche Ann"aherung an das Bauwerk, kein Copy Paste aus B"uchern; 
Kompetente und dichte Vorstellung vom Bau vermitteln. \\
\textbf{Thema:} Das Alte Museum (Schinkel)
Instruktiver Bildgebrauch (Nicht hin und her schwenken) $\rightarrow$ vergleichend (2 Bilder gleichzeitig) |
Schr"ag zum Bauwerk stellen, nicht davor! (\emph{kommunikatives Dreieck} $\rightarrow$ Gruppe soll Geb"aude und Vortragenden sehen) | 
Schinkel war fasziniert aber auch angewiedert von Rohheit der antiken Tempel. |
Es gibt selten B"ucher nur "uber ein bestimmtes Thema, daher muss man die Quellen 
Fachterminologie ist wichtig, Details | Besprechen, ob bestimmte Bauelemente auf die Zeit hinweisen | Was ist das wichtigste in dem Bild? $\rightarrow$ damit anfangen!
Zum original gehen
Zeit: 6 Minuten\\
\textbf{Struktur:} Einleitung (kurze Einf"hrung) | Hauptteil: Kerndaten (K"unstlerin, Zeit, Provinienz, Erhaltungszustand, Entstehungskontext) -- Beschreibung (in eigenen Worten, kein Selbstzweck: in den roten Faden eingegliedert, Fokus auf Kernthemen) -- Deutung (Intention (des Auftraggebers), Stilgeschichtliche Einordnung) | Fazit (Zusammenfassung, Kernthese wiederholen und damit abschlie\ss en)\\
\textbf{Hausarbeit:} Es liegt nahe, "uber das Vortragsthema zu schreiben

\subsection{Tutorium Wissenschaftliches Arbeiten (09.11.16)}

\subsubsection{Vortrag Madonna in der Kirche}

Madonna in der Kirche (Jan van Eyck), 1415-1435, 31x14cm | 
"Ol auf Eichenholz, Gem"aldegalerie |
Zu sehen: Mutter Maria mit Kind, Jesus, auf \emph{ihrem} rechten Arm |
Blick des Betrachters wird nach rechts in die Tiefe geleitet |
Im Hintergrund ist M"onch mit Engel zu sehen |
Langhaus-/Mittelschiffwand mit sechsteiliger Triphoriengalerie mit Laufgang |
Durch das Kirchenportal f"allt unnat"urliches Licht (aus Norden) |
Maria f"allt "ubernat"urlich gro\ss~aus und sie hebt sich farblich stark von dem farblich eher monotonen Hintergund der Kirche ab. | 
Maria hat den B"ogen der Schiffwand entlang geneigten Kopf mit jugendlich, zartem Gesicht und tr"agt eine Krone. Sie steht zentral. | Bauelement weisen auf vorreformatiorische Zeit hin

\subsubsection{Vortrag Paul Cezanne}

Paul Cezanne (1839-1906) Aix en Provence, Jung Romantiker, sp"ater Impressionist
Stilleben mit BLumen und Fr"uchten alte Nationalgalerie |
Referentin teilt Bild auf in Mittelsenkrechte und Mittel..., sowie positive und negative Diagonale |
Leichte Aufsicht auf schmalen Tisch der im Zentrum positioniert ist, bildet zweite Basis "uber dem Rand | 
Birne und Pflanze: Verk"orperung von Bildr"aumen | 
Bild ist harmonisch aufgebaut, viel parallel aufgebaut, gr"un und wei\ss~dominieren
Grober Farbauftrag, trotzdem detailliert. Ruhige, harmonische Stimmung, ganzheitlich Geschlossen $\rightarrow$ typisch f"ur Impressionismus

%\subsubsection{Das wissenschaftliche Arbeiten}


\section{Medien der Architektur III \\(09.11.16)}

\subsection{S"aulen}

\textbf{Grundmerkmale}:\\Untersch. Durchmesser, Proportionierungen (Modul $\rightarrow$ Verhaltnis Durchmesser zu H"ohe), Basen (Fu\ss teil), Kapitelle (Kopfteil) | Namen der S"aulenordnungen: GriechIonisch, Dorisch, Korinthisch, .. {\color{red}(?)} | Wichtig zu wissen: Klassische \emph{Superposition} der S"aulenordnungen war in der Antike \emph{nicht} kanonisch, es war \emph{nicht immer so}


\subsubsection{Grundbegriffe}

\begin{description}[leftmargin=!,labelwidth=\widthof{\bfseries Perlenstab}]
  \item[S"aule] Walzenf"ormige [sich nach oben leicht verj"ungende], meist aus Basis, Schaft und Kapitell bestehende senkrechte St"utze eines Bauwerks, die aber auch frei stehend dekorativen Zwecken dienen kann
  \item[Ornament] Verzierung, schm"uckendes Muster an einem [k"unstlerischen] Gegenstand oder an einem Bauwerk
  \item[Geb"alk] Gesamtheit von [Stein]balken und horizontalen Bauelementen über einer S"aule (Auflager) | besteht aus Architraph, Fries und Gesims (von unten nach oben)
  \item[Fries] Mit plastischen oder gemalten Ornamenten und figürlichen Darstellungen ausgestaltete Fl"ache als Gliederung und Schmuck einer Wand
  \item[Metope] Im Geb"alkfries des dorischen Tempels mit Triglyphen wechselnde, fast quadratische, bemalte oder mit Reliefs verzierte Platte aus gebranntem Ton oder Stein
  \item[Kapitell] Oberer Abschluss einer S"aule, eines Pfeilers oder eines Pilasters
  \item[Entasis] Das kaum merkliche Dickerwerden des Schaftes antiker S"aulen nach der Mitte zu
  \item[Schaft] Gerader, lang gestreckter, schlanker Teil eines Gegenstandes (der bei Werkzeugen, Waffen häufig der Handhabung dient)
  \item[Triglyph] Am Fries des dorischen Tempels mit Metopen abwechselndes dreiteiliges Feld
  \item[Echinus] Die Deckplatte mit dem S"aulenschaft verbindender Wulst am Kapitell einer dorischen S"aule
  \item[Abakus] \emph{Deckplatte}
  \item[Perlenstab] Rundum laufende Verzierung, besonders zwischen Schaft und Kapitell einer Säule(\emph{Astragal})
  \item[Eierstab] Eine fortlaufende Zierleiste mit friesartigem Ornament, meist an Fassaden oder Säulen
\end{description}

\subsubsection{Detailbegriffe}

\begin{description}[leftmargin=!,labelwidth=\widthof{\bfseries Zahnschnitt}]
  \item[Zahnschnitt] Abstrakt geometrischer Fries
  \item[Konsole] Vorsprung (als Teil einer Wand, Mauer), der etwas tr"agt oder auf dem etwas aufgestellt werden kann
  \item[Vorlage] S"aulen die von der Wand abstehen aber noch eine Wandbindung haben.
  \item[Freis"aule] Als Freis"aulen werden S"aulen bezeichnet, die frei im Raum, in einem Raumteil oder einer Wandvertiefung (Ricetto-Motiv) stehen und keine tragende Funktion übernehmen. Freis"aulen werden als Zier- oder Gliederungselement verwendet und k"onnen dabei auch Statuen oder "Ahnliches tragen. Als Pfeiler wird eine frei stehende St"utze auch als Freipfeiler bezeichnet.
\end{description}

\subsubsection{S"aulenordnungen}

Die Wahl der S"aulenordnung repr"asentieren Wichtigkeit $\rightarrow$ sind nicht nur eine Frage "asthetischer Natur\\

\begin{description}[leftmargin=!,labelwidth=\widthof{\bfseries Griechisch-Doris}]
  \item[Griechisch-Dorisch] Genannt \emph{Dorika} Kannelierte S"aule, Kapitell: Halsring, tellerf"ormiger Echinus, Abakus (Deckplatte), Geb"alk besteht aus glattem Architraf, geschm"uckter Fries und Gesims (von unten nach oben) 
  \item[R"omisch-Dorisch] kr"aftigerer Halsring, schwacher Echinus, sonst wie Griechisch-Dorisch
  \item[Toskanisch] ungeschm"uckter Fries, gedrungenste Ordnungen $\rightarrow$ niedrigste Ordnung
  \item[Ionisch] Profilierter Architraf, Nackter Fries, Gesims mit Perl- und Eierst"aben; Basis kann unterschiedlich sein, typisch attische Basis (Wulst -- Kehle -- Wulst)
  \item[Korinthisch] Kanneluren bilden Steg; Kapitell ist kelchf"ormig mit aufgelegten Arkantusbl"attern [B"arenklau] mit St"angeln, die auf L"ucke gesetzt sind: Wuchsvorgang wird skulptural dargestellt | Ganz oben: Abakusbl"ute
  \item[Komposit] Kombiniert Korinthische mit dem oberen Abschluss der Ionischen S"aulenordnung

\end{description}

\subsubsection{Reduktionsformen}

\begin{description}[leftmargin=!,labelwidth=\widthof{\bfseries Halbs"aulenvorlage}]
  \item[Halbs"aulenvorlage] Halbe S"aule (nicht freistehend sondern Teil der Wand)
  \item[Pilaster] Flacher Mauerstreifen der alles gleich einer S"aule hat (Basis, Kapitell..)
  \item[Lisene] Flacher Mauerstreifen ohne Kapitell oder Basis, nur noch ein Streifen
  \item["Adikula] Zwei Pilaster mit Giebel; Form eines Hausquerschnittes
\end{description}

\subsection{Wichtige Architekten}


\begin{description}[leftmargin=!,labelwidth=\widthof{\bfseries Le Corbusie}]
  \item[Le Corbusier] Wollte sch"opferisch mit der Antike umgehen. Er suchte nach dem idealen Ma\ss~und orientierte sich dabei am Menschen. Wichtiges Buch: \emph{Hin zu einer modernen Architektur}
\end{description}


\subsection{Grenzen der Baubeschreibung}

\textbf{Beispiel}: \emph{Erbbegr"abnis Wissinger} in Stahnsdorf:\\ Man kann nur mit \emph{Assoziationen} arbeiten: \emph{Sieht aus wie...}, \emph{Erinnert an...} (Trotzdem analytisch sein!)

Grimmzentrum Erinnert an Antike, doch die S"aulenordnung ist komplett modern, also minimalistisch. Moduleinheit ist nicht der Mensch (wie bei Le Corbusier), sondern das B"ucherregal.



\subsection{Tutorium zu Primus II (16.11.16)}
Nichts Besonderes

\section{Bautechnik und Material\\(16.11.16)}

\subsection{Grundbegriffe}

\begin{description}[leftmargin=!,labelwidth=\widthof{\bfseries Stahlbetonskelettbau}]
  \item[Streifenfundament] Durchgehendes Fundament das bis nach oben hin durchgeht
  \item[Einzelfundament] (Punktfundament) F"ur S"aulen etc.
  \item[Mauerwerk] Historisches Mauerwerk ist "ublicherweise ein Zweischalenmauerwerk, wo die Mitte bef"ullt ist mit Gemischen; Gegensatz heutzutage: Energetisches Mauerwerk | 
  \item[Quader] Glatt bearbeitet: in Rechte Winkel geschnitten; Vorteile: Sehr Pr"azises Vermauern mit Brandschlag; Brolüre??
  \item[Backstein] Synthetischer Stein (Wenn es keine Steinbr"uche in der N"ahe gibt); Lehm wird in (Holz-) Formen eingestrichen und im Brennofen gebrannt (?) | Heute industriell | Im Expressionismus waren Fehler wie schwarze Stellen, \emph{Versinterungen}, etc. sehr beliebt (`Ofenmischung' $\rightarrow$ versinten leichter)
  \item[Binder \& L"aufer] Bei Quadern wie bei Backsteinen: Kurze (Binder, gucken nach au\ss en) und Lange Steine (L"aufer, parallel zur Wand) $\rightarrow$ Beispiel: alternierende L"aufer- und Binderschichten
  \item[Rustizierung] Wenn Oberfl"ache der Bauglieder durch starke Fugen getrennt sind (Bsp.: \emph{Diamantquader})
  \item[Spoliation] "Ubertragen von Materialien eines Baus in einen anderen; \emph{Spolien} sind Bauteile und andere "Uberreste wie Teile von Reliefs oder Skulpturen, Friese und Architravsteine, S"aulen- oder Kapitellreste, die aus Bauten "alterer Kulturen stammen und in neuen Bauwerken wiederverwendet werden. | Wird sowohl aus Liebhaberei, als auch als Kriegstroph"ae benutzt
  \item[Standger"ust]
  \item[Fliegendes Ger"ust]
  \item[Ger"ustloch]
  \item[Amierung] Bewehrungsstahl oder Betonstahl, fr"uher auch Armierungseisen oder Moniereisen, dient als Bewehrung (Verst"arkung) von Stahlbetonbauteilen und wird nach dem Einbau in die Schalung mit Beton vergossen.
  \item[Stahlbetonskelettbau] Statik beruht komplett auf dem Betonskelett mit den Moniereisen als Bewehrung
  \item[Schalung (Beton)] ??
  \item[Binder] Ist ein horizontales Konstruktionselement und kann, je nach Verwendungszweck, aus verschiedenen Baustoffen wie z.B. Beton, Holz, Stahl hergestellt sein.
\end{description}

\subsection{Spolienbeispiele}

\begin{description}[leftmargin=!,labelwidth=\widthof{\bfseries San Giorgio Maggiorei}]
  \item[Markusdom] \emph{Basilica di San Marco} in Venedig: Spolien aus Konstantinopel (heutiges Istanbul) beinhaltet 1204 geraubte S"aulen als Erinnerung an die Eroberung der Stadt
  \item[San Giorgio Maggiore] Venedig, von Andrea Palladio; Bau hat innen quadratischem Pfeiler mit Halbs"aulenvorlagen zum Seitenschiff, Palladio nutzte unterschiedliche Vorlagen:
  \item[Neue Wache] in Berlin von Karl Friedrich Schinkel: Dorische S"aulen ohne Basis $\rightarrow$ ging sch"opferisch damit um
\end{description}


\subsection{Gew"olbe}


\begin{description}[leftmargin=!,labelwidth=\widthof{\bfseries Kreuzrippengew"olbe}]
  \item[Gew"olbe] Eine nach oben hin gew"olbte Geb"audedecke, die nicht –- wie etwa eine Balkendecke –- flach auf den W"anden aufliegt.
  \item[Unechtes Gew"olbe] Schritt f"ur Schritt gehen die Steine immer mehr zusammen (Bsp: Palanque/Mykene der Maya) | eine Vorform des echten Gewölbes als oberer Abschluss eines Raumes bezeichnet. Parallel dazu gibt es schon seit frühester Zeit Kragkuppeln, Kragsteinkuppeln oder falsche Kuppeln (Auch \emph{Kraggew"olbe, Falsches Gew"olbe})
  \item[W"olbung] mit Lehrbogen/Lehrger"ust ??
  \item[Tonnengew"olbe] Gew"olbe, dessen Querschnitt einen Halbkreis darstellt, mit zwei gleich langen parallelen Widerlagern | mit Gurten, rund- oder spitzbogige Tonne
  \item[Kreuzgratgew"olbe] aus zwei sich rechtwinklig "uberschneidenden Tonnengew"olben bestehendes Gew"olbe | gurtlos
  \item[Kreuzrippengew"olbe] Durch selbsttragende Rippen (Kreuzrippen) gebildet und gehalten. Die Rippen kreuzen sich dabei wie die Diagonalen in einem Rechteck; sie leiten die Druck- und Schubkr"afte des Gew"olbes auf die Pfeiler ab. | In Beschreibung: \emph{ist so und so profiliert} | seit etwa 1100 
\end{description}

\subsection{Mauertechniken}

\begin{description}[leftmargin=!,labelwidth=\widthof{\bfseries Fachwerk}]
  \item[Fachwerk] Skelettbauweise, tragendes Skelett ist aus Holzbalken, Leerstellen werden mit unterschiedlichen (oft minderen) Materialien bef"ullt (Lehm, Stroh, Backstein etc.) und dann verputzt. Man sieht die tragenden Elemente, also das Holzskelett | `\emph{Deutsche Gem"utlichkeit}' | `Moderne' Version \emph{Eisenfachwerk}: Tragendes Skelett aus Stahl/Eisen, nur horizontal und vertikal, und Leerstellen meist mit Backsteinen bef"ullt
  \item[Beton] Synthetischer Baustoff; Bindemittel: Zement | Gesteinsk"ornungen/Zuschl"age: Kies, Sand, Zugabewasser (Auch anderes kann beigemischt je nach dem, was man haben will, kann auch eingef"arbt werden). Wird in alle m"oglichen Formen gegossen | Ortbeton: wird vor Ort gegossen); Pr"afabrikat: k"onnen auf Serie hergestellt werden | \emph{opus caementicium}: Schalung, in die hinein gegossen wird, bleibt als Mauer stehen | Erst seit den 20ern. im gro"sen Stil genutzt, da meist auch eher noch nur im Innenraum sichtbar | in den 50ern/60ern wollte man die Form (N"agel, Jahresringe etc.) der Bretter, in die der Beton gegossen wurde sehen; das Beton sollte nicht noch gegl"attet werden | Le Corbusier nutzte Beton sehr gern, besonders \emph{Spritzbeton} (Bsp: Kloster \emph{La Tourette}) | \emph{Formsteine}: Serielle Fertigung (h"aufig Beton), Beispiel Plattenbau; sp"ater dann auch mit `\emph{historischer Anmutung}' m"oglich | \emph{Waschbeton}: Gesteinsk"ornung aus mit Kies, die dann abgewaschen wird sodass an der Aussenseite die Kiesk"ornung sichtbar wird | Typisch 90er/00er Jahre:Glatter Beton aber mit kreisrunden L"ochern (R"odelloch/Schalungsanker), die Struktur geben (daher keine Balkenabdr"ucke oder K"ornung); Wichtigstes Beispiel: Tadao Andō, Vulkanasche als Zuschlag: Sehr weiche, seidige, fast marmorne Oberfl"ache; Orientiert sich an klassisch japanischer Architektur (Gr"osse von Tatamiplatten); In Berlin: Krematorium Baumschulenweg (1996-1999)
 \item[Eisen] Wichtigste Arten: Gusseisen, Gewalztes Eisen/Profilstahl (z.B. Stahltr"ager/I-Tr"ager/U-Tr"ager etc.) | \emph{Stahl}: Eisen in einer Legierung (\emph{durch Zusammenschmelzen entstandenes Gemisch verschiedener Metalle}), die aufgrund ihrer Festigkeit, Elastizit"at, chemischen Best"andigkeit gut verarbeitet, geformt, geh"artet werden kann | Beispiele: Erste Gusseisenbr"ucke, \emph{The Iron Bridge} in Coalbrookdale, Schottland (ohne Ornamente, \emph{Nacktkonstruktion}) | \emph{Tour Eiffel} in Paris | \emph{Gare du Nord} (1864-1865) | \emph{Crystal Palace} (Weltausstellung 1851) im Hyde Park in London: das erste Mal ohne steinernem Ger"ust drum herum | Schon seit dem Mittelalter
  \item[Blob] Blob Architektur (\emph{Binary Large Object})
  \item[Fassade] Von Innen nach Au"sen: Kern -- W"armed"ammung -- Vorgeh"angte (nicht tragende) Fassade (Steintapete/-platten; Marmorplatten) $\rightarrow$ \emph{Inkrustation} | Wandverkleidung mit meist repetitivem Muster | Beispiele: Azulejos (Portugal), Moscheen Istanbul
\end{description}

\subsection{Tutorium Bildrechereche und -verwendung (23.11.16)}

\textbf{Bildbeschaffung}\\
Qualit"atsmerkmale: Hohe Aufl"sung (dpi-Zahl ist bedeutungslos) | digitale Bildrecherche: prometheus, digitale Bildsammlungen (Museum, Archive), offline-Bilder aus Literatur, Museumsdatenbanken, Google Art Project\\

\noindent\textbf{Bildrecht und Bildnachweis}\\
Urheberrecht (gilt ohne Anmeldung) $\neq$ Copyright (Druckerlaubnis $\rightarrow$ Nutzungsrecht)\\
Wissenschaftsschranke: Verwendung innerhalb der Universit"at


\section{Bautechnik und Material II\\(23.11.16)}

\textbf{Case Study: Moderne Kirchen mit Glasw"anden}\\
Vorbild der Sainte-Chapelle (Paris) aus dem 13. Jh. f"ur moderne Kirchen wie Ged"achtniskirche (1959-61, Egon Eiermann) in Berlin oder Notre-Dame (1922-23)in Le Raincy deren Au"senw"ande auch zu weiten Teilen aus Glas bestehen | 
Durch die st"arkere Tragkraft der Materialien konnten die W"ande entlastet werden und noch mehr Glas verwendet werden. | Glas Horizontales Windeisen; Bleirouten (?) halten Glaspartikel zusammen aber unterstreichen die Fig"urlichkeit; Glasmalerei durch feine Pinselstriche geben dem Bild Volumen\\

\noindent\textbf{Case Study: Glashaus von Bruno Taut}\\
Bei Werkbundausstellung in K"oln (1914); Als Leistungsbeweis der deutschen Glaskonstruktion (Farbe, Form, Festigkeit) | Doppelschalige Konstruktion: Innere und "Au"sere Glasschale; Glasbausteine finden "uberall Verwendung

\subsection{Glas}

\begin{description}[leftmargin=!,labelwidth=\widthof{\bfseries Transluzentes G.}]
  \item[Glas] Als Stilmittel f"ur Farbigkeit, als Trennung nach Au"sen (Wand, Fenster)
  \item[Curtain Wall] selbsttragende Wand aus Stahl \& Glas | Beispiele: Bauhaus Dessau, Seagram Building) klassisch aus den 20ern von Gropius, van der Rode; Jedoch gibt es auch schon Bauten aus dem Ende des 19. Jhs. (Giengen an der Brenz, Fabrik Steiff)
  \item[Drahtglas] Flachglas mit Drahtraster
  \item[Glasbausteine] Innen Hohl, manchmal auch farbig, relativ lichtintensiv, l"asst aber nur wenig erahnen, was dahinter ist
  \item[Betonglas](aus 'Dalles de Verre', zugebrochene, 3-4cm dicke Glasst"ucke, in Stahlform eingesetzt, Zwischenr"aume mit Beton bef"ullt), ca. 20er, Beispiel: Audincourt, Sacre Coeur, Taufkapelle (1951) 
  \item[Opakes Glas] \emph{Milchglas}: lichtintensiv, aber l"asst keinen Einblick zu |  Beispiel: Kunsthaus Bregenz (1997)
  \item[Transluzentes G.] ??
  \item[Glasfassade] Haltetechniken der Glasfassade: \emph{Hinterspannte Glasfassade} (gespannte Seile die von Au"sen die Kr"ummung h"alt) und \emph{Punkthalterung}
  \item[Medienfassade] Durch Displays oder computergesteuerte Leuchtmittel wandelbar | \emph{Darstellend} (Wiedergabe vorgefertigter Inhalte), \emph{Reagierend} (Installation die Daten aus der Umgebung aufnimmt und verarbeitet) oder \emph{Interaktiv} (Einflussnahme der PassantInnen)
\end{description}

\subsection{Beispiele f"ur neue Bautechniken}

\begin{description}[leftmargin=!,labelwidth=\widthof{\bfseries Blockbauweisei}]
  \item[Blockbauweise] Aufstapeln von Holzbrettern
  \item[Asphalt] Kapelle in Etsdorf (Oberpfalz), 2001/02
  \item[Biokunststoff] TU Stuttgart: Treppenhaus aus dreieckigen statisch tragenden Platten (2013)
  \item[Carbonbeton] Statt Stahlbeton: Kein Rosten und leichteres Material
\end{description}

\section{Bauaufgaben und Hauptbauwerke\\(30.11.16)}

Architekturgeschichte = Stilgeschichte?

\subsection{Abgrenzung des Faches}

Historisch bedingte Inklusionen und Exklusionen | Noch lange nicht fertig; Forschung der Kunstgeschichte ist konstanter Prozess der (Neu-)Entdeckung, Hinterfragung; Austausch der Kulturen | Pioniere: Adolf Goldschmidt, Richard Harmann (?) | Empfehlung: Francis Ching's (?) Versuch eines Lexikons | Jenseits der Abgrenzung: Architektonische Formen, deren Geschichte, Ausbreitung und Transformation | Klassische Epocheneinteilung in der Architekturgeschichte: Wilfried Koch (Baustilkunde, 1982); Jedoch: verl"auft die Stilgeschichte wirklich Schritt f"ur Schritt, Stil nach Stil (mit Fr"uh-, Hoch- und Sp"atphasen)? | Man sollte eine Sensibilit"at daf"ur haben, dass alle historischen Modelle problembehaftet sind. | Stilbegriffe sind interessant auch wegen ihrer \emph{eigenen} Entstehungsgeschichte; Beispiel: Begriff der \emph{Romanik} (ca. 1000 -- 1200) erst sehr sp"at entwickelt, denn f"ur lange Zeit verworfen als einfach geschmacklos |

%\newpage
%\section{"Uber den Professor}
%Prof. Dr. Kai Kappel

%\begin{figure}[h]
%	\centering
%	\includegraphics[width=0.5\textwidth]{images/template.png}
%	\caption{Template Bild}
%	\label{fig:template}
%\end{figure}


%\begin{description}[leftmargin=!,labelwidth=\widthof{\bfseries Bla}]
%  \item[Bla] ...
%\end{description}


\end{document}
