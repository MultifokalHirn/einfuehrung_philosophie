\documentclass[]{scrartcl}
\usepackage{graphicx}
\usepackage{color}
\usepackage{german}
\usepackage{hyperref}
%\pagestyle{headings}

\begin{document}

\title{
	\includegraphics*[width=0.75\textwidth]{images/hu_logo.png}\\
	\vspace{24pt}
	Westliches und "Ostliches\\Denken im Vergleich:\\Sokrates, Zoroaster\\Buddha, Konfuzius}
\subtitle{Proseminar WS 16/17\\
          Dr. Bettina Fr"ohlich\\
          Philosophisches Institut I \\ 
          Humboldt Universit"at zu Berlin}
\author{Lennard Wolf\\
        \href{mailto:lennard.wolf@student.hu-berlin.de}{lennard.wolf@student.hu-berlin.de}}
\maketitle
\begin{abstract}

In kulturvergleichenden Diskursen werden h"aufig die Differenzen zwischen westlichem und "ostlichem Denken betont. Das Konzept einer diskursiven Rationalit"at, die Idee der individuellen Freiheit und Pers"onlichkeit und die Wissensorientierung werden dabei meist als spezifische Merkmale des westlichen Denkens angesehen und von den Wahrnehmungs- und Denkmustern der "ostlichen Kulturen abgegrenzt. Im Seminar wollen wir diese Sichtweise anhand der verschiedenen Denktraditionen "uberpr"ufen. Im Zentrum stehen Denker der sogenannten Achsenzeit: Sokrates, Zoroaster, Buddha und Konfuzius. Wir werden sokratische Dialoge des platonischen Fr"uhwerks, die Gathas aus dem Avesta, Texte aus dem Pali-Kanon und Passagen aus den Gespr"achen des Konfuzius lesen. Die Texte sollen hinsichtlich der methodologischen und epistemologischen Pr"amissen sowie der ethischen und anthropologischen Konzepte ausgewertet und vergleichend betrachtet werden.

\end{abstract}
\newpage

\tableofcontents
\listoffigures
\newpage


\section{Einf"uhrungssitzung\\(20.10.16)}
\subsection{Kulturvergleich}

\begin{itemize}
  \item Kulturaustausch wird immer allt"aglicher
  \item Wir sind nicht mehr nur Teil unserer Nationen, sondern auch Teil der Menschheit
  \item Interkultureller Dialog bedarf einer Basis
  \item Kulturvergleich dient ebendiesem, sowie der Entfaltung von Potenzialen in fremdkultureller Texte f"ur die eigene Kultur
\end{itemize}

\subsection{Interkulturelle Philosophie}
\begin{itemize}
  \item Ethno-/Eurozentrismus typisch f"ur 17.-19. Jahrhundert, \\''Philosophie beginnt im alten Griechenland''
  \item Wurde dann aufgebrochen, besonders durch die \emph{interkulturelle Philosophie} seit den 80ern: ''Andere Kulturen haben auch philosophische Fragestellungen und Ideen!''
  \item $\rightarrow$ China \& Indien \emph{auch} Ursprungsorte der Philosophie
  \item Karl Jaspers: \emph{Die gro\ss en Philosophen} (1957)
  \item Klasenap (?): \emph{Die Religionen der Menschheit} (1954)
  \item Kulturrelativismus (Kulturen sind nicht komparabel) vs. Universalismus
  \item 3. Weg: \emph{Relativer Universalismus}
  \item Trennlinie Philosophie und Religion?
\end{itemize}


\newpage

\section{Platon, Apologie\\(27.10.16)}


%\begin{figure}[h]
%	\centering
%	\includegraphics[width=0.5\textwidth]{images/template.png}
%	\caption{Template Bild}
%	\label{fig:template}
%\end{figure}



\newpage
\section{"Uber die Dozentin}
Dr. Bettina Fr"ohlich ist..


\end{document}
