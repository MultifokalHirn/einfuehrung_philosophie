\documentclass[emulatestandardclasses]{scrartcl}
\usepackage{graphicx}
\usepackage{color}
\usepackage[ngerman]{babel}
\usepackage{hyperref}
\usepackage{fullpage}
\usepackage{calc} 
\usepackage{enumitem}
\usepackage{titlesec}
\newcommand{\todo}[1]{\textcolor{red}{TODO: #1}\PackageWarning{TODO:}{#1!}}
\date{\vspace{-3ex}}
\begin{document}

\title{
	\includegraphics*[width=0.75\textwidth]{ErstesSem/images/hu_logo.png}\\
	\vspace{24pt}
	Aristoteles\\Nikomachische Ethik}
\subtitle{VEV SS 17\\
          Dr. Philipp Br"ullmann\\
          Philosophische Fakult"at I \\ 
          Humboldt Universit"at zu Berlin}
\author{Lennard Wolf\\
        \small{\href{mailto:lennard.wolf@student.hu-berlin.de}{lennard.wolf@student.hu-berlin.de}}}
\maketitle
\begin{abstract}

Die Nikomachische Ethik des Aristoteles ist eine der interessantesten, einflussreichsten und meistdiskutierten Schriften zur praktischen Philosophie. Sie behandelt eine Vielzahl wichtiger Themen (wie etwa das gute Leben, Tugend, Verantwortung, Gerechtigkeit, Freundschaft, Willensschw"ache, Lust und Erziehung) und entwirft eine Konzeption der Ethik, die systematisch ernstzunehmen ist. Wer sich f"ur praktische Philosophie interessiert, sollte die Nikomachische Ethik kennen. In unserer Vorlesung werden wir uns (a) einen "uberblick "uber die zentralen Thesen und Argumente der Schrift verschaffen und dabei (b) einige Interpretationsprobleme und Forschungskontroversen kennenlernen. Außerdem werden wir (c) versuchen, die Nikomachische Ethik in ihren Kontext, d.h. den Kontext der antiken Ethik und der Aristotelischen Philosophie, einzuordnen und nicht zuletzt (d) immer wieder die Frage stellen, wie sich der Ansatz des Aristoteles zu anderen Ans"atzen in der normativen Ethik verh"alt.
\end{abstract}
\newpage

\tableofcontents
\listoffigures
\newpage


\section{Die Frage nach dem Gl"uck [I 1-5]\\(25.04.17)}

\subsection{Zusammenfassung erste Sitzung}

\begin{itemize}
  \item Unterscheidung Esoterisch/Exoterisch : Ver"offentlicht/Vorlesungen
  \item "`Gl"uck"': objektiv, kontingent
  \item "`Tugend"': technische Definition
  \item Ziel: Tugendhaft zu \emph{werden}, nicht blo"s Theorie kennen
  \item Enger Zusammenhang zwischen Tugend und Gl"uck
  \item Ethik hat hier nichts mit Moral zu tun
\end{itemize}

\noindent \textbf{Ziel der Sitzung:} Wie ist die Frage nach dem Gl"uck formuliert/wie ist das Problem beschrieben?

\subsection{\emph{eudaimonia}}

\begin{description}[leftmargin=!,labelwidth=\widthof{\bfseries Frage nach dem Gl"uck}]
  \item[Frage nach dem Gl"uck] Welches ist das beste Leben, das man f"uhren kann?
\end{description}


\subsection{Die Herausforderung des Amoralisten}

\begin{itemize}
  \item Frage nach dem guten Leben kam schon in Platons fr"uhen Dialogen vor (Sokrates)
  \item Sokrates versucht zu zeigen, dass der Gerechte ein besseres Leben hat als der Ungerechte (es ist f"ur \emph{ihn} besser, gerecht zu sein).
  \item Problem: intuitiv ist ungerecht sein doch von Vorteil f"ur einen
  \item Hirte Gyges (\emph{Politeia}): Ring kann unsichtbar machen, Hirte findet und kann heimlich Unrecht tun. "`W"urden wir das nicht auch alle machen?"'
  \item Herausforderung des Amoralisten: Warum sollen wir dann "uberhaupt gerecht sein?
  \item Genauer: einziger Vorteil eines gerechten Lebens ist Sicherheit vor Strafe
  \item Antwort von Sokrates: Gerechtigkeit ist ein bestimmter Zustand der Seele. Der Ungerechte schadet seiner Seele, er ist quasi \emph{krank}. (Unordnung in der Seele, die Teile der Seele tun nicht das wof"ur sie da sind)
\end{itemize}

\subsection{Aristoteles und der Amoralist}

\begin{itemize}
  \item Aufgabe: Zeigen, dass auch konsequent eigeninteressierte Menschen einen Grund haben, gerecht zu sein
  \item Aristoteles wendet sich nicht an den Amoralisten
  \item Eignet sich nicht f"ur Menschen, die sich nicht von ihren Leidenschaften bestimmen lassen, sowie sittenhaft und gut erzogen sind ($\rightarrow$ gemeinsame Grundlage)
  \item $\rightarrow$ was hei"st das f"ur die Moralpsychologie/das Begr"undungsprojekt?
\end{itemize}

\subsection{Die Frage nach dem Gl"uck}

\subsubsection{Das Anerkannte und das Umstrittene [I 2-3]}

\begin{itemize}
  \item Aristoteles fragt sich: Wor"uber besteht Einigkeit/Uneinigkeit?
  \item \textbf{Uneinigkeit}: Was ist das Gl"uck? | Vielfalt und Relativit"at der Antworten | Lebensformen... Lust (Menge): sklavenartig; Ehre (Politiker): abh"angig von anderen; Gutheit: w"are mit Unt"atigkeit vereinbar; Betrachtung (theoria): ???; Gelderwerb: Mittel zum Zweck
  \item \textbf{Einigkeit}: \emph{Das Gl"uck ist das h"ochste durch Handeln erreichbaren G"uter} (gut leben, handeln; sich wohl befinden)
  \item Diese Unterscheidung ist Grundlage f"ur Frage: Was ist das h"ochste Gut?
\end{itemize}

\subsubsection{G"uter und Ziele [I 1]}

\begin{itemize}
  \item Kede Kunst, Lehre, Handlung und Entschluss scheint irgendein Gut zu erstreben.
  \item G"uter sind Ziele; Das h"ochste Gut ist ein oberstes Ziel [I 1]
  \item Wenn $a$ das Ziel der T"atigkeit $b$ ist, dann ist $a$ "`besser"' (ein h"oheres Gut) als $b$. | Wenn $a$ ein Ziel ist, das wir umwillen eines h"ohren Ziels $b$ ist, dann ist $b$ "`besser"' (ein h"oheres Gut) als $a$.  | (G"uter k"onnen verglichen werden)
  \item H"ochstes Gut: Etwas das wir immer um seiner selbst willen und nie um einer anderen Sache willen erstreben und um dessentwillen wir alles andere erstreben
  \item Die \emph{eudaimonia} ist solch ein h"ochstes Gut
\end{itemize}

\subsubsection{Es gibt keine Idee des Guten [I 4]}

\begin{itemize}
  \item Begriff des h"ochsten Guts in Verkn"upfung mit Platons Ideenlehre: Gegenstand der \emph{gut} im "`h"ochsten Ma"se"' aufweist
  \item $\rightarrow$ Gradueller Unterschied (im Kontrast zu [I 1])
  \item Aristoteles weist diesen Ansatz zur"uck, da G"uter zu verschieden sind; Kennen des abstrakten Guten hilft nichts f"ur den Erwerb der einzelnen, konkreten G"uter
  \item $\rightarrow$ \emph{Nicht hintergehbare Verschiedenheit der G"uter}. Die Ethik muss dieser Verschiedenheit gerecht werden.
\end{itemize}

\subsubsection{Die drei Kriterien [I 5]}

\noindent \textbf{Kriterien die die eudaimonia als h"ochstes Gut auszeichnen:}

\begin{enumerate}
  \item Wir w"ahlen alles um des guten Lebens willen, aber niemals das gute Leben um einer anderen Sache willen.
  \item Die \emph{eudaimonia} ist "`autark"', d.h. wenn wir sie besitzen, dann bed"urfen wir keiner weiteren Dinge (betrifft auch andere Menschen).
  \item Die \emph{eudaimonia} l"asst sich nicht durch die Hinzuf"ugung weiterer G"uter verbessern.
\end{enumerate}

\noindent $\rightarrow$ Vorrauss"atzungen f"ur [I 6 ff.] eine eigene Bestimmung des Gl"ucks zu entwickeln

\subsection{Begriffe}

\begin{description}[leftmargin=!,labelwidth=\widthof{\bfseries \emph{eudaemonia}}]
  \item[\emph{eudaimonia}] \emph{Nicht}: Gl"ucksgef"uhl, oder Zufallsgl"uck | \emph{Sondern}: Ein insgesamt gelungenes Leben | W"ortlich: Unter "`gutem daimon"', unter einem guten Stern stehen (auch: \emph{Gl"uckseligkeit})
  \item[\emph{ergon}] Eigent"umliche Leistung | Ziel (Ergebnis -- muss nicht positiv sein!) einer T"atigkeit (\emph{techne}) | Um zu erkl"aren, was ein bestimmtes \emph{technite} ist, kommt man nicht umhin sein jeweiliges \emph{ergon} zu benennen
  \item[\emph{energeia}] Aktualit"at
  \item[\emph{dynamis}] M"oglichkeit/Potenzial
\end{description}

\subsection{Tutorium}

Wir verwenden die griechischen Ausdr"ucke


\section{Bestimmung des Gl"ucks [I 6-12]\\(02.05.17)}

\subsection{Das \emph{ergon} Argument [I 6]}

\subsubsection{Die Argumentstruktur}

\begin{description}[leftmargin=!,labelwidth=\widthof{\bfseries P3}]
  \item[P1] F"ur alle Gegenst"ande, die ein \emph{ergon} besitzen, liegt das \emph{agathon} und das \emph{eu} im \emph{ergon.}
  \item[P2] Der Mensch besitzt ein \emph{ergon}.
  \item[P3] Das \emph{ergon} des Menschen besteht in der T"atigkeit der Seele entsprechend der Vernunft. (Weil die Vernunft sein \emph{idios} ist)
  \item[K] Das Gute f"ur den Menschen liegt in der T"atigkeit der Seele entsprechend der Vernunft.
\end{description}

\subsubsection{Probleme mit dem Arguemnt}

\begin{itemize}
  \item Debatten "uber starke/schwache Lesart von \textbf{P1}
  \item Hat \textbf{P2} eine Begr"undung? Plausibel?
  \item Beruht das Argument auf einem "Aquivokationsfehlschluss? (Gut mit mehreren Bedeutungen?)
  \item Beruht das Argument auf einem problematischem Essentialismus? (Warum sollte es ein "`Wesen"' des Menschen geben? Warum sollte dieses normative Konsequenzen haben? | Moral basierend auf essentialistischen Pr"amissen ist schwierig)
  \item Ist das \emph{ergon} des Menschen richtig bestimmt? (Es gibt andere T"atigkeiten und F"ahigkeiten zu geben, die den Menschen vor anderen Lebewesen auszeichnen.)
\end{itemize}

\subsubsection{Eine deflation"are Lesart}

\begin{itemize}
  \item PLS COPY FROM FILE
\end{itemize}


\subsection{Vergleich mit den bestehenden Meinungen [I 8-12]}

\subsubsection{"Ubereinstimmung und Modifikation [I 8-9]}

\begin{itemize}
  \item Sein Vorschlag sollte "ubereinstimmen mit der gr"o"seren Menge der existierenden Meinungen
  \item Sie \textbf{stimmt "uberein} mit...
  \item PLS CPY
  \item "Au"sere Umst"ande k"onnen entweder $(a)$ Werkzeuge, oder $(b)$ Tr"ubung des Gl"ucks bedeuten. (Doch wie damit umgegangen werden soll steht im n"achsten Abschnitt.)
\end{itemize}

\subsubsection{Das Problem der "au"seren Ungl"ucksf"alle [I 10-11]}

\begin{itemize}
  \item PLS CPY
\end{itemize}

\subsection{Begriffe}

\begin{description}[leftmargin=!,labelwidth=\widthof{\bfseries \emph{makarios}}]
  \item[\emph{agathon}] Das Gute
  \item[\emph{eu}] Das "`auf gute Weise"' (Adverb von \emph{agathon})
  \item[\emph{psych\={e}}] Seele
  \item[\emph{idios}] Das einem Ding eigent"umliche
  \item[\emph{aret\={e}}] Tugend, Gutheit (noch nicht gekl"art)
  \item[\emph{makarios}] Seelig
  \item[\emph{athlios}] Elendig
\end{description}


\subsection{Tutorium}

\begin{description}[leftmargin=!,labelwidth=\widthof{\bfseries P3}]
  \item[P1] Ein "`abschlie"sendes Ziel"' (\emph{teleios}) ist ein solches, das niemals zum Zwecke eines anderen Zieles verfolgt wird.
  \item[P2] Das Gl"uck ist ein Ziel menschlichen Handelns (\emph{praxis}), das niemals zum Zwecke eines anderen Zieles verfolgt wird.
  \item[P3] Es gibt nur ein "`abschlie"sendes Ziel"' (\emph{teleios}) menschlichen Handelns (\emph{praxis}).
  \item[K] Das Gl"uck ist das "`abschlie"sende Ziel"' (\emph{teleios}) menschlichen Handelns (\emph{praxis}).
\end{description}


\section{Zwei Arten des Gl"ucks? [I 13; X 6-9]\\(09.05.17)}

\subsection{Lekt"urenotizen}

\noindent \textbf{[I 13]} Gutheit (\emph{aret\={e}}) = \emph{menschliche} Gutheit $\rightarrow$ Seele hat vernunftlosen (vegetativen) und vern"unftigen Bestandteil $\rightarrow$ Gutheit des vernunftlosen Bestandteils ist allen Tieren gleich und somit nicht relevant f"ur die \emph{menschliche} Gutheit $\rightarrow$ Es gibt in der Seele etwas das gegen die Vernunft wirkt (wie der Gegenmuskel) $\rightarrow$ Beide Bestandteile der Seele sind derart zweiteilig $\rightarrow$ Entsprechend wird die Gutheit auch zweigeteilt, und zwar in Tugenden des \emph{Denkens} und Tugenden des \emph{Charakters}\newline

\noindent \textbf{[X 6]} Gl"uck (\emph{eudaemonia}) ist keine Disposition (sonst k"onnte der Gute auch immerzu schlafen) $\rightarrow$ Handlungen der Gutheit: "uber diese hinaus sucht man nichts $\rightarrow$ Aber dies scheint auch f"ur Vergn"ugung zu gelten, und diese allein kann nicht gut sein, da f"ur die anderes vernachl"assigt wird (K"orper, Arbeit) $\rightarrow$ Das vernachl"assigte muss wieder erarbeitet werden und so m"ussen daraufhin Dinge getan werden, die nicht nur mit dem Gl"uck zum Ziele unternommen werden $\rightarrow$ Vergn"ugung ist also nicht Ziel, sondern lediglich ein Mittel daf"ur, Kraft f"ur gute T"atigkeiten zu sammeln \newline

\noindent \textbf{[X 7]} Betrachtung ist die h"ochste T"atigkeit, da ihr Gegenstand die h"ochsten Dinge sind, sie die kontinuierlichste ist, sie ernsthaft ist und ihr Weisheit (\emph{sophia}) beigemischt ist, welche die h"ochste Lust ist. Zudem ist sie autark und wird um ihrer selbst willen geliebt. | Die Betrachtung kann aber nicht die vollkommen einzige Bet"atigung sein, zudem h"atte dies etwas g"ottliches, un\emph{menschliches} -- keine menschliche Bet"atigung ist vollst"andig. Trotzdem ist es die "`gl"uckseligste"' aller T"atigkeiten.\newline

\noindent \textbf{[X 8]} Sekund"are Form des Gl"ucks wird erlangt durch guten Charakter, ausgedr"uckt durch gute Bet"atigung (\emph{energeia}): gerechtes, tapferes, angemessenes, kluges, richtiges Handeln. Solche Tugend ist aber getrennt von der Tugend des intuitiven Denkens (\emph{nous}, siehe [X 7]), welche die prim"are Form des Gl"ucks ist. Jede gute \emph{energeia} hat bestimmte Vorraussetzungen, so braucht der Tapfere einen starken K"orper usw. -- die Betrachtung aber hat keine/wenige Vorraussetzung. Die vollkommene Tugend liegt zugleich im Vorsatz und der damit verbundenen Handlung selbst. | Dass die Betrachtung die prim"are Form des Gl"ucks ist zeigt sich auch darin, dass die G"otter nicht handeln wie die Menschen (der Gedanke allein ist l"acherlich), aber ja insgesamt gl"uckselig sind, und somit wohl allein der Betrachtung fr"ohnen m"ussen.\newline

\noindent \textbf{[X 9]} Das gl"uckselige Leben bedarf bestimmter Umst"ande (genug zu essen usw.), doch nie im "Uberma"s: "`Auch mit bescheidenen Mitteln kann man n"amlich in Aus"ubung der Tugend handeln"', es gilt sich prim"ar im Sinn der Tugend zu bet"atigen. So brauch man, um den Gl"ucklichen zu finden, nicht nur bei den Reichen und M"achtigen suchen. Ob alles Gesagte zum Gl"uck wirklich wahr ist, l"asst sich nur durch Taten nachpr"ufen. | Die G"otter erfreuen sich an dem ihnen "ahnlichen, also dem Betrachtendem, dem Weisen. Dieser ist der gl"ucklichste und von den G"ottern am meisten geliebt.

\subsection{Begriffe}

\begin{description}[leftmargin=!,labelwidth=\widthof{\bfseries \emph{eudaemonia}}]
  \item[\emph{agathon}] Das Gute
\end{description}

\subsection{Tutorium}


\section{-? [II]\\(09.05.17)}

\textbf{[II 1]} \emph{aret\={e}} des Denkens (\emph{diano\={e}tik\={e}}) durch Belehrung, \emph{aret\={e}} des Charakters (\emph{\={e}thik\={e}}) durch Gewohnheit (\emph{ethos}) $\rightarrow$ Letztere nicht durch Natur bestimmt (weil Natur nicht durch Gew"ohnung ver"andert werden kann), sondern \emph{erm"oglicht}: F"ahigkeit geht der T"atigkeit voraus, aber T"atigkeit geht der Tugend voraus! So ist ab Beginn des Lebens jede \emph{energeia} ein Schritt in Richtung gl"uckseligem Leben, oder weg davon. \newline

\noindent \textbf{[II 2]} Was erlangen wir gute \emph{praxis}? Jede \emph{energeia} ist auf angemessene Weise zu vollf"uhren, d.h. ohne Mangel oder "Uberma"s, welche jeweils die Tugenden zerst"oren, w"ahrend die Mitte sie erh"alt. Richtige Erziehung hat zum Ziel, dass der Mensch Lust bei tugendhafter, und Unlust bei untugendhafter \emph{praxis} empfindet (so wird durch Bestrafung im Kindeskopf Unlust mit der schlechter \emph{energeia} verkn"upft). Tugend wird f"alschlicherweise durch \emph{apatheia} definiert, der wahrhaft gl"uckseelige empfindet aber durch die Wahl des Richtigen (d.i. das Werthafte, das N"utzliche, das Angenehme) Lust, und durch die Wahl des Falschen (d.i. das Niedrige, das Sch"adliche, das Unangenehme) Unlust. Aus der Lust an der charakterlichen Tugend (\emph{aret\={e} \={e}thik\={e}}) entsteht mehr charakterliche Tugend, das Umgekehrte gilt genauso.\newline

\noindent \textbf{[II 3]} Doch wie soll gute \emph{praxis} vollbracht werden, wenn der Mensch noch gar nicht gut ist? Tugend wird erlernt durch das Kopieren der \emph{praxis} von Tugendhaften, dann wird es nach einige "Ubung Teil von einem, da man es zu seinem eigenen Wissen z"ahlen kann. Die \emph{praxis} ist erst dann tats"achlich gut, wenn sie von einer guten Person ausgef"uhrt werden. Theorie und Philosophie kann nicht zu Tugend f"uhren, nur \emph{praxis}.\newline

\noindent \textbf{[II 4]} Was ist die \emph{aret\={e} \={e}thik\={e}}? Kandidaten (da nur sie in Seele vorkommen): Affekt (\emph{pathos}, von Lust und Unlust begleitetes Gef"uhl), Anlage (\emph{dynamis}, Voraussetzung f"ur Affekte), Disposition (\emph{hexis}, Veranlagung zur Empf"anglichkeit f"ur gewisse Affekte) | Tugend/Laster sind nicht Affekte, denn f"ur letztere wird man nicht gelobt; sie sind nicht Anlage, denn Anlage hat man ohne Vorsatz, und auch f"ur sie wird man nicht gelobt. Durch das Ausschlussverfahren m"ussen Tugenden und Laster Dispositionen sind. \newline

\noindent \textbf{[II 5]} \emph{aret\={e}} betrifft Handlung \emph{und} Handelnden. Das Mittlere in der \emph{praxis} ist nicht f"ur jeden das Gleiche, es bezieht sich nicht auf die \emph{praxis} selber, sondern auf uns (z.B.: einer \emph{braucht} mehr Essen als ein anderer). Die Tugend hat mit all dem zu tun, wo es "Uberma"s und Mangel geben kann (siehe [II 6]). "`Menschen sind gut auf eine Art, schlecht auf viele"'. \newline

\noindent \textbf{[II 6]} (Ist in sich schon so auf den Punkt, dass ich die Definition der Tugend nicht kopieren werde.) | Nicht jede Handlung hat "Uberma"s oder Mangel $\rightarrow$ Diebstahl kann nie "Uberma"s oder Mangel haben, es ist in sich schlecht. Diese k"onnen entsprechend nie tugendhaft sein.\newline

\noindent \textbf{[II 7]} "`Nur"' Aufz"ahlung von Tugenden; Beispiel: Tapferkeit ist Mitte von Furcht und Mut.\newline

\noindent \textbf{[II 8]} Die drei Dispositionen sind "Uberma"s, Mangel und Mitte, wobei nur das Letzte eine Tugend ist. Alle drei sind einander entgegengesetzt, und so nennt der Feige den Tapferen (f"alschlicherweise) einen Tollk"uhnen, und der Tollk"uhne einen Feigen! Das wozu wir eine Neigung haben, erscheint uns weiter entfernt von der Mitte, als das andere Extrem.\newline

\noindent \textbf{[II 9]} Gutes Handeln ist selten, lobenswert und edel, da es durch das in [II 8] erw"ahnte Ph"anomen als besonders fern und schwer empfunden wird. Da das eine Extrem im Bezug zu einem selber weniger fehlerhaft ist, ist dieses zuerst aufzusuchen, um sich dann an die Mitte heranzutasten!\newline


\noindent \textbf{Frage 1:} Wie ist in [II 4] auszuschlie"sen, dass es nicht neben \emph{pathos}, \emph{dynamis} und \emph{hexis} noch anderes in der Seele gibt? Es kann sich ja auch nicht um eine Metapher handeln, sonst w"are das verwendete Ausschlussverfahren nicht zul"assig -- wie kommt er gerade auf diese drei?\newline

\noindent \textbf{Frage 2:} Die Argumentation der "`zweitbesten"' Fahrt in [II 9] leuchtet stark ein, doch ist in (nicht nur) meiner Erfahrung in ihr auch h"aufig der R"ucksprung in ein nur noch viel st"arkeres Entgegengesetztes versteckt, welches die Dividenden ausgezahlt haben will. Zu was w"urde Aristoteles hier raten?

Tugend ist eine Disposition die sich in Vors"atzen "au"sert

Differenz Disposition und Hanldung aufgrund von DIspositionen scheint schwammig


\section{Gerechtigkeit [V]\\(31.05.17)}

\textbf{Versucht, das Verh"altnis der Gerechtigkeit zu den "ubrigen Tugenden zu bestimmen. Beachtet dabei die Differenzierung in Gerechtigkeit im allgemeinen und speziellen Sinn, die Aristoteles einführt.}\\

Universale Gerechtigkeit: 

Partikulare Gerechtigkeit: 

\noindent \textbf{Betrachtet analog das Verh"altnis der Ungerechtigkeit zu den übrigen Lastern!}\\

\noindent \textbf{Welche Vorstellung von Verteilungsgerechtigkeit hat Aristoteles (Kap. 6 + 7 Anfang)? Wie "uberzeugend findet ihr diese bzw. habt ihr einen Gegenvorschlag oder stimmt ihr Aristoteles zu?}\\

\noindent \textbf{Welche Rolle spielt Geld in Aristoteles' Gerechtigkeitstheorie (Kap. 8)?}\\

\noindent \textbf{In Kapitel 15 wirft Aristoteles die Frage auf, ob man ungerecht gegen sich selbst sein kann. Wie beantwortet er diese Frage vor dem Hintergrund der vorangehenden Kapitel und warum? Welche spezifische Differenz der Gerechtigkeit zu den anderen Tugenden wird dabei ausgedr"uckt?}\\

Inwiefern ist jemand, der gerecht ist, tapfer?

\begin{itemize}
  \item Gerechtigkeit ist die charakterliche Gutheit
  \item Sie ist einzuteilen in die universale Gerechtigkeit und die partikulare Gerechtigkeit (distributiv [Verteilung], korrektiv [Schadensausgleich], reziprok [Tausch])
  \item Gerechtigkeit hat keine Affekte und es gibt weder "Uberma"s noch Mangel
  \item 
\end{itemize}

\section{III}

\subsection{Lekt"urenotizen}

\textbf{[III 1]} Gewolltes: Lob, Ungewolltes: Mitleid $\rightarrow$ Unterschied muss klar gemacht werden, auch f"ur Rechtsprechung n"otig | Unterscheidung legt Wert auf Zeitpunkt der Handlung | Gewolltes: Wenn Handlung ihren Ursprung in Person selber findet; Ungewolltes: aus Zwang (Beweggrund \emph{au"serhalb}), Unwissenheit; Problematisch: aus Angst (aber eher gewollt) | F"ur Niedriges erh"alt man Lob, wenn es mit teurem Ziel aufgewogen ist $\rightarrow$ Unedel ist der, der Niedriges aufgrund von unedelen Zielen tut.\newline

\noindent \textbf{[III 2]} Handeln aufgrund von Unwissenheit bez"uglich bestimmter Bedingungen (besonders Mittel und Zweck) ist nicht gegen das Wollen, solches Handeln ist \emph{ohne Wollen}, was sogar \emph{gegen das Wollen} sein kann | Pers"onliche Unwissenheit ist der Grund f"ur ungerechtes und schlechtes Handeln \newline

\noindent \textbf{[III 3]} Gewolltes Handeln ist also ein solches, dessen Ursprung in der Person selber liegt, die zudem alle Bedingungen der Situation kennt | Handlung aus Begierde und Erregung ist auch gewollt (sonst k"onnten Tiere nicht wollen)\newline

\noindent \textbf{[III 4]} Vorsatz $\subset$ Gewolltes | Vors"atze sind nicht Teil des Vernunftlosen (Begierde/Erregung etc.) und auch keine W"unsche oder Meinungen | Vors"atze gehen mit "Uberlegen \emph{logos} und Denken einher\newline

\noindent \textbf{[III 5]} "Uberlegen bezieht sich auf Dinge, die in unserer Macht stehen (durch Handeln), daher nicht die Ziele, sonder jenes, das zum Ziel f"uhrt ("`Welche Handlung ist wie die Beste?"') | Grundlage ist die Untersuchung auf M"oglichkeit | Der Vorsatz bezieht sich auf die \emph{"uberlegte Handlung}\newline

\noindent \textbf{[III 6]} W"unsche beziehen sich aber auf die Ziele, welche jeweils ein (subjektives) Gut sind $\rightarrow$ wird durch eine Handlung ein ungewolltes Ziel erreicht, kann dieses nicht gew"unscht gewesen sein \newline

\noindent \textbf{[III 7]} Ziel ist gew"unscht, entsprechende Handlung vors"atzlich $\rightarrow$ Handlung ist \emph{gewollt} | Da wir vors"atzlich (nicht) Handeln, liegt die Tugend bei uns und damit, ob wir gut sind oder schlecht | Niemand ist gl"uckselig per Zufall, jeder ist schlecht aufgrund des eigenen Wollens | Dies ist im Recht repr"asentiert: Strafe f"ur gewolltes, bewusstes Handeln (man kann f"ur Unwissenheit verantwortlich sein) | Wer nur w"unscht, tugendhaft zu sein, wird dadurch nicht tugendhaft | Charakterliche und k"orperliche Schlechtheit (Untugendhaftigkeit, Krankheit) geht aus Gewolltem hervor, da man Vorkehrungen h"atte treffen k"onnen | Wenn die Tugenden auf dem Wollen beruhen, beeinflussen wir die Dispositionen, und kontrollieren damit das Wollen\newline

\noindent \textbf{[III 8]} Zusammenfassung: Tugend ist Mitte der Dispositionen, welche durch das in unserer Kontrolle stehende Handeln ver"andert werden\newline


\noindent \textbf{Frage 1:} Wieso kann Aristoteles immer Lob und Tadel der anderen als Argument f"ur die Kategorisierung von Begriffen wie "`Vosatz"' und "`Tugend"' anf"uhren? \newline

\noindent \textbf{Frage 2:} Warum ist es nach Aristoteles bei einer Handlung aus Angst nicht ganz deutlich, dass sie gewollt ist? Umst"ande zwingen einen doch immer zur Handlung im Leben, was macht die Angst da anders? 

\subsection{Aristoteles Vier Ursachen}

\begin{enumerate}
  \item Form (\emph{causa formalis})
  \item Materie (\emph{causa materialis}) -- erm"oglicht Individualit"at
  \item Wirkung (\emph{causa efficiens})
  \item Ziel (\emph{causa finalis})
\end{enumerate}

Hylemorphismus


\newpage
\section{"Uber den Professor}
Prof. Mustermann ist..


%\begin{figure}[h]
%	\centering
%	\includegraphics[width=0.5\textwidth]{images/template.png}
%	\caption{Template Bild}
%	\label{fig:template}
%\end{figure}

\end{document}
