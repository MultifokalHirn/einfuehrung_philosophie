\documentclass[]{scrartcl}
\usepackage{graphicx}
\usepackage{color}
\usepackage[ngerman]{babel}
\usepackage{hyperref}
\usepackage{fullpage}
\usepackage{calc} 
\usepackage{enumitem}
\usepackage{titlesec}
\newcommand{\todo}[1]{\textcolor{red}{TODO: #1}\PackageWarning{TODO:}{#1!}}
\begin{document}

\title{
	\includegraphics*[width=0.75\textwidth]{images/hu_logo.png}\\
	\vspace{24pt}
	Wittgenstein:\\Tractatus Logico-Philosophicus}
\subtitle{Proseminar WS 16/17\\
          Dr. Jasper Liptow\\
          Philosophisches Institut I \\ 
          Humboldt Universit"at zu Berlin}
\author{Lennard Wolf\\
        \href{mailto:lennard.wolf@student.hu-berlin.de}{lennard.wolf@student.hu-berlin.de}}
\maketitle
\begin{abstract}

In seinem Tractatus logico-philosophicus (1921) pr"asentiert Ludwig Wittgenstein in atemberaubend dichter Form "Uberlegungen zum Wesen der Sprache, der Logik, der Welt und nicht zuletzt der Philosophie selbst. Seine Charakterisierung des Tractatus im Vorwort lautet kurz: \emph{Das Buch behandelt die philosophischen Probleme und zeigt – wie ich glaube – dass die Fragestellung dieser Probleme auf dem Missverst"andnis der Logik unserer Sprache beruht. Man k"onnte den ganzen Sinn des Buches etwa in die Worte fassen: Was sich "uberhaupt sagen l"asst, l"asst sich klar sagen; und wovon man nicht reden kann, dar"uber muss man schweigen.} Diese Zeilen deuten bereits an, warum der Tractatus einen der entscheidenden Schritte auf dem Weg zu einer inhaltlichen und methodischen Orientierung der Philosophie an der Sprache darstellt, wie sie die Philosophie des 20. Jahrhunderts gepr"agt hat (linguistic turn). 
Im Seminar wollen wir zum einen versuchen, den Gedankengang des Tractatus anhand einer genauen Lekt"ure zu rekonstruieren, und gucken, welche der Schmuckst"ucke aus dieser \emph{jewel box of insights} (wie der amerikanische Philosoph Wilfrid Sellars den Tractatus einmal nennt) bis heute ihren Glanz bewahrt haben.


\end{abstract}
\newpage

\tableofcontents
\newpage


\section{Einf"uhrende Sitzung\\(31.10.16)}

\textbf{Zu lesen:} 

\emph{Wittgenstein's Tractatus logico-philosophicus: A Reader's Guide} von Roger White

Selbst gefunden: 
\emph{Sprache und Wirklichkeit in Wittgensteins Tractatus} von Rolf-Albert Dietrich | \url{https://books.google.no/books?id=mwe1KLQM3bYC}

\subsection{Kontext}

In einer Liste seiner Einfl"usse war Schopenhauer der einzige Philosoph, den Wittgenstein auff"uhrte. Am wichtigsten waren jedoch Frege und Russell und deren Versuche zum Beweis des Logizismus.

\begin{description}[leftmargin=!,labelwidth=\widthof{\bfseries 2}]
  \item[Frege] War Vertreter des \emph{Logizismus} und verfolte daher das Ziel zu zeigen, dass die Axiome der Arithmetik sich aus den Axiomen der Logik herleiten lassen. Er entwickelte in seiner \emph{Begriffsschrift} die Pr"adikatenlogik erster Stufe. In seinem Buch \emph{Die Grundlagen der Arithmetik}, welches die Frage `\emph{Was sind Zahlen?}' thematisiert, stellt er das \emph{Kontextprinzip} vor, demnach Begriffe nur im Zusammenhang eines Satzes etwas bedeuten. So erh"alt etwa der Begriff `Stein' erst eine Bedeutung, wenn er im Elementarsatz `$x$ ist ein Stein' auftritt. In \emph{Grundgesetze der Arithmetik} wollte er den Logizismus voll implementieren, doch die Einf"uhrung der Mengentheorie als Axiom der Logik produzierte ein Paradox (\emph{Russelsche Antinomie}), welches von Russell erkannt wurde (z.B. Die Menge er Mengen, die sich nicht selbst beinhalten).
  \item[Russell] Selber Logizist, versuchte er mit seinem Freund Whitehead das System von Frege in der \emph{Principia Mathematica} weiterzuf"uhren und dabei das oben genannte Paradox zu vermeiden. Dazu schw"achte er das Axiom der Menge ab indem er Typen einf"uhrte. Eine Menge eines niederen Typus konnte somit keine Menge des selben oder h"oheren Typen enthalten. Durch dieses abgeschw"achte Axiom konnte aber die Arithmetik nicht mehr hergeleitet werden, weshalb Russell noch drei weitere Axiome einf"uhren musste.
  \item[Wittgensteins Einw"ande] Wittgenstein empfand die neu eingef"uhrten Axiome als k"unstlich und fragte sich, woher wir wissen k"onnen, dass sie Axiome sind. Entsprechend lassen sich folgende Fragen als Ausgangspunkt für den \emph{Tractatus} sehen: \textbf{(1)} \emph{Was l"asst sich von Russells Typentheorie halten?} und \textbf{(2)} \emph{Was l"asst sich "uber eine logische Wahrheit sagen?}. Zu letzterer Frage liesse sich sagen, dass eine logische Wahrheit \emph{notwendig} ist und \emph{a priori} erkennbar ist.
\end{description}

\subsection{Vorwort}

\emph{Das Buch will also dem Denken eine Grenze ziehen, oder vielmehr --	
 nicht dem Denken, sondern dem Ausdruck der Gedanken: Denn um dem Denken eine Grenze zu ziehen, m"ußten wir beide Seiten dieser Grenze denken k"onnen.} Wittgensteins Hauptanliegen ist es, die Philosophie von Unsinn und Verwirrung zu bereinigen, denn \emph{[d]ie meisten S"atze und Fragen, welche "uber philosophische Dinge geschrieben worden sind, sind nicht falsch, sondern unsinnig. Wir k"onnen daher Fragen dieser Art "uberhaupt nicht beantworten, sondern nur ihre Unsinnigkeit feststellen. Die meisten Fragen und S"atze der Philosophen beruhen darauf, dass wir unsere Sprachlogik nicht verstehen.} (4.003)

Die Grenzziehung in der Sprache wird die Grenzziehung zwischen \emph{sinnvollen}, \emph{sinnlosen} und \emph{unsinnigen} S"atzen sein.

\subsection{Die S"atze der ersten Gliederungsebene}
\begin{description}[leftmargin=!,labelwidth=\widthof{\bfseries 12}]
  \item[1] Die Welt ist alles, was der Fall ist.
  \item[2] Was der Fall ist, die Tatsache, ist das Bestehen von Sachverhalten.
  \item[3] Das logische Bild der Tatsachen ist der Gedanke.
  \item[4] Der Gedanke ist der sinnvolle Satz.
  \item[5] Der Satz ist eine Wahrheitsfunktion der Elementars"atze.\\
(Der Elementarsatz ist eine Wahrheitsfunktion seiner selbst.)
  \item[6] Die allgemeine Form der Wahrheitsfunktion ist: $[~\bar{p},~\bar{\xi},~N(\bar{\xi})~]$.\\
Dies ist die allgemeine Form des Satzes.
  \item[7] Wovon man nicht sprechen kann, dar"uber muss man schweigen.
\end{description}

\section{Gegenstand, Sachverhalt, Tatsache: 1 bis 2.063\\(07.11.16)}
\subsection{Lekt"urenotizen}
\subsubsection{Gedanken und Fragen}
\textbf{Gedanken}

Wittgenstein scheint in diesem Abschnitt zum einen eine terminologische Grundlage f"ur den Rest des Textes legen zu wollen, zum anderen werden aber auch schon gewissen ontologische Aussagen getroffen, zum Beispiel durch \emph{Die Welt ist die Gesamtheit der Tatsachen, nicht der Dinge.} (1.1). Ich stelle mir die nun folgende Struktur des Textes so vor, dass erst eine Ontologie entworfen wird und dann gezeigt wird, wie die Sprache (und damit das Denken?) mit der in dieser Ontologie entworfenen `Welt' in Beziehung steht. \newline
\\
\textbf{Fragen}

\begin{itemize}
  \item \emph{Eines kann der Fall sein oder nicht der Fall sein und alles "ubrige bleibt gleich} (1.21) | Soll das hei\ss en, dass Tatsachen voneinander unabh"angig sind? $\rightarrow$ steht in (2.061), aber was hei\ss t das? Wie spielt Zeit/Kausalit"at da rein?
  \item \emph{Die Substanz der Welt besteht unabh"angig von dem, was der Fall ist.} (2.024) Hei\ss t das, dass ohne eine Konfiguration die Substanz ein formloser `Blob' ist?
  \item Unterschied Welt und Wirklichkeit? Welt hat Substanz, Wirklichkeit ist nur eine `Wahrheitstafel'? 
\end{itemize}

\vspace{10pt}
\subsubsection{Neue Begriffe}

\begin{description}[leftmargin=!,labelwidth=\widthof{\bfseries Konfiguration}]
  \item[Welt] \emph{Alles} was der Fall ist. Die Gesamtheit \emph{aller} Tatsachen, d.h.  bestehenden Sachverhalte.
  \item[Tatsache] Das Bestehen (das \emph{Wahrsein}) von Sachverhalten. (\emph{Was der Fall ist}). Befinden sich im logischen Raum und sind immer wahr. [\emph{fact}]
  \item[Gegenstand] Ein einfaches Ding das nicht teilbar ist. $\rightarrow$ Auch \emph{Ding} genannt. | Um ihn zu \emph{kennen}, muss ich alle seine internen Eigenschaften kennen, d.h. auch s"amtliche M"oglichkeiten seines Vorkommens in Sachverhalten (diese liegen in seiner Natur). | K"onnen nur im Kontext von Sachverhalten vorkommen. | Identit"at ist keine Relation zwischen Gegenst"anden.
  \item[Sachverhalt] Eine Verbindung von Gegenst"anden. Sachverhalte k"onnen wahr und nicht wahr sein und sind voneinander unabh"angig. [\emph{atomic fact}]
  \item[Sachlage] Ein Komplex aus Sachverhalten.
  \item[Struktur] \emph{Die Art und Weise, wie die Gegenst"ande im Sachverhalt zusammenh"angen, ist die Struktur des Sachverhaltes.} (2.032)
  \item[Form des ??] \emph{Die Form} \emph{ist die M"oglichkeit der Struktur.} (2.033)  \item[Form des Gegenstands] \emph{Die Form} ist \emph{die M"oglichkeit seines Vorkommens in Sachverhalten}. (2.0141)
  \item[Substanz] Die Substanz der Welt ist gebildet durch die (unteilbaren) Gegenst"ande. Sie besteht unabh"angig von dem, was der Fall ist. (s. 2.024)
  \item[Konfiguration] Die Konfiguration der Gegenst"ande bildet den Sachverhalt. Die ist das Wechselnde, Unbest"andige (im Gegensatz zu den Gegenst"anden).
  \item[Wirklichkeit] \emph{Das Bestehen und Nichtbestehen von Sachverhalten ist die Wirklichkeit.} (2.06) | \emph{Sind alle Gegenst"ande gegeben, so sind damit auch alle m"oglichen Sachverhalte gegeben.} (2.0124) | \emph{Die gesamte Wirklichkeit ist die Welt.} (2.063) {\color{red}(Was w"are halbe Wirklichkeit?)}
\end{description}



\subsection{Sitzung}

\begin{itemize}
  \item Tatsachen sind immer in Form von S"atzen anzutreffen, w"ahrend Gegenst"ande singul"are Terme sind. 
  \item `Beispielwelt':\\\emph{Dinge}: $a, b, F, G$\\\emph{Sachverhalte}: $a$ ist $F$, $a$ ist nicht $F$,  $a$ ist $G$, $b$ ist $G$, $b$ ist nicht $F$\\\emph{Tatsachen}:  $a$ ist $F$ und $G$, $b$ ist $G$
  \item Alle letzten Elemente, in die die Welt zerf"allt, sind Tatsachen.
  \item Wir k"onnen sehr viel weniger sagen als wir denken.
  \item Sachverhalte + Logik = Tatsachen
  \item Q: \emph{Wie} atomar sind denn die Gegenst"ande? Ist ein Tisch ein Gegenstand?\\A: Ein Tisch ist ein \emph{Komplex}, aber es scheint von der Betrachtungs-/Abstraktionsebene abzuh"angen. Wittgenstein l"asst das sehr weit offen f"ur Interpretation.
  \item Q: Wann ist etwas im logischen Raum?\\A: Wenn es in logischen Beziehungen stehen kann (Die Sonne kann nicht im Widerspruch zu etwas stehen).
\end{itemize}

\section{Tatsache und Bild: 2.1 bis 3.01\\(14.11.16)}


\subsection{Hausaufgaben}

\begin{enumerate}
  \item "Uber Argumentation nachdenken, warum es Gegenst"ande geben muss. (2.02 bis 2.023)
  \item Erl"autern sie Wittgensteins Gedanken, dass ein Bild kein Gegenstand sondern eine Tatsache ist.
  \item Warum muss ein Bild mit dem Abgebildeten die Form der Abbildung gemein haben?
  \item Warum l"asst sich die Form der Abbildung nicht abbilden?
\end{enumerate}


\subsection{Lekt"urenotizen}
\textbf{Gedanken zum Text}

Wittgenstein scheint...

\vspace{10pt}
\textbf{Neue Begriffe}

\begin{description}[leftmargin=!,labelwidth=\widthof{\bfseries Sachverhalt}]
  \item[Bild] ...
  \item[Tatsache] 
\end{description}

\textbf{Fragen}

\begin{itemize}
  \item 
\end{itemize}

\section{Satz und Name: 3 bis 3.263\\(21.11.16)}


\subsection{Lekt"urenotizen}
\textbf{Gedanken zum Text}

Wittgenstein scheint...

\vspace{10pt}
\textbf{Neue Begriffe}

\begin{description}[leftmargin=!,labelwidth=\widthof{\bfseries Sachverhalt}]
  \item[Bild] ...
  \item[Tatsache] 
\end{description}

\textbf{Fragen}

\begin{itemize}
  \item 
\end{itemize}

\begin{center}
    \begin{tabular}{  l | l }
    \textbf{Wirklichkeit} & \textbf{Sprache}\\ \hline
    Gegenstand & Name \\ 
    Sachverhalt & Elementarsatz \\
    Sachlage & Satz \\
    \end{tabular}
\end{center}


\section{Primat des Satzsinns: 3.3 bis 3.5\\(28.11.16)}


\subsection{Lekt"urenotizen}
\textbf{Gedanken zum Text}

Wittgenstein scheint...

\vspace{10pt}
\textbf{Neue Begriffe}

\begin{description}[leftmargin=!,labelwidth=\widthof{\bfseries Sachverhalt}]
  \item[Bild] ...
  \item[Tatsache] 
\end{description}

\textbf{Fragen}

\begin{itemize}
  \item 
\end{itemize}



\newpage
\section{Anhang}

\begin{figure}[h]
	%\centering
	\includegraphics[width=1\textwidth]{images/tractatus-structur.png}
	\caption{Der Gesamtaufbau des Tractatus. Quelle: \emph{Sprache und Wirklichkeit in Wittgensteins Tractatus} von Rolf-Albert Dietrich}
	\label{fig:struct}
\end{figure}


\newpage
\section{"Uber den Dozenten}
Dr. Jasper Liptow absolvierte 1996 seinen Magister an der Universit"at Hamburg, promovierte in Gie\ss en mit einer Arbeit zum Thema \emph{Gebrauchstheorien der Bedeutung} bei Prof. Martin Seel und ist Privatdozent.


\begin{figure}[]
	\centering
	\includegraphics[width=0.32\textwidth]{images/liptow.jpg}
	\caption{Dr. Jasper Liptow. Quelle: \url{https://www.uni-frankfurt.de/45457854/liptow.jpg}}
	\label{fig:liptow}
\end{figure}

%\begin{figure}[h]
%	\centering
%	\includegraphics[width=0.5\textwidth]{images/template.png}
%	\caption{Template Bild}
%	\label{fig:template}
%\end{figure}

\end{document}
