\documentclass[a4paper, emulatestandardclasses, 12pt]{scrartcl}
\usepackage{graphicx}
\usepackage{fullpage}
%\usepackage{parskip}
\usepackage{color}
\usepackage[ngerman]{babel}
\usepackage{hyperref}
\usepackage{calc} 
\usepackage{enumitem}
\usepackage{titlesec}
%\pagestyle{headings}
\usepackage{setspace} %halbzeilig
\usepackage[authoryear,round]{natbib}
\bibliographystyle{natdin}

%\titleformat{name=\section,numberless}
%  {\normalfont\Large\bfseries}
%  {}
%  {0pt}
%  {}
\date{\vspace{-3ex}}
\begin{document}

\title{\vspace{5ex}
	\includegraphics*[width=0.72\textwidth]{images/hu_logo.png}\\
	\vspace{30pt}
	\scshape\LARGE{Kurzessay zur\\Gebrauchstheorie der Sprache}}
	
	\subtitle{\vspace{20pt}Einf"uhrung in die Sprachphilosophie\\
          \vspace{6pt}
          Tutorium Benjamin\\}


\author{\vspace{-4pt}Lennard Wolf\\
        \small{\href{mailto:lennard.wolf@student.hu-berlin.de}{lennard.wolf@student.hu-berlin.de}}}      

\maketitle

\vspace{\fill}

\begin{minipage}[b]{\textwidth}
    \centering
    \onehalfspacing
    \large   
    01. Februar 2017\\
    Wintersemester 2016/2017

    \vspace{-20mm} 
\end{minipage}%
\thispagestyle{empty}
\newpage
\clearpage
\setcounter{page}{1}

\begin{onehalfspace} 

\noindent\textbf{$(o)$ Einleitung}

\noindent In dieser Arbeit m"ochte ich erl"autern, weshalb eine Gebrauchstheorie wie sie Ludwig Wittgenstein vorgestellt hat unumg"anglich ist. Desweiteren versuche ich aufzuzeigen, warum diese jedoch vorangegangene Bedeutungstheorien nicht abl"ost, sondern einen Platz zu ihrer Seite einnimmt. 

Dazu werde ich wie folgt vorgehen. In Abschnitt $(i)$ erl"autere ich ein Problem mit der klassischen Bedeutungstheorie von W"ortern. Es folgt in $(ii)$ eine Darstellung der alternativen Gebrauchstheorie nach Wittgenstein. In $(iii)$ zeige ich, dass sein antisystematischer Ansatz jedoch nicht mit dem typischen Erwerb von Fremdsprachenkenntnissen vereinbar ist. Daher pl"adiere ich in $(iv)$ f"ur eine Gebrauchstheorie, die zudem auch systematische Ans"atze erlaubt. 
\vspace{5mm}

\noindent\textbf{$(i)$ Klassische Bedeutungstheorie}

\noindent In seinem Buch "`Philosophische Untersuchungen"' \citep{wittgenstein1963tractatus} stellt Wittgenstein zu Beginn eine klassische\footnote{Diese Bezeichnung stammt vom Autor.} Bedeutungstheorie vor, wie sie zum Beispiel Augustinus vertreten habe. Diese besagt: "`Jedes Wort hat eine Bedeutung. Diese Bedeutung ist dem Wort zugeordnet. Sie ist der Gegenstand, f"ur welchen das Wort steht"' (PU 1). Anhand des Beispiels von "`Gem"usesalat"' m"ochte ich nun zeigen, dass dies eine unzureichende Theorie f"ur die Darstellung des Sprachgebrauchs ist.

Das Wort "`Gem"usesalat"' h"atte nach einer klassischen Bedeutungstheorie eine eindeutig definierte Bedeutung, n"amlich die der Kategorie jener Salate, die zum Gro"steil aus Gem"use zusammengestellt werden. Bei "`Fruchtsalat"' w"urde es sich analog verhalten. Jetzt wollen wir den Tomatensalat einer dieser beiden Kategorien zuordnen: Gingen wir von der genannten Kategoriendefinition aus, so m"ussten wir sagen, dass der Tomatensalat ein Fruchtsalat und kein Gem"usesalat ist. Dies ergibt sich daraus, dass die Bedeutung von "`Fruchtsalat"' eine Kategorie ist, die den Tomatensalat einschlie"st, w"ahrend die Bedeutung von "`Gem"usesalat"' diesen ausschlie"st. Problematisch hieran ist, dass diese Kategorisierung jedoch f"ur die meisten deutsch sprechenden Personen unintuitiv ist, vielleicht sogar als falsch angesehen wird.\footnote{Es ist eine g"angige, jedoch falsche Auffassung, dass Tomaten als Gem"use zu z"ahlen sind. Es handelt sich bei diesen jedoch um Fr"uchte.} 

Es zeigt sich also, dass eine feste Verbindung von Wort und Bedeutung nicht immer eine ad"aquate Darstellung des Sprachgebrauchs ist. Aus diesem Grund entwickelte Ludwig Wittgenstein die seine \emph{Gebrauchstheorie}.

\newpage %\vspace{5mm}
\noindent\textbf{$(ii)$ Antisystematische Gebrauchstheorie nach Wittgenstein}	

\noindent auch das antisystematische hier erläutern

\vspace{5mm}
\noindent\textbf{$(iii)$ Probleme des antisystematischen Ansatzes}	

\noindent 

\vspace{5mm}
\noindent\textbf{$(iv)$ Synthese}	

\noindent 

\end{onehalfspace}

\bibliography{sprachphilo-ha-3.bib}

\end{document}
