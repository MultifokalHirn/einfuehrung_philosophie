\documentclass[a4paper, emulatestandardclasses, 12pt]{scrartcl}
\usepackage{graphicx}
\usepackage{fullpage}
%\usepackage{parskip}
\usepackage{color}
\usepackage[ngerman]{babel}
\usepackage{hyperref}
\usepackage{calc} 
\usepackage{enumitem}
\usepackage{titlesec}
%\pagestyle{headings}
\usepackage{setspace} %halbzeilig
\usepackage[authoryear,round]{natbib}
\bibliographystyle{natdin}

\date{\vspace{-3ex}}
\begin{document}

\title{%\vspace{5ex}
	\includegraphics*[width=0.1\textwidth]{images/hu_logo2.png}\\
	\vspace{50pt}
	\scshape\LARGE{Von den unphysikalischen Aspekten\\der Bluetooth-Technologie}}
	
	\subtitle{Eine Verteidigung des Physikalismus}
\author{Lennard Wolf\\
        \small{\href{mailto:lennard.wolf@student.hu-berlin.de}{lennard.wolf@student.hu-berlin.de}}}      
%Abstract?

\maketitle

\vspace{\fill}

\begin{minipage}[b]{\textwidth}
    \centering
    \onehalfspacing
    \large   
    %30. November 2016\\
    Einf"uhrung in die Philosophie\\
    Wintersemester 2016/2017

    \vspace{-20mm} 
\end{minipage}%
\thispagestyle{empty}
\newpage
\clearpage
\setcounter{page}{1}

\begin{onehalfspace} 


\noindent\textbf{$(i)$ Einleitung}

\noindent In dieser Arbeit m"ochte ich zwei Einw"ande zu Frank Jacksons Argumentation gegen den Physikalismus darlegen. Der erste bezieht auf das Wesen der Information die eine Person erlangt, wenn sie das erste Mal Farben erblickt und der zweite auf die von Jackson angef"uhrte Schlussfolgerung, dass solche Information ein Beweis f"ur das Existieren von Nichtphysikalischem ist.

%Dazu werde ich wie folgt vorgehen. In $(ii)$ rekonstruiere ich Jacksons Argumentation und in $(iii)$ und $(iv)$ stelle ich jeweils einen der Einw"ande dar. In $(v)$ folgt die Konklusion.
\vspace{2mm}

\noindent\textbf{$(ii)$ Jacksons Argumentation}

\noindent In seinem Essay "`Epiphenomenal Qualia"' \citep{jackson1982epiphenomenal} argumentiert Frank Jackson gegen den \emph{Physikalismus}, der besagt, dass es nichts gibt, das nicht physikalisch ist. Jackson sagt zu Beginn, dass aus dem Physikalismus folgen m"usse, dass alle \emph{richtigen} Informationen "uber die Welt sich ausschlie"slich auf etwas physikalisches beziehen muss. Durch ein Gedankenexperiment m"ochte er daraufhin zeigen, dass es jedoch Informationen gibt, die sich \emph{nicht} auf etwas physikalisches beziehen, wodurch "uber den Modus tollens gezeigt w"are, dass der Physikalismus nicht stimmen kann. 

Das Gedankenexperiment verl"auft folgenderma"sen: Mary ist eine Wissenschaftlerin die sich auf Farbforschung spezialisiert hat, jedoch in ihrem Leben noch nie tats"achlich Farben gesehen hat. Durch ihre lange und gr"undliche Forschung konnte sie \emph{alle} physikalischen Informationen "uber Farben erlangen. Doch wenn Mary zum ersten Mal Farben selber sieht, so Jackson, lerne sie doch etwas dazu. Diese neue Information beziehe sich laut Jackson folglich nicht auf etwas Physikalisches und der Physikalismus w"are folglich falsch.\\ 

\noindent Mein erster Einwand richtet sich gegen Annahmen, die dem Gedankenexperiment zugrundeliegen und der zweite gegen die Schlussfolgerung von dem Szenario auf die Existenz nichtphysikalischer Dinge. 

\vspace{5mm}
\noindent\textbf{$(iii)$ Erster Einwand: Feuer und Farbe}

\noindent Jackson irrt sich meiner Meinung nach in der Annahme, dass Mary beim ersten Erblicken von Farben tats"achlich etwas \emph{"uber Farben} lernt. Vielmehr denke ich, dass sie etwas \emph{"uber sich selbst} lernt, n"amlich wie es f"ur sie ist, wenn bestimmte mit Farben in Verbindung gebrachte Wellenl"angen von Licht auf ihre Retina treffen. Dies m"ochte ich veranschaulichen, indem ich das Gedankenexperiment um Mary auf das Feuer "ubertrage. 

Feuer hat aus bestimmten Gr"unden an bestimmten Stellen eine bestimmte Temperatur. Dies ist eine der vielen physikalischen Information "uber das Feuer, die Mary nat"urlich kennen w"urde. Wenn sie nun aber das erste Mal ihre Hand ins Feuer legt und den Schmerz sp"urt, dann lernt sie nicht, dass das Feuer hei"s \emph{ist}, sondern dass es sich hei"s \emph{anf"uhlt}, denn "`hei"s"' ist ein relativer Begriff, der nur in Beziehung zu etwas anderem eine Bedeutung hat. Man muss also unterscheiden zwischen Informationen "uber solche Eigenschaften, die das Feuer wie die Farben objektiv, das hei"st messbar, haben und jene, die man ihnen aufgrund von sinnlicher Erfahrung zuschreibt. Informationen letzterer Art beziehen sich meiner Meinung nach aber nicht mehr nur auf das betreffende Ding als solches, sondern darauf, \emph{wie etwas ist}. Und da Mary eben nur neue Informationen dieser Art erlangt, hat sich nichts "uber nichtphysikalische Eigenschaften von Feuer und Farbe gelernt. 

Sollte Jackson gemeint haben, dass die neuen Informationen sich gar nicht nur auf die Farben beziehen sollten, dann w"are sein Argument nicht mehr schl"ussig, da dann angenommen werden m"usste, dass zwangsl"aufig alles sinnlich wahrgenommene nichtphysikalische Informationen erzeugt. Er w"urde dann also nichtphysikalische Informationen beweisen, indem er sie annimmt.

\vspace{5mm}
\noindent\textbf{$(iv)$ Zweiter Einwand}

\noindent Desweiteren bin ich nicht der Auffassung, dass sich aus Informationen der Art, wie Jackson sie sich vorstellt, schlie"sen l"asst, dass es Dinge gibt, die nicht physikalischer Natur sind. %Dies m"ochte ich an dem folgenden Beispiel erl"autern.
Damit diese Schlussfolgerung m"oglich ist, muss es \emph{ausgeschlossen} sein, dass diese neuen Informationen sich auf etwas physikalisches beziehen, da sie ansonsten aus der Luft gegriffen ist. 	

Bluetooth ist ein standardisiertes Funkverfahren im UHF-Frequenzbereich. Mit unseren derzeitigen sinnlichen M"oglichkeiten sind wir nicht in der Lage, Bluetooth Verbindungen wahrzunehmen. Da diese Technologie von Menschen entwickelt wurde k"onnen wir uns aber sicher sein, dass uns alle physikalischen Informationen "uber Bluetooth zug"anglich sind: Wie die Funkwellen erzeugt werden, welche Frequenz sie haben, wie das Kommunikationsprotokoll aussieht und so weiter. Man stelle sich nun vor, dass ein Bluetooth-Empf"anger entwickelt wird, der Menschen eingepflanzt werden kann und sich mit ihrem Nervensystem verbindet, um Bluetooth \emph{erfahrbar} zu machen. 

\vspace{5mm}
\noindent\textbf{$(v)$ Konklusion}

\noindent 




\end{onehalfspace}
%\nocite{*}
\bibliography{einfuehrung-i-d-p-essay}

\end{document}
