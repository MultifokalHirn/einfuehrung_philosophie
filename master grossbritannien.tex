\documentclass[a4paper]{article}
\usepackage{graphicx}
\usepackage{fullpage}
%\usepackage{parskip}
\usepackage{color}
\usepackage[ngerman]{babel}
\usepackage{hyperref}
\usepackage{calc} 
\usepackage{enumitem}
\usepackage{titlesec}
\usepackage{bussproofs}
\usepackage[export]{adjustbox}
%\pagestyle{headings}

\titleformat{name=\section,numberless}
  {\normalfont\Large\bfseries}
  {}
  {0pt}
  {}
\date{\vspace{-3ex}}
\begin{document}

\title{
    \vspace{-30pt}
	\includegraphics*[width=0.1\textwidth,left]{ErstesSem/images/hu_logo2.png}\\
	\vspace{-10pt}
	Master Grossbritannien}
\author{Lennard Wolf\\
        \small{\href{mailto:lennard.wolf@student.hu-berlin.de}{lennard.wolf@student.hu-berlin.de}}}
\maketitle
\vspace{-4pt}

\section*{Bewerbung}
\large

MLitt in St Andrews / Stirling (posh, renommiertes Phil. Dep., sehr klein)
MLitt Logic weniger veranstaktungen als philosophy
Einj"ahrig


Cambridge Master: Nur Essays, keine Kurse! ("`research post-graduate course"') 

Gegensatz dazu: Taught Master

Oxford: gr"osstes Philosophisches Institut der Welt, d.h. extrem gro"ses Spektrum an Themen

philosophicalgourmet.com f"ur Uni-Listen sortiert nach Themen  

--------------

Wenn Essay sehr gut ist sind Noten m"oglicherweise nur sekund"ar

--------------

empfehlungsschreiben genauso wichtig wie essay


----

toefl wird überall akzeptiert, IELTS ist mit Menschen



Versicherung f"ur EU-STudis kostenlos, 


Stipendienbewerbungsfristen sind meist erst recht lang \emph{nach} der Annahme, aber man kann nicht damit rechnen, angenommen zu werden

rhodes scholarship extrem aufwendig!


daad-stipendien-bewerbung meist lange (3 mon) vor bewerbung an uni, empfehlungsschreiben müssen per post!

daad will den toefl! ist in bonn im märz wird man persönlich eingeladen, 2 empfehlungsschreiben sind gut!

daad fragt welches freizeitangebot man nutzen will, d.h. man muss einiges über land und uni wissen



tinyurl.com/ycyrf7cl

\end{document}
