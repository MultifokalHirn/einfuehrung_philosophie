\documentclass[a4paper, 12pt]{article}
%\usepackage{CJKutf8} % japanese
\usepackage{graphicx}
\usepackage{hyperref}
\usepackage{fullpage}
%\usepackage{parskip}
\usepackage{color}
\usepackage[ngerman]{babel}
\usepackage{hyperref}
\usepackage{calc} 
\usepackage{enumitem}
\usepackage[utf8]{inputenc}
\usepackage{titlesec}
%\pagestyle{headings}
\usepackage{setspace} %halbzeilig
\usepackage[style=authoryear-ibid,natbib=true]{biblatex}
\usepackage[hang]{footmisc}
\setlength{\footnotemargin}{-0.8em}
%\bibliographystyle{natdin}
\addbibresource{debatten-ha2.bib}
\DeclareDatamodelEntrytypes{standard}
\DeclareDatamodelEntryfields[standard]{type,number}
\DeclareBibliographyDriver{standard}{%
  \usebibmacro{bibindex}%
  \usebibmacro{begentry}%
  \usebibmacro{author}%
  \setunit{\labelnamepunct}\newblock
  \usebibmacro{title}%
  \newunit\newblock
  \printfield{number}%
  \setunit{\addspace}\newblock
  \printfield[parens]{type}%
  \newunit\newblock
  \usebibmacro{location+date}%
  \newunit\newblock
  \iftoggle{bbx:url}
    {\usebibmacro{url+urldate}}
    {}%
  \newunit\newblock
  \usebibmacro{addendum+pubstate}%
  \setunit{\bibpagerefpunct}\newblock
  \usebibmacro{pageref}%
  \newunit\newblock
  \usebibmacro{related}%
  \usebibmacro{finentry}}

%\titleformat{name=\section,numberless}
%  {\normalfont\Large\bfseries}
%  {}
%  {0pt}
%  {}
\date{\vspace{-3ex}}
\begin{document}

\title{\vspace{5ex}
	\includegraphics*[bb=0 0 720 200, width=0.72\textwidth]{ErstesSem/images/hu_logo.png}\\
	\vspace{30pt}
	\scshape\LARGE{Zusammenfassung II}\\\Large{Orientalism}\vspace{20pt}}
	


\author{Regionalwissenschaftliche Debatten\\
	\vspace{7pt}
          Dozent: Prof. Dr. phil. Vincent Houben\\\vspace{4pt}Lennard Wolf\\
        \small{Matrikelnummer: 583052}\\
        \small{E-Mail: lennard.wolf@hu-berlin.de}}

        %\href{mailto:lennard.wolf@student.hu-berlin.de}{lennard.wolf@student.hu-berlin.de}}}      

\maketitle

\vspace{\fill}

\begin{minipage}[]{0.92\textwidth}
    \centering
    \onehalfspacing
    \large   
    25. November 2017\\
    Wintersemester 17/18

    \vspace{-20mm} 
\end{minipage}%
\thispagestyle{empty}
\newpage
%\clearpage
%\thispagestyle{empty}
%\tableofcontents
%\newpage
\setcounter{page}{1}

\begin{onehalfspace} 

%\noindent\textbf{Zusammenfassung}

% Überlegen Sie sich Zwischenüberschriften zu den einzelnen Abschnitten des Textes!

% Erarbeiten Sie sich aus jedem Abschnitt zwei Kernaussagen!

% Versuchen Sie, eine Definition des Konzeptes "Orientalismus" zu formulieren!

\noindent Zu Beginn stellt der Autor drei Bedeutungen für den Orientalismusbegriff vor. Erstens als Forschungsfeld, in dem Wissenschaftler*innen sich mit der Kultur und Geopolitik des Orients beschäftigen, zweitens als ontologische Unterscheidung zwischen Okzident und Orient, die sich von der Kunst bis zu den Sozialwissenschaften breit gemacht hat, und drittens als institutionalisierte Domination des Orients, die vor allem von England und Frankreich herrührt. Im Verlauf des Buches wird sich Said an Foucaults Begriff des "`Diskurses"' orientieren, um aufzuzeigen, dass keine ungefärbte, freie Beschäftigung mit dem Orient möglich war und ist, und inwiefern Wissen und Macht im Orientalismus zusammenhängen. Orient und Okzident sind vom Menschen erdachte Konzepte, deren Entstehung untrennbar ist von Macht und Formen der Hegemonie. Orientalismus ist folglich keine unschuldige Fantasie von exotischen Ländern und Leuten, sondern eine Ansammlung von herabschauenden, imperialistischen Theorien und Praktiken, die sich tief in das Denken der westlichen Kultur eingeprägt hat. 

Auf die Einführung zum Orientalismusbegriff folgt eine Reflexion über die Unterscheidung zwischen politischem und "`reinem"', das heißt von der Meinung der Wissenschaftler*innen ungefärbtem, Wissen. Die Vorstellung, wahres Wissen wäre immer unpolitisch, lässt außen vor, dass kein Mensch seinem soziopolitischen Kontext entfliehen und so "`objektives"' Wissen generieren kann. Folglich sei in allen Auseinandersetzungen des Westens mit dem Orient der Orientalismus anzutreffen, der durch politische Interessen und geistige Kultur zu einem breitgestreuten Narrativ wurde. Somit ist es anzutreffen in literarischen Werken von Balzac, zu politikphilosophischen wie jenen von Mill und Marx. Für Said ergeben sich daraus Fragen nach den Motivationen hinter der Entstehung des Orientalismus, nach der Art und Weise, wie geistiges Leben Orientalismus geformt und am Leben erhalten hat, und wie mit ihm umzugehen sei. 

Anschließend bespricht Said noch nein Vorgehen bei der Wahl der Grundlagentexte, der Entscheidung, sich auf England und Frankreich zu konzentrieren, und seine Beschäftigung mit der Oberfläche und Oberflächlichkeit des orientalistischen Diskurses.

Im letzten Abschnitt wird die Unterscheidung zwischen \emph{latentem}, das heißt im Grunde unbewusstem, und \emph{manifestem}, das heißt offen betriebenem, Orientalismus. Während manifester Orientalismus sich zum Beispiel in der Beschreibung fremder Völker als "`unzivilisiert"', "`rückwärtsgerichtet"' oder "`degeneriert"' äußert, verbirgt sich latenter Orientalismus im Denken über die fremden Menschen als edle Wilde.



\end{onehalfspace}
\nocite{*}
%\bibliography{merleau-ponty-essay}
\printbibliography
\end{document}
