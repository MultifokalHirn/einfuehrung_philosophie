\documentclass[a4paper]{article}
\usepackage{graphicx}
\usepackage{fullpage}
%\usepackage{parskip}
\usepackage{color}
\usepackage[ngerman]{babel}
\usepackage{hyperref}
\usepackage{calc} 
\usepackage{enumitem}
\usepackage{titlesec}
\usepackage{bussproofs}
\usepackage[export]{adjustbox}
%\pagestyle{headings}

\titleformat{name=\section,numberless}
  {\normalfont\Large\bfseries}
  {}
  {0pt}
  {}
\date{\vspace{-3ex}}
\begin{document}

\title{
    \vspace{-30pt}
	\includegraphics*[width=0.1\textwidth,left]{ErstesSem/images/hu_logo2.png}\\
	\vspace{-10pt}
	Aristoteles: Nikomachische Ethik}
\author{Lennard Wolf\\
        \small{\href{mailto:lennard.wolf@student.hu-berlin.de}{lennard.wolf@student.hu-berlin.de}}}
\maketitle
\vspace{-4pt}

\section*{Lekt"urenotiz [I 13; X 6-9]}
\large

\textbf{[I 13]} Gutheit (\emph{aret\={e}}) = \emph{menschliche} Gutheit. Seele hat vernunftlosen (vegetativen) und vern"unftigen Bestandteil $\rightarrow$ Gutheit des vernunftlosen Bestandteils ist allen Tieren gleich und somit nicht relevant f"ur die \emph{menschliche} Gutheit. Es gibt in der Seele etwas, das gegen die Vernunft wirkt (wie der Gegenmuskel) $\rightarrow$ Beide Bestandteile der Seele sind derart zweiteilig $\rightarrow$ Entsprechend wird die Gutheit auch zweigeteilt, und zwar in Tugenden des \emph{Denkens} und Tugenden des \emph{Charakters}.\newline

\noindent \textbf{[X 6]} Gl"uck (\emph{eudaemonia}) ist keine Disposition (sonst k"onnte der Gute auch immerzu schlafen). Handlungen der Gutheit: "uber diese hinaus sucht man nichts $\rightarrow$ Aber dies scheint auch f"ur Vergn"ugung zu gelten, und diese kann nicht allein gut sein, da f"ur sie anderes vernachl"assigt wird (K"orper, Arbeit) $\rightarrow$ Das Vernachl"assigte muss wieder erarbeitet werden und so m"ussen daraufhin Dinge getan werden, die nicht nur mit dem Gl"uck zum Ziele unternommen werden $\rightarrow$ Vergn"ugung ist also nicht Ziel, sondern lediglich ein Mittel daf"ur, Kraft f"ur gute/tugendhafte T"atigkeiten zu sammeln. \newline

\noindent \textbf{[X 7]} Betrachtung ist die h"ochste T"atigkeit, da ihr Gegenstand die h"ochsten Dinge sind, sie die kontinuierlichste ist, sie ernsthaft ist und ihr Weisheit (\emph{sophia}) beigemischt ist, welche die h"ochste Lust ist. Zudem ist sie autark und wird um ihrer selbst willen geliebt. | Die Betrachtung kann aber nicht die vollkommen einzige Bet"atigung sein, zudem h"atte dies etwas g"ottliches, un\emph{menschliches} -- keine menschliche Bet"atigung ist vollst"andig. Trotzdem ist es die "`gl"uckseligste"' aller T"atigkeiten.\newline

\noindent \textbf{[X 8]} Sekund"are Form des Gl"ucks wird erlangt durch guten Charakter, ausgedr"uckt durch gute Bet"atigung (\emph{energeia}): gerechtes, tapferes, angemessenes, kluges, richtiges Handeln. Solche Tugend ist aber getrennt von der Tugend des intuitiven Denkens (\emph{nous}, siehe [X 7]), welche die prim"are Form des Gl"ucks ist. Jede gute \emph{energeia} hat bestimmte Vorraussetzungen, so braucht der Tapfere einen starken K"orper usw. -- die Betrachtung aber hat keine/wenige Vorraussetzung. Die vollkommene Tugend liegt zugleich im Vorsatz und der damit verbundenen Handlung selbst. | Dass die Betrachtung die prim"are Form des Gl"ucks ist zeigt sich auch darin, dass die G"otter nicht handeln wie die Menschen (der Gedanke allein ist l"acherlich), aber ja insgesamt gl"uckselig sind, und somit wohl allein der Betrachtung fr"ohnen m"ussen.\newline

\noindent \textbf{[X 9]} Das gl"uckselige Leben bedarf bestimmter Umst"ande (genug zu essen usw.), doch nie im "Uberma"s: "`Auch mit bescheidenen Mitteln kann man n"amlich in Aus"ubung der Tugend handeln"', es gilt sich prim"ar im Sinn der Tugend zu bet"atigen. So brauch man, um den Gl"ucklichen zu finden, nicht nur bei den Reichen und M"achtigen suchen. Ob alles Gesagte zum Gl"uck wirklich wahr ist, l"asst sich nur durch Taten nachpr"ufen. | Die G"otter erfreuen sich an dem ihnen "ahnlichen, also dem Betrachtendem, dem Weisen. Dieser ist der Gl"ucklichste und von den G"ottern am meisten geliebt.\newline

\noindent \textbf{Frage 1:} Wenn Vorsatz und Handlung beide zugleich relevant sind, um eine Handlung vollkommen tugendhaft zu machen, kann es dann Situationen geben, in denen vollkommen tugendhafte Handlungen unm"oglich sind? \newline

\noindent \textbf{Frage 2:} Ich verstehe den Begriff des intuitiven Denkens (\emph{nous}) nicht recht -- ist er gleichzusetzen mit der Betrachtung? Oder handelt es sich dabei um die Voraussetzung f"ur die Betrachtung, wie die Beine f"ur das Gehen?
\end{document}
