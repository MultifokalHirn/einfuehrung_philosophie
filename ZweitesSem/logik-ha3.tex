\documentclass[a4paper]{article}

\usepackage{rotating}
\usepackage{qtree}
%\usepackage{KMcalc} %Lennard



%KM-Kalkül, geordnetes paar und gather-Umgebung für Formeln:
\newcommand{\UB}{$\und$B}
\newcommand{\UE}{$\und$E}
\newcommand{\OrE}{$\oder$E}
\newcommand{\OrB}{$\oder$B}
\newcommand{\EB}{$\ex$B}
\newcommand{\EE}{$\ex$E}
\newcommand{\paar}[1]{\left \langle #1 \right \rangle}
\newcommand{\gth}[2]{\begin{gather*} #1 \label{#2} \end{gather*}}
\newcommand{\gthd}[2]{\begin{gather} \begin{gathered}#1 \label{#2}\end{gathered}\end{gather}}

%objektsprachliche Symbole:
\newcommand{\und}{\wedge}
\newcommand{\oder}{\vee}
\newcommand{\then}{\rightarrow}
\newcommand{\eq}{\leftrightarrow}
\newcommand{\uu}{\cup}
\newcommand{\C}{\cap}
\newcommand{\TM}{\subseteq}
\newcommand{\all}{\forall}
\newcommand{\ex}{\exists}
% \neg gibt es schon

%Zahlbereichsymbole, Potenzmenge und Sprachnamen
\newcommand{\NN}{\mathbb{N}}
\newcommand{\QQ}{\mathbb{Q}}
\newcommand{\RR}{\mathbb{R}}
\newcommand{\Pp}{\mathcal{P}}
\newcommand{\Ll}{\ensuremath{\mathcal{L}}}
\newcommand{\pair}[1]{\langle #1 \rangle}
\newcommand{\quine}[1]{\ulcorner #1 \urcorner}
\newcommand{\mq}[1]{\mlq #1 \mrq} % steht für "math quotes"
\newcommand{\sse}{\subseteq}
\newcommand{\gesch}{\cap} % steht für "geschnitten"
\newcommand{\verin}{\cup} % Habe hier mit Absicht einen Buchstaben weggelassen, ähnlich wie \infty
\newcommand{\set}[1]{\{#1\}}
\newcommand{\LPL}{\ensuremath{\mathcal{L}_{PL}}}
\newcommand{\LAL}{{\ensuremath{\mathcal{L}_{AL}}}} % zusätzliche „{}“, um es in Subskripts nutzen zu können
\newcommand{\FmLAL}{\ensuremath{\mathcal{F}m_\LAL}} % zur besseren Lesbarkeit der AL-Definitionen
\newcommand{\SKLAL}{\ensuremath{\mathcal{SK}_\LAL}} % zur besseren Lesbarkeit der AL-Definitionen
\newcommand{\MIb}{{\pair{M, I}, \beta}} % zur Benutzung im Math-Mode


%metasprachliche Symbole:
\newcommand{\Land}{
    \raisebox{-0.12em}{ \begin{turn}{90} $\eqslantgtr$ \end{turn} }
}
\newcommand{\Lor}{
     \raisebox{-0.12em}{ \begin{turn}{90} $\eqslantless$ \end{turn} }
}
\newcommand{\Then}{
    \Rightarrow
}
\newcommand{\Gdw}{
    \Leftrightarrow
}
\newcommand{\Neg}{
    \neg\hspace*{-0.5em}\neg
}
\newcommand{\Forall}{
    \raisebox{0.18em}{\scriptsize{\textbackslash}}\hspace*{-0.175em}\forall
}
\newcommand{\Exists}{
    \exists\hspace*{-0.45em}\exists
}

\newcommand{\nehT}{\Leftarrow}

\DeclareMathSymbol{\mlq}{\mathord}{operators}{``}
\DeclareMathSymbol{\mrq}{\mathord}{operators}{`'}
\newcommand{\concat}{\raisebox{0.45em}{\scalebox{0.7}{$\smallfrown$}}}

%models gibt es schon
%\

\newcommand{\deduces}{\vdash}
\newcommand{\sidew}[1]{\begin{sideways} #1 \end{sideways}}
%Anführungszeichen

\newcommand{\qleft}{\ulcorner}
\newcommand{\qright}{\urcorner}
\newcommand{\qc}[1]{\qleft #1 \qright}
\newcommand{\anf}[1]{`#1'}
\newcommand{\manf}[1]{\text{`}#1\text{}}
\newcommand{\tmanf}[1]{\text{`#1'}}
%\renewcommand{\models}{\vDash}
\newcommand{\Anf}[1]{„#1“}
\newcommand{\cn}{\/^{\smallfrown}}
%Kopf


\usepackage{stmaryrd}
\usepackage{graphicx}
\usepackage{fullpage}
%\usepackage{parskip}
\usepackage{color}
\usepackage[ngerman]{babel}
\usepackage{hyperref}
\usepackage{calc} 
\usepackage{enumitem}
\usepackage{titlesec}
\usepackage{bussproofs}
\usepackage[export]{adjustbox}
%\pagestyle{headings}
\usepackage{amssymb} % fuer logik

\titleformat{name=\section,numberless}
  {\normalfont\Large\bfseries}
  {}
  {0pt}
  {}
\date{\vspace{-3ex}}
\begin{document}

\title{
    \vspace{-30pt}
	\includegraphics*[width=0.1\textwidth,left]{ErstesSem/images/hu_logo2.png}\\
	\vspace{-10pt}
	Einf"uhrung in die Logik -- Aufgabenblatt 3}
\author{Lennard Wolf\\
        \small{\href{mailto:lennard.wolf@student.hu-berlin.de}{lennard.wolf@student.hu-berlin.de}}}
\maketitle
\vspace{-4pt}

\section*{Aufgabe 1}
\large

\textbf{a) }
\vspace{4pt}

Zu zeigen:

\vspace{2pt}
($\circ$) \hspace*{1em} $(A \Land \Neg B) \Then \Neg (A \Then B)$

\vspace{2pt}
Angenommen:

\vspace{2pt}
(1) \hspace*{1em} $A \Land \Neg B$

\vspace{2pt}
Zu zeigen:

\vspace{2pt}
($\dagger$) \hspace*{1em} $\Neg (A \Then B)$

\vspace{2pt}
Angenommen (zur \emph{reductio ad absurdum} $\Neg$($\dagger$)): 

\vspace{2pt}
(2) \hspace*{1em} $A \Then B$
\vspace{2pt}

Wir wissen aus (1): 

\vspace{2pt}
(3) \hspace*{1em} $A$

\vspace{2pt}
Aus (2) und (3) per \emph{modus ponens}: 

\vspace{2pt}
(4) \hspace*{1em} $B$

\vspace{10pt}
(1) $\lightning$ (4). 

Durch die \emph{reductio} ist gezeigt, dass $\Neg \Neg (\dagger$), also  ($\dagger$), aus (1) folgt. Damit ist ($\circ$) gezeigt.
\vspace{14pt}



\noindent \textbf{b) }
\vspace{4pt}

Zu zeigen:

\vspace{2pt}
($\circ$) \hspace*{1em} $(A \Lor B) \Then (A \Then \Neg B)$

\vspace{2pt}
Angenommen:

\vspace{2pt}
(1) \hspace*{1em} $A \Lor B$

\vspace{2pt}
Zu zeigen:

\vspace{2pt}
($\dagger$) \hspace*{1em} $A \Then \Neg B$

\vspace{2pt}
Oder-Beseitigung aus (1): 

\vspace{2pt}
(2) \hspace*{1em} $A$

\vspace{2pt}
Aus ($\dagger$) und (2) per \emph{modus ponens}: 

\vspace{2pt}
(3) \hspace*{1em} $\Neg B$

\vspace{2pt}
Oder-Beseitigung aus (1): 

\vspace{2pt}
(4) \hspace*{1em} $B$

\vspace{10pt}
(3) $\lightning$ (4). 

Durch den Widerspruch ist gezeigt, dass ($\dagger$) nicht aus (1) folgen kann. Damit ist gezeigt, dass ($\circ$) logisch falsch ist.

Beispiels"atze: "`Ich bin ein Mensch"' $\Lor$ "`Ich bin ein Tier"' $\Then$ "`Ich bin ein Mensch und ich bin nicht ein Tier"'. 


\vspace{14pt}

\noindent \textbf{c) }
\vspace{4pt}

Zu zeigen:

\vspace{2pt}
($\circ$) \hspace*{1em} $A \Lor (B \Then A)$

\vspace{2pt}
Angenommen:

\vspace{2pt}
(1) \hspace*{1em} $A$

\vspace{2pt}
Zu zeigen:

\vspace{2pt}
($\dagger$) \hspace*{1em} $B \Then A$

\vspace{2pt}
Angenommen (zur \emph{reductio ad absurdum} die Kontraposition $\Neg$($\dagger$)): 

\vspace{2pt}
(2) \hspace*{1em} $\Neg B \Then \Neg A$

\vspace{2pt}
Angenommen: 

\vspace{2pt}
(3) \hspace*{1em} $\Neg B$

\vspace{2pt}
Aus (2) und (3): 

\vspace{2pt}
(4) \hspace*{1em} $\Neg A$

\vspace{10pt}
(1) $\lightning$ (4). 

Es wurde gezeigt, dass $\Neg (\dagger$) nicht mit (1) "`verodert"' werden kann. Durch die \emph{reductio} ist ($\circ$) gezeigt.

\vspace{14pt}

\noindent \textbf{d) }
\vspace{4pt}

Zu zeigen:

\vspace{2pt}
($\circ$) \hspace*{1em} $\Neg (A \Lor B) \Then (\Neg A \Land \Neg B)$

\vspace{2pt}
Angenommen:

\vspace{2pt}
(1) \hspace*{1em} $\Neg (A \Lor B)$

\vspace{2pt}
Zu zeigen:

\vspace{2pt}
($\dagger$) \hspace*{1em} $\Neg A \Land \Neg B$

\vspace{2pt}
Oder-Beseitigung aus (1): 

\vspace{2pt}
(2) \hspace*{1em} $\Neg A$

\vspace{2pt}
Oder-Beseitigung aus (1): 

\vspace{2pt}
(3) \hspace*{1em} $\Neg B$

\vspace{2pt}
Und-Einf"uhrung aus (2) und (3): 

\vspace{2pt}
(4) \hspace*{1em} $\Neg A \Land \Neg B$

\vspace{10pt}
($\dagger$) $\Gdw$ (4). 

($\dagger$) folgt also logisch aus (1), womit ($\circ$) gezeigt w"are.

\newpage
\noindent \textbf{e) }
\vspace{4pt}

Zu zeigen:

\vspace{2pt}
($\circ$) \hspace*{1em} $(A \Land \Neg A) \Then B$

\vspace{2pt}
Angenommen:

\vspace{2pt}
(1) \hspace*{1em} $A \Land \Neg A$

\vspace{2pt}
Zu zeigen:

\vspace{2pt}
($\dagger$) \hspace*{1em} $B$

\vspace{2pt}
Und-Beseitigung aus ($1$):

\vspace{2pt}
(2) \hspace*{1em} $A$

\vspace{2pt}
Und-Beseitigung aus ($1$):

\vspace{2pt}
(3) \hspace*{1em} $\Neg A$

\vspace{2pt}
Oder-Einf"uhrung aus (2):

\vspace{2pt}
(4) \hspace*{1em} $A \Lor B$

\vspace{2pt}
\emph{Modus tollens} aus (3) und (4): 

\vspace{2pt}
(5) \hspace*{1em} $B$

\vspace{10pt}
($\dagger$) $\Gdw$ (5). 

($\dagger$) folgt also logisch aus (1), womit ($\circ$) gezeigt w"are.



\section*{Aufgabe 2}


\begin{description}[leftmargin=!,labelwidth=\widthof{\bfseries g)}]
  \item[a)] $\Forall x (x$ ist sch"on)
  \item[b)] $\Exists x (x$ ist ein Drache $\Land~x$ kann fliegen)
  \item[c)] $\Forall x (x$ ist ein Drache $\Then x$ lebt nicht vegetarisch)
  \item[d)] $\Exists x (x$ ist ein Stern $\Land~x$ ist nicht hell)
  \item[e)] $\Forall x (x$ ist eine nat"urliche Zahl $\Then  (x$ ist gerade $\Lor  x$ ist ungerade))
  \item[f)] $x$ ist eine Stadt $\Gdw x$ hat mehr als 1999 Einwohner*innen
  \item[g)] ($\Neg \Exists x (x$ ist ein Einhorn)) $\Then$ $\Forall y (y$ ist ein*e Naturforscher*in $\Then y$ irrt sich nicht)
\end{description}

\end{document}
