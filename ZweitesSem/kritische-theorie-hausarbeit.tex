\documentclass[a4paper, 12pt]{article}
%\usepackage{CJKutf8} % japanese
\usepackage{graphicx}
\usepackage{hyperref}
\usepackage{fullpage}
%\usepackage{parskip}
\usepackage{color}
\usepackage[ngerman]{babel}
\usepackage{hyperref}
\usepackage{calc} 
\usepackage{enumitem}
\usepackage[utf8]{inputenc}
\usepackage{titlesec}
%\pagestyle{headings}
\usepackage{setspace} %halbzeilig
\usepackage[style=authoryear-ibid,natbib=true]{biblatex}
\usepackage[hang]{footmisc}
\setlength{\footnotemargin}{-0.8em}
%\bibliographystyle{natdin}
\addbibresource{kritische-theorie-hausarbeit.bib}
\DeclareDatamodelEntrytypes{standard}
\DeclareDatamodelEntryfields[standard]{type,number}
\DeclareBibliographyDriver{standard}{%
  \usebibmacro{bibindex}%
  \usebibmacro{begentry}%
  \usebibmacro{author}%
  \setunit{\labelnamepunct}\newblock
  \usebibmacro{title}%
  \newunit\newblock
  \printfield{number}%
  \setunit{\addspace}\newblock
  \printfield[parens]{type}%
  \newunit\newblock
  \usebibmacro{location+date}%
  \newunit\newblock
  \iftoggle{bbx:url}
    {\usebibmacro{url+urldate}}
    {}%
  \newunit\newblock
  \usebibmacro{addendum+pubstate}%
  \setunit{\bibpagerefpunct}\newblock
  \usebibmacro{pageref}%
  \newunit\newblock
  \usebibmacro{related}%
  \usebibmacro{finentry}}

%\titleformat{name=\section,numberless}
%  {\normalfont\Large\bfseries}
%  {}
%  {0pt}
%  {}
\date{\vspace{-3ex}}


\begin{document}

\title{\vspace{5ex}
	\includegraphics*[bb=0 0 720 200, width=0.72\textwidth]{ErstesSem/images/hu_logo.png}\\
	\vspace{30pt}
	\scshape\LARGE{Regressiver Fortschritt}\\\vspace{5pt}\Large{Zur Dialektik der Aufklärung und dem allgegenwärtigen Fortschrittsdrang}\\\vspace{20pt}}
	


\author{Kritische Theorie der Gesellschaft: Horkheimer und Adorno (PS)\\
	\vspace{7pt}
          Dozent: Dr. Arnd Pollmann\\\vspace{4pt}Lennard Wolf\\
        \small{Matrikelnummer: 583052}\\
        \small{E-Mail: \href{mailto:lennard.wolf@hu-berlin.de}{lennard.wolf@hu-berlin.de}}\\
        \small{Telefonnummer: +49 176 5687 4131}\\
        \small{Studiengang: B.A. Philosophie}\\
        \small{Modul: Praktische Philosophie}}

\maketitle

\vspace{\fill}

\begin{minipage}[]{0.92\textwidth}
    \centering
    \onehalfspacing
    \large   
    30. September 2017\\
    Sommersemester 2017

    \vspace{-20mm} 
\end{minipage}%
\thispagestyle{empty}
\newpage
%\clearpage
%\thispagestyle{empty}
%\tableofcontents
%\newpage
\setcounter{page}{1}

\begin{onehalfspace} 

\noindent\textbf{$(o)$ Einleitung}

\noindent Das 20. Jahrhundert ist nun schon seit einigen Jahren vorbei. Seine Schrecken bleiben für die einen als stetes Mahnmal präsent, für die anderen könnte die "`dämliche Bewälti\-gungs\-politik"'\footnote{Wortlaut des AfD-Fraktionsvorsitzenden im Thüringer Landtag, Björn Höcke. Siehe \url{https://www.tagesschau.de/inland/hoecke-rede-105.html}.} endlich mal aufhören. Dem Anschein nach hat sich seit dem zweiten Weltkrieg in Deutschland auf jeden Fall vieles verändert. Der Bundesinnenminister spricht heute sogar von einer "`Leitkultur"' und von "`aufgeklärten Patrioten"'. Die Deutschen seien also aufgeklärt und repräsentieren eine moderne Kultur, an deren Leitung man sich halten könne. Doch was heißt es überhaupt, "`aufgeklärt"' zu sein? Hielt sich die Gesellschaft zu Hitlers Zeiten nicht auch für aufgeklärt? Welchen Fortschritt haben die Deutschen seither gemacht, und wohin leiten sie eigentlich? 

Die Sozialwissenschaftler und Philosophen Max Horkheimer und Theodor W. Adorno haben sich unter anderem mit der vorgeblichen Widersprüchlichkeit zwischen der Aufklärung und den Grausamkeiten der Nationalsozialisten beschäftigt. Sie waren Vertreter der \emph{Kritischen Theorie}, die sich mit den herrschenden Strukturen und Ideologien in der Gesellschaft kritisch auseinanderzusetzen versucht und Sozialwissenschaften nicht bloß als von außen beobachtende Messarbeit betreiben will. In dieser Hausarbeit möchte ich ihre Überlegungen zur Dialektik der Aufklärung begreiflich machen und daraufhin näher auf den Begriff des Fortschritts eingehen, zum einen um am Ende in Frage zu stellen, dass wir tatsächlich in aufgeklärten Zeiten leben und zum anderen um zu begründen, weshalb vorbehaltsloser Fortschrittsglaube ebenso problematisch ist, wie radikaler Konservatismus.

Dafür werde ich wie folgt vorgehen. Im eher theoretisch gehaltenen Abschnitt $(i)$ führe ich in die Idee der Dialektik der Aufklärung ein. Es folgt in Abschnitt $(ii)$ eine kritische Auseinandersetzung mit dem für die Aufklärung zentralen Fortschrittsgedanken, und dem heutzutage allgegenwärtigen Drang, ihm zu folgen. Abschnitt $(iii)$ schließt die Hausarbeit zusammenfassend ab.

\vspace{5mm}


\noindent\textbf{$(i)$ Dialektik der Aufklärung}

\noindent \emph{Dialektik der Aufklärung}, 1944 im kalifornischen Exil der Autoren fertig gestellt, gilt als eines der wirkmächtigsten\footnote{\Cite[Vgl.][S. 249]{jaeggi}.} Werke der Kritischen Theorie. Ihr Ausgangspunkt ist die Frage, "`warum die Menschheit, anstatt in einen wahrhaft menschlichen Zustand einzutreten, in eine neue Art von Barbarei versinkt"'\footnote{\Cite[Siehe][S. 1]{dialektik-der-aufklaerung}.}. Wenn doch die Aufklärung, die sich programmatisch den "`Ausgang des Menschen aus seiner selbst verschuldeten Unmündigkeit"'\footnote{\Cite[Siehe][S. 481]{kant}.}, oder in anderen Worten, eine Befreiung von Gesellschaft und Verstand, auf die Fahne geschrieben hat, wie konnte es unter anderem zu den Schrecken des zweiten Weltkriegs kommen, vor denen die Autoren geflohen sind? Diese scheinbare Widersprüchlichkeit führt Horkheimer und Adorno zu dem Versuch einer \emph{Aufklärung der Aufklärung}\footnote{\Cite[Vgl.][S. 406]{habermas}.}, die zeigen soll, dass die barbarischen Zustände nicht \emph{trotz} des aufklärerischen Denkens ausbrachen, sondern dass sie gerade dessen \emph{notwendige Konsequenz} sind. Die Aufklärung der Aufklärung vollziehen die Autoren auf zwei Ebenen. Zum einen üben sie Kritik an der hegemonialen Ideologie, das heißt der allumfassenden, allem gesellschaftlichen Handeln zugrunde liegenden bürgerlichen Gesinnung, indem sie deren Widersprüchlichkeiten aufzeigen. Zum anderen geben sie der Kritik noch ein selbstreflexives Moment hinzu. Sie fragen nach der Möglichkeit der Kritik selbst, da diese selber, durch die Absicht der Verbesserung der Umstände, der kritisierten zweckrationalen Denkweise verhaftet bleibt. Die sich aus diesen Untersuchungen ergebende Kernthese der \emph{Dialektik der Aufklärung} ist, dass die Aufklärung, in der Form, die sie zumindest um 1944 hat, zum "`totalen Betrug der Massen"'\footnote{\Cite[Siehe][S. 49]{dialektik-der-aufklaerung}.} geworden ist und ihrem Ziel, der Freiheit, so nicht näher kommen kann. 

Die \emph{Fragmente} des Werkes sind eine Vorrede, ein geschichtsphilosophischer Einfüh\-rungs\-essay mit dem Titel "`Begriff der Aufklärung"', zwei Exkurse "`Mythos und Aufklärung"' und "`Aufklärung und Moral"', zwei zeitdiagnostische Essays zu Kulturindustrie und Antisemitismus, sowie eine angehängte Sammlung von Aufzeichnungen und Entwürfen. Diese schon in seinem Titel angedeutete fragmentarische Struktur des Buches findet sich auch in den einzelnen Essays selber wieder, weshalb Jürgen Habermas dem Buch eine "`eher unübersichtliche Form der Darstellung"'\footnote{\Cite[Siehe][S. 406]{habermas}.} beimisst und Rahel Jaeggi den Argumentationsgang als "`eliptisch [sic]"'\footnote{\Cite[Siehe][S. 250]{jaeggi}.} anmutend bezeichnet. Dieser Umstand erschwert die Lektüre des Werkes und verschleiert auch zuweilen die Intention der Autoren.

Es lohnt sich, zu Beginn die im Buchtitel befindlichen Begriffe "`Dialektik"' und "`Aufklärung"' zu klären. "`Dialektik"' bezieht sich Adorno zufolge sowohl auf die Struktur des behandelten Gegenstands, als auch auf die dadurch notwendige Methode.\footnote{\Cite[Vgl.][S. 9]{dialektik}.} Eine dialektische Philosophie versucht die Dinge zu begreifen, nicht indem sie sie definiert und unserem geistigen Begriff von ihr gleichsetzt, sondern indem sie die untersuchte Sache und unseren Begriff von ihr in einem lebendigen Wechselspiel aneinander prüft und so auch immerzu wandelt. Dialektik unterstellt, dass die Dinge keine voneinander getrennte Identität haben, die es nur genau genug zu erkennen gilt. Stattdessen bestehen sie vielmehr aus inneren und äußeren Gegensätzen, Verknotungen und Widersprüchen, so wie die Nacht nicht denkbar ist ohne ihren Konterpart, den Tag. Die Autoren versuchen dementsprechend die Aufklärung zu verstehen, indem sie eine nie endende Korrektur an der Beziehung zwischen unserem Verständnis von ihr und dabei an der Aufklärung selbst, vollziehen. Als diese Korrektur solle sich ihnen zufolge die Aufklärung auch selber verstehen, um sich von den ihr noch immanenten Rückständen der Unvernunft zu befreien. "`Aufklärung"' ist für Horkheimer und Adorno die \emph{Negation}, oder Aufhebung, des \emph{Mythos}. Dieser bildet ihren Nährboden, von dem sie sich frei zu machen versucht, in dem sie jedoch verwurzelt bleibt. Und so kommt es zur "`Verschlingung von Mythos und Aufklärung"'\footnote{Siehe \Cite{habermas}.}, in der Aufklärung selbst zum Mythos wird. Das bisherige Versäumnis der Aufklärung, sich entgültig vom Mythos zu trennen, führt sie, so die Autoren, in die unweigerliche Selbstzerstörung.

Ihren Ursprung findet die Aufklärung in dem Ziel, allen Aberglauben, allen Mythos aus der Welt zu verbannen und so den Glauben mit Wissen auszutauschen. Sie ist der Prozess der "`Bezwingung mythischer Gewalten"'\footnote{\Cite[Siehe][S. 408]{habermas}.}, der Entzauberung. Bezüglich des Ursprungs von Mythen zeigen die Autoren zu Beginn, dass diese aber immer schon aus aufklärerischem Handeln entstanden sind. Der Mensch, als stets wachsames Lebewesen, hat ein durch Angst hervorgerufenes Bedürfnis, die Welt um sich herum zu verstehen, um durch Vorausschau sich besser vor Gefahren schützen zu können. Durch die Weitergabe von Erfahrungen an die Nachkommen, konnte so ein Kollektivwissen aufgebaut werden, das sicherlich zum größten Teil empirischer Natur war. Sollte man sich erzählt haben, die Welt sei eine flache Scheibe, mit Land in der Mitte, das umgeben ist von Wasser, das möglicherweise an den Rändern hinunterfliesst, so war das weniger primitiver Aberglaube, als ein rein empirisches Forschungsergebnis, dessen Lücken mit aus dem Gegebenen extrapolierten, logisch erscheinenden Ergänzungen gefüllt werden. "`Der Mythos wollte berichten, nennen, den Ursprung sagen: damit aber darstellen, festhalten, erklären"'\footnote{\Cite[Siehe][S. 14]{dialektik-der-aufklaerung}.}, und so wurde aus den Berichten über die Generationen hinweg autoritäres "`Wissen"', das mit aus den Hinzudichtungen entstandenen Mystifikationen gespickt war. Die sozialen, wie verstandesmäßigen Folgen waren Unfreiheit und Barbarei, da den Menschen vorgegeben wurde, wie die Dinge sind, wie sie zu sein haben und daraus folgend, wie die Menschen sich dem anzupassen haben. Aus solchen Mythen nährt sich die Aufklärung und nahm sich dem Ziel an, den Menschen und seinen Verstand aus der Herrschaft zu befreien. Doch dieses Programm ist in seinem Kern das selbe wie jenes, mit dem der zu bekämpfende Mythos einst entstanden ist, dessen Inhalte die Aufklärung mit subjektiver Projektion verwechselt.\footnote{\Cite[Vgl.][S. 12]{dialektik-der-aufklaerung}.} Dieser Ursprung ist der Trieb zur Selbsterhaltung und damit auch die Angst vor der Beherrschung durch die Natur, der in einstigem Mythos noch durch eine ehrfürchtige Selbstanpassung entgegengewirkt wurde. In der Aufklärung aber durch Beherrschung der Natur. 

Die Autoren identifizieren die zweckgerichtete, \emph{instrumentelle} Vernunft als ein fundamentales Problem der Aufklärung. Für sie reicht Wissen, durch das der Mensch Natur bändigen und sich dieser besser anpassen kann, auch schon nicht mehr aus. Wissen muss nicht nur die Möglichkeit zu reaktivem Handeln erfüllen, sondern aktive Kreation ermöglichen, es muss ein Instrument sein. Es gilt daher: Ich kenne die Dinge, wenn ich sie manipulieren und machen kann - \emph{Wissen ist Macht}.\footnote{\Cite[Vgl.][S. 15]{dialektik-der-aufklaerung}.} Der aufklärende Mensch erhöht sich so zu einem göttlichen Gebieter über die Natur und die Welt.\footnote{\Cite[Vgl.][S. 15]{dialektik-der-aufklaerung}.} Um diese Machtposition zu begründen und zu behalten, wird allen Erkenntnissen, wie auch denen der Naturwissenschaften, immer weiter der Maßstab der Verwertbarkeit angelegt. So geht heutzutage sogar ein Thema wie die Abstrakte Algebra nur als lehrwürdig durch, weil es die \emph{abstract thinking skills} der Studierenden steigert, nicht aber zum Beispiel wegen ihres ästhetischen \emph{Eigenwertes}, denn Eigenwert gibt es für die instrumentelle Vernunft nicht. Alles in der Welt wird gemessen und damit messbar, alles Lebendige wird unlebendig und definierbar gemacht. Und so radikalisiert sich die Angst, die einst das Unlebendige zum Lebendigen emporhob, und ruft nach einem vor nichts halt machenden, alles durch Abstraktion komparabel oder gleich machenden Allwissen, für das es kein Unbekanntes und Unbenanntes mehr geben darf.\footnote{\Cite[Vgl.][S. 22 f.]{dialektik-der-aufklaerung}.} Der technische und gesellschaftliche Fortschritt, der kein Blick nach vorne, sondern nach hinten wurde, betrügt sich selbst, indem er das positivistische Welt\emph{bild} mit der Welt an sich verwechselt. In der mathematisierten Welt ist alles schon vorherbestimmt, das mir unbekannte hat bestimmt irgendwer schon erforscht, der mir sagt was es ist, und Probleme lasse ich lieber von Algorithmen lösen - Verstand und Fantasie verkümmern\footnote{\Cite[Vgl.][S. 42]{dialektik-der-aufklaerung}.} im Angesicht der objektivierten, determinierten Welt. Das konstante Differenzieren der Begriffe, das den Kern der Erkenntnis ausmachen soll, führen aber ebenso entweder zur Relativierung von Wahrheit und Moral oder zu Vorstellungen von der Erklärbarkeit von Moral anhand der Naturwissenschaften\footnote{Siehe hierzu beispielsweise: Harris, Sam (2010). \emph{The Moral Landscape: How Science Can Determine Human Values}. New York: Free Press.}, da diese ja die objektive Wahrheit beschreiben. Indem die Betrachtungen der Wissenschaften immer weiter auf teilnahmsloses Datensammeln herunter gebrochen werden, wird den Menschen nicht nur die Möglichkeit der qualitativen Erkenntnis genommen, sondern auch jede Hoffnung darauf, dass die Dinge anders sein könnten. In diesem Anspruch, die einzig mögliche Wahrheit im Handeln wie im Denken zu sein, zeigt sich, dass "`Aufklärung [...] totalitär wie nur irgendein System"'\footnote{\Cite[Siehe][S. 31]{dialektik-der-aufklaerung}.} ist. 

Die instrumentelle Vernunft objektiviert und objektifiziert nicht nur die Natur und sich selbst, sondern auch den Menschen selber, den sie auf einen quantifizierbaren, austauschbaren Haufen von Atomen reduziert. Dieser kann stets einer Maschine gleich optimiert werden, sei es bezüglich des Aussehens, der Fitness oder der inneren Ausgeglichenheit. Damit das Ziel der Freiheit erreicht werden kann, werden Werkzeuge wie Demokratie und Selbstkontrolle eingesetzt, in deren Namen das Individuum aber wieder angepasst, das heißt beherrscht und freiwillig fügsam gemacht wird. "`But lo! men have become the tools of their tools"'\footnote{\Cite[Siehe][S. 37]{walden}.}. Der Mensch wird also seiner selbst Herr, und nimmt damit nur die Position einer vermittelnden Instanz der Beherrschung durch die Natur ein. Weil das geltende Wissen, dem ein objektiver Wahrheitswert unterstellt wird, als eins mit den "`naturgegebenen"', da einzig als valide anerkannten Machtstrukturen der aufgeklärten Gesellschaft wahrgenommen wird,\footnote{\Cite[Vgl.][S. 411]{habermas}.} und deren tatsächliche Kontingenz selten Beachtung findet, wird das Individuum um das eigene, freie Denken beraubt, ohnmächtig in das Kollektiv gezwungen und somit um das eigentliche Versprechen seiner Befreiung betrogen. Im Namen der Freiheit wird also Unfreiheit geschaffen und aus dem ursprünglichen Aufbruch in die freie Gesellschaft, deren Verstand sich selbst und die Welt mit Klarheit erblicken sollte, wurde ein Weg in sich selbst widersprechende Verblendungszusammenhänge, die alles und jeden unterdrücken. So schlägt die erkenntnistheoretische Bewegung der "`Aufklärung in Mythologie zurück"',\footnote{\Cite[Siehe][S. 6]{dialektik-der-aufklaerung}.} und gleicht einem vor sich selbst die Augen verschließenden, ziellos durch die Gegend rasenden und alles in seinem Weg zertrampelnden Riesen\footnote{\Cite[Vgl.][S. 25]{fortschritt}.}. Die abscheulichen Massenmorde und totalitären Regime des 20. Jahrhunderts sind also keinesfalls von der Aufklärung getrennt, sondern die letzte Konsequenz eines unreflektierten Fortschrittswahns.

Diese "`Dialektik der Aufklärung"' ist die Kernbeobachtung des so betitelten Werkes. Sie ist allerdings keineswegs als eine destruktive Kritik zu verstehen, die zu einer Überwindung der Aufklärung hinführen will. Horkheimer und Adorno sind der Ü\-ber\-zeu\-gung, "`daß die Freiheit in der Gesellschaft vom aufklärenden Denken unabtrennbar ist"'\footnote{\Cite[Siehe][S. 3]{dialektik-der-aufklaerung}.} und dass die so "`an Aufklärung geübte Kritik [...] einen positiven Begriff von ihr vorbereiten [soll], der sie aus ihrer Verstrickung in blinder Herrschaft löst"'\footnote{\Cite[Siehe][S. 6]{dialektik-der-aufklaerung}.}. Die Aufklärung soll ihr mythologisches Erbe abgeben und über Mythoskritik hinausgehende Selbstreflexion betreiben. Erst die Aufklärung, die sich nicht selbst mythologisiert, so scheint die These gemeint zu sein, wird den Menschen in die Freiheit leiten können. Doch wie solch eine Aufklärung aussehen könnte, schildern die Autoren nicht. Max Horkheimer schrieb passend dazu später einmal: "`Ich bekenne mich zur kritischen Theorie; das heißt, ich kann sagen, was falsch ist, aber ich kann nicht definieren, was richtig ist"'\footnote{\Cite[Siehe][S. 150]{gesellschaft}.}. Es bleibt also den Lesenden überlassen, was aus alledem für sie persönlich und für die Gesellschaft als Ganzes folgt. 25 Jahre später schrieb Adorno in \emph{Minima Moralia} dazu, wie man sich überhaupt noch in der Gesellschaft verhalten könne: "`Das einzige, was sich verantworten läßt, ist, den ideologischen Mißbrauch der eigenen Existenz sich zu versagen und im übrigen privat so bescheiden, unscheinbar und unprätentiös sich zu benehmen, wie es längst nicht mehr die gute Erziehung, wohl aber die Scham darüber gebietet, daß einem in der Hölle noch die Luft zum Atmen bleibt"'\footnote{\Cite[Siehe][S. 24]{minima}.}. Solche Sätze zeugen eher von einer düsteren Hoffnungslosigkeit, und hinterlassen einen Nachgeschmack \`{a} la "`Wie man's macht, macht man's falsch"'. Wie kann der Mensch, der zurückhaltend lebt, denn überhaupt die von den Autoren so negativ beschriebenen Umstände ändern? Oder ist dies schon gar nicht mehr möglich? Heute scheinen die Massenmorde in den Konzentrationslagern und Gulags zwar vorerst überwunden, doch Krieg und Unterdrückung regieren weiterhin vielerorts. Immer weniger Menschen müssen hungern auf der Welt, Kindersterblichkeit sinkt von Jahr zu Jahr, doch Fremdenfeindlichkeit hat seither nicht nachgelassen. Die im Essay "`Kulturindustrie - Aufklärung als Massenbetrug"'\footnote{\Cite[Siehe][S. 128--176]{dialektik-der-aufklaerung}.} von Horkheimer und Adorno vorgetragene Kritik an der Verschmelzung der Kultur- und Herrschaftsinstitutionen wirkt heute im Rückblick noch sehr viel relevanter zu sein, was darauf hindeutet, dass der Massenbetrug über die damaligen Zustände hinausgewachsen ist.

\vspace{5mm}
\noindent\textbf{$(ii)$ Fortschritt}

\noindent Fortschritt ist die versuchte Grundbewegung der Aufklärung. Ganz abstrakt ist sie die Veränderung von einem Zustand weg, hin zu einem besseren. Das Neue muss \emph{besser} sein, denn ansonsten gäbe es keinen \emph{Fort}schritt, sondern lediglich eine Veränderung, oder gar einen \emph{Rück}schritt. Zwei ganz basale Fragen, die ein Fortschrittsvorhaben beantworten müsste, wären daher, wieso das Bestehende unzureichend sei und inwiefern das Neue besser wäre. Hieraus entstehen schon zwei grundlegende Probleme. Es ist nämlich fraglich, welches Maß für "`besser"' und "`schlechter"' angelegt wird und wo dieses Maß überhaupt angelegt werden soll, da das Neue während der Planung noch gar nicht eingetreten ist. Überhaupt kann erst von der Richtung einer Bewegung gesprochen werden, wenn man die Gesamtheit aller Zustände kennt. Etwas als "`fortschrittlich"' zu bezeichnen ist also tendenziell erstmal problematisch. 

Dies ändert jedoch nichts daran, dass man überall von "`Fortschritt"' hört, sei er intellektueller, technischer, medizinischer, gesellschaftlicher oder persönlicher Natur. "`Fortschrittliche Länder"' haben sich zur Aufklärung bekannt und damit Demokratie, Gleichberechtigung und Säkularismus zu ihren Grundfesten gemacht, da diese in der modernen, bürgerlichen Gesellschaft als "`fortschrittlich"' - lies: besser als ihre Alternativen - wahrgenommen werden. Die deutsche "`Leitkultur"' sei fortschrittlich, denn sie ist ja "`aufgeklärt"'. Überhaupt sind "`aufgeklärte"' Gesellschaften von Fortschritt ganz begeistert. Wenn er nicht schon allgegenwärtig ist, sollte er es auf jeden Fall schleunigst sein. Als "`technischen Fortschritt"' bezeichnet man heutzutage gemeinhin alles, was es vorher noch nicht gab. Besonders in der Medizin und der Telekommunikationstechnik kann Fortschritt gar nicht zu schnell kommen. Wenn er auf gesamtgesellschaftlicher Ebene zu besonders einschneidenden Veränderungen führt, wird zum Beispiel im Silicon Valley, der vermeintlichen Hochburg des Fortschritts, auch gern von \emph{disruption} gesprochen. In den Worten des in Frankfurt am Main, dem Ursprungsort der Kritischen Theorie, geborenen Milliardärs Peter Thiel in seiner "`Gründerbibel"' \emph{Zero to One}: "`Horizontal [...] progress means copying things that work — going from 1 to \emph{n}. [....] Vertical [...] progress means doing new things — going from 0 to 1"'\footnote{\Cite[Siehe][S. 6]{thiel}.}. Solcher \emph{Zero to One} Fortschritt, eine \emph{disruption}, ist für Thiel, wie für viele andere in der "`Gründerszene"', besonders wünschenswert. Dies führt aber zum Kernproblem des Fortschrittsglaubens, nämlich dass ein möglicher Fortschritt in einer gesellschaftlichen Sphäre einen Rückschritt in anderen bedeuten kann. Wenn beispielsweise \emph{startups} wie Uber sich zum Ziel nehmen, Märkte wie das Taxigeschäft zu "`disrupten"' und behaupten, dadurch einen Fortschritt für die Gesellschaft zu erzielen, verschleiern sie nur ihr zerstörerisches Handeln. Die Ineffizienzen der laufenden Taxiunternehmen reichen ihnen als Begründung aus, um weltweit tausenden Taxifahrer*innen ihre Lebensgrundlage zu nehmen, Fahrer*innen in ärmeren Ländern durch einen Autokauf in die Armut zu treiben und all das mit dem Endziel, in Zukunft sowieso nur selbstfahrende Autos zu verwenden. Hinter den philanthropisch anmutenden Weltverbesserungshymnen solcher \emph{startups} versteckt sich am Ende meistens nur die kapitalistische Profitgier und hinter dem vermeintlichen Fortschritt die rücksichtslose Zerstörung des Bestehenden. "`Der Rationalisierungsfortschritt wird mit einer zum äußersten gesteigerten Inhumanität erkauft"'\footnote{\Cite[Siehe][S. 389]{hetzel2011adorno}.}.

In der \emph{Dialektik der Aufklärung} nahmen sich Horkheimer und Adorno ein kritisches Denken vor, "`das auch vor dem Fortschritt nicht innehält"'\footnote{\Cite[Siehe][S. IX ("`Zur Neuausgabe"')]{dialektik-der-aufklaerung}.}. Dieses soll aber, wie schon bei der totalitär gewordenen Aufklärung, den Fortschritt nicht verneinen, sondern vielmehr zu einem besseren Selbstverständnis, und damit aus der Blindheit heraus führen. Für solch ein Vorhaben muss sich die Grundstruktur des Fortschrittsgedankens erst einmal klar gemacht werden. Wie die Aufklärung im allgemeinen, zieht Fortschritt seinen Wert nicht aus sich selbst, sondern aus dem Überwinden des "`Schlechten"' und triumphiert auf diese Weise in der Negation des Überwundenen.\footnote{\Cite[Vgl.][S. 638]{fortschritt}.} Er ist zugleich Werkzeug und Ziel der Aufklärung, und dadurch zu einem geradezu metaphysischen Imperativ geworden, einer \emph{creatio continua}. Da aber niemand so recht weiß, wohin eigentlich fortgeschritten wird, begnügt man sich damit, einfach vom Bestehenden hinfort zu schreiten, wodurch jeder Schritt zum Fortschritt wird. Dieser fluchtartigen Bewegung fehlt aber der bewusste Fokus auf das größere Ziel der Freiheit, und das größere Interesse am Überwinden einzelner Probleme führt somit zum "`rastlos mühselige[n] Fortschritt ins Unendliche"'\footnote{\Cite[Siehe][S. 32]{dialektik-der-aufklaerung}.}. So schaut die instrumentelle Vernunft, mit dem verschwommenen Bild der Freiheit im Hinterkopf, verächtlich nach hinten auf das Überwundene, mit dem Bewusstsein, dass der derzeitige Zustand ebenso defizitär ist und macht sich schnell daran, auch diesen hinter sich zu lassen. 

Die Grundbedürfnisse der Menschen haben sich seit Jahrtausenden wahrscheinlich nicht sonderlich verändert. Stattdessen aber die Verstrickung in Herrschaftsverhältnisse, die uns dazu bewegen, von unseren Grundbedürfnissen zu abstrahieren und uns den derzeitigen Gepflogenheiten anzupassen, zumal kaum ein Bedürfnis stärker ist, als Teil der Gruppe zu sein. Das Eigenheim, einst zum Schutz vor Niederfall und zum Aufbewahren von Essen und Werkzeugen ausgelegt, muss heute eine komfortable Ausstattung samt Deckenheizung und Bodenventilator haben, sowie als bürgerlicher Repräsentant für den eigenen Wohlstand herhalten. Vieler technischer Fortschritt befriedigt heute lediglich das universelle Bedürfnis nach Neuem, weder befreit er, noch erfüllt er ein Grundbedürfnis. Dies führt zum Ritual des jährlich neuen iPhones, wie auch zu den stets steigenden Scheidungszahlen. Die Frage danach, ob der Austausch denn überhaupt wirklich nötig sei, erübrigt sich sowieso, da das vom Smartphone- oder Liebesmarkt stets angebotene Neue die Hoffnung darauf aufleuchten lässt, dass die Dinge besser werden. Beim Fantasieren über das neue Handy oder den neuen Liebespartner ist weniger das ursprüngliche Bedürfnis, das heißt das mobile Surfen oder die tiefe Verbundenheit im Fokus, sondern der potenzielle Kontrast vom Neuen zum Alten, also die längere Akkulaufzeit oder das bessere Aussehen. "`Schon erscheinen die älteren Häuser rings um die Betonzentren als Slums, und die neuen Bungalows am Stadtrand verkünden schon wie die unsoliden Konstruktionen auf internationalen Messen das Lob des technischen Fortschritts und fordern dazu heraus, sie nach kurzfristigem Gebrauch wegzuwerfen wie Konservenbüchsen"'\footnote{\Cite[Siehe][S. 128]{dialektik-der-aufklaerung}.}. 

Und gerade diese "`Dekadenz ist der Nervenpunkt, wo die Dialektik des Fortschritts"'\footnote{\Cite[Siehe][S. 627]{fortschritt}.} zutage tritt, denn Dekadenz und Tabu bilden das Spannungsfeld, den der unreflektierte Fortschritt im Bewusstsein der Menschen aufspannt. Eine Verneinung des Fortschritts, geäußert im Tabu, ist gleichzeitig ein Ruf nach Unfreiheit und damit eine Verneinung der Aufklärung. Sie kommt einem Zetern über die ach so leidlichen Zustände gleich, das wieder nur die "`guten, alten"', unfreien Zeiten mythologisiert. Aus dem Tabu spricht aber nicht die Unvernunft, sondern die zweckrationale Vernunft selber, die in sich selbst die Dekadenz erkennt und sich folglich selbst einschränken, beziehungsweise beherrschen muss.\footnote{\Cite[Vgl.][S. 628]{fortschritt}.} So vereint die instrumentelle Vernunft den befreienden und beherrschenden Impuls, und der von ihr angestoßene Fortschritt bleibt in sich zerrissen. Kommt es zum Fortschritt in der Befreiung einer Gesellschaftssphäre, so wird eine sich da­ge­gen­stem­mende Bewegung hervorkommen und/oder die Befreiung in die Machtstrukturen aufgesogen und zu deren Zementierung missbraucht. Seitdem also zum Beispiel in vielen Teilen der Welt die sexuelle Befreiung fortgeschritten ist, so kam es zum einen als Gegenreaktion in einigen Gesellschaften zur Radikalisierung der sexuellen Repression, zum anderen wurde sie vom Kapitalismus aufgenommen, der sie als kaufbares Produkt verwertbar machte und so seine Herrschaftsposition verstärkte.\footnote{Auf diese Kräftigung der Gleichschaltung und Unterdrückung durch befreiende, oppositionelle Bewegungen wies zum Beispiel auch \citet{marcuse} eindringlich hin.} Ein allgemeines Wahlrecht erhöht die Anzahl der Menschen, die die gewählten Parteien legitimieren. Eine Frauenquote in Aufsichtsräten duldet patriarchale Unternehmensstrukturen. Breite Akzeptanz und Vermarktung radikaler Kunst verhindert jede Möglichkeit eines tatsächlich kritischen Moments im Kunstwerk. Je gleicher die Menschen, desto gleicher ihre Umstände, desto fester das Gesellschaftsgefüge, desto ohnmächtiger die Opposition. Auf diese Weise schlägt die Erhöhung der Toleranz für das Fremde in Bestätigung des Bestehenden um. So "`involviert Anpassung an die Macht des Fortschritts den Fortschritt der Macht [...]. Der Fluch des unaufhaltsamen Fortschritts ist die unaufhaltsame Regression"'\footnote{\Cite[Siehe][S. 42]{dialektik-der-aufklaerung}.}.

Es folgt daraus aber nicht, dass vor Missständen die Augen verschlossen werden sollen. Unterdrückende Machstrukturen müssen weiterhin, im eigentlichen Sinne der Aufklärung, aufgebrochen werden. Doch die blinde Zerstörung dieser Strukturen käme zum einen wieder nur der Naturbeherrschung gleich, zum anderen könnte sie den soeben beschriebenen Rückschritt in die Unfreiheit befeuern. Seit der Veröffentlichung der \emph{Dialektik der Aufklärung} hat sich hinsichtlich selbstreflexiver Aufklärung und Fortschritt aber scheinbar nicht allzu viel verändert. Auf der einen Seite wird stark für soziale Gleichberechtigung gekämpft, sodass "`diversity"' inzwischen überall zum \emph{buzzword} geworden ist. Apple, eines der größten Unternehmen der Welt, hat einen offen homosexuellen CEO, Googles Geschäftsführer kommt aus Indien und Deutschland wird von einer Frau regiert. An der Oberfläche hat es also Fortschritte in der Toleranz und Akzeptanz von Frauen und Minderheiten gegeben. Aber erst recht dadurch, dass sie in ihren Machtpositionen der hegemonialen Ideologie des Kapitalismus zustimmen, können Konzerne und Regierungen seither ihre Autorität und Unanfechtbarkeit nur immer weiter zementieren. Die bedingungslose Offenheit gegenüber Migranten hat zudem im Umkehrschluss die Tore für immer mehr Fremdenhass geöffnet. Und so ist es auch nicht weiter verwunderlich, dass der Milliardär Donald Trump zum amerikanischen Präsidenten, und die rechtspopulistische AfD 2017 mit 12,7\% zur drittstärksten Partei im deutschen Bundestag gewählt werden konnte.  

Es bleibt demnach weiterhin die Frage, wie denn Fortschritt ohne solche Regression aussehen könnte, beziehungsweise, was man als einzelne Person tun kann, um regressiven Fortschritt zu bekämpfen. Interessant zu betrachten ist hierzu beispielsweise Alex Karp, der Mitgründer und Geschäftsführer von Palantir, dem nach Uber und Airbnb höchstbewerteten \emph{startup} im Silicon Valley. Karp hat seinen Doktor am Frankfurter Institut für Sozialforschung gemacht\footnote{Siehe hierzu Karps Dissertation: Karp, Alexander C. (2002). \emph{Aggression in der Lebenswelt: Die Erweiterung des Parsonsschen Konzepts der Aggression durch die Beschreibung des Zusammenhangs von Jargon, Aggression und Kultur}. Frankfurt am Main: Johann Wolfgang-Goethe Universität.} und führt mit Palantir ein geheimniskrämerisches Unternehmen, das \emph{big-data}-Analysesoftware für Geheimdienste und Privatunternehmen anbietet. Dass eines Tages mächtige digitale Werkzeuge zur Überwachung und Zukunftsprognose existieren würden, war schon zu Beginn der digitalen Revolution vorauszusehen - dieser technische "`Fortschritt"' war gar nicht mehr abzuwenden. Was aber abzuwenden ist, ist dass solche Werkzeuge als dystopische Unterdrückungsapparate verwendet werden. Auch wenn es nicht notwendiger Weise so sein muss, so kann ich mir gut vorstellen, dass Alex Karp in seiner Zeit am IfS ein tiefes Bewusstsein für die Dialektik der Aufklärung entwickelt hat, und nun Einfluss und ethisches Bewusstsein zur Abwehr des Ärgsten verwendet. Adornos Vorschlag dafür, was selbstreflexiver Fortschritt sein könnte, ließe sich auch in diesem Sinne auslegen. Ihmzufolge müsste man "`Fortschritt in den Widerstand gegen die immerwährende Gefahr des Rückfalls"'\footnote{\Cite[Siehe][S. 638]{fortschritt}.} verwandeln, sodass er gerade die Gegenbewegung zum blinden Fortschreiten werde. 

Der regressive Fortschritt ist vor einiger Zeit ins Rollen gekommen und wird vorerst auch immer weiter rollen. Eine selbstreflexive Aufklärung könnte sich zum Ziel machen, mit ruhiger Hand und steter Abwägung, im Bewusstsein der Dialektik des Fortschritts, die Regression bremsen, sich weiter mit sich selbst kritisch auseinanderzusetzen und auf diese Weise an sich selbst genesen. Aufklärung kann immer noch nicht rein an sich selbst triumphieren, denn "`Freiheit ist selbst für die freiesten der bestehenden Gesellschaften erst noch herzustellen"'\footnote{\Cite[Siehe][S. 98]{marcuse}.}. Und so folgt wohl auch heute noch eine Diagnose, die am Ende der Kants nicht allzu unähnlich bleibt: "`Wenn denn nun gefragt wird: Leben wir jetzt in einem aufgeklärten Zeitalter? so ist die Antwort: Nein, aber wohl in einem Zeitalter der Aufklärung"'\footnote{\Cite[Siehe][S. 491]{kant}.}.

\vspace{5mm}

\noindent\textbf{$(iii)$ Konklusion}

\noindent In den vorangegangenen Seiten habe ich zum einen versucht, die Grundideen der Dialektik der Aufklärung darzulegen und zum anderen, den darin verfilzten Fortschrittsdrang, der zum modernen Mythos geworden ist, sowie seine Schattenseiten im Blick auf die heutige Zeit zu untersuchen. Ich habe gezeigt, wie Aufklärung durch die Vereinigung des Wahrheits- und Herrschaftsmonopols totalitär geworden ist und mit ihrer instrumentellen Vernunft den Menschen statt zu befreien wieder in Ketten legt. Daraus soll jedoch nicht folgen, dass die Aufklärung an sich problematisch sei, sondern nur noch nicht über sich selbst aufgeklärt wurde. Es folgte eine Beschreibung des modernen Fortschrittsglaubens, der durch seine Verknotung mit der instrumentellen Vernunft in den Menschen eine innere Entzweiung in Tabu und Dekadenz antreibt. 

Zusammenfassend lässt sich sagen, dass auch im 21. Jahrhundert die Aufklärung nicht sehr viel aufgeklärter ist als vorher. Deutschland mag seit dem Krieg wichtige Schritte hin zur Gleichberechtigung vorgenommen haben, doch stets im Rahmen der bestehenden, hegemonialen Ideologie, wodurch diese nur umso mehr als richtig, gut und unanfechtbar zementiert wurde. Eine wirklich aufgeklärte Aufklärung steht also noch immer aus.


%\noindent\textbf{$(iii)$ Die Dialektik der Kalifornische Ideologie}
%
%\begin{itemize}
%
%  \item Bsp: amerikanischen Institutionen wie Stanford im Bereich der Computer Science gang und gebe ist, wo die Institute sogar Namen	 von erfolgreichen Leute nhaben (Gates Computer Science Building, Hewlett Teaching Center) 
%  \item Die Herrschaftsstrukturen des kapitalistischen Marktes werden (möglicherweise murrend) hingenommen,
%  \item david byrne text
%  \item Facebook kämpft für connection, doch gleichzeitig sind sie die enabler für Trump und co
%  \item zu zeiten obamas sind politik und silicon valley immer weiter zusammengewachsen: Libertarianism 
%  \item problem: in amerika sind wegen satz 230 blabla die webseiten nicht für ihre inhalte verantwortlich
%  \item das größte taxi unternehmen besitzt keine taxis, das größte medienunternehmen produziert keine eigenen medien
%  \item die shopping daten werden zum politischen einfluss genutzt, so zeigt sich die vershcmelzung der medienindustrie, der wirtschaft und der politik, zu einem großen datenhaufen, der nutzbar wird, um leute in diesen drei, nicht wirklich mehr getrennten handlungsbereichen, zu beeinflussen
%  \item menschen werden von facebook emotional stark beeinflusst. es zeigt sich dass diese beeinflussung auf alle 3 bereiche einfluss haben müsste. bots sind durch viral machen von posts in der lage, die leute zu beeinflussen. 
%\end{itemize}
%

\newpage

\end{onehalfspace}
\nocite{*}
\printbibliography
\end{document}
