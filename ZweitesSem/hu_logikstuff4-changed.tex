
\usepackage{rotating}
\usepackage{qtree}
%\usepackage{KMcalc} %Lennard



%KM-Kalkül, geordnetes paar und gather-Umgebung für Formeln:
\newcommand{\UB}{$\und$B}
\newcommand{\UE}{$\und$E}
\newcommand{\OrE}{$\oder$E}
\newcommand{\OrB}{$\oder$B}
\newcommand{\EB}{$\ex$B}
\newcommand{\EE}{$\ex$E}
\newcommand{\paar}[1]{\left \langle #1 \right \rangle}
\newcommand{\gth}[2]{\begin{gather*} #1 \label{#2} \end{gather*}}
\newcommand{\gthd}[2]{\begin{gather} \begin{gathered}#1 \label{#2}\end{gathered}\end{gather}}

%objektsprachliche Symbole:
\newcommand{\und}{\wedge}
\newcommand{\oder}{\vee}
\newcommand{\then}{\rightarrow}
\newcommand{\eq}{\leftrightarrow}
\newcommand{\uu}{\cup}
\newcommand{\C}{\cap}
\newcommand{\TM}{\subseteq}
\newcommand{\all}{\forall}
\newcommand{\ex}{\exists}
% \neg gibt es schon

%Zahlbereichsymbole, Potenzmenge und Sprachnamen
\newcommand{\NN}{\mathbb{N}}
\newcommand{\QQ}{\mathbb{Q}}
\newcommand{\RR}{\mathbb{R}}
\newcommand{\Pp}{\mathcal{P}}
\newcommand{\Ll}{\ensuremath{\mathcal{L}}}
\newcommand{\pair}[1]{\langle #1 \rangle}
\newcommand{\quine}[1]{\ulcorner #1 \urcorner}
\newcommand{\mq}[1]{\mlq #1 \mrq} % steht für "math quotes"
\newcommand{\sse}{\subseteq}
\newcommand{\gesch}{\cap} % steht für "geschnitten"
\newcommand{\verin}{\cup} % Habe hier mit Absicht einen Buchstaben weggelassen, ähnlich wie \infty
\newcommand{\set}[1]{\{#1\}}
\newcommand{\LPL}{\ensuremath{\mathcal{L}_{PL}}}
\newcommand{\LAL}{{\ensuremath{\mathcal{L}_{AL}}}} % zusätzliche „{}“, um es in Subskripts nutzen zu können
\newcommand{\FmLAL}{\ensuremath{\mathcal{F}m_\LAL}} % zur besseren Lesbarkeit der AL-Definitionen
\newcommand{\SKLAL}{\ensuremath{\mathcal{SK}_\LAL}} % zur besseren Lesbarkeit der AL-Definitionen
\newcommand{\MIb}{{\pair{M, I}, \beta}} % zur Benutzung im Math-Mode


%metasprachliche Symbole:
\newcommand{\Land}{
    \raisebox{-0.12em}{ \begin{turn}{90} $\eqslantgtr$ \end{turn} }
}
\newcommand{\Lor}{
     \raisebox{-0.12em}{ \begin{turn}{90} $\eqslantless$ \end{turn} }
}
\newcommand{\Then}{
    \Rightarrow
}
\newcommand{\Gdw}{
    \Leftrightarrow
}
\newcommand{\Neg}{
    \neg\hspace*{-0.5em}\neg
}
\newcommand{\Forall}{
    \raisebox{0.18em}{\scriptsize{\textbackslash}}\hspace*{-0.175em}\forall
}
\newcommand{\Exists}{
    \exists\hspace*{-0.45em}\exists
}

\newcommand{\nehT}{\Leftarrow}

\DeclareMathSymbol{\mlq}{\mathord}{operators}{``}
\DeclareMathSymbol{\mrq}{\mathord}{operators}{`'}
\newcommand{\concat}{\raisebox{0.45em}{\scalebox{0.7}{$\smallfrown$}}}

%models gibt es schon
%\

\newcommand{\deduces}{\vdash}
\newcommand{\sidew}[1]{\begin{sideways} #1 \end{sideways}}
%Anführungszeichen

\newcommand{\qleft}{\ulcorner}
\newcommand{\qright}{\urcorner}
\newcommand{\qc}[1]{\qleft #1 \qright}
\newcommand{\anf}[1]{`#1'}
\newcommand{\manf}[1]{\text{`}#1\text{}}
\newcommand{\tmanf}[1]{\text{`#1'}}
%\renewcommand{\models}{\vDash}
\newcommand{\Anf}[1]{„#1“}
\newcommand{\cn}{\/^{\smallfrown}}
%Kopf

