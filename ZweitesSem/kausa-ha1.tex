\documentclass[a4paper]{article}
\usepackage{graphicx}
\usepackage{fullpage}
%\usepackage{parskip}
\usepackage{color}
\usepackage[ngerman]{babel}
\usepackage{hyperref}
\usepackage{calc} 
\usepackage{enumitem}
\usepackage{titlesec}
\usepackage{bussproofs}
\usepackage[export]{adjustbox}
%\pagestyle{headings}

\titleformat{name=\section,numberless}
  {\normalfont\Large\bfseries}
  {}
  {0pt}
  {}
\date{\vspace{-3ex}}
\begin{document}

\title{
    \vspace{-30pt}
	\includegraphics*[width=0.1\textwidth,left]{ErstesSem/images/hu_logo2.png}\\
	\vspace{-10pt}
	Kausales Schlie"sen}
\author{Lennard Wolf\\
        \small{\href{mailto:lennard.wolf@student.hu-berlin.de}{lennard.wolf@student.hu-berlin.de}}}
\maketitle
\vspace{-4pt}

\section*{Aufgabe 1}
\large

\textbf{Aufgabenstellung}\\ Man betrachte folgendes Beispiel: Nach einem gemeinsamen Ausflug, bei dem sie sich wegen des schlechten Wetters stark verk"uhlen, erkranken die Zwillinge Anna und Hannah an einem grippalen Infekt mit Fieber. Beide h"uten brav das Bett und nehmen ein acetylsalicyls"aurehaltiges Medikament in gleicher Dosierung und H"aufigkeit ein. W"ahrend es Anna bald besser geht, tritt bei Hannah infolgedessen "Ubelkeit auf - eine Nebenwirkung, die laut Hersteller bei 1 bis 10 von 100 Anwendern auftrittt.

Der Text vertritt die Auffassung, dass immer wenn Ereignisse kausal erzeugt sind, das Determinismusprinzip gilt, das kurz als "`Gleiche Ursachen, gleiche Wirkungen"' gefasst wird (vgl. Baumgartner u. Gra"shoff 2004-3 S. 70).

Inwiefern eignet sich das Beispiel, die G"ultigkeit des Determinismusprinzips zu kritisieren? Stellen Sie dazu das Determinismusprinzip dar und begr"unden Sie unter Bezugnahme darauf Ihre Entscheidung. Umfang: 1-2 Seiten.\newline

\noindent \textbf{Antwort}\\ Das Determinismusprinzip besagt, dass bei gleichen/"aquivalenten Ursachentypen die gleichen/"aquivalenten Wirkungstypen instantiiert werden. Wenn also in einer Situation $A$ Ereignis-Instanzen der Typen $X, Y$ und $Z$ vorkommen und in einer Situation $B$ ebenso, wenn auch andere, Ereignis-Instanzen der Typen $X, Y$ und $Z$ vorkommen, so folgt aus dem Determinismusprinzip, dass die aus $A$ instantiierten Wirkungstypen mit den aus $B$ instantiierten Wirkungstypen "aquivalent sind.

Im Besonderen w"urde das f"ur die in der Aufgabenstellung beschriebene Situation folgendes bedeuten: Wenn davon auszugehen w"are, dass in den Situationen von Anna und Hannah alle Ursachentypen "aquivalent waren, dann w"are dem Determinismusprinzip zufolge nach der Einnahme des Aspirins entweder bei beiden M"adchen "Ubelkeit aufgetreten, oder bei keinem der beiden. Jedoch ist nur Hannah "ubel geworden und so l"asst sich der oben genannten Annahme zufolge erst einmal behaupten, dass das Determinismusprinzip daher nicht g"ultig sein kann.

Sollte dies der Fall sein so m"usste man sich fragen, was denn aber "`schief gelaufen"' sein muss, dass Hannah "ubel geworden ist, beziehungsweise warum Anna gerade \emph{nicht} "ubel geworden ist. Da aber das Determinismusprinzip in dieser jetzigen "`Ursachensuche"' als falsch anzusehen ist, lie"se sich aber scheinbar kein zufriedenstellendes Ergebnis finden, da sich Ursache-Wirkung-Verbindungen nicht generalisieren lie"sen, und man somit in Erkl"arungsnot bei jeder gegebenen Antwort k"ame. Dem k"onnte man aber entgegnen, dass es doch m"oglich sei, dass alle kausalen Zusammenh"ange \emph{zuvorderst} durch Wahrscheinlichkeiten charakterisiert seien, und daher eben aufgrund des "`W"urfelspiels"' des Universums Hannah erkrankt ist, und Anna nicht.

Solch eine Theorie w"are sicherlich vorstellbar, und auch vereinbar mit Theorien "uber die stochastische Natur der Quantenmechanik. Doch ohne ein Determinismusprinzip w"are Nachdenken "uber Kausalzusammenh"ange extrem verkompliziert -- was nat"urlich in keinster Weise ausschlie"sendes Argument ist. Zudem scheint das Determinismusprinzip durch die Natur der Reproduzierbarkeit von physikalischen Ph"anomenen in streng parametrisierten Experimenten, meines Wissens nach zumindest in Kontexten au"serhalb der Quantenmechanik, "`erwiesen"' zu sein. 

Unter der Voraussetzung des Prinzips m"usste in der Situation um Hannah und Anna folglich folgendes der Fall gewesen sein: Da die Wirkungen unterschiedlich waren, m"ussen die Ursachen unterschiedlich gewesen sein. Dies ist auch sofort nachvollziehbar, denn nur weil die M"adchen Zwillinge sind, hei"st das noch lange nicht, dass ihre K"orper exakt gleich funktionieren. So scheint es, dass man wahrscheinlich genauer auf die k"orperlichen Reaktionen der beiden Zwillinge schauen muss, um dann zu erkennen, dass die M"adchen einfach, und damit die Ausgangssituationen, zu unterschiedlich waren, um von gleichen/"aquivalenten Ursachentypen sprechen zu k"onnen.

Trotzdem stellt sich allgemein die Frage, \emph{wie} genau man eigentlich nachschauen muss, um eine ausreichende "`Aufl"osung"' der Betrachtung nach Ursachen-/Wirkungstypen zu haben. Und k"onnte sich nicht herausstellen, dass diese "`Aufl"osung"' am Ende nur die h"ochstm"ogliche sein k"onnte, und es so etwas wie "`Ursachen- und Wirkungstypen"' nur in der Theorie gibt?


\end{document}
