\documentclass[a4paper]{article}
\usepackage{graphicx}
\usepackage{fullpage}
%\usepackage{parskip}
\usepackage{color}
\usepackage[ngerman]{babel}
\usepackage{hyperref}
\usepackage{calc} 
\usepackage{enumitem}
\usepackage{titlesec}
\usepackage{bussproofs}
\usepackage[export]{adjustbox}
%\pagestyle{headings}

\titleformat{name=\section,numberless}
  {\normalfont\Large\bfseries}
  {}
  {0pt}
  {}
\date{\vspace{-3ex}}
\begin{document}

\title{
    \vspace{-30pt}
	\includegraphics*[width=0.1\textwidth,right]{ErstesSem/images/hu_logo2.png}\\
	\vspace{-10pt}
	Aristoteles: Nikomachische Ethik}
\author{Lennard Wolf\\
        \small{\href{mailto:lennard.wolf@student.hu-berlin.de}{lennard.wolf@student.hu-berlin.de}}}
\maketitle
\vspace{-4pt}

\section*{Essay -- Struktur}
\large

\textbf{(i) Einleitung} 

\begin{itemize}
  \item (Kurze) Darstellung des Kontextes des thematisierten Problems innerhalb der N.E.\footnote{W"are es angebracht meine per"onliche Motivation zur Befassung mit dem Thema zu nennen?}
  \item $\rightarrow$ Wodurch kommt das Problem auf?
  \item $\rightarrow$ Welche Bedeutung hat die L"osung des Problems f"ur die Gesamttheorie
  \item Beschreibung des Vorgehens
\end{itemize}

\noindent \textbf{(ii) Darstellung des Problems}

\begin{itemize}
  \item Erl"auterung anhand eines eigenen Beispiels
\end{itemize}


\noindent \textbf{(iii) Rekonstruktion der L"osung in der N.E.}

$\rightarrow$ Wertfrei!

Kurze Darstellung von verwandten Stellen in der Metaphysik (M I.1, M IX.8) zur St"utzung des Arguments, Vergleich mit Ansatz in bez"uglich des Leierspielens\newline

\noindent \textbf{(iv) Diskussion} 

\begin{itemize}
  \item Problematisierung der Unklarheit der Originaltextstelle
  \item Leuchtet die L"osung ein? Oder weicht sie dem Problem nur einfach aus? 
  \item Falls Problematisch: Andere L"osungsans"atze? Scheinproblem? Gesamttheorie defizit"ar?
  \item Falls nachvollziehbar: Was lernen wir durch sie "uber Tugend? 
  \item $\rightarrow$ Kunstfertigkeit/Wissenschaft?
\end{itemize}


\noindent \textbf{(v) Ausblick} 

\begin{itemize}
  \item Zusammenfassung der Erkenntnisse
  \item Einordnung in das Leben der Lesenden (im Sinne der N.E.)
\end{itemize}



\end{document}
