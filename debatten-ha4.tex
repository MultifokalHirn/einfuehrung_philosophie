\documentclass[a4paper, 12pt]{article}
%\usepackage{CJKutf8} % japanese
\usepackage{graphicx}
\usepackage{hyperref}
\usepackage{fullpage}
%\usepackage{parskip}
\usepackage{color}
\usepackage[ngerman]{babel}
\usepackage{hyperref}
\usepackage{calc} 
\usepackage{enumitem}
\usepackage[utf8]{inputenc}
\usepackage{titlesec}
%\pagestyle{headings}
\usepackage{setspace} %halbzeilig
\usepackage[style=authoryear-ibid,natbib=true]{biblatex}
\usepackage[hang]{footmisc}
\setlength{\footnotemargin}{-0.8em}
%\bibliographystyle{natdin}
\addbibresource{debatten-ha4.bib}
\DeclareDatamodelEntrytypes{standard}
\DeclareDatamodelEntryfields[standard]{type,number}
\DeclareBibliographyDriver{standard}{%
  \usebibmacro{bibindex}%
  \usebibmacro{begentry}%
  \usebibmacro{author}%
  \setunit{\labelnamepunct}\newblock
  \usebibmacro{title}%
  \newunit\newblock
  \printfield{number}%
  \setunit{\addspace}\newblock
  \printfield[parens]{type}%
  \newunit\newblock
  \usebibmacro{location+date}%
  \newunit\newblock
  \iftoggle{bbx:url}
    {\usebibmacro{url+urldate}}
    {}%
  \newunit\newblock
  \usebibmacro{addendum+pubstate}%
  \setunit{\bibpagerefpunct}\newblock
  \usebibmacro{pageref}%
  \newunit\newblock
  \usebibmacro{related}%
  \usebibmacro{finentry}}

%\titleformat{name=\section,numberless}
%  {\normalfont\Large\bfseries}
%  {}
%  {0pt}
%  {}
\date{\vspace{-3ex}}
\begin{document}

\title{\vspace{5ex}
	\includegraphics*[bb=0 0 720 200, width=0.72\textwidth]{ErstesSem/images/hu_logo.png}\\
	\vspace{30pt}
	\scshape\LARGE{Zusammenfassung IV}\\\Large{The End of Elsewhere:\\Writing Modernity Now}\vspace{20pt}}
	


\author{Regionalwissenschaftliche Debatten\\
	\vspace{7pt}
          Dozent: Prof. Dr. phil. Vincent Houben\\\vspace{4pt}Lennard Wolf\\
        \small{Matrikelnummer: 583052}\\
        \small{E-Mail: lennard.wolf@hu-berlin.de}}

        %\href{mailto:lennard.wolf@student.hu-berlin.de}{lennard.wolf@student.hu-berlin.de}}}      

\maketitle

\vspace{\fill}

\begin{minipage}[]{0.92\textwidth}
    \centering
    \onehalfspacing
    \large   
    15. Januar 2018\\
    Wintersemester 17/18

    \vspace{-20mm} 
\end{minipage}%
\thispagestyle{empty}
\newpage
%\clearpage
%\thispagestyle{empty}
%\tableofcontents
%\newpage
\setcounter{page}{1}

\begin{onehalfspace} 

%\noindent\textbf{Zusammenfassung}

% Überlegen Sie sich Zwischenüberschriften zu den einzelnen Abschnitten des Textes!

% Erarbeiten Sie sich aus jedem Abschnitt zwei Kernaussagen!

% Versuchen Sie, eine Definition des Konzeptes "Orientalismus" zu formulieren!

\noindent 
\emph{The End of Elsewhere: Writing Modernity Now} ist ein 2011 erschienener Essay, in dem Carol Gluck anhand eines Beispiels zeigt, wie durch globalere Geschichtsschreibung der Begriff der Moderne globaler gefasst werden kann.

Die Moderne von heute äußert sich global betrachtet durch Nationalstaatlichkeit, Urbanisierung und politische Partizipation der Bürger*innen. Sie erlaubt keine Alternative zu sich, sodass Entwicklung, Emanzipation und Menschenrechte zum globalen Imperativ werden. Regionen, die nicht Schritt halten, werden verfemt. Die Moderne ist ein Phänomen, das auch besonders Reformer und Revolutionäre anzieht, und, in ihrer Verflechtung von Theorie und Geschichte, für diese meist sowohl der gedankliche Ausgangspunkt, als auch in `verbesserter Form' das universelle Ziel ist. Da sie nicht nur in Europa aufgetaucht ist, besteht Theoriebildung zur Moderne nicht nur bei westlichen Klassikern, sondern auch in nichtwestlichen historischen Abhandlungen, denen zwangsläufig solche Theorien immanent sind. Aus diesem Grund erkundet die Autorin die Meiji-Zeit (1868-1912) in Japan, um den Blick auf Modernisierungsprozesse zu weiten.

Die Menschen in Japan zur Meiji-Zeit nahmen die Entwicklung zur `Zivilisation' als schnelle Umwälzung von allem wahr. Doch die Modernisierung ist nicht `plötzlich einfach so' gekommen, sondern durch ein Zusammenspiel historischer Kontingenzen, die zu einer Zentralstaatlichkeit führten. Parallel dazu machten Regionen in Europa ähnliche Umstrukturierungen zu Nationalstaaten durch. Daher kann die Modernisierung der Welt zeitlich gesehen nicht als von Europa ausgehend betrachtet werden. Trotzdem hielt eine solche Sicht auch in Japan ein, weshalb der antimoderne Nationalismus der 1930er dort zwangsläufig gleich ein antiwestlicher war.

Der Protagonist der Moderne war für Japan immer der Staat, der durch Reformen die Leute bildete und ihnen eine gute Lebensgrundlage gab, doch zum Beispiel das vom Staat aufoktroyierte Schulsystem musste am Ende von den Eliten finanziert, und von den Bürger*innen selbst anerkannt werden. In solchen Beispielen zeigt sich, dass der Staat nicht der einzige Protagonist der Modernisierung war.

Modernisierung ist kein gerader Weg in eine fixe Richtung, sondern ein improvisierender \emph{trial and error} Prozess. Viele Menschen wurden daher zurückgelassen, politische Erdbeben führten zu immer neuen Taktiken und Institutionen waren meist nicht auf einander abgestimmt. Der Weg, den die Moderne in Japan ging war folglich nicht vorbestimmt oder unvermeidlich, aber jede einzelne Schritt hatte seinen Einfluss auf den nächsten.

Gluck schlägt den Begriff "`\emph{blended modernity}"' zur Beschreibung der Moderne als Ansammlung global unterschiedlich aussehender, improvisierter Prozesse vor. Dass Kremation die häufigste Bestattungsform in Japan wurde, ist ein Beispiel für solch improvisierten Prozess. Kremation ist weder eine neue noch alte, weder buddhistische noch westliche Tradition, sondern ein Phänomen, das aus einem \emph{blend} (Mix) von verschiedenen Umständen hervorgegangen ist. Die Welt ist nun voll von \emph{blended modernities}, die alle keine schlechten Imitate einer "`echten Moderne"' sind, sondern alle Hybride, bedingt durch historische und kulturelle Kontingenzen.

Abschließend lädt die Autorin Historiker*innen dazu ein, ihrem Beispiel folgend Theorien über die Moderne durch einen geweiteten geografischen Blick aufzustellen, um sie wirklich in globalen Begriffen begreifen zu können.

%Sie bleibt ja aber offensichtlich vollkommen in der Moderne hängen (was ja nicht schlimm ist)
%Moderne ist ein edlritch das über sich selbst nachdenkt, aber ultimative angst vor seinem tod hat


\end{onehalfspace}
\nocite{*}
%\bibliography{merleau-ponty-essay}
\printbibliography
\end{document}
