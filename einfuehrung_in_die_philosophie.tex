\documentclass[]{scrartcl}
\usepackage{graphicx}
\usepackage{color}
\usepackage{german}
\usepackage{hyperref}
%\pagestyle{headings}

\begin{document}

\title{
	\includegraphics*[width=0.75\textwidth]{images/hu_logo.png}\\
	\vspace{24pt}
	Einf"uhrung in die Philosophie}
\subtitle{Vorlesung WS 16/17\\
          Prof. Dr. Thomas Schmidt\\
          Philosophisches Institut I \\ 
          Humboldt Universit"at zu Berlin}
\author{Lennard Wolf\\
        \href{mailto:lennard.wolf@student.hu-berlin.de}{lennard.wolf@student.hu-berlin.de}}
\maketitle
\begin{abstract}

Die Vorlesung bietet eine Einf"uhrung in f"ur die Philosophie charakteristische Fragestellungen, Denkweisen und Theorieans"atze sowie in Begriffe, deren Kenntnis zum philosophischen Handwerkszeug geh"ort. Im Vordergrund steht die exemplarische Auseinandersetzung mit ausgew"ahlten philosophischen Sachproblemen aus verschiedenen Teilgebieten der Philosophie.

\end{abstract}
\newpage

\tableofcontents
\listoffigures
\newpage


\section{Einf"uhrung / Logische Prop"adeutik\\(20.10.16)}

\subsection{Einf"uhrung}
\subsubsection{Was ist Philosophie?}

\begin{itemize}
  \item Professoren sind sehr uneinig "uber diese Frage
  \item Kein \emph{Laberfach}, sondern ''Verstehen, was die Welt im Inneren zusammenh"alt'' (Faust)
  \item Sie ist der Ursprung aller Einzelwissenschaften -- bleibt nichts "ubrig mehr nach der Zergliederung?
  \item Wissenschaft ist h"aufig nur \emph{Hingucken}, Philosophie ist zus"atzlich \emph{Nachdenken}\\(''I sit down and think.'' $\rightarrow$ \emph{armchair philosophy})
  \item \emph{Willensfreiheit vs. Determinismus} ist klassisch philosophische Fragestellung
\end{itemize}

\subsubsection{Willensfreiheit vs. Determinismus}

\begin{itemize}
  \item Empirie zeigt, dass alles deterministisch verl"auft
  \item Ob Willensfreiheit mit diesem Determinismus vereinbar ist ist keine empirische Frage (Status quo der Philosophen ist, dass es vereinbar ist)
\end{itemize}

\subsubsection{Philosophische Fragestellungen / Teilgebiete}

\begin{itemize}
  \item \emph{Ontologie}: Was ist die Struktur der Realit"at / Was gibt es?
  \item \emph{Erkenntnistheorie}: Was ist Wissen? Was f"ur Arten dessen gibt es?
  \item \emph{Metaphysik}: Was ist Wahrheit? (???)
  \item \emph{Normative / Werttheoretische Fragen}: Was ist ein gutes Leben? Wie soll man handeln?
\end{itemize}

\subsubsection{Philosophie und ihre Geschichte}

\begin{itemize}
  \item Anders als bei anderen Wissenschaften: Klassische philosophische Theorien sind auch heute noch ernst zu nehmen
  \item Historisches Bewusstsein ist in der Philosophie essenziell (\emph{elder contemporary view}, "altere Zeitgenossen wie Aristoteles sollte man sch"atzen k"onnen)
  \item Forschung in der Geschichte der Philosophie ist aber eher unerheblich, es geht vielmehr um tats"achliche Sachfragen
\end{itemize}

\subsection{Konzeption der Vorlesung}

\subsubsection{Was die Vorlesung bietet}

\begin{itemize}
  \item Eine Art philosophisches Glossar
  \item Teilgebiete werden durch Problemstellungen n"aher gebracht
\end{itemize}


\subsubsection{Was die Vorlesung nicht bietet}

\begin{itemize}
  \item ?????
\end{itemize}

\subsubsection{Was die Vorlesung von uns erwartet}

\begin{itemize}
  \item \emph{Meinung haben ist leicht, Meinung begr"unden ist schwer!}
  \item Aktives Nachdenken in der Vorlesung!
\end{itemize}

\subsection{Logische Prop"adeutik: Was ist ein gutes Argument?}

\subsubsection{Was sind Argumente?}

Begr"undungen f"ur Meinungen / Sind Antworten auf \emph{Warum}-Fragen zu Meinungen

\subsubsection{Begr"undungen vs. Erkl"arungen}

Begr"undungen sind gefragt, wenn die Konklusion fraglich ist, w"ahrend bei Erkl"arungen der Wahrhaftigkeit der Konklusion außer Frage steht


\subsubsection{Struktur von Argumenten}
\textbf{Normalform:}  Pr"amissen (P1 -- Pn) $\rightarrow$ Konklusion (K)


\subsubsection{Was ist ein gutes Argument?}

???
\newpage

\section{Metaphysik I: Was gibt es?\\(27.10.16)}


\section{Metaphysik II: Was gibt es?\\(27.10.16)}

%\begin{figure}[h]
%	\centering
%	\includegraphics[width=0.5\textwidth]{images/template.png}
%	\caption{Template Bild}
%	\label{fig:template}
%\end{figure}



\newpage
\section{"Uber den Professor}
Prof. Mustermann ist..


\end{document}
