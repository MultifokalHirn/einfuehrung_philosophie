\documentclass[a4paper, 12pt]{article}
%\usepackage{CJKutf8} % japanese
\usepackage{graphicx}
\usepackage{hyperref}
\usepackage{fullpage}
%\usepackage{parskip}
\usepackage{color}
\usepackage[ngerman]{babel}
\usepackage{hyperref}
\usepackage{calc} 
\usepackage{enumitem}
\usepackage[utf8]{inputenc}
\usepackage{titlesec}
%\pagestyle{headings}
\usepackage{setspace} %halbzeilig
\usepackage[style=authoryear-ibid,natbib=true]{biblatex}
\usepackage[hang]{footmisc}
\setlength{\footnotemargin}{-0.8em}
%\bibliographystyle{natdin}
\addbibresource{debatten-ha3.bib}
\DeclareDatamodelEntrytypes{standard}
\DeclareDatamodelEntryfields[standard]{type,number}
\DeclareBibliographyDriver{standard}{%
  \usebibmacro{bibindex}%
  \usebibmacro{begentry}%
  \usebibmacro{author}%
  \setunit{\labelnamepunct}\newblock
  \usebibmacro{title}%
  \newunit\newblock
  \printfield{number}%
  \setunit{\addspace}\newblock
  \printfield[parens]{type}%
  \newunit\newblock
  \usebibmacro{location+date}%
  \newunit\newblock
  \iftoggle{bbx:url}
    {\usebibmacro{url+urldate}}
    {}%
  \newunit\newblock
  \usebibmacro{addendum+pubstate}%
  \setunit{\bibpagerefpunct}\newblock
  \usebibmacro{pageref}%
  \newunit\newblock
  \usebibmacro{related}%
  \usebibmacro{finentry}}

%\titleformat{name=\section,numberless}
%  {\normalfont\Large\bfseries}
%  {}
%  {0pt}
%  {}
\date{\vspace{-3ex}}
\begin{document}

\title{\vspace{5ex}
	\includegraphics*[bb=0 0 720 200, width=0.72\textwidth]{ErstesSem/images/hu_logo.png}\\
	\vspace{30pt}
	\scshape\LARGE{Zusammenfassung III}\\\Large{The Politics of Knowledge:\\Or, How to Stop Being Eurocentric}\vspace{20pt}}
	


\author{Regionalwissenschaftliche Debatten\\
	\vspace{7pt}
          Dozent: Prof. Dr. phil. Vincent Houben\\\vspace{4pt}Lennard Wolf\\
        \small{Matrikelnummer: 583052}\\
        \small{E-Mail: lennard.wolf@hu-berlin.de}}

        %\href{mailto:lennard.wolf@student.hu-berlin.de}{lennard.wolf@student.hu-berlin.de}}}      

\maketitle

\vspace{\fill}

\begin{minipage}[]{0.92\textwidth}
    \centering
    \onehalfspacing
    \large   
    08. Januar 2018\\
    Wintersemester 17/18

    \vspace{-20mm} 
\end{minipage}%
\thispagestyle{empty}
\newpage
%\clearpage
%\thispagestyle{empty}
%\tableofcontents
%\newpage
\setcounter{page}{1}

\begin{onehalfspace} 

%\noindent\textbf{Zusammenfassung}

% Überlegen Sie sich Zwischenüberschriften zu den einzelnen Abschnitten des Textes!

% Erarbeiten Sie sich aus jedem Abschnitt zwei Kernaussagen!

% Versuchen Sie, eine Definition des Konzeptes "Orientalismus" zu formulieren!

\noindent 
\emph{The Politics of Knowledge: Or, How to Stop Being Eurocentric} ist ein 2014 erschienener Essay, in dem sich Sanjay Seth mit drei unterschiedlichen Argumenten gegen Eurozentrismus beschäftigt. Nach Ansicht der Eurozentristen hat die Moderne und ihre Verbreitung ihre Wurzeln in Europa und dem Protestantismus. Daher sei Eurozentrismus keine Voreingenommenheit, sondern schlicht auf historischen Wahrheiten basierend. Dem widersprechen jedoch verschiedene Stimmen, auf unterschiedliche Weise. Zum einen sei Kapitalismus, als Basis für Europas Fortschrittlichkeit, keine europäische Kreation, sondern ein Geflecht aus weltweiten Beziehungen, die weit vor der Aufklärung entstanden sind. Andere zeigen auf, dass Asien und Europa bis ins späte 18. Jahrhundert ökonomisch vergleichbar waren. Als weniger empirische Gegendarstellung zum Eurozentrismus meinen postkoloniale Theorien, dass die Mittel zur Analyse der nichtwestlichen Welt immer schon eurozentristisch gefärbt sind, und sie damit den “Rest der Welt” nicht bloß versuchen objektiv zu verstehen, sondern ihn gleichzeitig überhaupt erst als solchen konstituieren. Der Autor geht diese drei Anfechtungen der Reihe nach durch. 

Kein europäischer Geist, sondern die durch die Entdeckung Amerikas sowie Kolonisierung und Eroberungen gebrachten Reichtümer und so gebrachten Mittel waren es, in ein Zeitalter des sich stets beschleunigenden Fortschritts einzutreten. Der Niedergang der bisher fortschrittlicheren und größeren asiatischen Wirtschaft fiel mit dem europäischen Aufschwung zusammen, dessen Bewegung in Richtung Spitze der Weltwirtschaft folglich eher kontingenter Natur war.

Als nächstes geht es um die sehr ähnlichen Umstände Asiens mit Europa um Ende des 18. Jahrhunderts herum. Starker Bevölkerungszuwachs und die Begrenztheit von Raum und Ressourcen stellten beide Regionen vor ähnliche Probleme, und so existiert die Ansicht, dass es gerade die Kolonien und gut gelegene Kohlebestände für das Heizen und den Betrieb von Maschinen waren, die England zum plötzlichen Wachstum verhalfen. Die von der Bevölkerung benötigten Ressourcenmengen hätte England nie allein bereitstellen können, aber konnten durch die Kolonien von außen herangebracht werden. So waren es eher “glückliche Umstände”, die Englands und darauf folgend Europas Aufstieg erklären, nicht die Gesinnung oder gar eine  intellektuelle Fortgeschrittenheit. 

Postkoloniale Theorie postuliert, dass nicht bloß die Erkenntnisse der Sozialwissenschaften eurozentrisch gefärbt sind, sondern ihre gesamte Herangehensweise, Grundvorstellungen und Urannahmen. Konzepte wie Kapital, Staat oder Gesellschaft sind möglicherweise nicht ohne Weiteres auf außereuropäische Geschichte und Regionen über\-tragbar, und folglich nicht geeignet für "`objektive"' Beschreibungen. Bei der Verwendung solcher Konzepte müssen andere Systeme und Denkweisen angepasst und übersetzt werden, wobei zwangsläufig vieles verloren gehen wird. Selbst Ideen wie die Unterscheidung zwischen Realität und Representation sind nicht “natürlich”, sondern kulturhistorisch bedingt und kontingent. Der Autor betont, dass aus solchen Gedanken kein Relativismus folge, sondern vielmehr nur ein Bewusstsein geschaffen werden muss für die Ursprünge, Färbungen und Grenzen all unserer Begriffe und Kategorien.

Abschließend folgt noch eine Besprechung des Gedankens, dass aus der Modernisierung der gesamten Welt überall die Akzeptanz westlicher Kategorien Einhalt nehmen und folglich postkoloniale Theorie irrelevant werde. Der Autor hält dagegen, dass es nicht nur einen Weg in die Moderne gebe, und daher aus Modernisierung keine Homogenisierung folgen müsse. Zudem könne der konstituierende Einfluss, sowie die "`Korrektheit"' von sozialwissenschaftlichen Betrachtungen immer nur unvollständig und perspektivisch bleiben, weshalb die Annahme der ersten beiden Argumente, dass die Sozialwissenschaften "`korrigierbar"' seien, schlussendlich nicht haltbar ist.


\end{onehalfspace}
\nocite{*}
%\bibliography{merleau-ponty-essay}
\printbibliography
\end{document}
