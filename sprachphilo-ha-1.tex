\documentclass[a4paper, emulatestandardclasses, 12pt]{scrartcl}
\usepackage{graphicx}
\usepackage{fullpage}
%\usepackage{parskip}
\usepackage{color}
\usepackage[ngerman]{babel}
\usepackage{hyperref}
\usepackage{calc} 
\usepackage{enumitem}
\usepackage{titlesec}
%\pagestyle{headings}
\usepackage{setspace} %halbzeilig
\usepackage[authoryear,round]{natbib}
\bibliographystyle{natdin}

%\titleformat{name=\section,numberless}
%  {\normalfont\Large\bfseries}
%  {}
%  {0pt}
%  {}
\date{\vspace{-3ex}}
\begin{document}

\title{\vspace{5ex}
	\includegraphics*[width=0.72\textwidth]{images/hu_logo.png}\\
	\vspace{30pt}
	\scshape\LARGE{Kurzessay zu Freges R"atsel}}
	
	\subtitle{\vspace{20pt}Einf"uhrung in die Sprachphilosophie\\
          \vspace{6pt}
          Tutorium Benjamin\\}


\author{\vspace{-4pt}Lennard Wolf\\
        \small{\href{mailto:lennard.wolf@student.hu-berlin.de}{lennard.wolf@student.hu-berlin.de}}}      

\maketitle

\vspace{\fill}

\begin{minipage}[b]{\textwidth}
    \centering
    \onehalfspacing
    \large   
    30. November 2016\\
    Wintersemester 2016/2017

    \vspace{-20mm} 
\end{minipage}%
\thispagestyle{empty}
\newpage
\clearpage
\setcounter{page}{1}

\section*{Wie Russells Modell der Kennzeichnungen Freges R"atsel l"ost, ohne von einer Sinnebene Gebrauch zu machen.}
\begin{onehalfspace} 

%\textbf{Fragestellung: Kann man mit Hilfe von Russells Theorie der Kennzeichnungen Freges R"atsel l"osen, ohne (wie Frege) die Existenz von Sinnen anzunehmen? Und wenn ja: Wie?}

Im Folgenden m"ochte zeigen, dass das von Russell in seinem Essay \emph{On Denoting}\footnote{Siehe \citep*{russell1905denoting}.} vorgestellte Modell der Kennzeichnungen Freges R"atsel zum einen l"ost, und zum anderen nach Ockhams Razor den von Frege selbst vorgestellten Konzepten von Sinn und Bedeutung vorzuziehen ist.

Dazu werde ich wie folgt vorgehen. In $(i)$ erl"autere ich Freges R"atsel anhand eines Beispiels. Darauf folgend gebe ich in $(ii)$ einen kurzen "Uberblick "uber Freges eigene L"osung und die dazu n"otigen Begriffe, welche er in seinem Aufsatz \emph{"Uber Sinn und Bedeutung}\footnote{Siehe \citep{sinnundbedeutung}.} eingef"uhrt hat. In $(iii)$ stelle ich Russells Kennzeichnungen vor und zeige, wie sie das R"atsel auf elegante Weise l"osen. Es folgt in $(iv)$ eine Gegen"uberstellung der beiden L"osungen, in der gezeigt wird, warum Russells Modell auf Freges Sinne verzichten kann. In $(v)$ schliesse ich den Kurzessay ab.\\ 
\noindent\textbf{$(i)$ Freges R"atsel}

In \citep{sinnundbedeutung} wird das R"atsel das erste Mal vorgestellt. Es befasst sich mit dem Problem des Erkenntnisgewinns aus Identit"atsaussagen wie $a = b$. Grundkenntnisse "uber Gleichungen lassen leicht darauf schlie"sen, dass bei einer wahren Aussage $a = b$ das $b$ mit einem $a$ ersetzt werden kann, sodass man $a = a$ erh"alt, was offensichtlich noch immer wahr ist.

Wenn man dies nun auf nat"urlichsprachliche Identit"atsaussagen wie "`Der CEO von Tesla ist der Gr"under von SpaceX."' anwendet, so erh"alt man "`Der CEO von Tesla ist der CEO von Tesla."'. Diese neue Aussage ist zwar \emph{a priori} richtig, doch sie erm"oglicht im Gegensatz zu der vorhergegangenen keine Erkenntnis mehr. Frege fragte sich nun, wo also der Unterschied zweier Begriffe $a$ und $b$ liegt, wenn ihr Gegenstand\footnote{In \citep{begriffundgegenstand} beschreibt Frege seine Unterscheidung zwischen dem \emph{Begriff} als rein sprachlichem Ph"anomen und dem \emph{Gegenstand} als das, was er repr"asentiert.} doch eigentlich der selbe sind.
\\\noindent\textbf{$(ii)$ Sinn$_{F}$ und Bedeutung$_{F}$}	

Um das R"atsel zu l"osen, f"uhrt Frege ein neues sprachphilosophisches Modell ein. In diesem haben Begriffe, oder \emph{Eigennamen} wie er sie nennt, zwei Interpretationsebenen, \emph{Sinn$_{F}$} und \emph{Bedeutung$_{F}$}\footnote{Ich werde von nun an "`Sinn$_{F}$"' und "`Bedeutung$_{F}$"' schreiben, wenn ich den jeweiligen Begriff \emph{im Fregeschen Sinn} meine.}. Diese lassen sich am besten durch ein Beispiel erkl"aren.

Die Bedeutung$_{F}$ des Eigennames "`der CEO von Tesla"' ist die Person Elon Musk, denn er ist jener Mensch in der objektiven Realit"at, auf den sich bezogen wird. Der Sinn$_{F}$ dieses Eigennamens w"are zum Beispiel  "`jene Person, die Gesch"aftsführerIn bei der Firma Tesla Motors ist"'. Mit dem Sinn$_{F}$ ist also die Art, in der der Begriff den Bezugsgegenstand sprachlich pr"asentiert, gemeint.

Die Eigennamen "`der CEO von Tesla"' und "`der Gr"under von SpaceX"' haben also die selbe Bedeutung$_{F}$, n"amlich Herrn Elon Musk, aber einen unterschiedlichen Sinn. Daher h"atte bei der Aussage "`Der CEO von Tesla ist der Gr"under von SpaceX."' ein Austauschen wie oben beschrieben zur Folge, dass zwar die Bedeutung$_{F}$ gleich bleibt, jedoch der Sinn$_{F}$ ver"andert werden w"urde. Freges Sinn$_{F}$ l"ost also das R"atsel nach der Quelle der Erkenntnis aus Identit"atsaussagen.
\\\noindent\textbf{$(iii)$ Kennzeichnungen$_{R}$}	

Russell l"ost das R"atsel mit seinem Konzept der \emph{Kennzeichnungen}$_{R}$\footnote{Ich werde von nun an "`Kennzeichnung$_{R}$"', wenn ich den Begriff \emph{im Russellschen Sinn} meine.} auf eine andere Art und Weise. Eine Kennzeichnung$_{R}$ 

\end{onehalfspace}

\bibliography{sprachphilo-ha-1}

\end{document}
