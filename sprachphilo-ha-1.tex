\documentclass[a4paper, emulatestandardclasses, 12pt]{scrartcl}
\usepackage{graphicx}
\usepackage{fullpage}
%\usepackage{parskip}
\usepackage{color}
\usepackage[ngerman]{babel}
\usepackage{hyperref}
\usepackage{calc} 
\usepackage{enumitem}
\usepackage{titlesec}
%\pagestyle{headings}
\usepackage{setspace} %halbzeilig

\titleformat{name=\section,numberless}
  {\normalfont\Large\bfseries}
  {}
  {0pt}
  {}
\date{\vspace{-3ex}}
\begin{document}

\title{
	\includegraphics*[width=0.72\textwidth]{images/hu_logo.png}\\
	\vspace{20pt}
	\scshape\LARGE{Einf"uhrung in die\\Sprachphilosophie}}
	
	\subtitle{\vspace{20pt}VEV WS 16/17\\
          Dr. Jasper Liptow\\
          Philosophisches Institut I \\ 
          Humboldt Universit"at zu Berlin}

\vspace{20pt}
\author{Lennard Wolf\\
        \small{\href{mailto:lennard.wolf@student.hu-berlin.de}{lennard.wolf@student.hu-berlin.de}}}
\vfill        
\date{\today}
\maketitle
\newpage


\section*{These..}
\begin{onehalfspace} 

Im Folgenden werde ich zeigen, dass das von Rosenberg angef"uhrte Gegenargument zu der These, dass weder $X$ noch $Y$ Schiff $T$ bedeuten$_{F}$, nicht schl"ussig ist. 

Welches das Schiff des Theseus ist, h"angt von Theseus selbst ab, denn `Schiff des Theseus' ist nur ein Name f"ur jenes Schiff, welches Theseus f"ur sein eigenes h"alt. Dieser Name kann f"ur eine gewisse Zeit nichts bedeuten$_{F}$, da z.B. Theseus sich in dieser Situation selbst nicht sicher sein k"onnte, welches der beiden denn nun \emph{sein} Schiff ist. In solch einem Moment w"urden tats"achlich weder $X$ noch $Y$ Schiff $T$ bedeuten$_{F}$, und Schiff $T$ w"are nicht \emph{verschwunden}, sondern h"atte tempor"ar einfach nur keine Bedeutung$_{F}$.

Entsprechend ist das von Rosenberg angef"uhrte \emph{R"atsel} nur eine sprachliche Verwirrung und der Vorschlag, dass weder $X$ noch $Y$ Schiff $T$ bedeuten$_{F}$, eine valide M"oglichkeit.

\end{onehalfspace}

\end{document}
