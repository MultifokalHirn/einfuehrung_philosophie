\documentclass[a4paper, emulatestandardclasses, 12pt]{scrartcl}
\usepackage{graphicx}
\usepackage{fullpage}
%\usepackage{parskip}
\usepackage{color}
\usepackage[ngerman]{babel}
\usepackage{hyperref}
\usepackage{calc} 
\usepackage{enumitem}
\usepackage{titlesec}
%\pagestyle{headings}
\usepackage{setspace} %halbzeilig
\usepackage[authoryear,round]{natbib}
\bibliographystyle{natdin}

%\titleformat{name=\section,numberless}
%  {\normalfont\Large\bfseries}
%  {}
%  {0pt}
%  {}
\date{\vspace{-3ex}}
\begin{document}

\title{\vspace{5ex}
	\includegraphics*[width=0.72\textwidth]{images/hu_logo.png}\\
	\vspace{30pt}
	\scshape\LARGE{Kurzessay zu Freges R"atsel}}
	
	\subtitle{\vspace{20pt}Einf"uhrung in die Sprachphilosophie\\
          \vspace{6pt}
          Tutorium Benjamin\\}


\author{\vspace{-4pt}Lennard Wolf\\
        \small{\href{mailto:lennard.wolf@student.hu-berlin.de}{lennard.wolf@student.hu-berlin.de}}}      

\maketitle

\vspace{\fill}

\begin{minipage}[b]{\textwidth}
    \centering
    \onehalfspacing
    \large   
    30. November 2016\\
    Wintersemester 2016/2017

    \vspace{-20mm} 
\end{minipage}%
\thispagestyle{empty}
\newpage
\clearpage
\setcounter{page}{1}

\begin{onehalfspace} 

\noindent\textbf{Fragestellung:}\\
\indent Kann man mit Hilfe von Russells Theorie der Kennzeichnungen Freges R"atsel l"osen, ohne (wie Frege) die Existenz von Sinnen anzunehmen?\\\indent Wenn ja: Wie? Wenn nein: Warum nicht?

\begin{center}
\vspace{-9pt}\line(1,0){450}
\end{center}


\noindent In dieser Arbeit m"ochte ich zeigen, wie das von Russell in seinem Essay \emph{On Denoting}\footnote{Siehe \citep*{russell1905denoting}.} vorgestellte Modell der Kennzeichnungen Freges R"atsel l"ost. %Weiterhin m"ochte ich mit Hilfe von \emph{Ockhams Rasiermesser}\footnote{"`Methodologisches Prinzip, demzufolge bei der Auswahl [...] von Theorien stets die einfachste und dennoch ihre Zwecke erf"ullende ausgew"ahlt werden soll."' \citep{ockham}} begr"unden, weshalb diese L"osung den von Frege selbst vorgestellten Konzepten von Sinn und Bedeutung vorzuziehen ist.

Dazu werde ich wie folgt vorgehen. In $(i)$ erl"autere ich Freges R"atsel anhand eines Beispiels. Darauf folgend gebe ich in $(ii)$ einen kurzen "Uberblick "uber Freges eigene L"osung und die dazu n"otigen Begriffe, welche er in seinem Aufsatz \emph{"Uber Sinn und Bedeutung}\footnote{Siehe \citep{sinnundbedeutung}.} eingef"uhrt hat. In $(iii)$ stelle ich Russells Kennzeichnungsmodell vor und zeige dann in $(iv)$ zum einen dass und zum anderen wie es das R"atsel l"ost, ohne von Sinnen Gebrauch machen zu m"ussen. %Es folgt in $(v)$ eine Gegen"uberstellung der beiden L"osungen, in der gezeigt wird, warum Russells Modell auf Freges Sinne verzichten kann.% In $(vi)$ schliesse ich den Kurzessay ab. 
\vspace{5mm}

\noindent\textbf{$(i)$ Freges R"atsel}

\noindent In \citep{sinnundbedeutung} wird das R"atsel das erste Mal vorgestellt. Es befasst sich mit dem Problem des Erkenntnisgewinns aus Identit"atsaussagen der Form $a = b$. Grundkenntnisse "uber Gleichungen lassen leicht darauf schlie"sen, dass bei einer wahren Aussage $a = b$ das $b$ mit einem $a$ ersetzt werden kann, sodass man $a = a$ erh"alt, was offensichtlich noch immer wahr ist.

Wenn man dies nun auf nat"urlichsprachliche Identit"atsaussagen wie "`Der CEO von Tesla ist der Gr"under von SpaceX."' anwendet, so erh"alt man "`Der CEO von Tesla ist der CEO von Tesla."'. Diese neue Aussage ist zwar \emph{a priori} richtig, doch sie erm"oglicht im Gegensatz zu der vorhergegangenen keine Erkenntnis mehr. Frege fragte sich nun, wo also der Unterschied zweier Begriffe $a$ und $b$ liegt, wenn ihr Gegenstand\footnote{In \citep{begriffundgegenstand} beschreibt Frege seine Unterscheidung zwischen dem \emph{Begriff} als rein sprachlichem Ph"anomen und dem \emph{Gegenstand} als das, was er repr"asentiert.} doch eigentlich der selbe ist.\vspace{3mm}
%\newpage

\noindent\textbf{$(ii)$ Sinn$_{F}$ und Bedeutung$_{F}$}	

\noindent Um das R"atsel zu l"osen, f"uhrt Frege ein neues sprachphilosophisches Modell ein. In diesem haben Begriffe, oder \emph{Eigennamen} wie er sie nennt, zwei Interpretationsebenen, \emph{Sinn$_{F}$} und \emph{Bedeutung$_{F}$}\footnote{Ich werde von nun an "`Sinn$_{F}$"' und "`Bedeutung$_{F}$"' schreiben, wenn ich den jeweiligen Begriff \emph{im Fregeschen Sinn} meine.}. Diese lassen sich am besten durch ein Beispiel erkl"aren.

Die Bedeutung$_{F}$ des Eigennames "`der CEO von Tesla"' ist die Person Elon Musk, denn er ist jener Mensch in der objektiven Realit"at, auf den sich bezogen wird. Der Sinn$_{F}$ dieses Eigennamens w"are zum Beispiel  "`jene Person, die Gesch"aftsführerIn bei der Firma Tesla Motors ist"'. Mit dem Sinn$_{F}$ ist also die Art, in der der Begriff den Bezugsgegenstand sprachlich pr"asentiert, gemeint.

Die Eigennamen "`der CEO von Tesla"' und "`der Gr"under von SpaceX"' haben also die selbe Bedeutung$_{F}$, n"amlich Herrn Elon Musk, aber einen unterschiedlichen Sinn. Daher h"atte bei der Aussage "`Der CEO von Tesla ist der Gr"under von SpaceX."' ein Austauschen wie oben beschrieben zur Folge, dass zwar die Bedeutung$_{F}$ gleich bleibt, jedoch der Sinn$_{F}$ ver"andert werden w"urde. Freges Sinn$_{F}$ l"ost also das R"atsel nach der Quelle der Erkenntnis aus Identit"atsaussagen. \vspace{3mm}

\noindent\textbf{$(iii)$ Kennzeichnungen$_{R}$}	

\noindent Russell l"ost das R"atsel mit seinem Konzept der \emph{Kennzeichnungen}$_{R}$\footnote{Ich verwende von nun an "`Kennzeichnung$_{R}$"', wenn ich den Begriff \emph{im Russellschen Sinn} meine.} auf eine andere Art und Weise. Bevor die L"osung vorgestellt werden kann, bedarf es aber zun"achst einer Einf"uhrung dieses Konzepts. 

Ein Ausdruck ist eine Kennzeichnung$_{R}$ genau dann wenn er aus einem \emph{bestimmten} oder \emph{unbestimmten Pronomen} (z.B. "`das"', "`eine"', "`alle"', "`kein"' etc.) und einer \emph{Nominalphrase} (z.B. "`Kennzeichnung"', "`geschlossene Tasche"', "`Schwester von Ludwig"' etc.) besteht. Entsprechend beinhaltet der Satz "`Der CEO von Tesla ist der Gr"under von SpaceX."' zwei Kennzeichnungen$_{R}$, und zwar "`der CEO von Tesla"' und "`der Gr"under von SpaceX"'. Ein Ausdr"uck wie "`der Sohn von Barack Obama"' ist auch eine Kennzeichnung$_{R}$, auch wenn er sich auf keinen real existierenden Gegenstand bezieht. 

Eigennamen gelten f"ur Russell nicht direkt als Kennzeichnungen$_{R}$, vielmehr sind sie als Abk"urzungen f"ur solche zu verstehen. "`Mark Zuckerberg"' steht also eigentlich f"ur eine Kennzeichung$_{R}$ wie "`die Person namens Mark Zuckerberg, die Facebook gegr"undet hat und ihr Studium in Harvard daf"ur abgebrochen hat"'. (vgl. 1. Vortrag in \cite{kripke1972naming})

Nun gilt es zu verstehen, wozu Russell dieses neue Konzept braucht. Er m"ochte verstehen, was die Logik hinter nat"urlichsprachlichen Aussagen ist und vorhergegangene Versuche dies zu tun, wie jener von Frege, weisen Unzul"anglichkeiten auf, die auch in \emph{On Denoting} beschrieben werden. Das Konzept der Kennzeichnung$_{R}$ erlaubt eine Zergliederung von Aussages"atzen in nichts weiter als \emph{Quantit"atsaussagen} und \emph{Pr"adikate}. Dadurch l"asst sich zum einen die genaue Struktur des originalen Satzes erkennen und ausnahmslos ein exakter Wahrheitswert (wahr oder falsch) zuordnen. Am besten l"asst sich dies durch ein Beispiel verstehen:\newpage

\begin{description}[leftmargin=!,labelwidth=\widthof{\bfseries Zergliederung}]
    \item[Originalsatz] Die Pr"asidentin der Vereinigten Staaten hat einen Sohn.
    \item[Zergliederung] Es gibt genau eine Entit"at, die Pr"asidentin der Vereinigten Staaten ist, und diese hat einen Sohn. 
\end{description}

Aus dem bestimmten Pronom "`die"' im Originalsatz folgt nach Russell zum einen eine Existenzaussage ($\rightarrow$ "`es gibt"') und zum anderen eine quantitative Spezifizierung ($\rightarrow$ "`genau eine"'). Die Nominalphrase "`Pr"asidentin der Vereinigten Staaten"' l"asst sich zu einem Pr"adikat umwandeln, n"amlich "`(...) ist die Pr"asidentin der Vereinigten Staaten"', und "`(...) hat einen Sohn"' ist schon eine. Aus letzterem folgt nun eine wichtige Erkenntnis "uber das Konzept: Da Kennzeichnungen$_{R}$  zu Pr"adikaten, die sich auf etwas beziehen m"ussen, zergliedert werden, \emph{sind sie f"ur sich genommen, das hei"st au"serhalb von S"atzen, bedeutungslos}. 

Die Pr"adikate, in die die Nominalphrasen zergliedert werden, sind Wahrheitsaussagen. Dies l"asst sich am besten durch eine Umformung zu Wahrheits\emph{fragen} erkennen ("`Hat (...) einen Sohn?"'). Die Existenzaussagen, die eine Unterklasse der Wahrheitsaussagen sind, beziehen sich auf ein unbekanntes, abstraktes Subjekt, das wir in dem Beispiel ganz einfach "`Entit"at"' genannt haben. Durch das Zusammensetzen dieser Wahrheits- und Existenzaussagen entsteht also eine Aussageform, die sich auf Wahrheit "uberpr"ufen, sowie mit Hilfe von Pr"adikatenlogik formalisieren l"asst.

\vspace{3mm}

\noindent\textbf{$(iv)$ Russells L"osung}

\noindent Die Aussage aus dem urspr"unglichen Beispiel l"asst sich nach Russell nun wie folgt zergliedern:

\begin{description}[leftmargin=!,labelwidth=\widthof{\bfseries Zergliederung}]
    \item[Originalsatz] Der CEO von Tesla ist der Gr"under von SpaceX.
    \item[Zergliederung] Es gibt genau eine Entit"at, die CEO von Tesla ist, und diese ist der Gr"under von SpaceX. 
\end{description}

Es werden nun zwei Pr"adikate aneinander gereiht, um zu schauen, ob es genau eine Entit"at gibt, die alle Eigenschaften erf"ullt. Die Erkenntnis entsteht in diesem Modell nun daraus, dass es eine Entit"at gibt, die beide Eigenschaften erf"ullt, im Gegensatz zu nur einer.  

Zudem l"asst sich nun klar erkennen, dass dieser Satz gar keine Identit"atsaussage der Form $a = b$ ist\footnote{Die Form l"asst sich pr"adikatenlogisch z.B. durch $\exists x~T(x) \wedge (\forall y~T(y) \rightarrow y = x) \wedge S(x) \wedge (\forall z~S(z) \rightarrow z = x)$ ausdr"ucken (die Funktionen $T(~)$ und $S(~)$ repr"asentieren hier die beiden Pr"adikate).}. Deshalb ist die erste Pr"amisse von Freges R"atsel falsch und das Problem somit gel"ost, ohne dass die Existenz von Sinnen angenommen werden musste. 

%Abschliessend m"ochte ich noch anmerken, dass \emph{Ockhams Rasiermesser}\footnote{"`Methodologisches Prinzip, demzufolge bei der Auswahl [...] von Theorien stets die einfachste und dennoch ihre Zwecke erf"ullende ausgew"ahlt werden soll."' \citep{ockham}} zufolge die L"osung von Russell der von Frege vorzuziehen ist, da sie nur auf einem grammatikalischen Substitutionsschema beruht, so auf unterschiedliche Interpretationsebenen f"ur Begriffe verzichtet und daher weniger Variablen enth"alt.


\vspace{3mm}

%\noindent\textbf{$(v)$ Gegen"uberstellung}



\end{onehalfspace}

\bibliography{sprachphilo-ha-1}

\end{document}
