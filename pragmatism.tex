\documentclass[emulatestandardclasses]{scrartcl}
\usepackage{graphicx}
\usepackage{color}
\usepackage[ngerman]{babel}
\usepackage{hyperref}
\usepackage{fullpage}
\usepackage[utf8]{inputenc}
\usepackage{calc} 
\usepackage{enumitem}
\usepackage{titlesec}
\newcommand{\todo}[1]{\textcolor{red}{TODO: #1}\PackageWarning{TODO:}{#1!}}
\date{\vspace{-3ex}}
\begin{document}

\title{
	\includegraphics*[width=0.75\textwidth]{ErstesSem/images/hu_logo.png}\\
	\vspace{24pt}
	Pragmatismus}
\subtitle{\vspace{10pt}Lesegruppe WS 17/18\\
          Philosophisches Institut I \\ 
          Humboldt Universit"at zu Berlin}
\author{Lennard Wolf\\
        \small{\href{mailto:lennard.wolf@student.hu-berlin.de}{lennard.wolf@student.hu-berlin.de}}}
\maketitle
\begin{abstract}
Pragmatism was a philosophical tradition that originated in the United States around 1870. The core of pragmatism was the pragmatist maxim, a rule for clarifying the contents of hypotheses by tracing their ‘practical consequences’. In the work of Peirce and James, the most influential application of the pragmatist maxim was to the concept of truth. But the pragmatists have also tended to share a distinctive epistemological outlook, a fallibilist anti-Cartesian approach to the norms that govern inquiry. It was not until the 1970s that interest in the writings of the Pragmatists became widespread and pragmatist ideas were recognized as able to make a major contribution to philosophy. (Source: \url{https://plato.stanford.edu/entries/pragmatism/})

\end{abstract}
\newpage

\tableofcontents
%\listoffigures
\newpage


\section{The Present Dilemma in Philosophy / What Pragmatism Means\\(26.10.17)}

\subsection{The Present Dilemma in Philosophy}

\begin{itemize}
  \item Source: \url{https://brocku.ca/MeadProject/James/James_1907/James_1907_01.html}
  \item Personal philosophy: "`it is our individual way of just seeing and feeling the total push and pressure of the cosmos."'
  \item Philosophy is at once the most sublime and the most trivial of human pursuits. It works in the minutest crannies and it opens out the widest vistas. It 'bakes no bread,' as has been said, but it can inspire our souls with courage; and repugnant as its manners, its doubting and challenging, its quibbling and dialectics, often are to common people, no one of us can get along without the far-flashing beams of light it sends over the world's perspectives. These illuminations at least, and the contrast effects of darkness and mystery that accompany them, give to what it says an interest that is much more than professional. - hach
  \item He trusts his temperament. Wanting a universe that suits it
  \item Modern people are caught between their spiritual needs and scientific standards. Where will these people look for in philosophy?
  \item You want a system that will combine both things, the scientific loyalty to facts and willingness to take account of them, the spirit of adaptation and accommodation, in short, but also the old confidence in human values and the resultant spontaneity, whether of the religious or of the romantic type. And this is. then your dilemma: you find the two parts of your quaesitum5 hopelessly separated. You find empiricism with inhumanism and irreligion; or else you find a rationalistic philosophy that indeed may call itself religious, but that keeps out of all definite touch with concrete facts and joys and sorrows.
  \item The different temperaments are both in some sense right and in some sense right, pragmatism will bring them together
\end{itemize}

\subsection{What Pragmatism Means}

\begin{itemize}
  \item Source: \url{https://brocku.ca/MeadProject/James/James_1907/James_1907_02.html}
  \item 
\end{itemize}


\newpage
%\section{"Uber den Professor}
%Matthias Schlo"sberger ist Heisenbergstipendiat der Deutschen Forschungsgemeinschaft
%an der Humboldt Universit"at zu Berlin mit dem Forschungsprojekt "`Die Erfahrung der Realit"at durch Widerstand"'.
%
%\begin{figure}[h]
%	\centering
%	\includegraphics[width=0.3\textwidth]{images/Matthias_Schlossberger.png}
%	\caption{Matthias Schlo"sberger}
%	\label{fig:MS}
%\end{figure}


%\begin{figure}[h]
%	\centering
%	\includegraphics[width=0.5\textwidth]{images/template.png}
%	\caption{Template Bild}
%	\label{fig:template}
%\end{figure}

\end{document}
