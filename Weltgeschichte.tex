\documentclass[emulatestandardclasses]{scrartcl}
\usepackage{graphicx}
\usepackage{color}
\usepackage[ngerman]{babel}
\usepackage{hyperref}
\usepackage{fullpage}
\usepackage{calc} 
\usepackage{enumitem}
\usepackage{titlesec}
\newcommand{\todo}[1]{\textcolor{red}{TODO: #1}\PackageWarning{TODO:}{#1!}}
\date{\vspace{-3ex}}
\begin{document}

\title{
	\includegraphics*[width=0.75\textwidth]{ErstesSem/images/hu_logo.png}\\
	\vspace{24pt}
	Hegels Theorie der Weltgeschichte}
\subtitle{Proseminar SS 18\\
          Dr. Dimitris Karydas\\
          Theologische Fakult"at \\ 
          Humboldt Universit"at zu Berlin}
\author{Lennard Wolf\\
        \small{\href{mailto:lennard.wolf@student.hu-berlin.de}{lennard.wolf@student.hu-berlin.de}}}
\maketitle
\begin{abstract}

Hegels Konzept von Weltgeschichte als Fortschritt im Bewusstsein der Freiheit wird im Zusammenhang des Systems des Geistes erläutert. Dem liegt ein Begriff der Geschichtlichkeit zugrunde, der so originär wie bis heute umstritten ist. Die zentralen damit verbundenen Konzepte und Motive von der Vernunft in der Geschichte, der Verwirklichung der Freiheit oder vom Ende der Geschichte werden im Mittelpunkt der Diskussion stehen. Die für Hegels Auffassung von Geschichte ausschlaggebenden Passagen aus den Vorlesungen über die Philosophie der Weltgeschichte, der Enzyklopädie und der Rechtsphilosophie werden gelesen, die auf moodle bereitgestellt werden.
\end{abstract}
\newpage

\tableofcontents
\listoffigures
\newpage


\section{Subjektiver und Objektiver Geist\\(24.04.18)}

\subsection{Organisatorisches}

\begin{itemize}
  \item Moodle-PW: historie (ab 27.04.) geist, manuskripte 
  \item Abgabe: Essay/Protokolle zu insgesamt 10 Seiten
  \item F"ur schein der Teilnahme ist ein Referat/ Protokoll abzugeben
  \item Vortrag am 08.05. Enz. 545 - 552
\end{itemize}

\subsection{Einf"uhrung}

\begin{itemize}
  \item Geist ist geschichtlich
  \item was wir sind wsind wir geschichtlich geworden
  \item Der Löwe versteht sich als Löwe (indem er sich als Löwe verhält ("`an sich"'), nicht reflektierend ("`für sich"')), ohne Bewusstsein zu haben
  \item Metareflektion darüber, wie der Mensch sich auf die Natur beziehen kann
  \item Reduplikation der Weltgeschichte anhand des Begriffs
  \item Subjektivität ist zweierlei: das Wesen des Subjekts (vom Subjekt ist erst in der Moderne zu reden; Das Bewusstsein dass alle \emph{rechtlich} frei sind etc. ist ein modernes) und Reflektion
  \item Die Tiere verhalten sich gesellschaftlich, nur sie reflektieren nicht darüber (Karydas: Hegel ist mit Darwin )
  \item Die Natur hat keine \emph{Geschichte} (was nicht heißt, dass sie keine Historie hat)
  \item Unterschied Historie Geschichte: Wird noch besprochen
  \item Feuerbachs Kernbegriff ist die "`Gattung"'
  \item Empfehlung: Tierphilosophie Hegels
  \item Die Idee: Der Algorithmus der Welt, was die Welt z
  \item Begriffe wie "`Idee"', "`Absolutes"' etc. wurden leider früher immer fetischisiert, doch sie sind auf die Welt zurückzuholen, sie sind ganz naiv, konkret gemeint
  \item die Bewegung vom Abstrakten zum Konkreten ist nach oben gerichtet (erste Negation?)
  \item es spielt das partikulare des Konkreten für das Allgemeine keine Rolle
  \item Der Geist ist das sich von Sich-von-sich-selbst-unterscheiden
  \item Bedürfnis zur Philosophie kommt aus dem Bedürfnis zu Begreifen
  \item der konkrete Geist ist die abstrakte Natur
  \item Der Sohn ist die Natur, die Idee ist der alte Mann
  \item  481: Meint gezeigt zu haben, dass der subjektive Geist ist durch die Form des Allgemeinen vermittelt
\end{itemize}



\section{Objektiver Geist II\\(08.05.18)}

\subsection{Erinnerung}

\begin{itemize}
  \item Übergang von Natur zu Geist
  \item Erst wenn er mit der Anthropologie fertig ist, 
  \item Seele ist noch ein tierischer Aspekt, weil die Reflexion auf sich noch fehlt (unmittelbares Verhältnis) - Eigene Aufhebung
  \item Psychologie hin zur abstrakten Gestalt des Absoluten, alles als Produkt des eigenen Tuns betrachtet wird (Bewegung zum Objektiven Geist)
  \item Philosophie des Rechts muss geklärt werden, um Geschichte verstehen zu können
  \item Das Buch zur Rechtsphilosophie ist eine Polemik gegen die deutsche Rechtsschule
  \item Verhältnis Der Zufalls zur Notwenidigkeit
  \item Das Absolute braucht einen Staat, weil ohne Staat kein Recht und ohne Recht kein objektiver Geist
  \item Es gibt kein Glück in der realen Welt
  \item Wer sich im Staat befindet der ist frei, egal welche Rechtsform besteht
  \item Wo es Zufall gibt, gibt es keine Allgemeinheit (!)
\end{itemize}

\subsection{Anmerkungen zum Übergang vom obj. zum abs.}

\begin{itemize}
  \item Volksgeist (Jena-Hegel): spielt später keine systematische Rolle - was war gemeint, ein Selbstverständnis, das geschichtlich, klimatisch geworden ist
  \item Dialektische Konstruktion muss nachvollzogen werden
  \item Kants ewiger Frieden als Gegenkonstruktion, von der sich Hegel abwendet - wäre unmöglich, da es eine abstrakte negation des krieges wär, wofür keine partikularinteressen für die staaten mehr existieren dürften
  \item Friedensverträge unterscheiden sich vom Naturrecht, vielmehr: konraktualistische Vorstellung vom Recht
  \item Das Recht ist kein Resultat eines Vertrags
  \item Friedensverträge sind die spekulative Bewegung aus Frieden und Krieg - es folgt, dass sie daher aber kein Recht sind
  \item "`Andere Länder andere Sitten"' - Hegel
  \item Wenn der Staat keine klaren Grenzen hat, werden die Menschen versklavt
  \item Marx Argument gegen Kapitalismus ist: Der Kapitalismus kann seine Grundfesten, Land und Kapital, nicht reproduzieren. Er kann nicht in ein Ganzes übergehen
  \item Philosophie kann nur im Tempel passieren
  \item 3 Arten der Geschichtsschreibung in den
  \item Weltgeschichte hier ist die objektive Geschichte und darin gehen die Rectsverhältnisse auf, und alle anderen Verhätnisse aus (Familie, Ökonomie)
  \item Der Staat ist eine Realgestalt des Absoluten
  \item Zur Weltgeschichte gehört beides: Die subjektiven Ereignisse sowie die partialgeschichten des Absoluten 
  \item Der Volksgeist kommt im Absoluten zur Geltung (Kunst, Religion etc.)
  \item Hegel et al haben die Französische Revolution auf den Begriff gebracht
  \item Nächstes mal: §§ zur bestimmung des objektiven geistes 481 bis 468
\end{itemize}


\section{Eigene Notizen}

\begin{itemize}
  \item Subjektiver Geist: Der Mensch als geistige Struktur (Psychologie (Kants Vermögen), Phänomenologie)
  \item Objektiver Geist: Menschliches Miteinander
  \item Absoluter Geist: Wissen des Göttlichen
  \item Bei Hegel ist Sittlichkeit bestimmt als "`Reihe von Pflichten, die wir haben, und die besagt, daß eine auf der Idee gegründete Gesellschaft gefördert und erhalten werden muß"'
  \item Der Staat ist dadurch ausgezeichnet, dass er als Ganzes alle Momente des menschlichen Lebens beinhaltet (familiäres, ökonomisches, politische Institutionen) und sich in seiner Entwicklung auf sich selbst bezieht (ebd.: 330, § 536). Der Staat ist dabei die höchste Weise der Vermittlung von Individualität und Gesellschaftlichkeit, und begründet, beinhaltet und umfasst die familiären und die ökonomischen Beziehungen der Menschen. 
  \item Vernunft ist schon immer da, sie muss nur immer weiter verwirklicht werden
  \item Zirkelschluss? "Daß in den Begebenheiten der Völker ein letzter Zweck das Herrschende, daß Vernunft in der Weltgeschichte ist, – nicht die Vernunft eines besonderen Subjekts, sondern die göttliche, absolute Vernunft, – ist eine Wahrheit, die wir voraussetzen; ihr Beweis ist die Abhandlung der Weltgeschichte selbst: sie ist das Bild und die Tat der Vernunft."'
  \item Ist das Baby dann am Anfang noch Natur
  \item Woher weiß ich dass mein Staat gegen die Befreiung des Geistes arbeitet
\end{itemize}



\section{Objektiver Geist II\\(22.05.18)}


\begin{itemize}
  \item Die Vernunft in der Geschichte
  \item Hat nie die begriffene Geschichte vorgetragen - ist dies überhaupt möglich
  \item Gewaltenteilung ist notwenidge Bedingung dafür, dass alle gleich vor dem Gesetz sind
  \item muss man dann sehen
  \item PR: 273; § 341
  \item von Schiller "`Resignation"': Weltgeschichte ist das Gericht
  \item Das Recht des Weltgeistes/allgemeinen Geistes: die vernünftigen Strukturen
  \item Schwierige Frage ist nach der Notwendigkeit und Zufälligkeit
  \item Vorwurf an ihn: Dass in seiner Geschitsauffassung alles gerechtfertigt war
  \item GW manuskript Band 18 S. 121-129, 1822/28
\end{itemize}

\section{Philosophie der Weltgeschichte - Einleitung 1822-28, S. 121-129\\(29.05.18)}

\subsection{Lektürenotizen}

\begin{itemize}
  \item Dreierlei Weisen des Geschichtsschreibens:
  \item Ursprüngliche Geschichte: seiendes in geistiges umwandeln: Erlebnisse festhalten. Bildung der Geschichte und Bildung des Autors sind das selbe
  \item Geschichtsschreibung braucht einen hohen Bildungsgrad im Volksgeist - und dass sie Geistlichkeit nicht isoliert, sondern im Akt des Geschichtemachens mit vorhanden sind
  \item Reflektierte Geschichte
\end{itemize}

\begin{itemize}
  \item Die Vernunft in der Geschichte ist der Versuch aus den Manuskripten und Mitschriften einen Gesamttext zu erstellen
  \item in der Weltgeschichte, nicht nur die Geschichte von Staaten (Partialgeschichten), sondern die Gestalten des Absoluten, die objektiviert werden, sie haben eine Positivität (Religion, Kunst, Philosophie)
  \item Partialgeschichten kommen mit rein, und es gibt einen Zusammenhang zwischen den Partialgeschichten (und auch untereinander) und dem Absoluten
  \item Wäre aber diese Verbindung herstellbar (ist es theoretisch tragbar und ist es ausführbar?) - die Welt nur als eine Totalität zu betrachten ist schwierig (Marx: die Welt kann ihre eigenen Grundlagen, die zufällig sind, nicht reproduzieren)
  \item Welt als Totalität müsste sich selbst hervorbringen - kein äußeres das einfluss hätte! | da aber Zufälle die Geschichte 
  \item Gott ist sein tun - Bei Hegel gibt es \emph{keine} Negative Theologie: deus absconditus (prinzipiellen Unerkennbarkeit Gottes) - Religion und Philosophie haben einen identischen Inhalt (Vernunft muss auch aus der Religion herausgefiltert werden)
  \item Der Begriff von Anaxagoras: Erstes Dokument der Philosophie (sagt dass der Lauf der Dinge zwar zeitlich ist, aber trotzdem vernünftig ist - vorsokratische Naturphilosophie (Heidegger hat hierzu einen Aufsatz geschrieben))
  \item Die Idealität der Vernunft muss auch Faktizität haben - so wie bei jeder Idee
  \item Als Ordnung der Vernunft ist die Dialektik strukturell begründet
  \item Nur die PHil. verfügt über das Wissen von vernünftigen Strukturen - das ist gerade ihr Geschäft (alles in Schlußform zu bringen (nicht Urteil) - Zufälligkeit ist notwendig)
  \item Wer die Welt vernünftig betrachtet, den betrachtet die Welt auch vernünftig (Rechtsphilosophie)
  \item Referat:
  \item Das was Weltgeschichte ist, dafür reicht unser Alltagsverständnis
  \item aber Philosophie der Weltgeschichte: Weltgeschichte denkend zu betrachten
  \item Es ist unmöglich die Geschichte so zu betrachten wie sie ist - daher ist Betrachtung mit der Vernunft legitimiert!
  \item Nur die Phil. kann den Gedanken der Vernunft benutzen | keine subjektive Idee oder Ideal, nicht bloß so unmächtig, um es nur bis zum Ideal zu bringen, nur als etwas besonderes in den Köpfen zu bleiben | und zwar als Ordnung der Geschichte
  \item Wer in die Sonne schaut wird blind (man steht vor der Abstraktion, und diese macht durch ihre Nivellierung blind vor den Partikularitäten)
  \item Mnemosyne: Göttin der Erinnerung
\end{itemize}

\begin{itemize}
  \item Antigone? dichthonische Ordnung (das Blutbad) vs. ... Ordnung (Recht)
  \item 
\end{itemize}



\section{Philosophie der Weltgeschichte - Einleitung 1822-28, S. 130-137\\(29.05.18)}

\subsection{Lektürenotizen}

\begin{itemize}
  \item 
\end{itemize}

%\section{Ideen zu Hausarbeit}
%
%\begin{itemize}
%  \item Es ist keine Whig history, da nicht gesagt ist, dass der Geist zwangsläufig die Freiheit erlangt, es ist vielmehr einfach eine Sache der Zeit
%  \item Indem der Fokus auf den subjektiven Geist zurückgelenkt wird, die Sittlichkeit als schädigend für den subjektiven Geist aberkannt wird und der objektive Geist zum Werkzeug für die erfüllung des subjektiven Willens umgekehrt wird, befinden wir uns in einem für den Geist regressiven Zeitalter
%  \item Zufall und Notwendigkeit beim Fortschritt, whig, moldbug
%  \item 
%\end{itemize}
%
%
%
%\newpage
%\section{"Uber den Professor}
%Prof. Mustermann ist..


%\begin{figure}[h]
%	\centering
%	\includegraphics[width=0.5\textwidth]{images/template.png}
%	\caption{Template Bild}
%	\label{fig:template}
%\end{figure}

\end{document}
