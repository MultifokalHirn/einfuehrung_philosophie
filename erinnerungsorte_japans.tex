\documentclass[a4paper]{article}

\usepackage[ngerman]{babel}
\usepackage{hyperref}
%\usepackage{fullpage}
\usepackage[utf8]{inputenc}
\usepackage[hang]{footmisc}
\setlength{\footnotemargin}{-0.8em}
\usepackage[style=authoryear-ibid,natbib=true]{biblatex}
\addbibresource{wahrnehmung.bib}
\date{\vspace{-2ex}}
\begin{document}

\title{Erinnerungsorte Japans\\Motivationsschreiben\vspace{-0.5ex}}

\author{Lennard Wolf}

\maketitle

\noindent Hiermit möchte ich, Lennard Wolf, mich für die Exkursion \emph{Erinnerungsorte Japans} bewerben. Ich werde zuerst meine Verbindung zu den Themengebieten Architektur und Japan erläutern und danach genauer auf die Exkursionsthemen eingehen. 

Zu Beginn meines Philosophiestudiums im Wintersemester 16/17 war mein Nebenfach die Kunst- und Bildgeschichte, weshalb ich bei Herrn Prof. Kappel das Seminar \emph{Architektur der Moderne} im Rahmen des Modul I besuchte, welches ich auch erfolgreich abgeschlossen habe. Ich hielt damals einen Vortrag zum Leben und Denken Le Corbusiers, der mich durch seine Radikalität sehr beeindruckte, aber dessen absoluter Traditionsbruch mir auch Sorgen bereitete. In der parallelen Vorlesung \emph{Einf"uhrung in die Geschichte der Architektur und des St"adtebaus} sprach Prof. Kappel dann von dem faszinierenden Tadao Andō, der, beeinflusst von Le Corbusier, einen modernistischen Stil verfolgte, aber immer mit Rückbezug auf traditionelle Strukturen und Formen, wie unter anderem die Tatami. Gerade diese Beziehung zwischen Tradition und Moderne, die in Japan heute auch als Spannungsfeld bezeichnet werden kann, interessiert mich seither sehr und ist auch in der modernen japanischen Philosophie ein großes Thema. Im darauf folgenden Semester wurde mir immer klarer, dass in diesem Lebensabschnitt mein Interesse an asiatischen Kulturen, und besonders die Japans, größer wurde als das an den europäischen. Aus diesem Grund entschied ich mich zum Wintersemester 17/18, Regionalwissenschaften Asien/Afrika als mein neues Nebenfach aufzunehmen. In dessen Rahmen habe ich nun den Sprachkurs \emph{Japanisch 1} belegt\footnote{Ich habe mich für das kommende Sommersemester schon für \emph{Japanisch 2} angemeldet.}, sowie auch das Seminar \emph{Einführung in die japanische Kultur und Gesellschaft} bei Prof. Leinss. Der in diesem Semester erhöhte Kontakt mit der japanischen Kultur hat mein Interesse immer weiter bestärkt, weshalb eine Reise nach Japan nun einen logischen nächsten Schritt für mich bedeuten würde.

Während meiner Auseinandersetzungen mit moderner Architektur stieß ich auf die Dokumentation \emph{KOCHUU}\footnote{Wachtmeister, Jesper (2003): \emph{KOCHUU - Japanese Architecture / Influence \& Origin}. Stockholm: Solaris Filmproduktion.}, die mich besonders auf die Möglichkeit eines harmonischen Verhältnisses zwischen moderner Architektur und Natur aufmerksam gemacht hat, sowie auch auf die damit verbundene Geistesgeschichte, in die traditionelle und moderne japanische Architektur eingebettet ist. Die aus dieser stammende Ästhetik, in japanischen Gärten besonders zum Vorschein kommend, strahlt einen wundersamen Zauber aus, da sie mir zugleich fremd und vertraut vorkommt. Besonders aufgrund dieses \emph{je ne sais quoi} der japanischen Gärten, dem ich auf die Spur kommen möchte, freue ich mich auch schon sehr auf das Seminar zu diesem Thema. Es ist mir ein großes Bedürfnis, japanische Architektur und Gärten einmal außerhalb von Bücherseiten und Bildschirmen selbst zu erleben. Im Rahmen des Seminars von Prof. Leinss habe ich mich im besonderen mit modernen intellektuellen Strömungen auseinandergesetzt, wo ich auf das schon oben angesprochene Spannungsfeld zwischen Tradition und modernem Gedankengut gestoßen bin. Ich konnte sehen, dass die Frage, was "`das Japanische"' denn nun sei, zu einer Art Unsicherheit über die eigene Identität geführt hat, weshalb zum Beispiel sowohl die identitätsessentialisierenden \emph{nihonjinron} (Japaner-Diskurse), als auch eine postmoderne, aber trotzdem idiosynkratische Popkultur in hohen Maßen konsumiert werden. Diese Frage nach Identität im Spannungsbogen zwischen Tradition und Moderne hatte nun sicherlich einen großen Einfluss auf die Architektur und das Lebensumfeld im Allgemeinen, weshalb ich auch schon sehr gespannt bin, hierzu mehr im diesbezüglichen Seminar zu erfahren. 

Diese Themen interessierten mich schon, bevor ich den Wechsel in die Regionalwissenschaften gemacht habe, weshalb ich mich also schon eine Weile mit ihnen beschäftigt habe. Über die Möglichkeit, dieses Interesse nun in den Seminaren zu vertiefen, freue ich mich jetzt schon sehr. Darüber hinaus dann an der Exkursion nach Japan teilzunehmen, in der diese Auseinandersetzungen kulminieren würden, wäre mir eine große Freude. Ich hätte außerordentliches Interesse daran, mich in einem solchen Kontext sowohl mit gleichgesinnten Studierenden aus den beiden Disziplinen auszutauschen, als auch durch Sie viele weitere Dinge zu erfahren und kennenzulernen.


\newpage
%\nocite{*}
\printbibliography

\end{document}