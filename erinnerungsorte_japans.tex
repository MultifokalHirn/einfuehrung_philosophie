\documentclass[a4paper]{article}

\usepackage[ngerman]{babel}
\usepackage{hyperref}
%\usepackage{fullpage}
\usepackage[utf8]{inputenc}
\usepackage[hang]{footmisc}
\setlength{\footnotemargin}{-0.8em}
\usepackage[style=authoryear-ibid,natbib=true]{biblatex}
\addbibresource{wahrnehmung.bib}
\date{\vspace{-2ex}}
\begin{document}

\title{Erinnerungsorte Japans\\Motivationsschreiben\vspace{-0.5ex}}

\author{Lennard Wolf}

\maketitle

\noindent Hiermit möchte ich, Lennard Wolf, mich für die Exkursion \emph{Erinnerungsorte Japans} bewerben. Ich werde zuerst meine Verbindung zu den Themengebieten Architektur und Japan erläutern und danach genauer auf die Exkursionsthemen eingehen. 

Zu Beginn meines Philosophiestudiums im Wintersemester 16/17 war mein Nebenfach die Kunst- und Bildgeschichte, weshalb ich bei Herrn Prof. Kappel das Seminar \emph{Architektur der Moderne} im Rahmen des Modul I besuchte, welches ich auch erfolgreich abgeschlossen habe. Ich hielt damals einen Vortrag zum Leben und Denken Le Corbusiers, der mich durch seine Radikalität sehr beeindruckte, aber dessen absoluter Traditionsbruch mir auch Sorgen bereitete. In der parallelen Vorlesung \emph{Einf"uhrung in die Geschichte der Architektur und des St"adtebaus} sprach Prof. Kappel dann von dem faszinierenden Tadao Andō, der, beeinflusst von Le Corbusier, einen modernistischen Stil verfolgte, aber immer mit Rückbezug auf traditionelle Strukturen und Formen, wie die Tatami-Matten. Gerade diese Beziehung zwischen Tradition und Moderne, die in Japan heute auch vielleicht als Spannungsfeld bezeichnet werden kann, interessiert mich seither sehr und ist auch in der modernen japanischen Philosophie ein großes Thema. Im darauf folgenden Semester wurde mir immer klarer, dass in diesem Lebensabschnitt mein Interesse an asiatischen Kulturen, und besonders die Japans, größer wurde als das an der europäischen. Daher entschied ich mich zum Wintersemester 17/18, Regionalwissenschaften Asien/Afrika als mein neues Nebenfach aufzunehmen, und in dem ich unter anderem nun den Sprachkurs \emph{Japanisch 1} belegt habe\footnote{Ich habe mich für das kommende Sommersemester schon für \emph{Japanisch 2} angemeldet.}, sowie auch das Seminar \emph{Einführung in die japanische Kultur und Gesellschaft} bei Prof. Leinss. Der in diesem Semester erhöhte Kontakt mit der japanischen Kultur hat mein Interesse immer weiter bestärkt und meinen Wunsch auf einen Besuch Japans sehr groß werden lassen.

Während meiner Auseinandersetzung mit moderner Architektur stieß ich auf die Dokumentation \emph{KOCHUU}\footnote{Wachtmeister, Jesper (2003): \emph{KOCHUU - Japanese Architecture / Influence \& Origin}. Stockholm: Solaris Filmproduktion.}, die mich besonders auf die Möglichkeit eines harmonischen Verhältnisses zwischen moderner Architektur und Natur aufmerksam gemacht hat, sowie auch auf die damit verbundene tiefe, philosophische Geistesgeschichte, in die traditionelle und moderne japanische Architektur eingebettet ist. Die aus dieser folgende Ästhetik, in japanischen Gärten besonders zum Vorschein kommend, strahlt einen wundersamen Zauber aus, obwohl ihre strikte Asymmetrie dem europäischen Gemüt doch eigentlich zuwider sein müsste. Aufgrund dieses \emph{je ne sais quoi} der japanischen Gärten, dem ich auf die Spur kommen möchte, freue ich mich auch schon besonders auf das Seminar zu diesem Thema. Diese Ästhetik möchte ich auf alle Fälle einmal außerhalb von Bücherseiten und Bildschirmen erleben.


 Eine Reise mit 


\newpage
%\nocite{*}
\printbibliography

\end{document}