\documentclass[emulatestandardclasses]{scrartcl}
\usepackage{graphicx}
\usepackage{color}
\usepackage[ngerman]{babel}
\usepackage{hyperref}
\usepackage{fullpage}
\usepackage[utf8]{inputenc}
\usepackage{calc} 
\usepackage{enumitem}
\usepackage{titlesec}
\newcommand{\todo}[1]{\textcolor{red}{TODO: #1}\PackageWarning{TODO:}{#1!}}
\date{\vspace{-3ex}}
\begin{document}

\title{
	\includegraphics*[width=0.75\textwidth]{ErstesSem/images/hu_logo.png}\\
	\vspace{24pt}
	Die Erfahrung der Realität\\durch Widerstand}
\subtitle{\vspace{10pt}Proseminar WS 17/18\\
          Matthias Schlo"sberger\\
          Philosophisches Institut I \\ 
          Humboldt Universit"at zu Berlin}
\author{Lennard Wolf\\
        \small{\href{mailto:lennard.wolf@student.hu-berlin.de}{lennard.wolf@student.hu-berlin.de}}}
\maketitle
\begin{abstract}
Häufig wird Erkenntnistheorie als Versuch der Rechtfertigung von Erkenntnis betrieben. Dass es möglich ist, von der Rechtfertigung einer Erkenntnis zu sprechen, verweist auf ein Problem, das vor allen Fragen der Rechtfertigung geklärt werden sollte. Bevor eine Erkenntnis gerechtfertigt werden kann, muss das in der Erkenntnis Erkannte erfahren werden. Dies, so die Arbeitshypothese des Seminars, gilt auch und insbesondere für die Erfahrung von Realität.
Eine mögliche Antwort auf die Frage, wie die Erfahrung der Realität gemacht wird, lautet: durch die Erfahrung der Widerständigkeit von etwas (der Welt, des Psychischen, des Physischen, des Anderen etc.). Ziel des Seminars ist es, diese Antwort bei einigen für dieses Thema klassischen Autoren zu untersuchen. Besonders wichtig wird es sein, einen klaren Begriff des Widerstands zu entwickeln, der an vielen verschiedenen Phänomenen erläutert werden kann.

\end{abstract}
\newpage

\tableofcontents
%\listoffigures
\newpage


\section{Einf"uhrung\\(19.10.17)}

\begin{itemize}
  \item Moodle-Passwort: Scheler
\end{itemize}


\subsection{Überblick über die Fragestellung}

\begin{itemize}
  \item Realismus vs. Idealismus
  \item Fragestellungen dieser Alternative: Ob wir Zugang zu Realität/Wirklichkeit haben und wenn ja, welchen?
  \item Realisten: Ja.
  \item Idealisten: Die Wirklichkeit wie sie an sich ist können wir nicht erfahren, sondern: das was wir als Wirklichkeit erfahren das ist immer Konstruktion des Subjekts (\emph{subjektiver Idealismus}) oder durch kollektive Konstruktion (\emph{Konstruktivismus})
  \item Essenzaussagen oder Existenzaussagen (was vs. dass) sind gekoppelt an Idealismus vs. Realismus
  \item Matrix: was oder dass Problem
  \item Descartes Ausgangsfrage: Wir müssen alles in Zweifel ziehen. Was bleibt übrig: ego das cogito machen tut (\emph{cogito existo})
  \item Wie können wir von unserer Existenz auf die Existenz der Welt schließen?
  \item Ist die Welt geträumt? Wie unterscheiden wir Traum und Wirklichkeit?
  \item Um zu fragen, ob das Wirklichkeit ist, muss ich mit dem Gegenteil bekannt sein.
  \item Ob das Urteil darüber wahr ist, ist eine andere Frage. (erst danach!)
  \item Klassische Frage: Gibt es Beweise dafür, ob die sogenannte außenwelt existiert?
  \item Übersieht, dass anderer Frage vorgelagert ist: Wie können wir dazwischen unterscheiden?
  \item Philosophie als Fundierungsprojekt! Unterscheidungsfrage ist noch zu klären!
  \item Damit wird die Frage Realismus vs. Idealismus nicht überflüssig! (wie Scheler, Carnap behauptet haben)
  \item Aber: Ändert dieses Fundierungsprojekt überhaupt etwas an der Unterscheidung?
  \item Ob wir in Erfahrungen, durch die wir mit der Unterscheidung vertraut werden, ob das wahr ist, sei erstmal dahingestellt. (macht nichts wenn wir vertauschen, denn wir sind trotzdem mit der Unterscheidung bekannt!)
  \item Dildheim: Über den Grund zur Annahme zur Existenz der Außenwelt 
  \item Repräsentationalismus: Annahme: Zugang zur Welt ist, dass sie vorgestellt ist, dass die Welt in meinem Bewusstsein, dann bedarf es eines Kriteriums dafür, dass die echte Welt 
  \item Naiver, Direkter und Kritischer Realismus
  \item Derealisierung: Die anderen sind da, aber nicht wirklich
  \item Inwiefern können wir uns die ganzen Sinne vorstellen?
  \item Bedeutung der Frage: In welchen Akten unterscheiden wir die Wahrnehmung als wirklich und unwirklich?
  \item Ist das zu kurz gekommen weil wir in erkenntnistheoretischen Fragen uns nur Rechtfertigungen beschäftigt haben und weil wir uns zu sehr an dem "`visozentrischen"' Modell orientiert haben?
  \item \emph{In diesem Seminar Hauptclaim}: Tastsinn wurde eminent vernachlässigt, obwohl dieser zunächst von der Wirklichkeit der Welt überzeugt ist
  \item Aus diesem Widerstand ziehen wir die tiefe Überzeugung, dass die Welt da ist!
  \item Haben Leute mit Derealisierungsproblemen Schwierigkeiten mit dem Widerstand?
  \item Gründen alle Erfahrungen der Welt in Widerständen?
  \item Entfremdungsphänomene: Hier ist jemand nicht mehr in der Lage, Widerstandserfahrungen zu machen.
  \item Vernunfteinsichten helfen nicht weiter! Sie können Problem fixieren, aber nicht lösen.
\end{itemize}

\subsection{Gestaltung des Semesters I}

\begin{itemize}
  \item Erster Schritt: Platons Höhlengleichnis - vergegenwärtigen, was das Grundproblem von Realismus/Idealismus ist
  \item Zweiter Schritt: Text, der entscheidender Bezugspunkt Entwicklung des Widerstandsarguments Grund Über den Grund zur Annahme zur Existenz der Außenwelt
  \item Dritter Schritt: Scheler: Hat am schärftsten artikuliert (auch vllt. David Katz ("`Aufbau der Tastwelt"') zur Anregung)
  \item Letztes Drittel: Merleau-Ponty, Waldenfels (responsive Ethik), Psychopathologische Phänomene (Thomas Fuchs (Karls Jaspers Lehrstuhl in Heidelberg))	
\end{itemize}

\section{Platons Höhlengleichnis\\(26.10.17)}

\subsection{Lektürenotizen}

\begin{itemize}
  \item Zwei Dinge die der Befreite sieht: Die Schatten, und die schattenwerfenden Dinge
  \item Erst seit er beides kennt, ergibt sich die Frage, welches denn nun die Realität sei
  \item Durch Gewöhnung kommt er in Schmerzen dann zum Erkennen des Neuen
  \item Er wird sich an den Spielen und Hierarchien der alten Freunde nicht mehr erfreuen können, es wäre nicht mehr so echt wie früher. So würde er sich lieber versklaven lassen in der neuen Wahrheit als König sein im Falschen.
  \item Diese alten würden ihn auch nicht ernst nehmen
\end{itemize}

\subsubsection{Fragen}

\begin{itemize}
  \item Ist nicht gerade die Bewegung der Erkenntnis unsere Grundlage für die Unterscheidung? Das heißt, wenn das Wissen das erste Mal als kontrafaktisch erkannt wird, dann ergibt sich schon die Möglichkeit der Frage nach Realität..?
\end{itemize}

\subsection{Gestaltung des Seminars II}

\begin{itemize}
  \item Dilthey, Scheler: Basalster Zugang zur Welt: Widerstand
  \item Erkenntnistheorie der Sinne und ihrer kognitiven Leistungen
  \item David Katz, Wechselspiel mit Scheler, Gestaltpsychologie
  \item Gestaltpsychologe: Jede Wahrnehmung ist eine Wahrnehmung eines Ganzen, dann erst werden die Bestandteile erkannt
  \item Dewey, Pragmatisten: Bestimmte Störungen (etwas klappt nicht) nötigt zur Reflektion (kann als Widerstand interpretiert werden) -- wäre ja sehr nah am Höhlengleichnis
  \item Waldenfels: Responsive Ethik - Wie kommt das normative in die Welt? Mögliche Antwort: Auf das Verhalten/Antlitz des anderen "`responsiv"' antworten
  \item Psychopathologie: Frage nach der Erfahrung der Realität, gleiche Frage, die sich auch für Psychopathologen stellt, wenn Patienten sagen dass die Welt nicht wirklich ist?
  \item Jede Erfahrung eines \emph{x} gründet im Widerstand dieses \emph{x}
\end{itemize}

\subsection{?}

\begin{itemize}
  \item Warum werden in bestimmen Richtungen Fragen nicht gestellt? Damit werden wir uns nicht auseinandersetzen
  \item Wenn ich von der Sterblichkeit erfahre, dann hat sich vorher die Frage nie gestellt
  \item Ist Erkenntnis rückgängig machbar?
\end{itemize}


\subsection{Notizen}

\begin{itemize}
  \item Dem Widerstand muss doch etwas voraus gehen?
  \item Descartes Meditationen sollte man gelesen haben!
  \item Blockchain erkennt sich selbst, definiert sich selbst, ist das was es von sich weiß
  \item Blockchain als selbstreferenzielle epistemologische Genealogie
\end{itemize}


\section{Wilhelm Dilthey: Beiträge zur Lösung der Frage vom Ursprung unseres Glaubens an die Realität der Aussenwelt und seinem Recht\\(26.10.17)}

\subsection{Lektürenotizen}

\begin{itemize}
  \item Satz der Phänomenalität: Alles was für mich da ist. ist Tatsache meines Bewusstseins
  \item Geht in Phänomenalismus über, wenn noch die Vorstellung hinzukommt, dass alle Dinge nur im Bewusstsein sind (aber: nicht solipsismus)
  \item Intellektualismus (?): Wir nehmen Dinge nicht wahr, sondern sie werden vom Verstand erst nach der Empfindung zusammengesetzt (unbewusster Schluß)
  \item Verbindung von Empfindung mit erworbenen Vorstellungen (woher kommen die?)
  \item -- Zusammenfassung --
  \item Realität der Au0enwelt ist die \emph{allgemeinste} Voraussetzung, welche allen Induktionen zugrunde liegt
  \item 
\end{itemize}

\subsection{Vorrede}

\begin{itemize}
  \item Erneuerungsepoche der Philosophie am Ende des 19. Jhds.
  \item Lebensphilosophie, Neukantianismus, Phänomenologie
  \item Kant, Hegel, Fichte, Schelling
  \item Am Ende des 19. Jhds. "`gehts wieder los"'
  \item PhdG $\rightarrow$ dann kann Philosophie nach Hegel nur noch Geschichtsphilosophie sein
  \item Die Eule der Menerve beginnt erst in der Dämmerung ihren Flug (jetzt wo der Geist zu sich gekommen ist, kann nachvollzogen werden)
  \item Linkshegelianer (Feuerbach, Marx et al): auch erst spät zu Ruhm gefunden
  \item Dilthey: Lebensphilosophie, Historismus ()
  \item Gilt als Relativist; Er meint, dass es keine lineare Entwicklung gibt. Es gäbe vielmehr bestimmte Grundarten, die Welt zu betrachten, diese Schließen sich gegenseitig aus, Weltanschauungsphilosophie
  \item Hat in Berlin gelehrt, berühmte Vorlesung zum 70. Geburtstag: Er sieht keine Möglichkeit die Anarchie der Philosophischen Systeme zusammenzuführen
  \item \emph{Begründer der Geisteswissenschaft}, da er den Begriff berühmt gemacht hat (Einleitung in die Geisteswissenschaften)
  \item Naturwissenschaften haben im 19. Jhd. unglaublichen Aufschwung gehabt; Dadurch: Naturalismus/Materialismus sehr beliebt (res extensa auf res extensa); alles andere sind dann nur noch Epiphänomene "`Kraft und Stoff"' (Marx: "`Vulgärmaterialisten"')
  \item Alle Wissenschaften, die nicht Naturwissenschaften sind (nicht Körper auf Körper)
  \item Kanonische Unterscheidung: 2 Erkenntnistheoretische Operationen, \emph{Erklären} und \emph{Verstehen} (Begriffe kontingent)
  \item Erklären: wie Körper auf Körper wirken; Verstehen: Leben versteht Leben/Psychisches versteht Psychisches (Seelenleben wird Gegenwärtig)
  \item Das eine ist nicht durch das andere zu ersetzen
  \item Geisteswissenschaften sollen sich um verstehen kümmern (aber das geht nicht ganz auf: biologische Verhaltenspsychologen)
  \item Anmerkung: Nach Dilthey - Andere Unterscheidungen: Naturwissenschaften (Gesetze; was ist (Körper)) und Kulturwissenschaften (Individuelles; Bedeutung: Werte/Normative)
  \item Große Fortsetzung gefunden in Sprachphilosophie: Ursachen (Körper) und Begründungen (Psychen)
\end{itemize}

\subsection{Argumentation des Textes}

\begin{itemize}
  \item 1. Frage: Ursprung des Glaubens an die Außenwelt
  \item 2. Frage: Ist diese berechtigt?
  \item "`Glaube"': ?
  \item Im Bewusstsein vs. 
  \item Ausgangssituation: Es könnte doch sein, dass alles was ich wahrnehme gar nicht wirklich existiert
  \item Warum sollte man schließen wollen?
  \item Widerstand: 
  \item Primat bestimmer Sinne
  \item Ist die Unterscheidung Außenleben und Innenleben dieselbe wie Außenwelt und Innenwelt (Hausarbeitsthema)
  \item Aufgabe für nächste Sitzung: Überlegen, sie wären blind, schon immer gewesen. Was für eine Raumvorstellung hat man dann? Wie konstituiert sich der Raum für einen Blinden/Wie ist diese möglich.
  \item Was sind Widerstandsempfingunen gemeint und wie breit ist der Begriff vom Widerstand bestimmt?
\end{itemize}


\begin{itemize}
  \item Wille ist die Summe aus Kausaldenken und Selbstbewusstsein
  \item Widerständigkeit als Grundlage für Designentscheidungen
  \item Je weniger Widerständig ein Objekt zur Veränderung ist, desto angenehmer ist die Arbeit   damit $\rightarrow$ Bsp: Text der durchsuchbar vs ein Text der nicht durchsuchbar ist
  \item "`ein bischen eklig"'
  \item Was heißt häßlichkeit für blinde?
\end{itemize}

Protokoll nächste Sitzung


\section{Dilthey III \\(16.11.17)}

\subsection{Dilthey}

\begin{itemize}
  \item Zentrale Frage: Was ist eine Empfindung?
  \item These: Empfindung auf Grundlage von Widerstand, Zwischenweg zwischen vermittelt und unvermittelt
  \item "`Wir haben gesehen dass die Annahme... Die Widerstandserfahrung nämlich entsteht "'
  \item Vermittelnde Denkvorgänge
  \item Komisch: Was soll dritter Weg zwischen 
  \item Doppeldeutigkeit des Begriffs "`Hören"': Sinnliche Wahrnehmung
  \item Ein und die selbe Empfindung kann mehrere Wies der Gegebenheit haben: 
  \item Gibt es reine Empfindung (ohne dass schon was aufgesetzt wird, "`rau"') oder ist immer alles vermittelt?
  \item Für alle Sinne gibt es Empfindungen, die keine Aufmerksamkeit bekommen und theoretisch da sind
  \item Dilthey: 1. Schritt: Reine Empfindung 2. Widerstandsinterpretation (nicht Schlußverfahren)
  \item Widerstand: Ist das aktiv oder passiv? Also, werde ich gehindert Dinge zu tun? Oder kann ich auch anderen Widerstand empfinden?
\end{itemize}

\subsection{Historie der Phänomenologie}

\begin{itemize}
  \item Historie: Dilthey ist Übergangsfigur, ganz stark von den cartesianischen Prämissen (Unterscheidung von Körper und Geist; Rückzug zum \emph{cogito}, wie kommt dieses zur Welt $\rightarrow$ Dilthey setzt da an: Wie kommt das \emph{cogito} zur Welt?) | Mit diesen Fragen hatte die cartesianische Tradition große Schwierigkeiten; sie ging von der reinen Empfindung aus, und dann wird diese durch das Denken (egal ob bewusst oder nicht) in einem zweiten Schritt bewertet
  \item Phänomenologie (um 1900): Wir müssen nochmal so eine cart. Operation vornehmen: Auf das Ursprüngliche, Unvermittelte, durch sich selbst Gegebene eingehen und dieses in Frage stellen. Grundlegende Bewegung Husserl et al: Reine Empfindung gibt es nicht und in tatsächlich müssen wir uns das Bewusstsein anders vorstellen: Nicht nach dem Abbildmodell, in dem wir eine Verdopplung der Welt haben (Wahrnehmung und Bewertung), dass wir Erfahrung anders als durch das Subjekt-Objekt Modell beschrieben werden muss!
  \item Ich bin immer schon bei diesem \emph{x} auf das ich gerichtet bin.
  \item Subjekt und Objekt sind notwendig aufeinander verwiesen (Intentionalität ist der Bezug auf ein Objekt)
  \item Intentionalität ist nicht "`Gerichtetheit"', denn dies impliziert \emph{Aktivität}!
  \item Husserl: 5. logische Untersuchung (Über Bewusstsein)
  \item Phänomenologische Bewegung beginnt um 1902: Husserls logische Untersuchungen mit Modell des Bewusstseins, sowie methodische Prinzipien; Johannes Daubert, münchener Student aus dem Zirkel von Theodor Lips, berichtete von Husserl und so begann in München reger Austausch über Husserls Thesen | Husserliana: Die Idee der Phänomenologie: Phänomenologische (Transzendentale) Reduktion: Philosophische Technik: Erlaubt es erst, den wahren Sinn der cartesianischen Zweifelsbetrachtung herauszuarbeiten. Erkenntnistheorie muss zeigen, wie Erfahrungsleistungen aufeinander aufbauen und am Anfang dürfen diesbezüglich keine Fehler machen. Bedingungen der Möglichkeit von Erkenntnis. $\rightarrow$ Husserl: \emph{Phänomenologische Reduktion}| Erfahrungsleistungen müssen rekonstruiert werden! SO sehen wir, dass es keine reine Empfindung gibt! | 1. "`Arg"'.: "`Ist doch so wenn man das ganze phänomenologisch betrachtet!"'(\emph{Wesensschau}) | 2. Arg.: Position des anderen destruieren: Wie kommt der Mensch zur Welt, wenn immer erstmal alles vermittelt ist? Wie ist der Übergang zu eklären? | Positive Theorie
  \item Husserl: Seinsgültigkeit der Welt wird ausgeklammert! Da spaltet sich die Bewegung schon! $\rightarrow$ Husserl: transzendentale Phänomenologie vs. Scheler et al: realistische Phänomenologie
  \item Husserl will diesen Prozess, dass Wirklichkeitserfahrung gründet in der Widerstandserfahrung, aufheben. Der Mensch könne die Erfahrung aufheben, asiatische Seelentechniken etc. aber das sei kognitiv gesehen etwas anderes
  \item 1920 gehts wirklich los: Erster fruchtbarer Niederschlag ist "`Erkenntnis und Arbeit"': Zur Philosophie der Wahrnehmung (Abschnitt \textbf{V}, Teil \emph{A} - Eigentlich spannendes Kapitel wird \textbf{VI} sein)
\end{itemize}

Nächstes Mal Protokoll!

\section{Scheler I}

Vorbemerkungen

\begin{itemize}
  \item Besserer Begriff für was Husserl meint: Erfahrung in ihrer Selbstgegebenheit; ( nicht reine empfindung) bedeutung: Wir gehen auf das fundierendste zurück
  \item cartesianisch:
  \item Korrektur Husserl als Methode die Reduktion einführt und dass die anderen das nicht gut finden; Besser: phänom. Reduktion im gegensatz zur zweifelsbetrahtung (im Modus der Meditation) ist eine Philosophische \emph{Einstellung}: Ein Modus, in dem wir alles was wir shcon wissen einklammern (zB was ist ein Körper). Husserl: nennt \emph{transzendentale} Reduktion, weil er die Seinsgeltung der Welt einklammert (weil die Welt zufällig ist) denn er will zum absoluten Wissen (später idealistische Reduktion)
  \item Empfehlung zu Einführung: SPiegellberg: the phenomenological movement erster band; Kürzer: Ferdinand Fellman; zahavi (sehr husserlzentriert) (?) einführung in die phänomenologie
  \item Nur zwei Namen präsent: Husserl, Scheler, Moritz Geiger, Reinach, Tender. Bei letzteren gibt es die reduktion nicht so sehr
  \item Vorwrot der ersten AUflage zeitschrift zur üphänomen forschung (publikationsorgan der bewegung)
  \item 2. Generation: Heideggers sein und zeit
  \item Was zeichnet die Phän Bewegung aus, was ist der Grundgedanke: Theorie des Bewusstseins: Das das Bewusstsein immer intentional. das Bewusstsein ist immer shcon gerichtet auf -> es gibt kein bewusstsein das erst bei sich ist, und dann auf etwas gerichtet ist (nicht wie kommen wir zur welt, wir sind immer schon zur welt)
  \item Vorwort erster Band. Schön aufbereitet: opencommonsphenomenology.com ophen.com
  \item Hintentionalität schwieriger begriff. usserl: mannigfache stufen der intentionalität: Beispiel: gefühle, die bestimmte intentionalität haben (man liebt \emph{jmd}); aber auch nicht intentionale gefühle (Traurigkeit, wo man nicht weiß woher das kommt) Husserl und Scheler: das bedeutet nicht dass diese traurigkeit nicht intentional ist! Intention bedeutet gerade nicht, dass das bzogen siein auf etwas, das beseelt sein, in urteilformen übersetzbar sien können; gerichtetheit kann auch unbestimmt sein..
  \item Leute die die reduktion machen wollen, leute scheler die an die reduktion glauben aber es nicht machen, sie sei nicht transzendental, asiatische seelentechniken, und da passiert was anderes als husserl sich da vorstellt, denn wir können die welt verschwinden lassen, aber dann verschwindet 
  \item scheler beispiel berühmt: die schwierigkeiten der reduktion werden geschickt vernachlässigt: quelle irgendwas mit kosmos) | geist ist was den menschen zum menschen macht und vom tier unterscheidet. Geist kann die fähigkeit zur ideation: geshcichte prinz buddha wächst von allen einflüssen der gemeinen gesellschaft abgeschirmt im elterlichen palast auf, in diesem abgeschirmten leben wird ern ihct konfrontiert mit dem phänomen der armut und dem phänomen der krankheit. er kennt die begriffe nicht. und er kennt die phäno nciht weil er sie nie erlebt hat. der prinz verlässt das elternhaus und sieht am wegesrand einen armen und einen kranken. Scheler: einbeispiel reicht aus, um das wesen der armut und das wesen der krankheit zu erfassen. -> phänomenlogisch erkenntnos. es geht nicht ohne erfahrung, zweitens die begriffe entstehen nicht indem wir ganz viele rfahrungen machen und durchschnittswerte machen sondern umgekehrt. Scheler: erkenntnis ist vorbegrifflich. der prinz ist ja nicht mit mit dem begriff bekannt. wenn er gefragt wird nahc dem wesen der armut, kann er reduktion machen, dann so tun als ob er das alles noch nicht kennen würde und die dinge für sich selbst reden lassen. Überall wo wir zu wesenschau machen 
  \item wirklich sein bedeutet widerstand erfahren, jede erfahrung des soseins gründet in der erfahrung des daseins von etwas. wirklichkeitserfahrung ist also fundamental für das sosein. husserl: umgekehrt, erst wenn ich die widerständigkeit der welt aufhebe, kann ich zum sosein kommen!
  \item Eine These des Textes: allein die these, dass wir zwischen geist und körper unterscheiden ist auch schon vorurteil einer tradition. wir wollen also diese begriffe einklammern, ebenso den der empfindung, der ist historisch belastet und kontaminiert. und so anfangen
\end{itemize}

V zur philosophie der wahrnehmung

\begin{itemize}
  \item müssen selektiv vorgehen, 
  \item vorgehen: thesen am text rekonstruieren
  \item Das große problem am empfindungsbegriff, er muss erklären, wie wir wissen von der welt haben können
  \item warum kann es nicht stimmen, dass wir erst nicht zur welt sind an x0 und zur welt kommen an einem zeitpunkt x1: wir kommen in zirkelprobleme wenn wir erklären wollen .. (???)
  \item 2 Lösungen: 1. das bezogen sein auf etwas ist immer schon ein bezogen sein auf gestalten! (im Text}: 2. Sagen: der phänomenlogische begriff des bewusstseins dass wir immer schon zur welt sind, (vorherige Sitzung)
  \item Definition für Gestalt geben. Gestaltbegriff ist sehr weit: Gestalten werden wahrgenommen, man kann auch gestalten fühlen. 
  \item Bei der Wahrnehmung dürfen wir nicht so vorgehen, dass wir kleine teile annehmen die wir zu einem größeren zusammensetzen, sondern umgekehrt. Empfindung ist also nur der rechenpfennig.
  \item Frage: warum ist es so, dass wir immer schon gestalten wahrnehmen?
  \item Was passiert, wenn ich aus farbklexen auf einem bild wenn ich weggehe ein gesamtbildung, eine gestalt auf einmal sehe.
  \item Beispiel: Stimmengewirr, wirkt wie Empfindung; sobald ich ein Gespräch habe nicht mehr (Denn: Worte sind Gestalt von Lauten);dies gilt auch für Sprachen die wir nicht sprechen (Denn schon etwas (auch fälschlich) $als$ Sprache zu hören ist eine Gestaltwahrnehmung) 
  \item Frage: Gibt es das was wir als reine Empfindung bezeichnen überhaupt?: Selbst Lärm nehmen wir als Lärm $von \ etwas$ wahr.
  \item Jede Form von Warhnehmung ist Gestaltwahrnehmung
\end{itemize}

Thesen

\begin{itemize}
  \item 322 unten: fragt noch einmal: was heißt denn jetzt was wir mit empfindung meinen; bedeutet empfindung denn noch etwas das existiert... Ja sie ist ein rechenpfennig
  \item 321 3 reize beispiel: keine klare verbindung zwischen reiz und empfindung
  \item Konstanzannahme: beispiel dass die sich nicht funktioniert: streicheln über den hinterkopf und dann fühlt es sich plötzlich ganz anders an; dabei ist der reiz ja eigentlich der selbe
  \item Ich kann keine reine empfindung haben. scheler: je reiner die empfindung wird, desto mehr verschwindet sie. 
  \item Bei gestlat gibt es nur an aus schalter, perspektivenwechsel. bei fühlen von etwas kann es graduell sein.
  \item Gestalten wahrnehmen, ist das angeboren Oder beigebracht? | Angeboren/Triebhaft, weil ansonsten nie zur gestalt gekommen werden kann. Konrad Lorenz spricht von den angeborenen Formen möglicher Erfahrung (-> Gestalten) | Kindchenschema
  \item Bestimmte Formen von Gestalten sind dann angeboren, Beispiel Autismus: Mimik nicht erkennbar; Universale Strukturen zur möglichkeit des Erkennens sind angeboren | Natürlich lernen babies erst wahrzunehmen
  \item Wahrnehmung immer als Gestalt; Empfindung 2 ideen: pure empfindung (fiktion), oder Teil von erlebnisse, erfahrung, cogitationes, haben immer einen empfindungscharakter (wie fühlt es sich denn an?); Empfindungen sind mitkonstitutiv..
  \item Ist diese Frage wichtig?
  \item Gibt Scheler uns eine Antwort?
\end{itemize}

Nächste Sitzung: 296, 297; Letzter Abschnitt VI 

\section{Scheler II\\30.11.2017}

\subsection{Anmerkungen}

\begin{itemize}
  \item Erkenntnistheorie als Grundlagenforschung in der Philosophie: als Projekt der Zurückführung um genau zu zeigen wie bestimmte erkenntnisleistungen aufeinander aufbauen (wie fundiert Erfahrung A die Erfahrung B? (wenn B dann A))
  \item Phänomenologisches Argument gegen logischen Schluß bei Aussagen über die Erfahrung
  \item Reduktion ist kein Gegenentwurf sondern eher eine Radikalisierung, nicht: "`ich zweifels jetzt mal A, B und C an."'
  \item Erfahrung brauch nicht symbolische Begriffe
  \item Kritik am Buddha Beispiel: 
  \item Grundsätzliche Schwierigkeiten für Philosphen: Dass wir in einer ganz konkreten Ontologie (Art und Weise, wie wir die Welt in verschiedene Substanzen unterteilen) aufwachsen; zB \emph{Res extensa} und \emph{Res cogitans}
\end{itemize}

\begin{itemize}
  \item Sphären als Fundierungstheorie
  \item Dilthey zwei Möglichkeiten wie Psyche funktioniert
  \item Summativ: Ich bau mir die Dinge zusammen
  \item Mir ist immer schon die gesamte Gestalt gegeben
  \item Schelers Grundargumentation: Es gibt keine Empfindung, die für sich ist
  \item Einwurf von Student: 2 Widerstände
  \item Scheler: sogenannte nichtidologische Theorie (einfache Formen des Bewusstseins, die nicht in der Lage sein müssen)
  \item Primäre Bewusstseinsimmanenz
  \item ekstatisches wissen: vitalistische Umschreiben der Intentionalität, ein Leben das auf die Welt ausgerichtet ist
  \item 4 Sphären, die chronologisch durchlebt werden müssen: 1. Es gibt \emph{etwas} | 2. Du-sphäre (Du Tisch) | 3. 
  \item In der ersten Sphäre gründet schon 
  \item Scheler: zunächst ist unser Bewusstsein ein strudel aus Ich-, du- unbewerteten Erlebnissen, und aus diesem entsteht erst später eine Ich-Du-Unterscheidung. Das Du ist vor dem ich
  \item Der Unterscheidung zwischen lebendig und nichtlebendig 
\end{itemize}

\begin{itemize}
  \item Leerer Blick: keine intentionalität, kein Bewusstsein?
  \item Schleyermacher: Religionsdefinition: die schlechthinnige Abhängigkeit 
  \item Crusoe, Cast Away als spannende Experimentierfelder
\end{itemize}

Nächste Sitzung: Wie intensivieren und entschläunigen, daher: wir lesen bis IV, das sphärenproblem: S 196; Idealismus Realismus Teil 2: 

Werkausgabe! wir noch hochgeladen

\newpage
%\section{"Uber den Professor}
%Matthias Schlo"sberger ist Heisenbergstipendiat der Deutschen Forschungsgemeinschaft
%an der Humboldt Universit"at zu Berlin mit dem Forschungsprojekt "`Die Erfahrung der Realit"at durch Widerstand"'.
%
%\begin{figure}[h]
%	\centering
%	\includegraphics[width=0.3\textwidth]{images/Matthias_Schlossberger.png}
%	\caption{Matthias Schlo"sberger}
%	\label{fig:MS}
%\end{figure}


%\begin{figure}[h]
%	\centering
%	\includegraphics[width=0.5\textwidth]{images/template.png}
%	\caption{Template Bild}
%	\label{fig:template}
%\end{figure}

\end{document}
