\documentclass[a4paper, 12pt]{article}
%\usepackage{CJKutf8} % japanese
\usepackage{graphicx}
\usepackage{hyperref}
\usepackage{fullpage}
%\usepackage{parskip}
\usepackage{color}
\usepackage[ngerman]{babel}
\usepackage{hyperref}
\usepackage{calc} 
\usepackage{enumitem}
\usepackage[utf8]{inputenc}
\usepackage{titlesec}
%\pagestyle{headings}
\usepackage{setspace} %halbzeilig
\usepackage[style=authoryear-ibid,natbib=true]{biblatex}
\usepackage[hang]{footmisc}
\setlength{\footnotemargin}{-0.8em}
%\bibliographystyle{natdin}
\addbibresource{kritische-theorie-hausarbeit.bib}
\DeclareDatamodelEntrytypes{standard}
\DeclareDatamodelEntryfields[standard]{type,number}
\DeclareBibliographyDriver{standard}{%
  \usebibmacro{bibindex}%
  \usebibmacro{begentry}%
  \usebibmacro{author}%
  \setunit{\labelnamepunct}\newblock
  \usebibmacro{title}%
  \newunit\newblock
  \printfield{number}%
  \setunit{\addspace}\newblock
  \printfield[parens]{type}%
  \newunit\newblock
  \usebibmacro{location+date}%
  \newunit\newblock
  \iftoggle{bbx:url}
    {\usebibmacro{url+urldate}}
    {}%
  \newunit\newblock
  \usebibmacro{addendum+pubstate}%
  \setunit{\bibpagerefpunct}\newblock
  \usebibmacro{pageref}%
  \newunit\newblock
  \usebibmacro{related}%
  \usebibmacro{finentry}}

%\titleformat{name=\section,numberless}
%  {\normalfont\Large\bfseries}
%  {}
%  {0pt}
%  {}
\date{\vspace{-3ex}}
\begin{document}

\title{\vspace{5ex}
	\includegraphics*[bb=0 0 720 200, width=0.72\textwidth]{ErstesSem/images/hu_logo.png}\\
	\vspace{30pt}
	\scshape\LARGE{Der Rückschritt im Fortschritt}\\\Large{Eine Analyse des Fortschrittsbegriffs in der Dialektik der Aufklärung im Anbetracht der Kalifornischen Ideologie}\\\vspace{20pt}}
	


\author{Kritische Theorie der Gesellschaft: Horkheimer und Adorno (PS)\\
	\vspace{7pt}
          Dozent: Dr. Arnd Pollmann\\\vspace{4pt}Lennard Wolf\\
        \small{Matrikelnummer: 583052}\\
        \small{E-Mail: lennard.wolf@student.hu-berlin.de}\\
        \small{Telefonnummer: +49 176 5687 4131}\\
        \small{Studiengang: B.A. Philosophie}\\
        \small{Modul: Praktische Philosophie}}

        %\href{mailto:lennard.wolf@student.hu-berlin.de}{lennard.wolf@student.hu-berlin.de}}}      

\maketitle

\vspace{\fill}

\begin{minipage}[]{0.92\textwidth}
    \centering
    \onehalfspacing
    \large   
    30. September 2017\\
    Sommersemester 2017

    \vspace{-20mm} 
\end{minipage}%
\thispagestyle{empty}
\newpage
%\clearpage
%\thispagestyle{empty}
%\tableofcontents
%\newpage
\setcounter{page}{1}

\begin{onehalfspace} 

\noindent\textbf{$(o)$ Einleitung}

\noindent 

\begin{itemize}
  \item 
  \item Der Aufruf in Californian Ideology, dass Europa nicht mitgehen soll erscheint in weiten Teilen gescheitert. 
  \item Zwar wird sich gegen Uber und AirbnB, den höchstgewerteten Startups, gewehrt, und facebook muss kommentare löschen etc, doch Occupy Wallstreet in Berlin dauerte nur kurz an, der CCC ist zu einer Party mit ein paar kritischen, aber folgelosen Präsentationen verkommen, die Piratenpartei befindet sich im Zerfall, die meistbesuchten Seiten sind… und ein Entfliehen von der ‘Datenkrake’ Amerika ist auch nicht im Sicht. 
\item Während die Kalifornische Ideologie ins geistige Sediment der virtuellen Klasse in Deutschland sickert, Privatinstitutionen wie das Hasso-Plattner-institut entstehen, an denen, wenn möglich, die nächsten deutschen Mark Zuckerbergs herangezüchtet werden sollen und macht sich ein immer größer werdendes Gefühl der Machtlosigkeit vonseiten der Bevölkerung wie der Politiker breit.
\item A und H kommen zum schluß, dass die rationalistische Form der AUfklärung äußerst problematisch ist
\item Ziel meiner Darstellung soll sein, den Reflex, auf alles Gute das die Aufklärung gebracht hat zu verweisen und es abzuwägen mit dem leid, als nicht haltbar zu entlarven.
\end{itemize}


Daf"ur werde ich wie folgt vorgehen. In Abschnitt $(i)$ beschreibe ich 
\vspace{5mm}

\noindent\textbf{$(i)$ Die Dialektik der Aufklärung}

\noindent "`Der Mythos wollte berichten, nennen, den Ursprung sagen: damit aber darstellen, festhalten, erklären"'\footnote{\Cite[Siehe][S. 14]{dialektik-der-aufklaerung}.}.

High Level View

\begin{itemize}
  \item Kritische Theorie als Gegenentwurf zur traditionellen, die Gesellschaftliche Verhältnisse als notwendig akzeptieren
  \item Bsp: amerikanischen Institutionen wie Stanford im 			Bereich der Computer Science gang und gebe ist, wo die Institute sogar Namen			von erfolgreichen Leute nhaben (Gates Computer Science Building, Hewlett 			Teaching Center) 
  \item Ziele: zunehmende Selbstbestimmung der Menschen, eine befreite Gesellschaft (?)
  \item Ziel der Dialektik der Aufklärung: Wie konnte die Aufklärung den Holocaust hervorbringen, wenn sie doch reiner Fortschritt sein will? Eine AUfklärung der Aufklärung, eine Kritik von innen.
  \item Vorgehen: Befangenheit, also mythologisches in der Aufklärung, aufzeigen. Doch machen nicht bei Marx'scher Ideologiekritik halt, sondern zeigen dass die Ideologiekritik auch ihrer Grundlage zu Hoffnunf entledigt wurde. Wollen aber an Aufklärung festhalten, da sie in sich den wahrhaften Kern der Suche nach Wahrheit in sich trägt.
  \item Ergebnis: Aufklärung hat, in seiner derzeitigen Form, notwendig regressive 				Nebenwirkungen (Vernunft wurde zweckbestimmt etc)
  \item totalisierte Kritik hinterlässt ein Gefühl der Ohnmacht. 'man kann auf lust hoffen in hölle' zeigt, wie die Ideologiekritik selber ihrer Macht enthoben wurde. Somit bleibt nichts mehr übrig, das getan werden kann. -> Silicon Valley versucht das gegenteil zu beweisen. Leute wie Alex Karp.
\end{itemize}

Low Level View

\begin{itemize}
  \item was ist aufklärung
  \item was ist mythos
  \item wie bedingt sich a aus m:  A speist sich aus M, hat damit aber keine eigenständige Existenz, Grundlage
  \item neue Grundlage die geschaffen wird: Rationalistischer Zweck Mittel Denken, historisch möglicherweise erklärbar aus Grundproblemen, wurden immer weiter angewandt auf alle neuen kontexte - struktur des kapitalismus - struktur der liebe
  \item  
\end{itemize}


\vspace{5mm}
\noindent\textbf{$(ii)$ Die Dialektik des Fortschritts}

\noindent 

\begin{itemize}
  \item was ist fortschritt
  \item die instrumentelle vernunft axiomatisert fortschritt
  \item 
\end{itemize}


\vspace{5mm}
\noindent\textbf{$(iii)$ Die Dialektik der Kalifornische Ideologie}

\noindent 

\begin{itemize}
  \item Tim Cook als homosexueller CEO des am wertvollsten gehandelten Unternehmens der welt hat sich öffentlich gegen Trump geäußert
  \item Die Herrschaftsstrukturen des kapitalistischen Marktes werden (möglicherweise murrend) hingenommen, um einen Fortschritt im Zusammensein 
  \item Fortschritt vs Disruption
  \item talking heads typs text
  \item 
\end{itemize}


\vspace{5mm}
\noindent\textbf{$(iv)$ Konklusion}

\noindent  



\end{onehalfspace}
\nocite{*}
\printbibliography
\end{document}
