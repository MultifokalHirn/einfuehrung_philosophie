\documentclass[a4paper, 12pt]{article}
%\usepackage{CJKutf8} % japanese
\usepackage{graphicx}
\usepackage{hyperref}
\usepackage{fullpage}
%\usepackage{parskip}
\usepackage{color}
\usepackage[ngerman]{babel}
\usepackage{hyperref}
\usepackage{calc} 
\usepackage{enumitem}
\usepackage[utf8]{inputenc}
\usepackage{titlesec}
%\pagestyle{headings}
\usepackage{setspace} %halbzeilig
\usepackage[style=authoryear-ibid,natbib=true]{biblatex}
\usepackage[hang]{footmisc}
\setlength{\footnotemargin}{-0.8em}
%\bibliographystyle{natdin}
\addbibresource{kritische-theorie-hausarbeit.bib}
\DeclareDatamodelEntrytypes{standard}
\DeclareDatamodelEntryfields[standard]{type,number}
\DeclareBibliographyDriver{standard}{%
  \usebibmacro{bibindex}%
  \usebibmacro{begentry}%
  \usebibmacro{author}%
  \setunit{\labelnamepunct}\newblock
  \usebibmacro{title}%
  \newunit\newblock
  \printfield{number}%
  \setunit{\addspace}\newblock
  \printfield[parens]{type}%
  \newunit\newblock
  \usebibmacro{location+date}%
  \newunit\newblock
  \iftoggle{bbx:url}
    {\usebibmacro{url+urldate}}
    {}%
  \newunit\newblock
  \usebibmacro{addendum+pubstate}%
  \setunit{\bibpagerefpunct}\newblock
  \usebibmacro{pageref}%
  \newunit\newblock
  \usebibmacro{related}%
  \usebibmacro{finentry}}

%\titleformat{name=\section,numberless}
%  {\normalfont\Large\bfseries}
%  {}
%  {0pt}
%  {}
\date{\vspace{-3ex}}


\begin{document}

\title{\vspace{5ex}
	\includegraphics*[bb=0 0 720 200, width=0.72\textwidth]{ErstesSem/images/hu_logo.png}\\
	\vspace{30pt}
	\scshape\LARGE{Regressiver Fortschritt}\\\Large{Zur Dialektik der Aufklärung und dem ihr zentralen Fortschrittsgedanken}\\\vspace{20pt}}
	


\author{Kritische Theorie der Gesellschaft: Horkheimer und Adorno (PS)\\
	\vspace{7pt}
          Dozent: Dr. Arnd Pollmann\\\vspace{4pt}Lennard Wolf\\
        \small{Matrikelnummer: 583052}\\
        \small{E-Mail: lennard.wolf@hu-berlin.de}\\
        \small{Telefonnummer: +49 176 5687 4131}\\
        \small{Studiengang: B.A. Philosophie}\\
        \small{Modul: Praktische Philosophie}}

        %\href{mailto:lennard.wolf@student.hu-berlin.de}{lennard.wolf@student.hu-berlin.de}}}      

\maketitle

\vspace{\fill}

\begin{minipage}[]{0.92\textwidth}
    \centering
    \onehalfspacing
    \large   
    30. September 2017\\
    Sommersemester 2017

    \vspace{-20mm} 
\end{minipage}%
\thispagestyle{empty}
\newpage
%\clearpage
%\thispagestyle{empty}
%\tableofcontents
%\newpage
\setcounter{page}{1}

\begin{onehalfspace} 

\noindent\textbf{$(o)$ Einleitung}

\noindent %wichtig unterschied Regression und Regress!

\begin{itemize}

\item Ziel meiner Darstellung soll sein, den Reflex, auf alles Gute das die Aufklärung gebracht hat zu verweisen und es abzuwägen mit dem leid, als nicht haltbar zu entlarven.?
   \item Was ist kritische theorie? Kritische Theorie als Gegenentwurf zur traditionellen, die Gesellschaftliche Verhältnisse als notwendig akzeptieren
  \item aus welchen fragen ist dsie entstanden?
  \item Wie fügt sich DdA in die KT als solche ein?
\end{itemize}

Begriffsklärung a la Mit "`Aufklärung"' ist nicht ein Zeitalter gemeint, sondern, grob gefasst, das geschichtsphilosophische Phänomen des Fortschrittsdrangs durch rationalen Diskurs.

Unterschied Aufklärung und Fortschritt: Fortschritt ist das Ziel der Aufklärung. Die Aufklärung nimmt sich Fortschritt als Werkzeug, um Fortschritt zu erreichen. achtung, sag ich unten auch nochmal so


Daf"ur werde ich wie folgt vorgehen. In Abschnitt $(i)$ gebe ich meine Auslegung der Methode und Erkenntnis der \emph{Dialektik der Aufklärung}. 

\vspace{5mm}


\noindent\textbf{$(i)$ Die Dialektik der Aufklärung}

\noindent Die \emph{Dialektik der Aufklärung}, 1944 im kalifornischen Exil der Autoren fertig gestellt, gilt als eines der wirkmächtigsten\footnote{\Cite[Vgl.][S. 249]{jaeggi}.} Werke der Kritischen Theorie. Ihr Ausgangspunkt ist die Frage, "`warum die Menschheit, anstatt in einen wahrhaft menschlichen Zustand einzutreten, in eine neue Art von Barbarei versinkt"'\footnote{\Cite[Siehe][S. 1]{dialektik-der-aufklaerung}.}. Wenn doch die Aufklärung, die sich programmatisch den "`Ausgang des Menschen aus seiner selbst verschuldeten Unmündigkeit"'\footnote{\Cite[Siehe][S. 481]{kant}.}, oder in anderen Worten, eine Befreiung von Gesellschaft und Verstand, auf die Fahne geschrieben hat, wie konnte es zu Schrecken des zweiten Weltkriegs kommen? Diese scheinbare Widersprüchlichkeit führt die Autoren zu dem Versuch einer \emph{Aufklärung der Aufklärung}\footnote{\Cite[Vgl.][S. 406]{habermas}.}, der zufolge die barbarischen Zustände nicht \emph{trotz} des aufklärerischen Denkens ausbrachen, sondern dass sie gerade dessen \emph{notwendige Konsequenz} sind. Die Aufklärung der Aufklärung vollziehen die Autoren auf zwei Ebenen. Zum einen üben sie Kritik an der hegemonialen Ideologie, das heißt der allumfassenden, allem gesellschaftlichen Handeln zugrunde liegenden bürgerlichen Gesinnung, indem sie deren Widersprüchlichkeiten aufzeigen. Zum anderen geben sie der Kritik noch ein selbstreflexives Moment hinzu. Sie fragen nach der Möglichkeit der Kritik selbst, da eine diese selber, aufgrund einer Absicht der Verbesserung der Umstände, in der kritisierten zweckrationalen Denkweise verhaftet bleibt. Die sich aus diesen Untersuchungen ergebende Kernthese der \emph{Dialektik der Aufklärung} ist, dass die Aufklärung, in der Form, die sie zumindest um 1944 hat, zum "`totalen Betrug der Massen"'\footnote{\Cite[Siehe][S. 49]{dialektik-der-aufklaerung}.} geworden ist. Die \emph{Fragmente} des Werkes sind eine Vorrede, ein geschichtsphilosophischer Einführungsessay mit dem Titel "`Begriff der Aufklärung"', zwei Exkurse "`Mythos und Aufklärung"' und "`Aufklärung und Moral"', zwei zeitdiagnostische Essays zu Kulturindustrie und Antisemitismus, sowie eine angehängte Sammlung von Aufzeichnungen und Entwürfen. Die schon in seinem Titel angedeutete fragmentarische Struktur des Buches findet sich auch in den einzelnen Essays selber wieder, weshalb Jürgen Habermas dem Buch eine "`eher unübersichtliche Form der Darstellung"'\footnote{\Cite[Siehe][S. 406]{habermas}.} unterstellt und Rahel Jaeggi den Argumentationsgang als "`eliptisch [sic]"'\footnote{\Cite[Siehe][S. 250]{jaeggi}.} anmutend bezeichnet. Dieser Umstand erschwert die Lektüre des Werkes und verschleiert auch zuweilen die Intention der Autoren.

Es lohnt sich, zu Beginn die im Buchtitel befindlichen Wörter "`Dialektik"' und "`Aufklärung"' zu klären. "`Dialektik"' bezieht sich nach Adorno sowohl auf die Struktur des behandelten Gegenstands, als auch auf die dadurch notwendige Methode.\footnote{\Cite[Vgl.][S. 9]{dialektik}.} Es wird also versucht, die Aufklärung zu verstehen, indem man eine nie stehenbleibende Korrektur von der Beziehung zwischen dem \emph{gedanklichen} Begriff und der Sache, auf die er sich bezieht, vollzieht. Da die Autoren anfangs aufzeigen, dass sich die Aufklärung als Negation des \emph{Mythos} versteht, der dadurch zugleich zu ihrem Nährboden wird, lässt sich schon eine dialektische "`Verschlingung von Mythos und Aufklärung"'\footnote{Siehe \Cite{habermas}.} erahnen. Der Mythos ist, so wird sich zeigen, die Wurzel der Aufklärung, doch deren bisherige Unfähigkeit, sich entgültig von ihm zu trennen, führt sie in die Selbstzerstörung.

Ihren Ursprung findet die Aufklärung in dem Ziel, allen Aberglauben, allen Mythos aus der Welt zu verbannen und so den Glauben mit Wissen auszutauschen. Sie ist der Prozess der "`Bezwingung mythischer Gewalten"'\footnote{\Cite[Siehe][S. 408]{habermas}.}, der Entzauberung. Bezüglich des Ursprungs von Mythen zeigen die Autoren zu Beginn, dass diese aber immer schon aus aufklärerischem Handeln entstanden sind. Der Mensch, als stets wachsames Lebewesen, hat ein durch Angst hervorgerufenes Bedürfnis, die Welt um sich herum zu verstehen, um durch Vorausschau sich besser vor Gefahren schützen zu können. Durch die Weitergabe von Erfahrungen an die Nachkommen, konnte so ein Kollektivwissen aufgebaut werden, das sicherlich zum größten Teil empirischer Natur war. Sollte man sich erzählt haben, die Welt sei eine flache Scheibe, mit Land in der Mitte, das umgeben ist von Wasser, das möglicherweise an den Rändern hinunterfliesst, so war das weniger primitiver Aberglaube, als ein rein empirisches Forschungsergebnis, dessen Lücken mit aus dem Gegebenen extrapolierten, logisch erscheinenden Ergänzungen gefüllt werden. "`Der Mythos wollte berichten, nennen, den Ursprung sagen: damit aber darstellen, festhalten, erklären"'\footnote{\Cite[Siehe][S. 14]{dialektik-der-aufklaerung}.}, und so wurde aus den Berichten über die Generationen hinweg autoritäres "`Wissen"', das mit aus den Hinzudichtungen entstandenen Mystifikationen gespickt war. Die sozialen, wie verstandesmäßigen Folgen waren Unfreiheit und Barbarei, da den Menschen vorgegeben wurde, wie die Dinge sind, wie sie zu sein haben und daraus folgend, wie die Menschen sich dem anzupassen haben. Aus solchen Mythen nährt sich die Aufklärung und nahm sich dem Ziel an, den Menschen und seinen Verstand aus der Herrschaft zu befreien. Doch dieses Programm ist in seinem Kern das selbe wie jenes, mit dem der zu bekämpfende Mythos einst entstanden ist, dessen Inhalte die Aufklärung mit subjektiver Projektion verwechselt.\footnote{\Cite[Vgl.][S. 12]{dialektik-der-aufklaerung}.} Dieser Ursprung ist der Trieb zur Selbsterhaltung und damit auch die Angst vor der Beherrschung durch die Natur, der in einstigem Mythos noch durch eine ehrfürchtige Selbstanpassung entgegengewirkt wurde. In der Aufklärung aber durch Beherrschung der Natur. 

Die Autoren identifizieren die zweckgerichtete, \emph{instrumentelle} Vernunft als ein fundamentales Problem der Aufklärung. Für sie reicht Wissen, durch das der Mensch Natur bändigen und sich dieser besser anpassen kann, auch schon nicht mehr aus. Wissen muss nicht nur die Möglichkeit zu reaktivem Handeln erfüllen, sondern aktive Kreation ermöglichen, es muss ein Instrument sein. Es gilt daher: Ich kenne die Dinge, wenn ich sie manipulieren und machen kann - \emph{Wissen ist Macht}.\footnote{\Cite[Vgl.][S. 15]{dialektik-der-aufklaerung}.} Der aufklärende Mensch erhöht sich so zu einem göttlichen Gebieter über die Natur und die Welt.\footnote{\Cite[Vgl.][S. 15]{dialektik-der-aufklaerung}.} Um diese Machtposition zu begründen und zu behalten, wird allen Erkenntnissen, wie auch denen der Naturwissenschaften, immer weiter der Maßstab der Verwertbarkeit angelegt. So geht heutzutage sogar ein Thema wie die Abstrakte Algebra nur als lehrwürdig durch, weil es die \emph{abstract thinking skills} der Studierenden steigert, nicht aber zum Beispiel wegen ihres ästhetischen \emph{Eigenwertes}, denn Eigenwert gibt es für die instrumentelle Vernunft nicht. Alles in der Welt wird gemessen und damit messbar, alles Lebendige wird unlebendig und definierbar gemacht. Und so radikalisiert sich die Angst, die einst das Unlebendige zum Lebendigen emporhob, und ruft nach einem vor nichts halt machenden, alles durch Abstraktion komparabel oder gleich machenden Allwissen, für das es kein Unbekanntes und Unbenanntes mehr geben darf.\footnote{\Cite[Vgl.][S. 22 f.]{dialektik-der-aufklaerung}.} Der technische und gesellschaftliche Fortschritt, der kein Blick nach vorne, sondern nach hinten wurde, betrügt sich selbst, indem er das positivistische Welt\emph{bild} mit der Welt an sich verwechselt. In der mathematisierten Welt ist alles schon vorherbestimmt, das mir unbekannte hat bestimmt irgendwer schon erforscht, der mir sagt was es ist, und Probleme lasse ich lieber von Algorithmen lösen - Verstand und Fantasie verkümmern\footnote{\Cite[Vgl.][S. 42]{dialektik-der-aufklaerung}.} im Angesicht der objektiven, determinierten Welt. Das konstante Differenzieren der Begriffe, das den Kern der Erkenntnis ausmachen soll, führen aber ebenso entweder zur Relativierung von Wahrheit und Moral oder zu Vorstellungen von der Erklärbarkeit von Moral anhand der Naturwissenschaften\footnote{Siehe hierzu beispielsweise: Sam Harris (2010). \emph{The Moral Landscape: How Science Can Determine Human Values}. New York: Free Press.}, da diese ja die objektive Wahrheit beschreiben. Indem die Betrachtungen der Wissenschaften immer weiter auf teilnahmsloses Datensammeln herunter gebrochen werden, wird den Menschen nicht nur die Möglichkeit der qualitativen Erkenntnis genommen, sondern auch jede Hoffnung darauf, dass die Dinge anders sein könnten. In diesem Anspruch, die einzig mögliche Wahrheit im Handeln wie im Denken zu sein, zeigt sich, dass "`Aufklärung [...] totalitär wie nur irgendein System"'\footnote{\Cite[Siehe][S. 31]{dialektik-der-aufklaerung}.} ist. 

Die instrumentelle Vernunft objektiviert und objektifiziert nicht nur die Natur und sich selbst, sondern auch den Menschen selber, den sie auf einen quantifizierbaren, austauschbaren Haufen von Atomen reduziert. Dieser kann stets einer Maschine gleich optimiert werden, sei es bezüglich des Aussehens, der Fitness oder der inneren Ausgeglichenheit. Damit das Ziel der Freiheit erreicht werden kann, werden Werkzeuge wie Demokratie und Selbstkontrolle eingesetzt, in deren Namen das Individuum aber wieder angepasst, das heißt beherrscht und freiwillig fügsam gemacht wird. "`But lo! men have become the tools of their tools"'\footnote{\Cite[Siehe][S. 37]{walden}.}. Der Mensch wird also seiner selbst Herr, und nimmt damit nur die Position einer vermittelnden Instanz der Beherrschung durch die Natur ein. Weil das geltende Wissen, dem ein objektiver Wahrheitswert unterstellt wird, als eins mit den "`naturgegebenen"', da einzig als valide anerkannten Machtstrukturen der aufgeklärten Gesellschaft wahrgenommen wird,\footnote{\Cite[Vgl.][S. 411 XXX]{habermas}.} und deren tatsächliche Kontingenz selten Beachtung findet, wird das Individuum um das eigene, freie Denken beraubt, ohnmächtig in das Kollektiv gezwungen und somit um das eigentliche Versprechen seiner Befreiung betrogen. Im Namen der Freiheit wird also Unfreiheit geschaffen und aus dem ursprünglichen Aufbruch in die freie Gesellschaft, deren Verstand sich selbst und die Welt mit Klarheit erblicken sollte, wurde ein Weg in sich selbst widersprechende Verblendungszusammenhänge, die alles und jeden unterdrücken. So schlägt die erkenntnistheoretische Bewegung der "`Aufklärung in Mythologie zurück"',\footnote{\Cite[Siehe][S. 6]{dialektik-der-aufklaerung}.} und gleicht einem vor sich selbst die Augen verschließenden, ziellos durch die Gegend rasenden und alles in seinem Weg zertrampelnden Riesen\footnote{\Cite[Vgl.][S. 25]{fortschritt}.}. Die abscheulichen Massenmorde und totalitären Regime des 20. Jahrhunderts sind also keinesfalls von der Aufklärung getrennt, sondern die letzte Konsequenz eines unreflektierten Fortschrittswahns.

Diese "`Dialektik der Aufklärung"' ist die Kernbeobachtung des so betitelten Werkes und keineswegs als eine destruktive Kritik zu verstehen, die zu einer Überwindung der Aufklärung hinführen will. Horkheimer und Adorno sind der Überzeugung, "`daß die Freiheit in der Gesellschaft vom aufklärenden Denken unabtrennbar ist"'\footnote{\Cite[Siehe][S. 3]{dialektik-der-aufklaerung}.} und dass die so "`an Aufklärung geübte Kritik [...] einen positiven Begriff von ihr vorbereiten [soll], der sie aus ihrer Verstrickung in blinder Herrschaft löst"'\footnote{\Cite[Siehe][S. 6]{dialektik-der-aufklaerung}.}. Die Aufklärung soll ihr mythologisches Erbe abgeben und über Mythoskritik hinausgehende Selbstreflexion betreiben. Erst die Aufklärung, die sich nicht selbst mythologisiert, so scheint die These gemeint zu sein, wird den Menschen in die Freiheit leiten können. Doch wie solch eine Aufklärung aussehen könnte diskutieren die Autoren nicht. Max Horkheimer schrieb dazu nachträglich passend: "`Ich bekenne mich zur kritischen Theorie; das heißt, ich kann sagen, was falsch ist, aber ich kann nicht definieren, was richtig ist"'\footnote{\Cite[Siehe][S. 150]{gesellschaft}.}. Es bleibt also den Lesenden überlassen, was aus alledem für sie persönlich und für die Gesellschaft als Ganzes folgt. 25 Jahre später schrieb Adorno in \emph{Minima Moralia} dazu, wie man sich überhaupt noch in der Gesellschaft verhalten könne: "`Das einzige, was sich verantworten läßt, ist, den ideologischen Mißbrauch der eigenen Existenz sich zu versagen und im übrigen privat so bescheiden, unscheinbar und unprätentiös sich zu benehmen, wie es längst nicht mehr die gute Erziehung, wohl aber die Scham darüber gebietet, daß einem in der Hölle noch die Luft zum Atmen bleibt"'\footnote{\Cite[Siehe][S. 24]{minima}.}. Solche Sätze zeugen eher von einer düsteren Hoffnungslosigkeit, und hinterlassen einen Nachgeschmack \`{a} la "`Wie man's macht, macht man's falsch"'. Wie kann der Mensch, der zurückhaltend lebt, denn überhaupt die von den Autoren so negativ beschriebenen Umstände ändern? Oder ist dies schon gar nicht mehr möglich? Heute scheinen die Massenmorde in den Konzentrationslagern und Gulags zwar vorerst überwunden, doch Krieg und Unterdrückung herrschen noch immer vielerorts. Immer weniger Menschen müssen Hungern auf der Welt, Kindersterblichkeit sinkt von Jahr zu Jahr, doch Fremdenfeindlichkeit hat seither nicht nachgelassen. Die die Kritik aus dem Essay "`Kulturindustrie"' wirkt heute im Rückblick noch sehr viel relevanter, als zur Zeit seiner Veröffentlichung, was eher darauf hindeutet, dass die Verblendungszusammenhänge nur noch allumfassender geworden sind. Es bleibt die Frage, ob die Menschheit seit 1944 einen "`Fortschritt"' gemacht hat, und was das für eine eine sich selbst aufklärende Aufklärung überhaupt bedeuten würde. Mit diesen Fragen werde ich mich nun im nächsten Abschnitt beschäftigen.

\vspace{5mm}
\noindent\textbf{$(ii)$ Die Dialektik des Fortschritts}

\noindent Fortschritt ist die versuchte Grundbewegung der Aufklärung. Ganz abstrakt ist sie die Veränderung von einem Zustand weg, hin zu einem besseren. Das Neue muss \emph{besser} sein, denn ansonsten gäbe es kein \emph{Fort}schritt, sondern nur eine Veränderung, oder gar ein \emph{Rück}schritt. Die zwei ganz basalen Fragen, die ein Fortschrittsvorhaben beantworten müsste, wären daher, wieso das Bestehende unzureichend sei und inwiefern das Neue besser wäre. Hieraus entstehen schon zwei grundlegende Probleme. Es ist nämlich fraglich, \emph{welches} Maß für "`besser"' und "`schlechter"' angelegt wird und \emph{wo} dieses Maß überhaupt angelegt werden soll, da das Neue ist während der Planung noch gar nicht eingetreten ist. Überhaupt kann erst von der Richtung einer Bewegung gesprochen werden, wenn man die Totalität aller Zustände kennt. Etwas als "`fortschrittlich"' zu bezeichnen ist also tendenziell erstmal problematisch. 

Dies ändert jedoch nichts daran, dass man überall von "`Fortschritt"' hört, sei er intellektueller, technischer, medizinischer, gesellschaftlicher oder persönlicher Natur. "`Fortschrittliche Länder"' haben sich zur Aufklärung bekannt und damit Säkularismus, Demokratie und Gleichberechtigung zu ihren Grundfesten gemacht, da diese in der modernen bürgerlichen Gesellschaft als "`fortschrittlich"', lies "`besser als ihre Alternativen"', wahrgenommen werden. Als "`technischen Fortschritt"' bezeichnet man heute gemeinhin alles, was es vorher noch nicht gab. Wenn diese auf gesellschaftlicher Ebene zu besonders einschneidenden Veränderungen führen (können), wird zum Beispiel im Silicon Valley, der Hochburg des technischen Fortschritts, auch gern von \emph{disruption} gesprochen. In den Worten des in Frankfurt am Main, dem Geburtsort der Kritischen Theorie, geborenen Milliardärs Peter Thiel in seiner "`Gründerbibel"' \emph{Zero to One}: "`Horizontal [...] progress means copying things that work — going from 1 to \emph{n}. [....] Vertical [...] progress means doing new things — going from 0 to 1"'\footnote{\Cite[Siehe][S. XXX]{thiel}.}. Solcher \emph{Zero to One} Fortschritt, solch eine \emph{disruption}, ist für Thiel, wie für viele andere in der "`Gründerszene"', äußerst wünschenswert. Dies führt aber zum Kernproblem des Fortschrittsglaubens, nämlich dass ein vermeintlicher Fortschritt in einer gesellschaftlichen Sphäre einen Rückschritt in einer anderen bedeuten kann. Wenn beispielsweise Startups wie Uber sich zum Ziel nehmen, Märkte wie das Taxigeschäft zu "`disrupten"' und behaupten, dadurch einen Fortschritt für die Gesellschaft zu erzielen, verschleiern sie ihr zerstörerisches Handeln. Die Ineffizienzen der laufenden Taxiunternehmen reichen als Begründung aus, um weltweit tausenden Taxifahrer*innen ihre Lebensgrundlage zu nehmen, Fahrer*innen in ärmeren Ländern durch einen Autokauf in die Armut zu treiben und all das mit dem Endziel, in Zukunft sowieso nur selbstfahrende Autos zu benutzen. Hinter den philanthropisch anmutenden Weltverbesserungshymnen solcher Unternehmen versteckt sich am Ende wieder nur die kapitalistische Profitgier, hinter dem Fortschritt nur die Zerstörung des Bestehenden und der Alternativen. "`Der Rationalisierungsfortschritt wird mit einer zum äußersten gesteigerten Inhumanität erkauft"'\footnote{\Cite[Siehe][S. 389]{hetzel2011adorno}.}.

% Struktur des Fortschritts

In der \emph{Dialektik der Aufklärung} nahmen sich Horkheimer und Adorno ein kritisches Denken vor, "`das auch vor dem Fortschritt nicht innehält"'\footnote{\Cite[Siehe][S. IX ("`Zur Neuausgabe"')]{dialektik-der-aufklaerung}.}. Dieses soll aber, wie schon allgemein im Bezug auf die Aufklärung, den Fortschritt nicht verneinen, sondern zu einem besseren Verständnis, und damit aus der Blindheit heraus führen. Für solch ein Vorhaben muss sich die Grundstruktur des Fortschrittsgedankens erst einmal klar gemacht werden. Wie die Aufklärung im Allgemeinen, zieht Fortschritt seinen Wert nicht aus sich selbst, sondern aus dem Überwinden des Schlechten und triumphiert auf diese Weise in der Negation des Überwundenen.\footnote{\Cite[Vgl.][S. 638]{fortschritt}.} Er ist zugleich Werkzeug und Ziel der Aufklärung, in der Moderne ist er zum metaphysischen Imperativ geworden, eine \emph{creatio continua}. Da aber niemand so recht weiß, wohin eigentlich fortgeschritten wird, begnügt man sich damit, einfach vom Bestehenden hinfort zu schreiten, wodurch jeder Schritt zum Fortschritt wird. Dieser fluchtartigen Bewegung fehlt aber der bewusste Blick auf ein größeres Ziel, und Fixierung auf das Überwinden des einzelnen Problems führt somit zum "`rastlos mühselige[n] Fortschritt ins Unendliche"'\footnote{\Cite[Siehe][S. 32]{dialektik-der-aufklaerung}.}. Die Aufklärung mag unterstellen, das Ziel ihres Fortschritts sei die Freiheit, doch dazu müssten die für die Befreiung genutzten Werkzeuge denn auch tatsächlich befreiender Natur sein und dürften nicht die Herrschaft des Menschen über sich selbst zur Konsequenz haben. Die instrumentelle Vernunft, mit dem verschwommenen Bild der Freiheit im Hinterkopf, schaut verächtlich nach hinten auf das Überwundene, mit dem Bewusstsein, dass der derzeitige Zustand genau so defizitär ist und folglich auch überwunden werden muss. 

%Weiter: wie entsteht der Regress

Die Grundbedürfnisse der Menschen haben sich seit Jahrtausenden wahrscheinlich nicht sonderlich verändert. Stattdessen aber die Komplexität von Herrschaftsverhältnissen, die auf hierarchischem Verhalten aufbaut, und uns dazu bewegt, unsere Grundbedürfnisse zu abstrahieren und den derzeitigen Gepflogenheiten anzupassen, denn kein Bedürfnis ist stärker, als Teil der Gruppe zu sein. Das Eigenheim, einst zum Schutz vor Niederfall und zum Aufbewahren von Essen und Werkzeugen ausgelegt, muss heute eine komfortable Ausstattung samt Deckenheizung und Bodenventilator haben, sowie als bürgerlicher Repräsentant für den eigenen Wohlstand herhalten. Vieler technischer Fortschritt befriedigt heute lediglich das universelle Bedürfnis nach Neuem, weder befreit er, noch erfüllt er ein Grundbedürfnis. Dies führt zum Ritual des jährlich neuen iPhones, wie auch zu den stets steigenden Scheidungszahlen. Die Frage danach, ob der Austausch denn überhaupt wirklich nötig sei, erübrigt sich sowieso, da das vom Smartphone- oder Liebesmarkt stets angebotene Neue die Hoffnung darauf aufleuchten lässt, dass die Dinge besser werden. Beim Fantasieren über das neue Handy oder den neuen Lebenspartner ist weniger das ursprüngliche Bedürfnis, also das mobile Surfen oder die tiefe Verbundenheit im Fokus, sondern der potenzielle Kontrast vom Neuen zum Alten, also die schnellere Ladezeit oder das größere Körperteil. "`Schon erscheinen die älteren Häuser rings um die Betonzentren als Slums, und die neuen Bungalows am Stadtrand verkünden schon wie die unsoliden Konstruktionen auf internationalen Messen das Lob des technischen Fortschritts und fordern dazu heraus, sie nach kurzfristigem Gebrauch wegzuwerfen wie Konservenbüchsen"'\footnote{\Cite[Siehe][S. 128]{dialektik-der-aufklaerung}.}. 

Aber gerade diese "`Dekadenz ist der Nervenpunkt, wo die Dialektik des Fortschritts vom Bewusstsein"'\footnote{\Cite[Siehe][S. 627]{fortschritt}.} sich zu eigen gemacht wird. Denn eine bloße Verneinung des Fortschritts wäre nicht im Sinne einer sich selbst aufklärenden Aufklärung. Sie käme eher einem Zetern über die ach so leidlichen Zeiten gleich, das wieder nur die guten alten Zeiten mythologisiert.



"`Demgegenüber involviert Anpassung an die Macht des Fortschritts den Fortschritt der Macht, jedes Mal aufs neue jene Rückbildungen, die nicht den mißlungenen sondern gerade den gelungenen Fortschritt seines eigenen Gegenteils überführen. Der Fluch des unaufhaltsamen Fortschritts ist die unaufhaltsame Regression"'\footnote{\Cite[Siehe][S. 42]{dialektik-der-aufklaerung}.}.

%Synthese bilden



"Insofern ließe sich sagen, der Fortschritt ereigne sich dort, wo er endet"'\footnote{\Cite[Siehe][S. 625]{fortschritt}.}.

Die ziellosigkeit des Fortschritts aufzeigen, zeigen, dass gerichteter Fortschritt aber nur sinn machen wird, und dass daher das selbstreflexive moment das einzige mittel ist, mit dem ein fortschritt in die freiheit funktionieren kann. Fortschritt, in dem ausgangs Zustand der Endpunkt der Geschichte sein könnte, und der kommende auch, doch deren potenzielle veränderung immer auch bekannt ist, was aber dies nicht als defizitär darstehen lässt.

Soll nach Adorno: (synthese): “`Fortschritt als Widerstand gegen die immerwährende Gefahr des Rückfalls”’ 638

%wie sieht das aus seitdem, Karp

Somit bleibt nichts mehr übrig, das getan werden kann. -> Silicon Valley versucht das gegenteil zu beweisen

Fortschrittsdenken mit Fokus auf das Soziale, wie die progressives in den USA: Apple, das am wertvollsten gehandelten Unternehmens der welt, hat einen homosexuellen CEO, Google einen indischen, Deutschland wird von einer Frau regiert  

Alex Karp: Doktor am Institut für Sozialforschung, heute Milliardär und CEO eines Datenanalyse Unternehmens.

\vspace{5mm}
%\noindent\textbf{$(iii)$ Die Dialektik der Kalifornische Ideologie}
%
%\begin{itemize}
%  \item Bsp: amerikanischen Institutionen wie Stanford im Bereich der Computer Science gang und gebe ist, wo die Institute sogar Namen	 von erfolgreichen Leute nhaben (Gates Computer Science Building, Hewlett Teaching Center) 
%  \item Die Herrschaftsstrukturen des kapitalistischen Marktes werden (möglicherweise murrend) hingenommen,
%  \item david byrne text
%  \item Facebook kämpft für connection, doch gleichzeitig sind sie die enabler für Trump und co
%  \item zu zeiten obamas sind politik und silicon valley immer weiter zusammengewachsen: Libertarianism 
%  \item problem: in amerika sind wegen satz 230 blabla die webseiten nicht für ihre inhalte verantwortlich
%  \item das größte taxi unternehmen besitzt keine taxis, das größte medienunternehmen produziert keine eigenen medien
%  \item die shopping daten werden zum politischen einfluss genutzt, so zeigt sich die vershcmelzung der medienindustrie, der wirtschaft und der politik, zu einem großen datenhaufen, der nutzbar wird, um leute in diesen drei, nicht wirklich mehr getrennten handlungsbereichen, zu beeinflussen
%  \item menschen werden von facebook emotional stark beeinflusst. es zeigt sich dass diese beeinflussung auf alle 3 bereiche einfluss haben müsste. bots sind durch viral machen von posts in der lage, die leute zu beeinflussen. 
%\end{itemize}
%
\noindent\textbf{$(iv)$ Konklusion}

\noindent 
Die Dialektik der Aufklärung ist nicht zu trennen von einer Dialektik des Fortschritts. Fortschritt ist die der Aufklärung unterliegende Struktur. Wenn man also eine Aufklärung betreiben will, die nicht in Regression endet, muss sich gefragt werden, wie Fortschritt denn eigentlich auszusehen habe, wo er hin führen soll soll, und warum der derzeitige Standpunkt nicht ausreichend sei.

Es folgt also eine Diagnose, die am Ende der Kants nicht zu unähnlich ist. "`Wenn denn nun gefragt wird: Leben wir jetzt in einem aufgeklärten Zeitalter? so ist die Antwort: Nein, aber wohl in einem Zeitalter der Aufklärung."'\footnote{\Cite[Siehe][S. 491]{kant}.}

\newpage

\end{onehalfspace}
\nocite{*}
\printbibliography
\end{document}
