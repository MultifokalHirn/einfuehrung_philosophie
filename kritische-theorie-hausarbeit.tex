\documentclass[a4paper, 12pt]{article}
%\usepackage{CJKutf8} % japanese
\usepackage{graphicx}
\usepackage{hyperref}
\usepackage{fullpage}
%\usepackage{parskip}
\usepackage{color}
\usepackage[ngerman]{babel}
\usepackage{hyperref}
\usepackage{calc} 
\usepackage{enumitem}
\usepackage[utf8]{inputenc}
\usepackage{titlesec}
%\pagestyle{headings}
\usepackage{setspace} %halbzeilig
\usepackage[style=authoryear-ibid,natbib=true]{biblatex}
\usepackage[hang]{footmisc}
\setlength{\footnotemargin}{-0.8em}
%\bibliographystyle{natdin}
\addbibresource{kritische-theorie-hausarbeit.bib}
\DeclareDatamodelEntrytypes{standard}
\DeclareDatamodelEntryfields[standard]{type,number}
\DeclareBibliographyDriver{standard}{%
  \usebibmacro{bibindex}%
  \usebibmacro{begentry}%
  \usebibmacro{author}%
  \setunit{\labelnamepunct}\newblock
  \usebibmacro{title}%
  \newunit\newblock
  \printfield{number}%
  \setunit{\addspace}\newblock
  \printfield[parens]{type}%
  \newunit\newblock
  \usebibmacro{location+date}%
  \newunit\newblock
  \iftoggle{bbx:url}
    {\usebibmacro{url+urldate}}
    {}%
  \newunit\newblock
  \usebibmacro{addendum+pubstate}%
  \setunit{\bibpagerefpunct}\newblock
  \usebibmacro{pageref}%
  \newunit\newblock
  \usebibmacro{related}%
  \usebibmacro{finentry}}

%\titleformat{name=\section,numberless}
%  {\normalfont\Large\bfseries}
%  {}
%  {0pt}
%  {}
\date{\vspace{-3ex}}
\begin{document}

\title{\vspace{5ex}
	\includegraphics*[bb=0 0 720 200, width=0.72\textwidth]{ErstesSem/images/hu_logo.png}\\
	\vspace{30pt}
	\scshape\LARGE{Die Regression im Fortschritt}\\\Large{Eine Beschreibung der Fortschrittsdialektik in der \emph{Dialektik der Aufklärung} im Anbetracht der Kalifornischen Ideologie}\\\vspace{20pt}}
	


\author{Kritische Theorie der Gesellschaft: Horkheimer und Adorno (PS)\\
	\vspace{7pt}
          Dozent: Dr. Arnd Pollmann\\\vspace{4pt}Lennard Wolf\\
        \small{Matrikelnummer: 583052}\\
        \small{E-Mail: lennard.wolf@student.hu-berlin.de}\\
        \small{Telefonnummer: +49 176 5687 4131}\\
        \small{Studiengang: B.A. Philosophie}\\
        \small{Modul: Praktische Philosophie}}

        %\href{mailto:lennard.wolf@student.hu-berlin.de}{lennard.wolf@student.hu-berlin.de}}}      

\maketitle

\vspace{\fill}

\begin{minipage}[]{0.92\textwidth}
    \centering
    \onehalfspacing
    \large   
    30. September 2017\\
    Sommersemester 2017

    \vspace{-20mm} 
\end{minipage}%
\thispagestyle{empty}
\newpage
%\clearpage
%\thispagestyle{empty}
%\tableofcontents
%\newpage
\setcounter{page}{1}

\begin{onehalfspace} 

\noindent\textbf{$(o)$ Einleitung}

\noindent %wichtig unterschied Regression und Regress!

\begin{itemize}
  \item Der Aufruf in Californian Ideology, dass Europa nicht mitgehen soll erscheint in weiten Teilen gescheitert. 
  \item Zwar wird sich gegen Uber und AirbnB, den höchstgewerteten Startups, gewehrt, und facebook muss kommentare löschen etc, doch Occupy Wallstreet in Berlin dauerte nur kurz an, der CCC ist zu einer Party mit ein paar kritischen, aber folgelosen Präsentationen verkommen, die Piratenpartei befindet sich im Zerfall, die meistbesuchten Seiten sind… und ein Entfliehen von der ‘Datenkrake’ Amerika ist auch nicht im Sicht. 
\item Während die Kalifornische Ideologie ins geistige Sediment der virtuellen Klasse in Deutschland sickert, Privatinstitutionen wie das Hasso-Plattner-institut entstehen, an denen, wenn möglich, die nächsten deutschen Mark Zuckerbergs herangezüchtet werden sollen und macht sich ein immer größer werdendes Gefühl der Machtlosigkeit vonseiten der Bevölkerung wie der Politiker breit.
\item Ziel meiner Darstellung soll sein, den Reflex, auf alles Gute das die Aufklärung gebracht hat zu verweisen und es abzuwägen mit dem leid, als nicht haltbar zu entlarven.
   \item Was ist kritische theorie? Kritische Theorie als Gegenentwurf zur traditionellen, die Gesellschaftliche Verhältnisse als notwendig akzeptieren
  \item aus welchen fragen ist dsie entstanden?
  \item Wie fügt sich DdA in die KT als solche ein?
\end{itemize}


Daf"ur werde ich wie folgt vorgehen. In Abschnitt $(i)$ gebe ich einen Überblick über die Methode und Erkenntnisse der \emph{Dialektik der Aufklärung}. 

\vspace{5mm}

\newpage

\noindent\textbf{$(i)$ Die Dialektik der Aufklärung}

\noindent Die \emph{Dialektik der Aufklärung}, entstanden aus Gretel Adornos Mitschriften von Gesprächen zwischen ihrem Ehemann Theodor W. Adorno und Max Horkheimer und 1944 im kalifornischen Exil veröffentlicht, gilt als zentrales Werk der Kritischen Theorie.\footnote{\Cite[Vgl.][S. 249]{jaeggi}.} Ihr Ausgangspunkt ist die Frage, "`warum die Menschheit, anstatt in einen wahrhaft menschlichen Zustand einzutreten, in eine neue Art von Barbarei versinkt"'\footnote{\Cite[Siehe][S. 1]{dialektik-der-aufklaerung}.}. Diese führt zu einer weitreichenden Prüfung der Aufklärung\footnote{Mit "`Aufklärung"' ist nicht ein Zeitalter gemeint, sondern, grob gefasst, das geschichtsphilosophische Phänomen des Fortschrittsdrangs durch rationalen Diskurs.}, die sich programmatisch den "`Ausgang des Menschen aus seiner selbst verschuldeten Unmündigkeit"'\footnote{\Cite[Siehe][S. 481]{kant}.} und somit eine Befreiung von Gesellschaft und Verstand auf die Fahne geschrieben hat. Die scheinbare Widersprüchlichkeit zwischen den im 20. Jahrhundert begangenen Schandtaten und dem ursprünglichen, und noch immer betonten Ziel einer freien Gesellschaft führt die Autoren zu dem Versuch einer "`Aufklärung der Aufklärung"', der zufolge die barbarischen Zustände nicht \emph{trotz} des aufklärerischen Denkens ausbrachen, sondern dass sie gerade dessen \emph{notwendige Konsequenz} waren. Die Aufklärung der Aufklärung vollziehen die Autoren auf zwei Ebenen. Zum einen üben sie klassische, aufklärerische Ideologiekritik, indem sie die Wiedersprüche im bürgerliche Denken und Verhalten aufzeigen. Zum anderen geben sie der Kritik noch ein selbstreflexives Moment hinzu. Sie fragen nach der Möglichkeit der Kritik selbst, da eine einfache Ideologiekritik selber, aufgrund der Absicht der Verbesserung der Umstände, in der Ideologie eines zweckrationalen Verstandes gefangen bleibt. Daraus ergibt sich die für die Kritische Theorie bezeichnende "`Kritik von Innen"', die sich ihrer unzulänglichen Subjektivität nur allzu bewusst ist. Die große Erkenntnis der \emph{Dialektik der Aufklärung} ist, dass die Aufklärung in ihrer derzeitigen Form zum "`totalen Betrug der Massen"'\footnote{\Cite[Siehe][S. 49]{dialektik-der-aufklaerung}.} verkommt. Die \emph{Fragmente} des Werkes sind eine Vorrede, ein geschichtsphilosophischer Einführungsessay mit dem Titel "`Begriff der Aufklärung"', zwei Exkurse "`Mythos und Aufklärung"' und "`Aufklärung und Moral"', zwei zeitdiagnostische Essays zu Kulturindustrie und Antisemitismus, sowie eine angehängte Sammlung von Aufzeichnungen und Entwürfen. Ziel dieses Abschnittes soll es sein, einen groben Überblick über den Einführungsessay, und damit über das Konzept der Dialektik der Aufklärung zu geben.  

Es lohnt sich, zu Beginn die im Buchtitel befindlichen Wörter "`Dialektik"' und "`Aufklärung"' zu klären. "`Dialektik"' bezieht sich sowohl auf die Struktur des behandelten Gegenstands, als auch auf die dadurch notwendig gewordene Methode. Es wird also versucht, die Aufklärung zu verstehen, indem man eine nie stehenbleibende Korrektur von der Beziehung zwischen dem \emph{gedanklichen} Begriff und der Sache, auf die er sich bezieht, vollzieht.  Da die Autoren anfangs aufzeigen, dass sich die Aufklärung als Negation des \emph{Mythos} versteht, der dadurch zugleich zu ihrem Nährboden wird, lässt sich schon eine dialektische "`Verschlingung von Mythos und Aufklärung"'\footnote{Siehe \Cite{habermas}.} erahnen. Der Mythos ist, so wird sich zeigen, die Wurzel der Aufklärung, doch deren bisherige Unfähigkeit, sich entgültig von ihm zu trennen, führt sie in die Selbstzerstörung.

Ihren Ursprung findet die Aufklärung in dem Ziel, allen Aberglauben, allen Mythos aus der Welt zu verbannen und so den Glauben mit Wissen auszutauschen. Bezüglich des Ursprungs von Mythen zeigen die Autoren zu Beginn, dass diese aber immer schon aus aufklärerischem Handeln entstanden ist. Der Mensch, als stets wachsames Lebewesen, hat ein durch Angst bedingtes Bedürfnis, die Welt um sich herum zu verstehen, um durch Vorausschau sich besser vor Gefahren schützen zu können. Durch die Weitergabe von Erfahrungen an die Nachkommen, konnte so ein kollektiver Erfahrungsschatz aufgebaut werden, der in weiten Teilen sicherlich empirischer Natur war. Sollte man sich erzählt haben, die Welt sei eine flache Scheibe, mit Land in der Mitte, das umgeben ist von Wasser, das möglicherweise an den Rändern hinunterfliesst, so war das weniger primitiver Aberglaube, als ein rein empirisches Forschungsergebnis, dessen Lücken mit aus dem Gegebenen extrapolierten, logisch erscheinenden Ergänzungen gefüllt werden. "`Der Mythos wollte berichten, nennen, den Ursprung sagen: damit aber darstellen, festhalten, erklären"'\footnote{\Cite[Siehe][S. 14]{dialektik-der-aufklaerung}.}, und so wurde aus den Berichten über die Generationen hinweg autoritäres "`Wissen"', das mit aus den Extrapolationen entstandenen Superstitionen gespickt war. Die sozialen, wie verstandesmäßigen Folgen waren die der Unfreiheit und Barbarei. Und aus diesem Mythos nährt sich die Aufklärung, mit dem Ziel, den Menschen und seinen Verstand zu befreien. Doch dieses Programm ist in seinem Kern das selbe wie jenes, mit dem der zu bekämpfende Mythos einst entstanden ist, dessen Inhalte die Aufklärung mit subjektiver Projektion verwechselt.\footnote{\Cite[Vgl.][S. 12]{dialektik-der-aufklaerung}.} Und dieser Ursprung ist der Trieb zur Selbsterhaltung und damit die Angst vor der Beherrschung durch die Natur, der in einstigem Mythos noch durch eine ehrfürchtige Selbstanpassung, und in der Aufklärung aber durch Beherrschung der Natur entgegengewirkt wird. So reicht Wissen, durch das der Mensch Natur bändigen und sich besser anpassen kann, nicht mehr aus. Wissen muss nicht nur einen reaktiven Zweck erfüllen, sondern aktive Kreation ermöglichen. Ich kenne die Dinge, da ich sie manipulieren und machen kann, mein Wissen ist Macht.\footnote{\Cite[Vgl.][S. 15]{dialektik-der-aufklaerung}.} Der aufklärende Mensch erhöht sich so zu einem göttlichen Gebieter über die Natur und die Welt.\footnote{\Cite[Vgl.][S. 15]{dialektik-der-aufklaerung}.} Alles in der Welt wird gemessen und damit messbar, alles Lebendige wird Unlebendig und definierbar gemacht. Und so radikalisiert sich die Angst, die einst das Unlebendige zum Lebendigen emporhob, und ruft nach einem vor nichts halt machenden, alles durch Abstraktion komparabel oder gleich machenden Allwissen, für das es kein Unbekanntes und Unbenanntes mehr geben darf.\footnote{\Cite[Vgl.][S. 22 f.]{dialektik-der-aufklaerung}.} Der technische und gesellschaftliche Fortschritt, der kein Blick nach vorne, sondern nach hinten wurde, betrügt sich selbst, indem er das positivistische Welt\emph{bild} mit der Welt an sich verwechselt. In der mathematisierten Welt ist alles schon vorherbestimmt, das mir unbekannte hat bestimmt irgendwer schon erforscht, der mir sagt was es ist, und Probleme lasse ich lieber von Algorithmen lösen - Verstand und Fantasie verkümmern\footnote{\Cite[Vgl.][S. 42]{dialektik-der-aufklaerung}.} im Angesicht der objektiven, determinierten Welt. Das konstante differenzieren der Begriffe, das den Kern der Erkenntnis ausmachen soll, führen aber ebenso entweder zur Relativierung von Wahrheit und Moral oder zu Vorstellungen von der Erklärbarkeit von Moral anhand der Naturwissenschaften\footnote{Siehe hierzu beispielsweise: Sam Harris (2010). \emph{The Moral Landscape: How Science Can Determine Human Values}. New York: Free Press.}, da diese ja die objektive Wahrheit beschreiben. Indem die Betrachtungen der Wissenschaften immer weiter auf teilnahmsloses Datensammeln herunter gebrochen werden, wird den Menschen nicht nur die Möglichkeit der qualitativen Erkenntnis genommen, sondern auch jede Hoffnung darauf, dass die Dinge anders sein könnten. In diesem Anspruch, das einzig mögliche zu sein, zeigt sich, dass "`Aufklärung [...] totalitär wie nur irgendein System"'\footnote{\Cite[Siehe][S. 31]{dialektik-der-aufklaerung}.} ist. Die \emph{instrumentelle}, das heißt die zweckgerichtete, naturbeherrschende Vernunft objektiviert und objektifiziert nicht nur die Natur, sondern auch den Menschen selber, den sie auf ein quantifizierbares, austauschbares Bündel von Atomen reduziert. Damit die "`aufgeklärten"' Ziele von Demokratie und Gleichheit erreicht werden können, muss dieser Mensch angepasst, das heißt beherrscht und freiwillig fügsam werden. Durch den "`objektiven Wahrheitswert"' des Wissens, und die "`naturgegebene"', weil einzig als valide anerkannte Herrschaftsstruktur der aufgeklärten Gesellschaft wird das Individuum um das eigene, freie Denken beraubt, ohnmächtig in das Kollektiv gezwungen und somit um das eigentliche Versprechen seiner Befreiung betrogen. Im Namen der Freiheit wird Unfreiheit geschaffen und aus dem ursprünglichen Aufbruch in die  freie Gesellschaft, deren Verstand sich selbst und die Welt mit Klarheit erblicken sollte, wurde ein Weg in sich selbst widersprechende Verblendungszusammenhänge, die alles und jeden unterdrücken. So schlägt die erkenntnistheoretische Bewegung der "`Aufklärung in Mythologie zurück"',\footnote{\Cite[Siehe][S. 6]{dialektik-der-aufklaerung}.} und gleicht einem vor sich selbst die Augen verschließenden, ziellos durch die Gegend rasenden und alles in seinem Weg zertrampelnden Riesen\footnote{\Cite[Vgl.][S. 25]{fortschritt}.}. Die abscheulichen Massenmorde und totalitären Regime des 20. Jahrhunderts sind also keinesfalls als von der Aufklärung getrennt zu sehen, sondern die letzte Konsequenz eines unreflektierten Fortschrittswahns.

Diese "`Dialektik der Aufklärung"' ist die Kernerkenntnis des so betitelten Werkes und keineswegs als eine Kritik zu verstehen, die zu einer Überwindung der Aufklärung hinführen will. Horkheimer und Adorno sind der Überzeugung, "`daß die Freiheit in der Gesellschaft vom aufklärenden Denken unabtrennbar ist"'\footnote{\Cite[Siehe][S. 3]{dialektik-der-aufklaerung}.} und dass die so "`an Aufklärung geübte Kritik [...] einen positiven Begriff von ihr vorbereiten [soll], der sie aus ihrer Verstrickung in blinder Herrschaft löst"'\footnote{\Cite[Siehe][S. 6]{dialektik-der-aufklaerung}.}. Die Aufklärung soll ihr mythologisches Erbe abgeben und über Ideologiekritik hinausgehende Selbstreflexion betreiben. Erst die Aufklärung, die sich nicht selbst mythologisiert, so scheint die These gemeint zu sein, wird den Menschen in die Freiheit leiten können. Doch die oben schon angesprochene "`Kritik von Innen"' reicht nicht aus, um die Autoren zu konstruktiven Vorschlägen zu befähigen, denn die in alle Dimensionen des Lebens eingesickerten Zwangs- und Verblendungszusammenhänge betrifft sie genau so wie alle anderen. Sätze wie Adornos "`Das einzige, was sich verantworten läßt, ist, den ideologischen Mißbrauch der eigenen Existenz sich zu versagen und im übrigen privat so bescheiden, unscheinbar und unprätentiös sich zu benehmen, wie es längst nicht mehr die gute Erziehung, wohl aber die Scham darüber gebietet, daß einem in der Hölle noch die Luft zum Atmen bleibt"'\footnote{\Cite[Siehe][S. XXX]{minima}.} zeugen von einer düsteren Hoffnungslosigkeit des Philosophen. Die Massenmorde in den Konzentrationslagern und Gulags scheinen vorerst überwunden, doch Krieg und Unterdrückung herrschen noch immer vielerorts. In Europa ist Fremdenfeindlichkeit wieder in den politischen Diskurs zurückgekehrt und die Kritik aus dem Essay "`Kulturindustrie"' erscheint uns heute noch sehr viel relevanter zu sein, als zur Zeit seiner Veröffentlichung. Die Lektüre der \emph{Dialektik der Aufklärung} ist zwar eine äußerst erhellende Erklärung für die herrschende Unvernunft und Barbarei, doch sie kann ebenso ein Gefühl von Ohnmacht und Ausweglosigkeit hinterlassen, sowie eine Ahnung, dass der nächste Holocaust vielleicht doch nicht so weit von uns entfernt ist. Mich persönlich beschleicht beim Befassen mit der Thematik weiterhin das Gefühl, dass egal wie ich handeln werde, ich werde falsch handeln.

\newpage

\vspace{5mm}
\noindent\textbf{$(ii)$ Die Dialektik des Fortschritts}

\noindent 

\begin{itemize}
  \item was ist fortschritt: eine pedantische Begriffsanalyse nichts sinnvoll siehe adorno: fortchritt. 
  \item Fortschritt kann auf individueller ebene betrachtet werden, jedoch werde ich mich hier auf die makroskopische Betrachtung beschränken
  \item Fortschritt wird intuitiv als Veränderung der Umstände der Menschen zu einem besseren hin betrachtet. Früher mussten die Früchte gesammelt werden, dann gab es Ackerbau. Früher war die Grippe tödlich, heute nicht der Rede wert. Dies sind technische Fortschritte. Gesellschaftliche Fortschritte sind Menschenrechte, Gleichberechtigung der Frau und enge internationale Zusammenarbeit. 
  \item Dies sind die häufigen Arguemnte zur stützung der Aufklärung. Dass der technische Fortschritt dem Menschenleben geholfen hat. Aber zu welchem Preis?
  \item als Kerngedanke der Aufklärung ist er dieser also in der Form ähnlich: Es geht um das Überwinden des Schlechten, und triumphiert so in der Negation des Überwundenen (vgl.fortschritt s 638) Es ist wieder ein Herrschaftsverhältnis: "`Modell des Fortschritts [...] ist die Kontrolle außer- und innermenschlicher Natur"'\footnote{\Cite[Siehe][S. 623]{fortschritt}.}
  \item die instrumentelle vernunft, die sich in der selbsterhaltung gründet, axiomatisert fortschritt und macht ihn zum metaphysischen Imperativ. Damit legt sie sich zur Grundlage aber auch immer schon das Misstrauen ihr selbst gegenüber. Indem von jedem gesellschaftlichen Zustand der Fortschritt ausgehen kann, ist jeder Zustand immer schon defizitär. Der Blick nach Vorn, die Unruhe, ist das einzig wahrnehmbare Jetzt.
  \item Ebenso enttarnt sich hier auch Ideologiekritik als Versuch des Fortschritts aus den Herrschaftsverhältnissen. 
  \item Adorno und Horkheimer unterstellen nun aber, dass jedem Fortschritt ein zwangsläufiges Rückschreiten anhaftet. belege
  \item Der erste, offensichtliche Regression ist schon in der Zerstörung des Bestehenden zu finden.
  \item Der Rationalisierungsfortschritt wird mit einer zum äußersten gesteigerten Inhumanität erkauft. S389 Hetzel
  \item Analgoie mit Riesen: Ist der Riese erstmal erwacht... 625
  \item Beispiel irgendwas in der Massendemokratie, beispiel aus der Kulturindustrie nehmmen
  \item jaeggi: Der so genannte Fortschritt der Menschheit schlage zuneh- mend in einen Prozess blind fortschreitender Natur- und Selbstbeherrschung um, der genau das zunichte zu machen drohe, was er einst verwirklichen sollte: „die Idee des Menschen“ (S. 13)
  \item Fortschritt entpuppt sich als ein fortschreiten der macht (achtung direktes pollmann zitat)
  \item Fortschrittsglaube
\end{itemize}

\newpage

\vspace{5mm}
\noindent\textbf{$(iii)$ Die Dialektik der Kalifornische Ideologie}

\noindent 

\begin{itemize}
  \item Somit bleibt nichts mehr übrig, das getan werden kann. -> Silicon Valley versucht das gegenteil zu beweisen. Leute wie Alex Karp.
  \item Untersucht wird nicht ein Abweg, auf den die Aufklärung geraten sei; inkriminiert nicht ihre unzureichende Verwirklichung (jaeggi 251)! SONDERN ist immer schon drin
  \item Verblendungszusammenhang
\end{itemize}

\begin{itemize}
  \item Tim Cook als homosexueller CEO des am wertvollsten gehandelten Unternehmens der welt hat sich öffentlich gegen Trump geäußert
  \item Bsp: amerikanischen Institutionen wie Stanford im Bereich der Computer Science gang und gebe ist, wo die Institute sogar Namen	 von erfolgreichen Leute nhaben (Gates Computer Science Building, Hewlett Teaching Center) 
  \item Die Herrschaftsstrukturen des kapitalistischen Marktes werden (möglicherweise murrend) hingenommen, um einen Fortschritt im Zusammensein 
  \item die totalität der aufklärung zeigt sich auch darin, dass google et al keine offices haben sondern einen campus
  \item Fortschritt vs Disruption
  \item talking heads typs text
  \item Fortschritt Peter Thiel: klassischer Fortschritt ist 1 zu n, während disruptiver Fortschritt 0 zu 1 sei.
  \item 'progressives' in amerika
  \item wollen nicht nur technischen Fortschritt, sondern die gesamte gesellschaft verändern
  \item Facebook kämpft für connection, doch gleichzeitig sind sie die enabler für Trump und co
  \item zu zeiten obamas sind politik und silicon valley immer weiter zusammengewachsen: Libertarianism 
  \item problem: in amerika sind wegen satz 230 blabla die webseiten nicht für ihre inhalte verantwortlich
  \item das größte taxi unternehmen besitzt keine taxis, das größte medienunternehmen produziert keine eigenen medien
  \item die shopping daten werden zum politischen einfluss genutzt, so zeigt sich die vershcmelzung der medienindustrie, der wirtschaft und der politik, zu einem großen datenhaufen, der nutzbar wird, um leute in diesen drei, nicht wirklich mehr getrennten handlungsbereichen, zu beeinflussen
  \item menschen werden von facebook emotional stark beeinflusst. es zeigt sich dass diese beeinflussung auf alle 3 bereiche einfluss haben müsste. bots sind durch viral machen von posts in der lage, die leute zu beeinflussen. 
  \item 
\end{itemize}


\vspace{5mm}
\noindent\textbf{$(iv)$ Konklusion}

\noindent Es folgt also eine Diagnose, die am Ende der Kants nicht zu unähnlich ist. "`Wenn denn nun gefragt wird: Leben wir jetzt in einem aufgeklärten Zeitalter? so ist die Antwort: Nein, aber wohl in einem Zeitalter der Aufklärung."'\footnote{\Cite[Siehe][S. 491]{kant}.}.

\newpage

\end{onehalfspace}
\nocite{*}
\printbibliography
\end{document}
