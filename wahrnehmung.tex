\documentclass[a4paper]{article}
\usepackage{graphicx}
%\usepackage{fullpage}
%\usepackage{parskip}
\usepackage{color}
\usepackage[utf8]{inputenc}
\usepackage[ngerman]{babel}

\usepackage{hyperref}
\usepackage{calc} 
\usepackage{enumitem}
\usepackage{titlesec}
\usepackage{bussproofs}
\usepackage[export]{adjustbox}
%\pagestyle{headings}

\titleformat{name=\section,numberless}
  {\normalfont\Large\bfseries}
  {}
  {0pt}
  {}
\date{\vspace{-3ex}}
\begin{document}

\title{
    \vspace{-30pt}
	\includegraphics*[width=0.1\textwidth,right]{ErstesSem/images/hu_logo2.png}\\
	\vspace{-10pt}
	Natur der Wahrnehmung – Kunst der Täuschung}
\author{Lennard Wolf}%\\\small{\href{mailto:lennard.wolf@student.hu-berlin.de}{lennard.wolf@student.hu-berlin.de}}}
\maketitle
\vspace{-4pt}

\section*{Motivationsschreiben}
\normalsize Im Sommersemester 2017 habe ich ein Seminar zu Merleau-Pontys Werk "`Phänomenologie der Wahrnehmung"' belegt, in das wir sehr intensiv eingestiegen sind und zu dem ich auch eine Hausarbeit geschrieben habe. Dort stieß ich das erste Mal auf das Konzept des Leibes, das mich bis heute nicht mehr los lässt. Dieses knüpft an heutigen Vorstellungen von embodiment in den Kognitionswissenschaften an, dass also Bewusstsein immer auch verkörpert ist. Mit der daraus resultierenden Untrennbarkeit von Denken und Wahrnehmen, sowie der neuen Stellung der Wahrnehmung als allerersten Schritt des Seins und der Erkenntnis setze ich mich derzeit in einem Seminar zur “Erfahrung der Realität durch Widerstand” weiter auseinander. Zentral ist hier die Frage, wie die Unterscheidung zwischen “real” und “nicht real” erkenntnistheoretisch und entwicklungspsychologisch zu erklären ist: Wie kommt es überhaupt zu der Vorstellung, dass etwas gar nicht wirklich so ist ist oder sein könnte? Derlei Fragen zum Thema Schein und Wirklichkeit setzen an der Möglichkeit zur Täuschung an, und damit aber auch an der Möglichkeit des Wahrnehmens und Vorstellens des Virtuellen, des nicht wirklichen. 

Mich interessiert hier besonders das aus dem Bedürfnis, stets in einer konsistenten Welt zu leben, entstehende Verhalten, uns selbst zu täuschen, Täuschungen von außen auf eine Art zu interpretieren, dass sie im Einklang ist mit den anderen Wahrnehmungen und unserer Erfahrung, sowie überhaupt Irrtümern zu unterlaufen, die aber weiterhin eine konsistente Welt erlauben. Gerade die ersten beiden Verhaltensweisen, in denen das Gehirn versucht, aus der Masse an Sinneseindrücken um jeden Preis ein sinnvolles Gesamtbild zusammenzufügen, sind für Überlegungen zum Embodiment in virtuellen Welten von zentraler Bedeutung. Ein klassisches Beispiel ist hier natürlich die Gummihand-Illusion, in der durch eine einfache Täuschung eine Gummihand für die eigene gehalten wird (vgl. Botvinick, Cohen 1998). Auf ähnliche Weise wird heute versucht, das Gehirn auszutricksen und die bisher nur audiovisuellen virtuellen Erfahrungen mit Oculus Rift und co. zu vollständigen Leibeserfahrungen weiter zu entwickeln (vgl. Slater et al. 20xx). Fortschritte in diese Richtung bringen aber nicht bloß bessere Technologien für die Unterhaltungsindustrie hervor, sondern ermöglichen beispielsweise auch neue Forschungsansätze für die experimentelle Kognitionswissenschaft, oder neue Behandlungsweisen in der psychiatrischen Therapie. Veröffentlichungen (http://publicationslist.org/melslater) des Experimental Virtual Environments for Neuroscience and Technology (EVENT) Labs der Universitat de Barcelona beschreiben erfolgreiche Experimente zu virtuellem embodiment, in denen der virtuelle Körper zum Beispiel der eines Kleinkinds () oder eines 

\nocite{*}

\bibliography{wahrnehmung}

\end{document}
