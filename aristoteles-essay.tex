\documentclass[a4paper, emulatestandardclasses, 12pt]{scrartcl}
\usepackage{graphicx}
\usepackage{hyperref}
\usepackage{fullpage}
%\usepackage{parskip}
\usepackage{color}
\usepackage[ngerman]{babel}
\usepackage{hyperref}
\usepackage{calc} 
\usepackage{enumitem}
\usepackage{titlesec}
%\pagestyle{headings}
\usepackage{setspace} %halbzeilig
\usepackage[authoryear,round]{natbib}
\bibliographystyle{natdin}

%\titleformat{name=\section,numberless}
%  {\normalfont\Large\bfseries}
%  {}
%  {0pt}
%  {}
\date{\vspace{-3ex}}
\begin{document}

\title{\vspace{5ex}
	\includegraphics*[width=0.72\textwidth]{ErstesSem/images/hu_logo.png}\\
	\vspace{30pt}
	\scshape\LARGE{Schwierigkeit, wie wir gut werden k"onnen, ohne es schon zu sein [AT]}}
	
	\subtitle{\vspace{20pt}Tutorium zu Aristoteles: Nikomachische Ethik (VEV)\\
	\vspace{6pt}
          Tutorin: Larissa Gniffke}


\author{\vspace{-4pt}Lennard Wolf\\
        \small{\href{mailto:lennard.wolf@student.hu-berlin.de}{lennard.wolf@student.hu-berlin.de}}}      

\maketitle

\vspace{\fill}

\begin{minipage}[b]{\textwidth}
    \centering
    \onehalfspacing
    \large   
    11. Juli 2017\\
    Sommersemester 2016

    \vspace{-20mm} 
\end{minipage}%
\thispagestyle{empty}
\newpage
%\clearpage
%\thispagestyle{empty}
%\tableofcontents
%\newpage
\setcounter{page}{1}

\begin{onehalfspace} 



\noindent\textbf{$(o)$ Einleitung}

\noindent In der Nikomachischen Ethik \citep{wolf2006nikomachische}, einer der esoterischen Schriften des Aristoteles, wird die Tugendethik des Philosophen beschrieben und begr"undet.

Dieser zufolge ist das h"ochste Ziel eines jeden\footnote{Aristoteles wendet sich in seinem Werk nur an M"anner, und so werde ich diesen Stil "ubernehmen. Diesbez"ugliche Problematiken werden hier nicht weiter thematisiert.} die \emph{eudaimonia}\footnote{Im Verlauf des Essays werde ich teilweise der Genauigkeit halber die griechischen Originalbegriffe verwenden, jedoch beim ersten Vorkommen die "Ubersetzung nach Ursula Wolf angeben.} (Gl"uck, Gl"uckseligkeit). Dies bedeutet, dass all unser Tun auf die \emph{eudaimonia} hinstrebt (siehe 1097 b)\footnote{In diesem Essay beziehe ich mich bei der Verwendung der Bekkerzahl durchgehend auf \citep{wolf2006nikomachische}.}. Dies Tun wird unterteilt in \emph{h\={e}don\={e}} (Lust), \emph{politikos} (Aus"ubung der Charaktertugend \footnote{Vgl. Wolf, U. (2006): \emph{Vorwort}. In: \citep{wolf2006nikomachische}.} %\citep[vgl. S. 12]{wolf2006nikomachische})
 und \emph{the\={o}r\={e}tikos} (Betrachtung) (vgl. 1095 b). Die B"ucher II bis VI der Nikomachischen Ethik befassen sich mit verschiedenen Aspekten der Charaktertugend und in Buch II, Abschnitt 3 wird die Frage aufgeworfen, "`wie wir gut werden k"onnen, ohne es schon zu sein"' (siehe 1105 a). Mit dieser Frage und der von Aristoteles gegebenen Antwort m"ochte ich mich in diesem Essay besch"aftigen.\newline

Daf"ur werde ich wie folgt vorgehen. In Abschnitt $(i)$ erl"autere ich die "`Schwierigkeit"' und in Abschnitt $(ii)$ rekonstruiere ich die gegebene L"osung. Daraufhin...

\vspace{5mm}

\noindent\textbf{$(i)$ "`Schwierigkeit, wie wir gut werden k"onnen, ohne es schon zu sein"'}

\noindent Der Titel von Buch II, 3, "`Schwierigkeit, wie wir gut werden k"onnen, ohne es schon zu sein"' ist von Ursula Wolf gew"ahlt und kommt nicht im Originaltext vor. Das in diesem Titel beschriebene Problem wird aber in Buch II, 3 selbst nicht tats"achlich besprochen. Vielmehr verneint Aristoteles nur den m"oglichen Einwand zum im Buch Vorhergegangenen, dass man doch meinen k"onnte, dass eine tugendhaft handelnde Person doch scheinbar "`automatisch"' schon tugendhaft ist. Diese Verneinung aber wirft nun das im Titel angesprochene, und im Folgenden beschriebene Problem auf.

Eine Handlung ist nach Aristoteles \emph{nur dann} tugendhaft\footnote{"`Gut"' und "`tugendhaft"' werden hier als austauschbar behandelt.}, wenn sie von einer tugendhaften Person ausgef"uhrt wird (vgl. 1105 b). Dies bedeutet, dass die Handlung in einer "`bestimmten Verfassung"' geschieht, und zwar wissend, vors"atzlich um der Handlung selbst willen und aus einer \emph{festen} Disposition heraus (siehe 1105 a). Jedoch wird Aristoteles zufolge Charaktertugend nur durch Gew"ohnung ausgebildet, das hei"st durch das Ausf"uhren von tugendhaften Handlungen (vgl. 1103 a). 

Hieraus ergibt sich aber das Folgende: Um tapfer\footnote{Tapferkeit ist eine \emph{aret\={e}} (Tugend), denn sie ist die Mitte aus "Ubermut und Feigheit (vgl. 1107 b).} handeln zu k"onnen muss ich eine tapfere Person sein, doch um eine tapfere Person zu werden muss ich tapfer handeln. Es scheint sich hier eine Unm"oglichkeit aufgetan zu haben: Wenn ich nicht schon tapfer bin, kann ich niemals tapfer sein. Doch Aristoteles sagt in 1103 a, "`dass keine der Tugenden des Charakters in uns von Natur aus (\emph{physei}) entsteht"', sondern dass wir vielmehr von Natur aus \emph{bef"ahigt} sind, die Charaktertugenden durch Gew"ohnung auszubilden. 

Wenn wir also zwar "`gezwungen"', aber bef"ahigt sind, Charaktertugenden erst auszubilden, so muss es einen nachvollziehbaren Weg geben dies zu tun, und dieser ist laut Aristoteles die tugendhafte \emph{praxis} (Handlung). Doch wie kann der noch nicht Tugendhafte sich durch tugendhafte \emph{praxis} zum Tugendhaften machen?

\vspace{5mm}
\noindent\textbf{$(ii)$ Tugendhafte Handlung}	

\noindent Zum weiteren Verst"andnis muss der Begriff der \emph{aret\={e}} gekl"art werden, das hei"st die Frage n"aher beleuchtet werden, wann oder warum eine Handlung tugendhaft ist. Zu Beginn von Buch II, 3 zeigt Aristoteles den Unterschied zwischen der \emph{aret\={e}} und dem  Herstellungswissen. Wer eine Grammatik gut beherrschen will, muss sich so lang in ihr bet"atigen, bis er selber das n"otige Wissen besitzt, um Grammatikexperte zu sein (vgl. 1105 a). Ein grammatisch richtiger Satz ist aber schon immer grammatisch richtig, egal ob die Person, die ihn "au"sert, Grammatikexperte ist oder nicht. Bei der \emph{aret\={e}} verh"alt es sich nach Aristoteles aber so, dass eine \emph{praxis} nur dann gut ist, wenn der Handelnde in einer guten \emph{Verfassung} handelt, das hei"st \emph{wissend}, \emph{vors"atzlich um der Handlung selbst willen} und aus einer \emph{festen und unver"anderlichen Disposition} heraus. Betrachten wir also diese drei Bedingungen.

Damit meine Handlung tugendhaft genannt werden kann, muss ich sie \emph{wissend} tun, ich kann also nicht aus Versehen tugendhaft handeln. Doch worauf bezieht sich das Wissen hier genau? Damit ich etwas meine "`Handlung"' nennen kann, muss ich doch um sie wissen, und so k"onnen zuf"allige Bewegungen meines K"orpers wie Stolpern gar nicht erst als "`Handlungen"' bezeichnet werden. Das "`Wissendsein"' scheint sich daher auf die Tugendhaftigkeit beziehen zu m"ussen. Wenn ich selber nicht wei"s, dass ich gerade tugendhaft gehandelt habe, dann habe ich auch nicht tugendhaft gehandelt. Doch diese Bedingung ist sehr nah an der zweiten Bedingung der Vors"atzlichkeit um der Handlung selbst willen, da ein Fall, in der diese erf"ullt ist, aber die erste nicht, unm"oglich erscheint aufgrund der Konsequenz der ersten aus der zweiten. Die erste Bedingung der Tugendhaftigkeit einer Handlung ist daher m"oglicherweise als redundant anzusehen. Die zweite leuchtet wiederum unmittelbar ein, denn wenn ich nicht vors"atzlich handle, empfinde ich keine Lust daraus, was aber ein Kernmerkmal des Tugendhaften ist (vgl. 1105 a). 

Die letzte Bedingung der \emph{festen und unver"anderlichen Disposition} scheint hier also aufgrund der Redundanz der ersten, und der unmittelbaren Klarheit der zweiten von zentraler Bedeutung zu sein. Was ist denn also diese "`Disposition"'? Tugendhafte Handlungen "`entstehen durch h"aufiges Tun des Gerechten und M"a"sigen"', also des Tugendhaften (siehe 1105 b). F"ur Aristoteles ist eine Handlung nicht einfach tugendhaft, weil sie einem festen "`Regelwerk der Tugendhaftigkeit"' folgt (wie ein grammatikalisch richtiger Satz im Bezug auf die Grammatik), sondern weil sie aus der tugendhaften \emph{Verfassung} heraus vollzogen wird. Die feste Disposition muss also durch eine \emph{andauernde}, tugendhafte "`Handlungsgeschichte"' entstehen. 

\vspace{5mm}
\noindent\textbf{$(iii)$ Die feste Disposition}	

\noindent Zum weiteren Vorgehen ist eine Kl"arung des Begriffs der "`festen Disposition"'  n"otig. Dispositionen (\emph{hexeis}) beziehen sich auf Affekte (Furcht, Zorn, Liebe etc., begleitet von Lust und Unlust), und zwar insofern als dass sie als "`Anf"alligkeiten"' f"ur die Affekte bezeichnet werden k"onnen (vgl. 1105 b). So bin ich im "Uberma"s f"ur die Furcht disponiert, wenn ich bei jedem Ger"ausch "angstlich zusammenzucke, oder im Mangel, wenn sich noch in den schlimmsten und gef"ahrlichsten Situationen nichts in mir r"uhrt. Aristoteles argumentiert daf"ur, dass die Tugend die Mitte zwischen den Lastern Mangel und "Uberma"s bez"uglich der Dispositionen f"ur die Affekte ist (vgl. 1107 a). Die Tugenden sind also gerade die Dispositionen (vgl. 1106 a).

"`Als Anzeichen der Dispositionen m"ussen wir die Lust (\emph{h\={e}don\={e}}) oder Unlust (\emph{lyp\={e}}) nehmen, die die Taten begleitet"' (1104 b). Eine \emph{feste} Disposition ist folglich eine solche, durch die eine Person unter "ahnlichen Bedingungen konstant zu der immergleichen Art von Handlung getrieben wird. Eine Person mit einer festen Disposition f"ur ein extremes "Uberma"s an Hass wird entsprechend in allen m"oglichen Situationen immerzu Hass versp"uren - es ist gerade \emph{ihre Natur, so zu sein}. 

\vspace{5mm}
\noindent\textbf{$(iv)$ Das Erreichen der festen Disposition}	

\noindent Durch die Kl"arung der Bedingungen f"ur die Tugendhaftigkeit einer Handlung, und der festen Disposition im Speziellen, l"asst sich die "`Schwierigkeit, wie wir gut werden k"onnen, ohne es schon zu sein"' nun schon viel klarer betrachten. Es ist gerade die Natur der tugendhaften Person, immerzu gut zu handeln. Da die Disposition aber nicht gottgegeben und in Stein gemei"selt sind (), sich also zuerst eine tugendhafte Disposition durch andauerndes, gutes Handeln erarbeiten, bevor eine ihrer Handlungen tats"achlich als tugendhaft gelten kann. Es stellt sich nun daher die Frage, wie dies erreicht werden kann. 



\begin{itemize}
  \item muss alles im Einklang sein (im Gegensatz zu Schuster)
  \item Vorsatz der Tugendhaftigkeit
  \item schritt für schritt handeln, wie der tugendhafte es würde
  \item Lust entwickeln?
  \item Disposition ist affekt druch handlung
  \item  Mitte zwischen Übermaß und Mangel muss FEST WERDEN
  \item keine anstrenung volle harmonie
\end{itemize}


\vspace{5mm}
%in II 2 wird die Gewöhnung erklärt
\vspace{3mm}

\end{onehalfspace}
\nocite{*}
\bibliography{aristoteles-essay}

\end{document}
