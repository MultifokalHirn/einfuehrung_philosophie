\documentclass[a4paper]{article}
\usepackage{graphicx}
\usepackage{fullpage}
%\usepackage{parskip}
\usepackage{color}
\usepackage[ngerman]{babel}
\usepackage{hyperref}
\usepackage{calc} 
\usepackage{enumitem}
\usepackage{titlesec}
\usepackage{bussproofs}
\usepackage[export]{adjustbox}
%\pagestyle{headings}

\titleformat{name=\section,numberless}
  {\normalfont\Large\bfseries}
  {}
  {0pt}
  {}
\date{\vspace{-3ex}}
\begin{document}

\title{
    \vspace{-30pt}
	\includegraphics*[width=0.1\textwidth,left]{ErstesSem/images/hu_logo2.png}\\
	\vspace{-10pt}
	Aristoteles: Nikomachische Ethik}
\author{Lennard Wolf\\
        \small{\href{mailto:lennard.wolf@student.hu-berlin.de}{lennard.wolf@student.hu-berlin.de}}}
\maketitle
\vspace{-4pt}

\section*{Lekt"urenotiz [II]}
\large

\textbf{[II 1]} \emph{aret\={e}} des Denkens (\emph{diano\={e}tik\={e}}) durch Belehrung, \emph{aret\={e}} des Charakters (\emph{\={e}thik\={e}}) durch Gewohnheit (\emph{ethos}) $\rightarrow$ Letztere nicht durch Natur bestimmt (weil Natur nicht durch Gew"ohnung ver"andert werden kann), sondern \emph{erm"oglicht}: F"ahigkeit geht der T"atigkeit voraus, aber T"atigkeit geht der Tugend voraus! So ist ab Beginn des Lebens jede \emph{energeia} ein Schritt in Richtung gl"uckseligem Leben, oder weg davon. \newline

\noindent \textbf{[II 2]} Was erlangen wir gute \emph{praxis}? Jede \emph{energeia} ist auf angemessene Weise zu vollf"uhren, d.h. ohne Mangel oder "Uberma"s, welche jeweils die Tugenden zerst"oren, w"ahrend die Mitte sie erh"alt. Richtige Erziehung hat zum Ziel, dass der Mensch Lust bei tugendhafter, und Unlust bei untugendhafter \emph{praxis} empfindet (so wird durch Bestrafung im Kindeskopf Unlust mit der schlechter \emph{energeia} verkn"upft). Tugend wird f"alschlicherweise durch \emph{apatheia} definiert, der wahrhaft gl"uckseelige empfindet aber durch die Wahl des Richtigen (d.i. das Werthafte, das N"utzliche, das Angenehme) Lust, und durch die Wahl des Falschen (d.i. das Niedrige, das Sch"adliche, das Unangenehme) Unlust. Aus der Lust an der charakterlichen Tugend (\emph{aret\={e} \={e}thik\={e}}) entsteht mehr charakterliche Tugend, das Umgekehrte gilt genauso.\newline

\noindent \textbf{[II 3]} Doch wie soll gute \emph{praxis} vollbracht werden, wenn der Mensch noch gar nicht gut ist? Tugend wird erlernt durch das Kopieren der \emph{praxis} von Tugendhaften, dann wird es nach einige "Ubung Teil von einem, da man es zu seinem eigenen Wissen z"ahlen kann. Die \emph{praxis} ist erst dann tats"achlich gut, wenn sie von einer guten Person ausgef"uhrt werden. Theorie und Philosophie kann nicht zu Tugend f"uhren, nur \emph{praxis}.\newline

\noindent \textbf{[II 4]} Was ist die \emph{aret\={e} \={e}thik\={e}}? Kandidaten (da nur sie in Seele vorkommen): Affekt (\emph{pathos}, von Lust und Unlust begleitetes Gef"uhl), Anlage (\emph{dynamis}, Voraussetzung f"ur Affekte), Disposition (\emph{hexis}, Veranlagung zur Empf"anglichkeit f"ur gewisse Affekte) | Tugend/Laster sind nicht Affekte, denn f"ur letztere wird man nicht gelobt; sie sind nicht Anlage, denn Anlage hat man ohne Vorsatz, und auch f"ur sie wird man nicht gelobt. Durch das Ausschlussverfahren m"ussen Tugenden und Laster Dispositionen sind. \newline

\noindent \textbf{[II 5]} \emph{aret\={e}} betrifft Handlung \emph{und} Handelnden. Das Mittlere in der \emph{praxis} ist nicht f"ur jeden das Gleiche, es bezieht sich nicht auf die \emph{praxis} selber, sondern auf uns (z.B.: einer \emph{braucht} mehr Essen als ein anderer). Die Tugend hat mit all dem zu tun, wo es "Uberma"s und Mangel geben kann (siehe [II 6]). "`Menschen sind gut auf eine Art, schlecht auf viele"'. \newline

\noindent \textbf{[II 6]} (Ist in sich schon so auf den Punkt, dass ich die Definition der Tugend nicht kopieren werde.) | Nicht jede Handlung hat "Uberma"s oder Mangel $\rightarrow$ Diebstahl kann nie "Uberma"s oder Mangel haben, es ist in sich schlecht. Diese k"onnen entsprechend nie tugendhaft sein.\newline

\noindent \textbf{[II 7]} "`Nur"' Aufz"ahlung von Tugenden; Beispiel: Tapferkeit ist Mitte von Furcht und Mut.\newline

\noindent \textbf{[II 8]} Die drei Dispositionen sind "Uberma"s, Mangel und Mitte, wobei nur das Letzte eine Tugend ist. Alle drei sind einander entgegengesetzt, und so nennt der Feige den Tapferen (f"alschlicherweise) einen Tollk"uhnen, und der Tollk"uhne einen Feigen! Das wozu wir eine Neigung haben, erscheint uns weiter entfernt von der Mitte, als das andere Extrem.\newline

\noindent \textbf{[II 9]} Gutes Handeln ist selten, lobenswert und edel, da es durch das in [II 8] erw"ahnte Ph"anomen als besonders fern und schwer empfunden wird. Da das eine Extrem im Bezug zu einem selber weniger fehlerhaft ist, ist dieses zuerst aufzusuchen, um sich dann an die Mitte heranzutasten!\newline


\noindent \textbf{Frage 1:} Wie ist in [II 4] auszuschlie"sen, dass es nicht neben \emph{pathos}, \emph{dynamis} und \emph{hexis} noch anderes in der Seele gibt? Es kann sich ja auch nicht um eine Metapher handeln, sonst w"are das verwendete Ausschlussverfahren nicht zul"assig -- wie kommt er gerade auf diese drei?\newline

\noindent \textbf{Frage 2:} Die Argumentation der "`zweitbesten"' Fahrt in [II 9] leuchtet stark ein, doch ist in (nicht nur) meiner Erfahrung in ihr auch h"aufig der R"ucksprung in ein nur noch viel st"arkeres Entgegengesetztes versteckt, welches die Dividenden ausgezahlt haben will. Zu was w"urde Aristoteles hier raten?
\end{document}
