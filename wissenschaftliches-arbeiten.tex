\documentclass[emulatestandardclasses]{scrartcl}
\usepackage{graphicx}
\usepackage{CJKutf8} % japanese
\usepackage{color}
\usepackage[ngerman]{babel}
\usepackage{hyperref}
\usepackage{fullpage}
\usepackage[utf8]{inputenc}
\usepackage{calc} 
\usepackage{enumitem}
\usepackage{titlesec}
\date{\vspace{-3ex}}
\begin{document}

\title{
	\includegraphics*[bb=0 0 720 200, width=0.72\textwidth]{ErstesSem/images/hu_logo.png}\\
	\vspace{25pt}
	Einführung in das\\wissenschaftliche Arbeiten}
\subtitle{\vspace{10pt}
			Prof. Dr. Michael Mann\\
			Proseminar WS 17/18\\
          Institut für Asien- und Afrikawissenschaften\\ 
          Humboldt Universit"at zu Berlin}
\author{Lennard Wolf\\
        \small{\href{mailto:lennard.wolf@hu-berlin.de}{lennard.wolf@hu-berlin.de}}}
\maketitle
\begin{abstract}
In diesem Seminar erarbeiten sich die Studierenden eine Reihe von zentralen Diskussionsthemen, die für das Studium des BA Regionalstudien Asien/Afrika entscheidend sind.

\end{abstract}
\newpage

%\tableofcontents
%\listoffigures
\newpage

\section{Allgemeines}

\subsection{Klausur}

\begin{itemize}
  \item Hausaufgaben!!
  \item Zusammenfassung! 20.11.2017
\end{itemize}


\section{Einführungsveranstaltung\\24.10.2017}

\begin{itemize}
  \item Theisen, Manuel René: "Ratschläge für einen schlechten wissenschaftlichen Arbeiter." In Wissenschaftliches Arbeiten, 245-247. München: Vahlen, 2008.
  \item "`Grundprinzipien einer wissenschaftlichen Arbeit"'
\end{itemize}



\section{Lesen wissenschaftlicher Texte\\07.11.2017}

\begin{itemize}
  \item Anforderungen wissenschaftlicher Texte / Unterschiede zu "Normaltexten"
  \item Lesestile/-techniken, Argumentationsstruktur
  \item Rost, Friedrich: "Wissenschaftliche Texte lesen, verstehen und verarbeiten." In: Lern- und Arbeitstechniken für das Studium, 5. akt. u. erw. Auflage, S. 177-195. Wiesbaden: VS Verlag für Sozialwissenschaften. 2008
  \item \textbf{Ceska, Edward A. und Michael Ashkenazi: "Piraterie vor den afrikanischen Küsten und ihre Ursachen." Aus Politik und Zeitgeschichte (APuZ) 34-35 (2009): 33-38.}
\end{itemize}


\section{Analyse und Interpretation am Beispiel des Textes "`Piraterie vor den afrikanischen Küsten und ihre Ursachen."'\\14.11.2017}

\begin{itemize}
  \item inhaltliche Besprechung des Textes
  \item Burchert, Heiko und Sven Sohr: "Praxis des wissenschaftlichen Arbeitens: eine anwendungsorientierte Einführung." 2. akt. und erg. Auflage, S. 62-65. München:Oldenbourg, 2008.
  \item \textbf{Ceska, Edward A. und Michael Ashkenazi: "Piraterie vor den afrikanischen Küsten und ihre Ursachen." Aus Politik und Zeitgeschichte (APuZ) 34-35(2009): 33-38.}
\end{itemize}


\newpage


\end{document}
