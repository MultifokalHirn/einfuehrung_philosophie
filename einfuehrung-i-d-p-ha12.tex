\documentclass[a4paper]{article}
\usepackage{graphicx}
\usepackage{fullpage}
%\usepackage{parskip}
\usepackage{color}
\usepackage[ngerman]{babel}
\usepackage{hyperref}
\usepackage{calc} 
\usepackage{enumitem}
\usepackage{titlesec}
\usepackage{bussproofs}
\usepackage[export]{adjustbox}
%\pagestyle{headings}

\titleformat{name=\section,numberless}
  {\normalfont\Large\bfseries}
  {}
  {0pt}
  {}
\date{\vspace{-3ex}}
\begin{document}

\title{
    \vspace{-30pt}
	\includegraphics*[width=0.1\textwidth,left]{images/hu_logo2.png}\\
	\vspace{-10pt}
	Einf"uhrung in die Philosophie}
\author{Lennard Wolf\\
        \small{\href{mailto:lennard.wolf@student.hu-berlin.de}{lennard.wolf@student.hu-berlin.de}}}
\maketitle
\vspace{0pt}

\section*{AB 12: Gerechtigkeit}
\large

%\vspace{10pt}
\noindent\textbf{(i) - Situation (gerecht)}\\
\noindent Der Staat gibt an \emph{alle} Eltern von Kindern einen festen, angemessenen Geldbetrag pro Kind f"ur die Erm"oglichung der Teilhabe an Kultur und Sport. Es existieren jedoch Sonderf"alle wie z.B. der folgende: 

Eltern von Kindern mit k"orperlichen Behinderungen bekommen zus"atzliche Betr"age f"ur die Teilhabe an Leibesert"uchtigungsm"oglichkeiten f"ur Menschen mit Behinderungen, da diese  tendenziell teurer sind.\\

\noindent\textbf{(i) - Begr"undung}\\
\noindent Kultur- und Sportveranstaltungen sind Grundpfeiler des menschlichen Zusammenlebens, damit ein Grundgut, und m"ussen daher allen von fr"uh auf erm"oglicht werden, was durch den vereinheitlichten (Grund-)Betrag der Kinderf"orderung versucht wird. Das Differenzprinzip, nachdem die Schlechtergestellten "`bevorteilt"' werden sollen, wird hier auch erf"ullt: Der Preisunterschied bei Sportprogrammen f"ur Menschen mit Behinderungen wird (so weit wie m"oglich) ausgeglichen. Ulpians \emph{suum cuique} w"are hier auch erf"ullt.\\
 
\noindent\textbf{(ii) - Situation (ungerecht)}\\
\noindent Um m"oglichst gerecht zu sein, entscheidet sich die Regierung, dass ab sofort Schulunterricht nur noch in Form von (Lern-)Filmvorf"uhrungen stattfindet, die in allen Schulen verteilt gleich sind. Es folgen nach jeder Vorf"uhrung standardisierte Multiple-Choice-Tests. F"ur Kinder mit Seh- und/oder H"orbehinderungen gibt es spezielle Fassungen der Filme und Tests.

Auf diese Weise werden alle Kinder des Landes den exakt gleichen Schulunterricht erhalten.\\

\noindent\textbf{(ii) - Begr"undung}\\
\noindent Die Gleichverteilung ist hier meiner Meinung ungerecht, da es nicht die individuellen Bed"urfnisse einzelner Personen ber"ucksichtigt. Weil es immer Kinder gibt, die entweder mehr oder weniger aufnahmef"ahig, sportlich, kreativ, etc. sind als andere, sollten diese auch eine ihnen entsprechende, individuelle F"orderung bekommen, da sie sich ansonsten in einem wie oben beschriebenen System langweilen oder "uberfordert f"uhlen w"urden.  
\end{document}
