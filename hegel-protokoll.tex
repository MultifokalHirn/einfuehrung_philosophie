\documentclass[a4paper, emulatestandardclasses]{scrartcl}
\usepackage{graphicx}
\usepackage{fullpage}
%\usepackage{parskip}
\usepackage{color}
\usepackage[ngerman]{babel}
\usepackage{hyperref}
\usepackage{calc} 
\usepackage{enumitem}
\usepackage{titlesec}
\usepackage{bussproofs}
\usepackage[export]{adjustbox}
%\pagestyle{headings}

\titleformat{name=\section,numberless}
  {\normalfont\Large\bfseries}
  {}
  {0pt}
  {}
\date{\vspace{-3ex}}
\begin{document}

\title{
    \vspace{-30pt}
	\includegraphics*[width=0.1\textwidth,right]{ErstesSem/images/hu_logo2.png}\\
	\vspace{-10pt}
	Sinnliche Gewi"sheit (16.05.17)}%}\\\vspace{10pt}\small{Lennard Wolf}}
	\subtitle{Hegels Ph"anomenologie des Geistes - ausgew"ahlte Kapitel, SS 17\\
          Dr. Dimitris Karydas\\
          Protokollant: Lennard Wolf}
\maketitle
\vspace{-30pt}

\section*{R"uckblick auf die Einleitung}
%\large
\begin{description}[leftmargin=!,labelwidth=\widthof{\bfseries Unterschied des Bewusstseins}]
  \item[Bewusstsein] Prozess der Umkehrung/Selbstpr"ufung des Bewusstseins, sowie seines Bewusstseinsgegenstandes, wodurch das Bewusstsein etwas anderes wird. Und diese Ver"anderung ist die Erfahrung. So bestreitet es einen Weg, "uber welchen es zur 	Selbstthematisierung kommt.
  \item[\emph{An sich f"ur es}] Das Bild das sich das Bewusstsein von etwas macht, sein Wissen. "Uber dieses kommt das Bewusstsein zur Erfahrung. Bild vom Gegenstand entspricht nicht dem, wie er f"ur sich selbst ist (Wahrheit).
  \item[Wir] Instanz des Autors/Lesers, die dem Bewusstsein "uber die Schulter schaut und "`redupliziert"' die Erfahrung des Bewusstseins im Buch/Lesen.
  \item[Unterschied des Bewusstseins] Unterschied zwischen dem \emph{an sich f"ur es} (Wissen) und dem, womit das Bewusstsein dieses Wissen vergleicht (Wahrheit).
  \item[Vollst"andigkeit der Form] Wird erreicht dadurch, dass durch die doppelte bestimmte Negation alles einseitige Wissen "uber den Gegenstand verworfen wird.
\end{description}


Die "`Ph"anomenologie des Geistes"' bestreitet einen Weg "uber welchen der theoretische und praktische Bezug des Bewusstsein zur Welt hin zum Selbstbezug vollst"andig res"umiert wird.

%\section*{Vortrag}


\section*{Sitzung}

\emph{Sagen wir, wir kaufen Hegel das zur sinnlichen Gewissheit Geschriebene ab, was haben wir gewonnen?}

Wir haben die \emph{Form der Wahrnehmung/des Aufzeigens} erfahren. Wir haben das abstrakte Allgemeine kennengelernt, von welchem das Bewusstsein beim Erkennen immer anf"angt. Das Bewusstsein nimmt zuerst unmittelbar (sinnlich) wahr, und (erst) wenn es etwas aufzeigen will (zum Beispiel beim Begreifen des Gegenstandes), muss es auf das Allgemeine rekurrieren. Wir haben von der Wahrhaftigkeit des Allgemeinen erfahren und seine Beziehung zum einfachen Einzelnen verstanden.\newline

\noindent \emph{Ist das, was Hegel macht, \emph{neu}?}

Das \emph{Neue} des Buches ist, dass die bisherigen Resultate, zu denen die Menschheitsgeschichte schon gekommen ist, sei es durch hohe Philosophie oder durch allgemeines Gedankengut, auf neue Art und Weise zusammen gestrickt werden -- eine "`Reduplikation der Erkenntnisee"'. Hegel benutzt zu diesem Zwecke Terminologie genau wie die gemeine Sprache sie benutzt.\newline

\noindent \emph{Dreierlei muss zum Verst"andnis eines jeden Abschnittes des Buches erkannt werden:} 

Mit welchem \textbf{Meinen} (\emph{Wissen}) wird begonnen, welche \textbf{Schritte} werden mit diesem Wissen durchgef"uhrt, und zu welcher \textbf{Erkenntnis} wird gelangt.\newline

"`\emph{Vergleichen wir das Verh"altnis, in welchem das Wissen und der Gegenstand zuerst auftrat, mit dem Verh"altnisse derselben, wie sie in diesem Resultate zu stehen kommen, so hat es sich umgekehrt. Der Gegenstand, der das Wesentliche sein sollte, ist nun das Unwesentliche der sinnlichen Gewi"sheit, denn das Allgemeine, zu dem er geworden ist, ist nicht mehr ein solches, wie er f"ur sie wesentlich sein sollte, sondern sie ist itzt in dem Entgegengesetzten, n"amlich in dem Wissen, das vorher das Unwesentliche war, vorhanden.}"'\newline


%\noindent \emph{Wie verh"alt es sich in diesem Kapitel?} 

\begin{description}[leftmargin=!,labelwidth=\widthof{\bfseries Negation der Negation}]
  \item[Meinen] Wissen als Wissen des Unmittelbaren, des Seienden, dessen was uns darbietet. ("`Jetzt ist das Sein der Nacht"')
  \item[Negation] Wenn das Bewusstsein von etwas Seiendem sprechen wollen, dann spricht es automatisch vom Allgemeinen, auch wenn es das Seiende meint. Durch Kenntnis des Prozesses der Ver"anderung wird das Seiende als sekund"ar erkannt. $\rightarrow$ Wissen als Wissen des abstrakt Allgemeinen ("`Jetzt ist ein Allgemeines"')
  \item[Negation der Negation] Das Jetzt ist aber bei jeder Aufzeigung ("`Jetzt ist 12 Uhr"') schon wieder vergangen, und also nur \emph{gewesen}. Aber wenn es "`nur"' \emph{gewesen} ist, dann \emph{ist} es nicht, und darum ging es uns aber. $\rightarrow$ "`Das Jetzt \emph{ist}"'.
  \item[Erkenntnis] Allein durch das Reflektieren "uber das Einfache und das daraus resultierende Erfahren, dass es ein Allgemeines ist, kommen wir zum wahrhaften Einfachen, "`welches im Anderssein bleibt, was es ist"' 
\end{description}

"`\emph{Es erhellt, da"s die Dialektik der sinnlichen Gewi"sheit nichts anders als die einfache Geschichte ihrer Bewegung oder ihrer Erfahrung, und die sinnliche Gewi"sheit selbst nichts anders als nur diese Geschichte ist. Das nat"urliche Bewu"stsein geht
deswegen auch zu diesem Resultate, was an ihr das Wahre ist, immer selbst fort, und macht die Erfahrung dar"uber; aber vergi"st es nur ebenso immer wieder, und f"angt die Bewegung von vorne an. }"'\newline

Es bleibt nun zu kl"aren, was die \emph{Form der Wahrnehmung} ist, d.h. was es bedeutet, etwas wahrzunehmen.

\section*{Offene Fragen}

\begin{itemize}
  \item Ist Sprache sekund"ar? Wie steht sie in Beziehung mit dem Bewusstsein?
  \item Zerbricht alles in diesem Buch, wenn auch nur an einer Stelle ein Fehlschluss gemacht wird? (W"are Fehlschluss nicht nur ein weiterer Schritt im Weg zur Erkenntnis? Es gibt nicht "`das System"'. Aber wie kann dann "uberhaupt Kritik an Hegel ge"ubt werden?)
\end{itemize}


\end{document}
