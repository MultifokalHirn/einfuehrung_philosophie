\documentclass[a4paper, emulatestandardclasses, 12pt]{scrartcl}
\usepackage{graphicx}
\usepackage{hyperref}
\usepackage{fullpage}
%\usepackage{parskip}
\usepackage{color}
\usepackage[ngerman]{babel}
\usepackage{hyperref}
\usepackage{calc} 
\usepackage{enumitem}
\usepackage{titlesec}
%\pagestyle{headings}
\usepackage{setspace} %halbzeilig
\usepackage[authoryear,round]{natbib}
\bibliographystyle{natdin}

%\titleformat{name=\section,numberless}
%  {\normalfont\Large\bfseries}
%  {}
%  {0pt}
%  {}
\date{\vspace{-3ex}}
\begin{document}

\title{\vspace{5ex}
	\includegraphics*[width=0.72\textwidth]{images/hu_logo.png}\\
	\vspace{30pt}
	\scshape\LARGE{Die Form als Verbindung von Sprache und Realit"at in Wittgensteins Tractatus [AT]}}
	
	\subtitle{\vspace{20pt}Seminar zu Wittgensteins Tractatus logico-philosophicus\\
	\vspace{6pt}
          Dozent: Dr. Jasper Liptow}


\author{\vspace{-4pt}Lennard Wolf\\
        \small{\href{mailto:lennard.wolf@student.hu-berlin.de}{lennard.wolf@student.hu-berlin.de}}}      

\maketitle

\vspace{\fill}

\begin{minipage}[b]{\textwidth}
    \centering
    \onehalfspacing
    \large   
    01. Mai 2017\\
    Wintersemester 2016/2017

    \vspace{-20mm} 
\end{minipage}%
\thispagestyle{empty}
\newpage
%\clearpage
%\thispagestyle{empty}
%\tableofcontents
%\newpage
\setcounter{page}{1}

\begin{onehalfspace} 



\noindent\textbf{$(o)$ Einleitung}

\noindent Im Tractatus logico-philosophicus \citep{wittgenstein1963tractatus} beschreibt Wittgenstein seine Theorie, wie es f"ur sprachliche Aussagen m"oglich ist, Tatsachen in der Welt abbilden zu k"onnen. Bertrand Russell schrieb in seiner Einf"uhrung zu der ersten englischsprachigen Ausgabe, dass diese Theorie die Grundlage f"ur Wittgensteins Kritik an der traditionellen Philosophie sei, derzufolge das Unwissen "uber die Art der "Ubereinstimmung von Realit"at und Sprache zu den bisherigen Theorien und Fragestellungen gef"uhrt hat \cite[vgl.][S. 7]{wittgenstein1922tractatus}. Doch Wittgensteins Thesen sind nicht nur wegen der daraus folgenden Kritik so interessant, sondern weil sie gerade eine fundamentale Theorie "uber die "`Harmonie"' zwischen Sprache und Realit"at darstellen \cite[vgl.][S. 1]{emiliani1999formsp}.

In dieser Hausarbeit m"ochte ich das Konzept der \emph{Form} erl"autern und zeigen, wie es die Grundlage schafft f"ur Wittgensteins Theorie der Abbildbarkeit von Realit"at durch Sprache. Zudem versuche ich zu zeigen, wie anhand dieser Theorie die Begriffe "`sinnvoll"', "`sinnlos"', "`wahr"', "`falsch"' und "`unsinnig"' im Tractatus zu verstehen sind.

%Daf"ur entwickelt er das Konzept der \emph{Form}, welches ihm zufolge die Grundlage f"ur die M"oglichkeit von Abbildung darstellt. Diese Idee m"ochte ich daher in dieser Hausarbeit näher erl"autern und zeigen, wie damit die Begriffe "`sinnvoll"', "`sinnlos"', "`wahr"', "`falsch"' und "`unsinnig"' zu verstehen sind. 

Daf"ur werde ich wie folgt vorgehen. In Abschnitt $(i)$ werde ich die f"ur das Verst"andnis der Idee der Form zentralen Begriffe einf"uhren. Daraufhin befasse ich mich in $(ii)$ mit der Form im Allgemeinen und wie sie zu verstehen ist. In $(iii)$ zeige ich, wie das Bild auf Grund seiner Form und der der Tatsache miteinander in Verbindung stehen k"onnen. In $(iv)$ und $(v)$ erl"autere ich jeweils, wie f"ur Wittgenstein Abbildungen sinnvoll, sinnlos und unsinnig sein k"onnen, und wie "`wahre"' sprachliche Aussagen m"oglich sind. Abschnitt $(vi)$ stellt die Konklusion dieses Essays dar.
\vspace{5mm}

%\begin{itemize}
%  \item Grundbegriffe
%  \item Die Form
%  \item Die Verbindung von Bild und Tatsache
%  \item Sinnvolle und sinnlose Abbildungen
%  \item Unsinnige Abbildungen, Unsagbares
%  \item Konklusion
%\end{itemize}
%\cite[vgl.][2.021]{wittgenstein1963tractatus}


\noindent\textbf{$(i)$ Grundbegriffe}

\noindent Zu Beginn des Tractatus entwickelt Wittgenstein eine Ontologie, deren Verst"andnis Voraussetzung f"ur das Verstehen des \emph{Form}-Konzepts ist. Grundbegriffe dieser Ontologie sind "`Gegenstand"', "`Sachverhalt"', "`Tatsache"', "`Welt"' und "`Bild"', welche ich in diesen Abschnitt erkl"aren werde.\footnote{In dieser Hausarbeit tue ich es Wittgenstein gleich und schreibe "uber Dinge, "uber die man (eigentlich) nicht sprechen kann, da es mir nicht anders m"oglich erscheint, "uber den Inhalt des Tractatus zu reden. Warum es \emph{unsagbares} geben soll, wird in Abschnitt $(v)$ besprochen.}

\emph{Gegenst"ande} sind die Grundbausteine der \emph{Welt} und bilden ihre \emph{Substanz} (siehe 2.021).\footnote{In dieser Hausarbeit werde ich zur besseren Lesbarkeit nur die Satznummern aus dem Tractatus angeben, welche sich dann immer auf \cite{wittgenstein1963tractatus} beziehen.} Sie k"onnen miteinander zu \emph{Sachverhalten} verbunden sein, und die \emph{M"oglichkeit} f"ur einen Gegenstand sich in einem bestimmten Sachverhalt zu befinden, ist bestimmt durch seine \emph{Form} (siehe 2.01, 2.0141). \emph{Tatsachen} sind bestehende Sachverhalte und die Welt ist die Gesamtheit der Tatsachen (siehe 2, 1.1). Wenn also Gegenst"ande in einem bestimmten Sachverhalt vern"upft sein k"onnen, und dies in der Welt so tats"achlich der Fall ist, dann kann von einer Tatsache gesprochen werden. Was genau aber ein Gegenstand ist, wird von Wittgenstein nicht weiter spezifiziert, da er dies auch nicht als seine Aufgabe betrachtete.\footnote{In \cite[S. 70]{malcolm2001ludwig} findet man diesbez"uglich eine Antwort Wittgensteins auf die Frage, warum er keine Beispiele f"ur Gegenst"ande im Tractatus gegeben habe: "`that it was not his business, as a logician, to try to decide whether this thing or that was a simple thing or a complex thing, that being a purely \emph{empirical} matter"'.} Es l"asst sich aber behaupten, dass zum Beispiel ein Stuhl nicht als Gegenstand zu bezeichnen sein m"usste, sondern als \emph{Komplex} von Gegenst"anden, die durch ihren Zusammenschluss in einem Sachverhalt ebendiesen ergeben (vgl. 5.5423). Gegenst"ande sind wom"oglich zu begreifen als die allerkleinsten Dinge, die es in der Welt gibt, da sie in 2.02 als \emph{einfach}, also als selber nicht weiter unterteilbar, beschrieben werden. Eine komplexe Tatsache wie das Sitzen einer Person auf einem Stuhl liesse sich theoretisch durchanalysieren und zerteilen in eine gro"se Menge von nicht mehr teilbaren Sachverhalten, also Verbindungen zwischen den fundamentalen Bestandteilen der Welt, bei den alten Griechen "`Atome"' genannt (siehe \citealt{sep-atomism-ancient}). 
%Diese von Wittgenstein als "`Gegenst"ande"' bezeichneten Atome sind ein Grundbaustein der "Uberlegungen der Denker*innen des logischen Atomismus. Jene versuchten, ein Verst"andnis der Welt zu erlangen, welches komplett mit Logik vereinbar und beschreibbar ist, 
Diese von Wittgenstein als "`Gegenst"ande"' bezeichneten Atome sind f"ur ihn allein durch logische "Uberlegungen zu erreichen und nicht durch physikalische Analysen\footnote{"`In Russell's opinion, what makes it appropriate to speak of logical atomism is that the atoms in question are to be arrived at by logical rather than physical analysis"' \citep{sep-wittgenstein-atomism}.}, weshalb sie auch als \emph{logischen Atome} bezeichnet werden k"onnen.\footnote{"`Logical atoms are the ultimate entities reached by the analysis of situations"' \cite[S. 47]{frascolla2007understanding}. Die Meinung, dass fundamentale Gegenst"ande tats"achlich existieren, schien Wittgenstein jedoch nicht schon immer klar vertreten zu haben: "`Wittgenstein's 1914–-1916 Notebooks bear witness to his incessant and tormented labour on the question of the existence of logically ultimate components of reality. Nonetheless, every doubt and hesitation appears to be overcome in the Tractatus"' \cite[S. 48]{frascolla2007understanding}.} %\emph{Wie} genau Farbigkeit, Masse etc. durch die Verbindung von Gegenst"anden entsteht, ist nicht 

\emph{Bilder} sind im Tractatus nicht nur als visuelle Darstellung zu verstehen, sondern als jeder Art Abbildung von Sachverhalten: "`Das Bild ist ein Modell der Wirklichkeit"' (2.12). Wittgenstein befasst sich im Speziellen mit sprachlichen Aussagen, beziehungsweise \emph{S"atzen}, die entsprechend eine Art von Bildern sind \cite[vgl.][S. 66]{mcguinness2002approaches}. Das rein logische Bild ist der \emph{Gedanke}, welcher sich im Satz sinnlich wahrnehmbar ausdr"uckt (siehe 3, 3.1). Ein Bild stellt einen Sachverhalt genau dann (erfolgreich) dar, wenn eine \emph{abbildende Beziehung}, also eine Zuordnung, zwischen den Elementen des Bildes und den Gegenst"anden des dargestellten Sachverhaltes besteht (siehe 2.13, 2.1513, 2.1514). Dies ist dann m"oglich, wenn Bild und Sachverhalt die \emph{Form der Abbildung} gemeinsam haben (siehe 2.17). Ziel dieser Hausarbeit wird es unter anderem sein, dieses Konzept zu erkl"aren. 

Die im Tractatus zentral behandelten Arten von Bildern sind Gedanken und S"atze, also sprachliche Aussagen. Im weiteren Verlauf dieser Hausarbeit werde ich mich auch nur noch auf diese beziehen.


\vspace{5mm}
\noindent\textbf{$(ii)$ Die Form}	

\noindent Beim Tractatus handelt es sich um eine \emph{philosophisch-logische} Abhandlung, und so sind die Gedanken in diesem Werk logischer Natur, nicht naturwissenschaftlicher. Wie genau Gegenst"ande und ihre Verbindungen zu Tatsachen Ph"anomene wie Raum, Zeit, Masse und Energie in der Welt hervorbringen wird nicht behandelt, weil es dabei um empirische Erkenntnisse ginge, und nicht um philosophisch-logische. So ist auch die Form eines Gegenstands nicht als seine bestimmt geformte Ausbreitung im Raum zu verstehen, sondern als eine rein \emph{logische Eigenschaft} des \emph{logischen Atoms} Gegenstand. Es ist nat"urlich denkbar, dass die Form eines Gegenstands auch seine bestimmt geformte Ausbreitung im Raum \emph{zur Folge} haben k"onnte, doch w"are dies ausschlie"slich ein Epiph"anomen (???) und au"serhalb des Themenbereichs des Tractatus. 

Jedoch haben nicht nur Gegenst"ande eine Form, sondern auch die Elemente von Bildern, wie zum Beispiel Namen oder sprachliche Ausdr"ucke im Allgemeinen. Wenn von der Form eines Gegenstands oder Bildelements die Rede ist, wird damit auf seine kombinatorischen M"oglichkeiten mit anderen Gegenst"anden/Bildelementen zu Sachverhalten beziehungsweise Bildern hingewiesen \cite[vgl.][S. 84]{emiliani1999formsp}. Die Form eines sprachlichen Ausdrucks w"are durch die Grammatik der benutzten Sprache bestimmt und liesse sich durch S"atze in denen er vorkommt \emph{zeigen}. Die Form eines Gegenstandes w"urde sich ebenso durch die Sachverhalte, in denen er vorkommt, \emph{zeigen}. Man k"onnte sie jeweils jedoch nicht beschreiben, denn "`um die logische Form darstellen zu k"onnen, m"ussten wir uns mit dem Satze au"serhalb der Logik aufstellen k"onnen, das hei"st au"serhalb der Welt"' (siehe 4.12). 


%\begin{itemize}
%  \item was ist logische form - jedenfalls wird sprachlichelogik, oder grammatische logik verwechselt mit tatsächlicher logik 
%  \item da die logik sich in der welt zeigt und wir keine unlogischen sätze sagen können muss alles eine logische form haben das darstellbar sein kann
%  \item Form der Abbildung: r"aumliches Bild kann r"aumliches Darstellen vgl Emil s 83
%  %\item logische form ist verallgemeinerung der Form der Abbildung (vllt eher abbildende Form: pictorial Form)
%\end{itemize}


\vspace{5mm}
\noindent\textbf{$(iii)$ Die Verbindung von Sprache und Realit"at}	
\vspace{3mm}
\begin{addmargin}[.065\linewidth]{.065\linewidth}% indent 0pt left, .5\linewidth right
\footnotesize

\noindent \emph{"`What relation must one fact (such as a sentence) have to another in order to be capable of being a symbol for that other? This last is a logical question, and is the one with which Mr Wittgenstein is concerned. He is concerned with the conditions for accurate Symbolism, i.e. for Symbolism in which a sentence `means' something quite definite."'} -- Bertrand Russell \cite[S. 7]{wittgenstein1922tractatus}

\end{addmargin}
\normalsize
\vspace{3mm}

Es scheint vern"unftig zu sein, anzunehmen, dass es eine "`Verbindung"' zwischen Sprache und Realit"at gibt \cite[vgl.][S. 1 f.]{emiliani1999formsp}. Wenn ein Satz eine Situation beschreibt, so muss er doch mit ihr verbunden sein, oder wenigstens etwas mit ihr gemein haben, oder ihr "ahnlich sein (vgl. 4.03, 2.16, 2.161). F"ur Wittgenstein stellte sich aus dieser intuitiven Annahme die Frage, worin diese Verbindung besteht. Er war der Meinung, dass eine Sprache, die etwas "`"uber die Welt ausdr"ucken kann, gewisse Eigenschaften der Welt vermittels dieser Eigenschaften, die sie besitzen mu"s"', \emph{spiegele} \cite[siehe][S. 209]{wittgenstein1963tractatus}. Das hei"st, dass Sprache und Welt gemeinsame Eigenschaften haben m"ussen, und diese meint Wittgenstein in der \emph{logischen Form der Abbildung} gefunden zu haben (siehe 2.2). Der Satz ist die sinnlich wahrnehmbare Form des Gedankens, des \emph{logischen Bildes der abgebildeten Tatsache} (siehe 4.03).

Doch wie repr"asentiert ein Gedanke, das logische Bild der Tatsache, diese? \citet{frascolla2007understanding} zufolge tut er dies durch einen Doppelmechanismus: Erstens stehen die Elemente des Gedankens als Symbole f"ur die Elemente der Tatsache ein und zweitens wird die logische Form der Tatsache von der logischen Struktur des Gedankens reproduziert.\footnote{Sinngem"a"se "Ubersetzung des Autors von "`a thought represents a situation by means of a double mechanism: (1) the elements of the thought play the role of being proxy for the components of the situation; and (2) the logical form of the situation is absorbed and reproduced in the structure of the thought (the distinction form/structure corresponds to the distinction possibility/actuality)"' \cite[S.46]{frascolla2007understanding}} the distinction form/structure corresponds to the distinction possibility/actuality Diese beiden Mechanismen, die es der Sprache erm"oglichen, die Welt zu \emph{spiegeln}, m"ussen nun genauer gekl"art werden.

Damit diese beiden Vorraussetzungen f"ur die Abbildung m"oglich sind, muss aber noch vorher etwas gelten: Das Bild muss eine \emph{Tatsache} sein (2.141), und ist daher gerade nicht, wie man intuitiv annehmen k"onnte, ein komplexer Gegenstand. Wittgenstein meint damit, dass XXXX frascolla S. 20

WIE KANN EIN SYMBOL EIN ELEMENT SYMBOLISIEREN? logische form

Das grammatikalische Symbol muss die selbe logische Form wie das korrespondierende Element im dargestellten Sachverhalt haben. Beispiel.

WIE KANN DIE STRUKTUR EINES GEDANKEN DIE LOGISCHE FORM REPRODUZIEREN? f. d. abbildung




%SILF s. 85 forms of sense: Syntax/Grammatik hat die Aufgabe den sprachlichen Ausdr"ucken den selben Kombinationsraum zu geben wie die ihn die korrespondierenden Gegenst"ande haben. Die grammatik gibt dem Ausdruck also seine logische Form. Entsprechend sind unm"ogliche Gegenstandsverbindungen/Sachverhalte sprachlich nicht korrekt auszudr"ucken. "Es kann nichts unlogisches gedacht werden"-Zitat


\vspace{5mm}
\noindent\textbf{$(iv)$ Sinnvolle und sinnlose S"atze}	

2.21
Das Bild stimmt mit der Wirklichkeit überein oder nicht; es ist richtig oder unrichtig, wahr oder falsch.

\vspace{5mm}
\noindent\textbf{$(v)$ Unsinnige S"atze und Unsagbares}	

Ein Satz soll also eine definitive Bedeutung haben, doch was sind S"atze ohne solch eine definitive Bedeutung? Was ist der Unterschied zu solchen die eine haben? MN 209

Um die logische Form darstellen zu können, müssten wir uns mit dem Satze außerhalb der Logik aufstellen können, das heißt außerhalb der Welt. 4.12

\vspace{5mm}
\noindent\textbf{$(vi)$ Kritik?}	

Problem ist dass nicht erkennbar ist ob Sprache tatsächlich die EIgenschaften der Welt spiegelt. Unsere Gehirne nehmen selektiv war und geben, wie auch andere Tiere, DIngen Bedeutung je nachdem wie wichtig und unwichtig sie für unser überleben sind. Beispiel: Frosch such nur nach etwas kleinem das sich bewegt (Insekt) und ist sehr leicht täuschbar, denn im Normalfall reicht diese Wahrnehmung. DIese Selektivität in der Wahrnehmung könnte großen einfluss auf unsere Sprache haben, da diese also vielmehr die \emph{Realität unserer Wahrnehmung} spiegeln würde, als die tatsächliche Realit"at!

\vspace{5mm}
\noindent\textbf{$(vii)$ Konklusion}

\end{onehalfspace}
\nocite{*}
\bibliography{tractatus-essay}

\end{document}
