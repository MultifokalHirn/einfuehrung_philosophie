\documentclass[a4paper, emulatestandardclasses, 12pt]{scrartcl}
\usepackage{graphicx}
\usepackage{fullpage}
%\usepackage{parskip}
\usepackage{color}
\usepackage[ngerman]{babel}
\usepackage{hyperref}
\usepackage{calc} 
\usepackage{enumitem}
\usepackage{titlesec}
%\pagestyle{headings}
\usepackage{setspace} %halbzeilig
\usepackage[authoryear,round]{natbib}
\bibliographystyle{natdin}

%\titleformat{name=\section,numberless}
%  {\normalfont\Large\bfseries}
%  {}
%  {0pt}
%  {}
\date{\vspace{-3ex}}
\begin{document}

\title{\vspace{5ex}
	\includegraphics*[width=0.72\textwidth]{images/hu_logo.png}\\
	\vspace{30pt}
	\scshape\LARGE{Die Form als Grundlage der Abbildbarkeit in Wittgensteins Tractatus [AT]}}
	
	\subtitle{\vspace{20pt}Seminar zu Wittgensteins Tractatus logico-philosophicus\\
	\vspace{6pt}
          Dozent: Dr. Jasper Liptow}


\author{\vspace{-4pt}Lennard Wolf\\
        \small{\href{mailto:lennard.wolf@student.hu-berlin.de}{lennard.wolf@student.hu-berlin.de}}}      

\maketitle

\vspace{\fill}

\begin{minipage}[b]{\textwidth}
    \centering
    \onehalfspacing
    \large   
    01. Mai 2017\\
    Wintersemester 2016/2017

    \vspace{-20mm} 
\end{minipage}%
\thispagestyle{empty}
\newpage
\clearpage
\thispagestyle{empty}
\tableofcontents
\newpage
\setcounter{page}{1}

\begin{onehalfspace} 



\noindent\textbf{$(o)$ Einleitung}

\noindent In seinem Tractatus logico-philosophicus \citep{wittgenstein1963tractatus} beschreibt der fr"uhe Wittgenstein, wie es f"ur sprachliche Aussagen im Speziellen, und Bildern im Allgemeinen, m"oglich ist, Tatsachen in der Welt abbilden zu k"onnen. Daf"ur entwickelt er das Konzept der \emph{Form}, welches ihm zufolge die Grundlage f"ur die M"oglichkeit von Abbildung darstellt. Diese Idee m"ochte ich in dieser Hausarbeit näher erl"autern und zeigen, wie damit die Begriffe "`sinnvoll"', "`sinnlos"', "`wahr"', "`falsch"' und "`unsinnig"' zu verstehen sind.

Daf"ur werde ich wie folgt vorgehen. In Abschnitt $(i)$ werde ich die f"ur das Verst"andnis der Idee der Form zentralen Begriffe einf"uhren. Daraufhin befasse ich mich in $(ii)$ mit der Form im Allgemeinen und wie sie zu verstehen ist. In $(iii)$ zeige ich, wie das Bild auf Grund seiner Form und der der Tatsache miteinander in Verbindung stehen k"onnen. In $(iv)$ und $(v)$ erl"autere ich jeweils, wie f"ur Wittgenstein Abbildungen sinnvoll, sinnlos und unsinnig sein k"onnen, und wie "`wahre"' sprachliche Aussagen m"oglich sind. Abschnitt $(vi)$ stellt die Konklusion dieses Essays dar.
\vspace{5mm}

%\begin{itemize}
%  \item Grundbegriffe
%  \item Die Form
%  \item Die Verbindung von Bild und Tatsache
%  \item Sinnvolle und sinnlose Abbildungen
%  \item Unsinnige Abbildungen, Unsagbares
%  \item Konklusion
%\end{itemize}
%\cite[vgl.][2.021]{wittgenstein1963tractatus}


\noindent\textbf{$(i)$ Grundbegriffe}

\noindent Zu Beginn des Tractatus beschreibt Wittgenstein eine Ontologie, deren Verst"andnis Voraussetzung f"ur das Verst"andnis des \emph{Form}-Konzepts ist. Die Grundbegriffe dieser Ontologie sind "`Gegenstand"', "`Sachverhalt"', "`Tatsache"', "`Welt"' und "`Bild"', welche ich nun kurz erkl"aren will.\footnote{In dieser Hausarbeit tue ich es Wittgenstein gleich und schreibe "uber Dinge, "uber die man (eigentlich) nicht sprechen kann, da es mir nicht anders m"oglich erscheint, "uber den Inhalt des Tractatus zu reden. Warum es \emph{unsagbares} geben soll, wird in Abschnitt $(v)$ besprochen.}

\emph{Gegenst"ande} sind die Grundbausteine des materiellen Universums, oder auch \emph{Welt}, und stellen ihre \emph{Substanz} dar (vgl. 2.021)\footnote{In dieser Hausarbeit werde ich zur besseren Lesbarkeit nur die Satznummern aus dem Tractaus angeben, welche sich dann immer auf \cite{wittgenstein1963tractatus} beziehen.}. Sie k"onnen miteinander zu \emph{Sachverhalten} verbunden sein, und die M"oglichkeit f"ur einen Gegenstand sich in einem bestimmten Sachverhalt zu befinden, ist bestimmt durch seine \emph{Form} (vgl. 2.01, 2.0141), doch dazu sp"ater mehr. \emph{Tatsachen} sind bestehende Sachverhalte und die Welt ist die Gesamtheit der Tatsachen (vgl. 2, 1.1).



\vspace{5mm}
\noindent\textbf{$(ii)$ Die Form}	

\noindent Die M"oglichkeit seines Vorkommens in Sachverhalten, ist die Form des Gegenstandes. (2.0141)

Form des Bildes?

Gemeinsamkeit

\end{onehalfspace}

\bibliography{tractatus-essay}

\end{document}
