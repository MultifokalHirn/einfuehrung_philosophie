\documentclass[a4paper, emulatestandardclasses, 12pt]{scrartcl}
\usepackage{graphicx}
\usepackage{fullpage}
%\usepackage{parskip}
\usepackage{color}
\usepackage[ngerman]{babel}
\usepackage{hyperref}
\usepackage{calc} 
\usepackage{enumitem}
\usepackage{titlesec}
%\pagestyle{headings}
\usepackage{setspace} %halbzeilig
\usepackage[authoryear,round]{natbib}
\bibliographystyle{natdin}

%\titleformat{name=\section,numberless}
%  {\normalfont\Large\bfseries}
%  {}
%  {0pt}
%  {}
\date{\vspace{-3ex}}
\begin{document}

\title{\vspace{5ex}
	\includegraphics*[width=0.72\textwidth]{images/hu_logo.png}\\
	\vspace{30pt}
	\scshape\LARGE{Die Form als Grundlage der Abbildbarkeit in Wittgensteins Tractatus [AT]}}
	
	\subtitle{\vspace{20pt}Seminar zu Wittgensteins Tractatus logico-philosophicus\\
	\vspace{6pt}
          Dozent: Dr. Jasper Liptow}


\author{\vspace{-4pt}Lennard Wolf\\
        \small{\href{mailto:lennard.wolf@student.hu-berlin.de}{lennard.wolf@student.hu-berlin.de}}}      

\maketitle

\vspace{\fill}

\begin{minipage}[b]{\textwidth}
    \centering
    \onehalfspacing
    \large   
    01. Mai 2017\\
    Wintersemester 2016/2017

    \vspace{-20mm} 
\end{minipage}%
\thispagestyle{empty}
\newpage
\clearpage
\thispagestyle{empty}
\tableofcontents
\newpage
\setcounter{page}{1}

\begin{onehalfspace} 



\noindent\textbf{$(o)$ Einleitung}

\noindent In seinem Tractatus logico-philosophicus \citep{wittgenstein1963tractatus} beschreibt der fr"uhe Wittgenstein, wie es f"ur sprachliche Aussagen im Speziellen, und Bildern im Allgemeinen, m"oglich ist, Tatsachen in der Welt abbilden zu k"onnen. Daf"ur entwickelt er das Konzept der \emph{Form}, welches ihm zufolge die Grundlage f"ur die M"oglichkeit von Abbildung darstellt. Diese Idee m"ochte ich in dieser Hausarbeit näher erl"autern und zeigen, wie damit die Begriffe "`sinnvoll"', "`sinnlos"', "`wahr"', "`falsch"' und "`unsinnig"' zu verstehen sind.

Daf"ur werde ich wie folgt vorgehen. In Abschnitt $(i)$ werde ich die f"ur das Verst"andnis der Idee der Form zentralen Begriffe einf"uhren. Daraufhin befasse ich mich in $(ii)$ mit der Form im Allgemeinen und wie sie zu verstehen ist. In $(iii)$ zeige ich, wie das Bild auf Grund seiner Form und der der Tatsache miteinander in Verbindung stehen k"onnen. In $(iv)$ und $(v)$ erl"autere ich jeweils, wie f"ur Wittgenstein Abbildungen sinnvoll, sinnlos und unsinnig sein k"onnen, und wie "`wahre"' sprachliche Aussagen m"oglich sind. Abschnitt $(vi)$ stellt die Konklusion dieses Essays dar.
\vspace{5mm}

%\begin{itemize}
%  \item Grundbegriffe
%  \item Die Form
%  \item Die Verbindung von Bild und Tatsache
%  \item Sinnvolle und sinnlose Abbildungen
%  \item Unsinnige Abbildungen, Unsagbares
%  \item Konklusion
%\end{itemize}
%\cite[vgl.][2.021]{wittgenstein1963tractatus}


\noindent\textbf{$(i)$ Grundbegriffe}

\noindent Zu Beginn des Tractatus beschreibt Wittgenstein eine Ontologie, deren Verst"andnis Voraussetzung f"ur das Verst"andnis des \emph{Form}-Konzepts ist. Grundbegriffe dieser Ontologie sind "`Gegenstand"', "`Sachverhalt"', "`Tatsache"', "`Welt"' und "`Bild"', welche ich in diesen Abschnitt erkl"aren werde.\footnote{In dieser Hausarbeit tue ich es Wittgenstein gleich und schreibe "uber Dinge, "uber die man (eigentlich) nicht sprechen kann, da es mir nicht anders m"oglich erscheint, "uber den Inhalt des Tractatus zu reden. Warum es \emph{unsagbares} geben soll, wird in Abschnitt $(v)$ besprochen.}

\emph{Gegenst"ande} sind die Grundbausteine des materiellen Universums, oder auch \emph{Welt}, und bilden ihre \emph{Substanz} (siehe 2.021).\footnote{In dieser Hausarbeit werde ich zur besseren Lesbarkeit nur die Satznummern aus dem Tractatus angeben, welche sich dann immer auf \cite{wittgenstein1963tractatus} beziehen.} Sie k"onnen miteinander zu \emph{Sachverhalten} verbunden sein, und die \emph{M"oglichkeit} f"ur einen Gegenstand sich in einem bestimmten Sachverhalt zu befinden, ist bestimmt durch seine \emph{Form} (siehe 2.01, 2.0141). \emph{Tatsachen} sind bestehende Sachverhalte und die Welt ist die Gesamtheit der Tatsachen (siehe 2, 1.1). Wenn also Gegenst"ande in einem bestimmten Sachverhalt vern"upft sein k"onnen, und dies in der Welt so tats"achlich der Fall ist, dann kann von einer Tatsache gesprochen werden. Was genau aber ein Gegenstand ist, wird von Wittgenstein nicht weiter spezifiziert, da er dies auch nicht als seine Aufgabe betrachtete.\footnote{In \cite[S. 70]{malcolm2001ludwig} findet man diesbez"uglich eine Antwort Wittgensteins auf die Frage, warum er keine Beispiele f"ur Gegenst"ande im Tractatus gegeben habe: "`that it was not his business, as a logician, to try to decide whether this thing or that was a simple thing or a complex thing, that being a purely \emph{empirical} matter"'.} Es l"asst sich aber behaupten, dass zum Beispiel ein Stuhl nicht als Gegenstand zu bezeichnen sein m"usste, sondern als \emph{Komplex} von Gegenst"anden, die durch ihren Zusammenschluss in einem Sachverhalt ebendiesen ergeben (vgl. 5.5423). Gegenst"ande sind wom"oglich zu begreifen als die allerkleinsten Dinge, die es in der Welt gibt, da sie in 2.02 als \emph{einfach}, also als selber nicht weiter unterteilbar, beschrieben werden. Eine komplexen Tatsache wie das Sitzen einer Person auf einem Stuhl liesse sich theoretisch durchanalysieren und zerteilen in eine gro"se Menge von nicht mehr teilbaren Sachverhalten, also Verbindungen zwischen den kleinsten Bestandteilen der Welt, bei den alten Griechen "`Atom"' genannt. Diese von Wittgenstein als "`Gegenst"ande"' bezeichneten "`Atome"' sind der Kern der "Uberlegungen der Denker*innen des logischen Atomismus, und werden daher auch \emph{logische Atome} genannt.\footnote{"`Logical atoms are the ultimate entities reached by the analysis of situations"' \cite[S. 47]{frascolla2007understanding}. Die Meinung, dass kleinste Gegenst"ande tats"achlich existieren, schien Wittgenstein jedoch nicht schon immer klar vertreten zu haben: "`Wittgenstein's 1914–-1916 Notebooks bear witness to his incessant and tormented labour on the question of the existence of logically ultimate components of reality. Nonetheless, every doubt and hesitation appears to be overcome in the Tractatus"' \cite[S. 48]{frascolla2007understanding}.} %\emph{Wie} genau Farbigkeit, Masse etc. durch die Verbindung von Gegenst"anden entsteht, ist nicht 

\emph{Bilder} sind im Tractatus nicht nur als visuelle Darstellung zu verstehen, sondern als jeder Art Abbildung von Sachverhalten. Wittgenstein befasst sich im speziellen mit sprachlichen Aussagen, beziehungsweise \emph{S"atzen}, die also auch als Bilder zu bezeichnen sind. Das rein logische Bild ist der \emph{Gedanke}, welcher sich im Satz sinnlich wahrnehmbar ausdr"uckt (siehe 3, 3.1). Ein Bild stellt einen Sachverhalt genau dann (erfolgreich) dar, wenn eine \emph{abbildende Beziehung}, also eine Zuordnung, zwischen den Elementen des Bildes und den Gegenst"anden des dargestellten Sachverhaltes besteht (siehe 2.13, 2.1513, 2.1514). Dies ist dann m"oglich, wenn Bild und Sachverhalt die \emph{Form der Abbildung} gemeinsam haben, was im n"achsten Abschnitt n"aher erl"autert wird. 


\vspace{5mm}
\noindent\textbf{$(ii)$ Die Form}	

\noindent 

\begin{itemize}
  \item wir reden von gegenständen nur im logischen raum
  \item form ist daher zwar zur veranschaulichung als puzzleteil zu verstehen, jedoch dann als mehrdimensionales, doch am ende geht es um die rein logische M"oglichkeit, mit anderen Gegenst"anden in logischen Beziehungen zu Sachverhalten verbunden sein zu k"onnen.
\end{itemize}



Die Vorraussetzung daf"ur, dass ein Bild einen Sachverhalt darstellen kann, ist die gemeinsame \emph{Form der Abbildung} von Bild und Sachverhalt. 


Die M"oglichkeit seines Vorkommens in Sachverhalten, ist die Form des Gegenstandes. (2.0141)

2.2
Das Bild hat mit dem Abgebildeten die logische Form der Abbildung gemein.

Form der Abbildung: 2.15

Gemeinsamkeit

\vspace{5mm}
\noindent\textbf{$(iii)$ Die Verbindung von Bild und Tatsache}	

\noindent "`a thought represents a situation by means of a double mechanism: (1) the elements of the thought play the role of being proxy for the components of the situation; and (2) the logical form of the situation is absorbed and reproduced in the structure of the thought (the distinction form/structure corresponds to the distinction possibility/actuality)"' \cite[S.46]{frascolla2007understanding}


\vspace{5mm}
\noindent\textbf{$(iv)$ Abbildungen}	

2.21
Das Bild stimmt mit der Wirklichkeit überein oder nicht; es ist richtig oder unrichtig, wahr oder falsch.

\end{onehalfspace}
\nocite{*}
\bibliography{tractatus-essay}

\end{document}
