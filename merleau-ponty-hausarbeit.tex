\documentclass[a4paper, 12pt]{article}
\usepackage{CJKutf8} % japanese
\usepackage{graphicx}
\usepackage{hyperref}
\usepackage{fullpage}
%\usepackage{parskip}
\usepackage{color}
\usepackage[ngerman]{babel}
\usepackage{hyperref}
\usepackage{calc} 
\usepackage{enumitem}
\usepackage[utf8]{inputenc}
\usepackage{titlesec}
%\pagestyle{headings}
\usepackage{setspace} %halbzeilig
\usepackage[style=authoryear-ibid,natbib=true]{biblatex}
\usepackage[hang]{footmisc}
\setlength{\footnotemargin}{-0.8em}
%\bibliographystyle{natdin}
\addbibresource{merleau-ponty-hausarbeit.bib}
\DeclareDatamodelEntrytypes{standard}
\DeclareDatamodelEntryfields[standard]{type,number}
\DeclareBibliographyDriver{standard}{%
  \usebibmacro{bibindex}%
  \usebibmacro{begentry}%
  \usebibmacro{author}%
  \setunit{\labelnamepunct}\newblock
  \usebibmacro{title}%
  \newunit\newblock
  \printfield{number}%
  \setunit{\addspace}\newblock
  \printfield[parens]{type}%
  \newunit\newblock
  \usebibmacro{location+date}%
  \newunit\newblock
  \iftoggle{bbx:url}
    {\usebibmacro{url+urldate}}
    {}%
  \newunit\newblock
  \usebibmacro{addendum+pubstate}%
  \setunit{\bibpagerefpunct}\newblock
  \usebibmacro{pageref}%
  \newunit\newblock
  \usebibmacro{related}%
  \usebibmacro{finentry}}

%\titleformat{name=\section,numberless}
%  {\normalfont\Large\bfseries}
%  {}
%  {0pt}
%  {}
\date{\vspace{-3ex}}
\begin{document}

\title{\vspace{5ex}
	\includegraphics*[bb=0 0 720 200, width=0.72\textwidth]{ErstesSem/images/hu_logo.png}\\
	\vspace{30pt}
	\scshape\LARGE{"`Das Wort selbst ist Tr"ager des Sinnes"'}\\\Large{Sinn, Ausdruck und Sprache bei Merleau-Ponty}\\\vspace{20pt}}
	


\author{Merleau-Ponty: Ph"anomenologie der Wahrnehmung (PS)\\
	\vspace{7pt}
          Dozent: Dr. Matthias Schlo"sberger\\\vspace{4pt}Lennard Wolf\\
        \small{Matrikelnummer: 12345}\\
        \small{E-Mail: lennard.wolf@student.hu-berlin.de}\\
        \small{Telefonnummer: +49 176 5687 4131}\\
        \small{Studiengang: B.A. Philosophie}\\
        \small{Modul: Theoretische Philosophie}}

        %\href{mailto:lennard.wolf@student.hu-berlin.de}{lennard.wolf@student.hu-berlin.de}}}      

\maketitle

\vspace{\fill}

\begin{minipage}[]{0.92\textwidth}
    \centering
    \onehalfspacing
    \large   
    30. September 2017\\
    Sommersemester 2017

    \vspace{-20mm} 
\end{minipage}%
\thispagestyle{empty}
\newpage
%\clearpage
%\thispagestyle{empty}
%\tableofcontents
%\newpage
\setcounter{page}{1}

\begin{onehalfspace} 

\noindent\textbf{$(o)$ Einleitung}

\noindent In seinem Buch "`Phänomenologie der Wahnehmung"' (\Cite{merleau1966phanomenologie})...


\vspace{5mm}

\noindent\textbf{$(i)$ Die Wortbildtheorie}

\noindent Um Merleau-Pontys Auffassung von Sprache besser einordnen zu können, lohnt es sich, zuerst jene zu thematisieren, gegen die er sich wendet. Dies reflektiert auch sein allgemeines Vorgehen in der "`Ph"anomenologie der Wahrnehmung"', stets zuerst die zum jeweiligen Thema passenden Theorien seitens des Empirismus und des Intellektualismus zu betrachten, dann argumentativ als unzulänglich abzuweisen, und daraufhin die eigene Theorie vorzustellen. 

Im Kapitel "`Der Leib als Ausdruck und die Sprache"' fasst er die Gegenpositionen unter dem Begriff "`Wortbildtheorie"' zusammen. Dieser zufolge sind Wortbilder\footnote{Französisch: "`images verbales"' (\Cite[Siehe][S. 203]{franzoesisch_phen}).} von wahrgenommenen Wörtern hinterlassene \emph{Spuren} in uns, die entweder physischer (Empirismus) oder psychischer (Intellektualismus) Natur sind.\footnote{\Cite[Vgl.][S. 208]{merleau1966phanomenologie}.} Diese Spuren könnten sich dem Empirismus zufolge zum Beispiel neuronal äußern, das heißt wenn ein Kind einen Ball sieht und die Eltern in Anwesenheit von diesem Ball immer "`Ball"' sagen, dann kommt es zu einer neuronalen Verbindung zwischen dem Ball als wahrgenommenes Ding, und "`Ball"' als wahrgenommenes Wort, sodass in Zukunft aus physischen Gründen beim Hören von "`Ball"' eine neuronale Assoziation \emph{feuert} und das Denken an den Ball hervorgerufen wird. Der intellektualistischen Betrachtungsweise zufolge seien Wörter für sich nur \emph{Symbole}, die wir aufgrund ihres Anteils in Wortbildern mit Gedanken assoziiert haben. Der \emph{Sinn} der Wörter liegt dieser Ansicht nach also \emph{hinter} ihnen, nämlich in den mit ihnen mental verbundenen Gedanken und Kategorien, die auch ohne Verbindung mit einem Wort existieren. Mit anderen Worten, Denken und Sprache sind getrennt voneinander.

Auch wenn sich Wortbildtheorien in ihren einzelnen Ausprägungen voneinander unterscheiden können, so lässt sich das Wortbild allgemein als Assoziation zwischen dem "`leeren"' Symbol (Wort) und dem Gemeinten festhalten. Was genau ein "`Gemeintes"' überhaupt \emph{ist}, ist nicht gerade offensichtlich und wird in verschiedenen Theorien auch unterschiedlich beantwortet, jedoch handelt es sich es auf jeden Fall um den vom Symbol getrennten, eigentlichen \emph{Sinn}.\footnote{Eine hiervon nicht allzu entfernte, von Ferdinand de Saussure beschriebene Theorie, die Merleau-Ponty in sp"ateren Schriften abweist, ist, dass sich der Sinn aus dem \emph{Sinnabstand} zwischen den Begriffen ergibt. Er schreibt: "`Wenn der Ausdruck A und der Ausdruck B ganz und gar keinen Sinn haben, so ist nicht ersichtlich, wie es zwischen ihnen einen Sinngegensatz geben kann"' \Citep[siehe][S. 53]{zeichen_merleau}.} 

\vspace{5mm}

\noindent\textbf{$(ii)$ Abgrenzung von der Wortbildtheorie}

\noindent Der Kerngedanke von Merleau-Pontys Sprachphilosophie ist, dass genau diese Trennung von Wort und Sinn fehlgeleitet ist. Im Verlauf dieser Hausarbeit möchte ich klären, was darunter zu verstehen ist, jedoch scheint es zunächst wichtig zu nennen, was damit \emph{nicht} gemeint ist, und was das grundlegende Problem ist, das Merleau-Ponty in der Wortbildtheorie sieht und ihn zu seiner Theorie der sinnerfüllten Wörter bringt.

Dass für den Menschen Wort und Sinn eine Einheit bilden, soll in der "`Ph"anomenologie der Wahrnehmung"' natürlich nicht heißen, dass die Person Peter für sich und sein von mir gehörter Eigenname "`Peter"' ein und das selbe sind. Hier geht Merleau-Ponty natürlich mit der Wortbildtheorie, sowie wohl den meisten Sprachphilosophen konform. Vielmehr wird es aber um die Einheit des Wortes "`Peter"' und meines \emph{Begriffs} von Peter, seiner "`emotionalen Essenz"'\footnote{\Cite[Siehe][S. 222]{merleau1966phanomenologie}.} \emph{für mich}, gehen. 

Merleau-Pontys Hauptkritik an der Wortbildtheorie ist vielmehr das Fehlen eines \emph{sprechenden Subjekts}, denn dort "`gibt [es] nur einen Wortfluß, ohne eine ihn führende Intention des Sprechens"'\footnote{\Cite[Siehe][S. 208]{merleau1966phanomenologie}.}. Unser Erlebnis vom Sprechen ist nämlich gerade das einer intentionalen Handlung: Wenn ich die Aussage einer anderen Person verneine, dann ist meine Verneinung doch ein mit Sinn durchtränkter, von meinem ganzen Sein erfüllter \emph{Akt}, nicht bloß die Produktion eines inhaltsleeren Geräusches. Und damit solch ein intentionales Sprechen möglich ist, bedarf es einer Konzeption von Wörtern als Werkzeuge für bedeutsame Handlungen. Für solch eine Konzeption \emph{müssen} Wörter gerade einen Sinn in sich tragen. Empirismus und Intellektualismus bieten dies nicht, weshalb in diesen Theorien kein intentionales Sprechen erklärt werden kann.

Dem Empirismus liegt ein physikalistisches Weltbild zugrunde, aus dem generell kein intentional handelndes Subjekt entspringen könne, geschweige denn ein intentional \emph{sprechendes} Subjekt. Vielmehr sei der Mensch entstanden aus mechanistischen Notwendigkeiten, und das Vernehmen von Schallwellenmustern (Wörtern) bringt in seinem Gehirn lediglich neuronale Assoziation mit sich, die mechanisch reproduziert werden, um tierische Bedürfnisse mitzuteilen. Das dies in starkem Widerspruch zu unserem Erleben von Sprechen steht, ist solch eine Theorie für den Phänomenologen Merleau-Ponty nicht haltbar.

In der intellektualistischen Theorie gebe es nur das \emph{denkende} Subjekt, das Wörter als Platzhalter für die dahinter liegenden, sinnbehafteten Gedanken benutzt und zur Kommunikation mit anderen ausspricht. Hier sei Sprache und Sprechen bloß ein Nebenprodukt des Denkens, sodass das Subjekt nur denkt, und die Gedanken dann nur noch in Symbole übersetzt. Dem entgegnet Merleau-Ponty, dass dies mit Erkenntnissen aus der Aphasieforschung im Widerspruch stünde, denen zufolge ein Verlust der Fähigkeit des Benennens von Dingen immer auch eine allgemeine Unfähigkeit des Erkennens dieser Dinge einhergeht.\footnote{\Cite[Vgl.][S. 208 f.]{merleau1966phanomenologie}.} Gäbe es nur denkende Subjekte, so wäre nicht erklärbar, warum zum Beispiel die Farbennamenamnesie, das heißt die Unfähigkeit, Farben korrekt zu benennen, nicht auch eine rein sprachliche Störung sein, und das Erkennen von Farben ungestört bleiben könnte. Kurzum, die vom Intellektualismus unterstellte Trennung von Sprache und Denken ist mit dem Merleau-Ponty bekannten Stand der Forschung in der Psychologie nicht vereinbar. Des Weiteren steht diese Theorie ebenso im Konflikt mit unserer alltäglichen Erfahrung, den Gedanken \emph{meistens} erst während des Sprechen zu entwickeln.

Um nun auf den Gedanken hinzuleiten, dass Wörter einen Sinn haben, muss zunächst ein Überblick über die Grundzüge der Philosophie Merleau-Pontys gegeben werden.

\vspace{5mm}

\noindent\textbf{$(iii)$ Die Welt und der Leib}

\noindent Spätestens seit Kant besteht die weit verbreitete Auffassung, dass die Welt \emph{an sich} etwas sei, auf das wir aufgrund unserer Endlichkeit keinen unvermittelten Zugriff h"atten, und dass die Welt, in der wir leben, immer nur eine Welt \emph{f"ur uns} sein k"onne, das heißt eine Art unzureichendes Modell der Realit"at, ein \emph{Schein}, ist. Merleau-Ponty schreibt, die \emph{reale} Welt sei "`das best"andige Sein, innerhalb dessen ich all meine Erkenntniskorrekturen vollziehe"'\footnote{\Cite[Siehe][S. 379]{merleau1966phanomenologie}.}. So scheint Merleau-Ponty dieser gängigen Ansicht zu folgen, und meint in der "`Ph"anomenologie der Wahrnehmung"' mit "`Welt"' stets jene \emph{unsere} Welt, in der wir leben, die \emph{uns wahrhaftig} ist und in der wir Dinge erfahren. 

Um eine Welt zu \emph{haben}\footnote{"`Eine Welt haben"' ist hier nicht im Sinne eines Besitzverh"altnisses zu verstehen, sondern im Sinne eines \emph{konstituierenden} Bezugs \Citep[vgl.][S. 207]{merleau1966phanomenologie}. Indem ich die Welt "`habe"', \emph{ist} sie erst und \emph{bin} ich als etwas in ihr.}, bedarf es Merleau-Ponty zufolge eines sogenannten \emph{Leibes}. Das Leibkonzept ist eine Art "`dritter Weg"' zur Lösung des Leib-Seele-Problems, der einen Leib-Seele-Dualismus, wie auch monistischen Physikalismus oder Psychismus verneint. Der Leib ist zu verstehen als \emph{inkarnierte Existenz}, er ist das "`Vehikel des Zur-Welt-Seins"'\footnote{\Cite[Siehe][S. 106]{merleau1966phanomenologie}.}. Ich \emph{bin} mein Leib und ich bin durch ihn. Unser Zur-Welt-Sein ist ausgezeichnet durch das Wahrnehmen von \emph{Dingen}. Diese konstituieren sich \emph{als} Dinge gerade erst im Bezug auf den Leib, indem sie auf ihn \emph{einwirken}, und er sie \emph{wahr}nimmt. "`Wahrnehmen ist [...] die Erfahrung des Entspringens eines immanenten Sinnes aus einer Konstellation von Gegebenheiten"'\footnote{\citep[S. 42]{merleau1966phanomenologie}.}. Der Leib ermöglicht mir das Empfinden von Gegebenheiten, und diese wirken auf mich als Phänomene ein, und verändern dadurch meinen Leib, das heißt mein ganzes Sein. Gerade weil ich die Phänomene als voneinander getrennte Dinge wahrnehme, sind sie für mich zwangsläufig immer schon mit Bedeutung erfüllt, ansonsten könnte ich sie nicht unterscheiden. Entsprechend ist f"ur Merleau-Ponty Sinn immer schon der Welt "`einverleibt"'. F"ur ihn bildet das Ding, das wir \emph{als solches} erkennen eine \emph{Bedeutungseinheit} - gerade seine Bedeutung macht es "uberhaupt erst zum Ding. Alles, das uns \emph{als} etwas begegnet, ist daher bedeutsam, tr"agt immer schon Sinn in sich.

Unser Bedeutungsverm"ogen, also die Eigenschaft, allem Wahrgenommenen immerzu einen Sinn zu geben, ist daher gerade die Essenz unseres Zur-Welt-Seins. "`Diese Bewegung, in der die Existenz eine faktische Situation sich zu eigen macht und verwandelt, nennen wir die Transzendenz"'\footnote{\Cite[Siehe][S. 202]{merleau1966phanomenologie}.}. Dass ich etwas \emph{als} Ding erfahre hei"st, dass ich mich in es und es in mich hineinlege, dass es mein Sein "andert, dass ich mit dem Ph"anomen in dem Augenblick \emph{koexistiere}\footnote{\Cite[Vgl.][S. 368]{merleau1966phanomenologie}.}. Und so ist sein Sinn nicht \emph{hinter} dem Ding, sondern "`verk"orpert sich in ihm"'\footnote{\Cite[Siehe][S. 370]{merleau1966phanomenologie}.}. Dieses Verh"altnis von Ding und Leib, in welchem beide einander bedingen, stellt f"ur Merleau-Ponty eine "`endg"ultige "Uberwindung der klassischen Entgegensetzung von Subjekt und Objekt"'\footnote{\Cite[Siehe][S. 207]{merleau1966phanomenologie}.} dar. Die Wahrnehmung der Welt ist also nicht vermittelt und interpretiert, sondern unvermittelt und wahrhaftig. Die Dinge \emph{sind} ihre Wirkung auf mich, und offenbaren mir so ihren Sinn.

\vspace{5mm}

\noindent\textbf{$(v)$ Der Ausdruck und das Wort}

\noindent Allgemein ist "`Ausdruck"' in der "`Ph"anomenologie der Wahrnehmung"' zu verstehen als die Manifestation von etwas in der Welt. Die Topfpflanze findet ihren Ausdruck für mich in ihrer Wahrnehmbarkeit durch meine Sinne. Das Einstechen einer Nadel in meinen Arm findet seinen Ausdruck im Schmerz den ich empfinde, und die Empfindung des Schmerzes drückt sich wiederum durch mein Aufschreien aus und wird somit sichtbar. Zentral für diesen Gedanken ist, dass der Ausdruck und das Ausgedrückte aber gerade eins sind. Es gibt das Ausgedrückte nicht ohne den Ausdruck und umgekehrt. Daher ist diese Trennung, die 

Wenn mir ein trauernder Mensch begegnet, so analysiere ich nicht erst mein Blickfeld, interpretiere die Tr"anen und das zerknitterte Gesicht als Symbole einer \emph{abstrakten} Traurigkeit, sondern ich nehme die Traurigkeit \emph{unvermittelt} wahr, denn die Tr"anen und das zerknitterte Gesicht \emph{sind} die Trauer. Sie sind der Ausdruck der Traurigkeit nach au"sen, und die andere Person erf"ahrt den Ausdruck der Traurigkeit nach innen. Gerade dieser Ausdruck aus dem Leib des anderen und die Impression auf meinen Leib \emph{sind} diese Emotion. Es ist "`das Wunder des Ausdrucks: im "Au"seren ein Inneres zu offenbaren"'\footnote{\Cite[Siehe][S. 370]{merleau1966phanomenologie}.}, der Ausdruck ist "`nicht lediglich eine "Ubersetzung, sondern eine Realisierung und Verwirklichung der Bedeutung selbst"'\footnote{\Cite[Siehe][S. 217]{merleau1966phanomenologie}.}. 

Dass ich Schmerz erfahre hei"st, dass sich mein Leib so "andert, dass ich zum einen den Schmerz sinnlich wahrnehme und dass ich zum anderen eine \emph{Geste mache}, also beispielsweise aufschreie und die betroffene Stelle mit den H"anden ber"uhre. Die Geste ist nicht zu trennen von der Erfahrung und umgekehrt. Ein Schauspieler, der Wut auf eine Weise vorspielt, dass wir sie ihm nicht abnehmen, \emph{empfindet sie auch nicht}. Doch wenn wir es einer anderen Schauspielerin abnehmen, so \emph{verstehen} wir ihr Geb"arde gerade weil sie \emph{wahr} ist. Es ist nicht blo"ser Zufall, dass Schauspieler*innen zuweilen an ihren Rollen zugrunde gehen. 

Innerhalb einer Kultur leben hei"st, dass die Dinge, und damit ihre Bedeutungen, f"ur uns in weiten Teilen dieselben sind, wie f"ur die anderen in der Kultur. Diese "`Kulturg"uter"' entwerfen eine Welt, die Teil unserer Welt ist. So verstehen wir die Gesten und Geb"arden der anderen Menschen in unserer Kultur, weil "`wir die von den beobachteten Zeichen vorgezeichnete Seinsweise uns zu eigen machen"'\footnote{\Cite[Siehe][S. 370]{merleau1966phanomenologie}.}. "`Die sexuelle Mimik des Hundes `verstehe' ich nicht, nicht zu reden vom Maik"afer oder der Gottesanbeterin"'\footnote{\Cite[Siehe][S. 219]{merleau1966phanomenologie}.}, denn die Welten der menschlichen Leiber ist jenen derlei anderer Lebewesen zutiefst verschieden. "`Der Sinn der also `verstandenen' Geste eines Anderen ist nicht hinter ihr gelegen, sondern f"allt zusammen mit der Struktur der von der Geb"arde entworfenen Welt"'\footnote{\Cite[Siehe][S. 220]{merleau1966phanomenologie}.}.

Der Leib ist für Merleau-Ponty "`schlechthin das Vermögen natürlichen Ausdrucks"'\footnote{\Cite[Siehe][S. 215]{merleau1966phanomenologie}.}, das heißt intentionalen Ausdrucks.

TODO was ist intentionalität?

\vspace{5mm}

\noindent\textbf{$(vi)$ Sprechen und Sprache}

\noindent 


\end{onehalfspace}
\nocite{*}
%\bibliography{merleau-ponty-essay}
\printbibliography
\end{document}
