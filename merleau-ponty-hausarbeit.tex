\documentclass[a4paper, 12pt]{article}
%\usepackage{CJKutf8} % japanese
\usepackage{graphicx}
\usepackage{hyperref}
\usepackage{fullpage}
%\usepackage{parskip}
\usepackage{color}
\usepackage[ngerman]{babel}
\usepackage{hyperref}
\usepackage{calc} 
\usepackage{enumitem}
\usepackage[utf8]{inputenc}
\usepackage{titlesec}
%\pagestyle{headings}
\usepackage{setspace} %halbzeilig
\usepackage[style=authoryear-ibid,natbib=true]{biblatex}
\usepackage[hang]{footmisc}
\setlength{\footnotemargin}{-0.8em}
%\bibliographystyle{natdin}
\addbibresource{merleau-ponty-hausarbeit.bib}
\DeclareDatamodelEntrytypes{standard}
\DeclareDatamodelEntryfields[standard]{type,number}
\DeclareBibliographyDriver{standard}{%
  \usebibmacro{bibindex}%
  \usebibmacro{begentry}%
  \usebibmacro{author}%
  \setunit{\labelnamepunct}\newblock
  \usebibmacro{title}%
  \newunit\newblock
  \printfield{number}%
  \setunit{\addspace}\newblock
  \printfield[parens]{type}%
  \newunit\newblock
  \usebibmacro{location+date}%
  \newunit\newblock
  \iftoggle{bbx:url}
    {\usebibmacro{url+urldate}}
    {}%
  \newunit\newblock
  \usebibmacro{addendum+pubstate}%
  \setunit{\bibpagerefpunct}\newblock
  \usebibmacro{pageref}%
  \newunit\newblock
  \usebibmacro{related}%
  \usebibmacro{finentry}}

%\titleformat{name=\section,numberless}
%  {\normalfont\Large\bfseries}
%  {}
%  {0pt}
%  {}
\date{\vspace{-3ex}}
\begin{document}

\title{\vspace{5ex}
	\includegraphics*[bb=0 0 720 200, width=0.72\textwidth]{ErstesSem/images/hu_logo.png}\\
	\vspace{30pt}
	\scshape\LARGE{"`Das Wort selbst ist Tr"ager des Sinnes"'}\\\Large{Sinn, Ausdruck und Sprache bei Merleau-Ponty}\\\vspace{20pt}}
	


\author{Merleau-Ponty: Ph"anomenologie der Wahrnehmung (PS)\\
	\vspace{7pt}
          Dozent: Dr. Matthias Schlo"sberger\\\vspace{4pt}Lennard Wolf\\
        \small{Matrikelnummer: 12345}\\
        \small{E-Mail: lennard.wolf@student.hu-berlin.de}\\
        \small{Telefonnummer: +49 176 5687 4131}\\
        \small{Studiengang: B.A. Philosophie}\\
        \small{Modul: Theoretische Philosophie}}

        %\href{mailto:lennard.wolf@student.hu-berlin.de}{lennard.wolf@student.hu-berlin.de}}}      

\maketitle

\vspace{\fill}

\begin{minipage}[]{0.92\textwidth}
    \centering
    \onehalfspacing
    \large   
    30. September 2017\\
    Sommersemester 2017

    \vspace{-20mm} 
\end{minipage}%
\thispagestyle{empty}
\newpage
%\clearpage
%\thispagestyle{empty}
%\tableofcontents
%\newpage
\setcounter{page}{1}

\begin{onehalfspace} 

\noindent\textbf{$(o)$ Einleitung}

\noindent In seinem Buch "`Phänomenologie der Wahnehmung"' (\Cite{merleau1966phanomenologie})...


\vspace{5mm}

\noindent\textbf{$(i)$ Die Wortbildtheorie}

\noindent Um Merleau-Pontys Auffassung von Sprache besser einordnen zu können, lohnt es sich, zuerst jene zu thematisieren, gegen die er sich wendet. Dies reflektiert auch sein allgemeines Vorgehen in der "`Ph"anomenologie der Wahrnehmung"', stets zuerst die zum jeweiligen Thema passenden Theorien seitens des Empirismus und des Intellektualismus zu betrachten, dann argumentativ als unzulänglich abzuweisen und schlussendlich den eigenen Ansatz vorzustellen. 

Im Kapitel "`Der Leib als Ausdruck und die Sprache"' fasst er die Gegenpositionen unter dem Begriff "`Wortbildtheorie"' zusammen. Dieser zufolge sind Wortbilder\footnote{Französisch: "`images verbales"' (\Cite[siehe][S. 203]{franzoesisch_phen}).} von wahrgenommenen Wörtern hinterlassene \emph{Spuren} in uns, die entweder physischer (Empirismus) oder psychischer (Intellektualismus) Natur sind.\footnote{\Cite[Vgl.][S. 208]{merleau1966phanomenologie}.} Diese Spuren könnten sich dem Empirismus zufolge zum Beispiel neuronal äußern, das heißt wenn ein Kind einen Ball sieht und die Eltern in Anwesenheit von diesem Ball immer "`Ball"' sagen, dann kommt es zu einer neuronalen Verbindung zwischen dem Ball als wahrgenommenes Ding, und "`Ball"' als wahrgenommenes Wort, sodass in Zukunft aus physischen Gründen beim Hören von "`Ball"' eine neuronale Assoziation \emph{feuert} und das Denken an den Ball hervorgerufen wird. Der intellektualistischen Betrachtungsweise zufolge seien Wörter für sich nur \emph{Symbole}, die wir aufgrund ihres Anteils in Wortbildern mit Gedanken assoziiert haben. Der \emph{Sinn} der Wörter liegt dieser Ansicht nach also \emph{hinter} ihnen, nämlich in den mit ihnen mental verbundenen Gedanken und Kategorien, die auch ohne Verbindung mit einem Wort existieren. Mit anderen Worten, Denken und Sprache sind voneinander getrennt.

Auch wenn sich Wortbildtheorien einander in ihren einzelnen Ausprägungen unterscheiden können, so lässt sich das Wortbild allgemein als Assoziation zwischen dem "`leeren"' Symbol (Wort) und dem Gemeinten festhalten. Was genau ein "`Gemeintes"' überhaupt \emph{ist}, ist nicht gerade offensichtlich und wird in verschiedenen Theorien auch unterschiedlich beantwortet, jedoch handelt es sich es auf jeden Fall immer um den vom Symbol getrennten, "`eigentlichen"' \emph{Sinn}.\footnote{Eine hiervon nicht allzu entfernte, von Ferdinand de Saussure beschriebene Theorie, die Merleau-Ponty in einer sp"ateren Schrift abweist, ist, dass sich der Sinn aus dem \emph{Sinnabstand} zwischen den Begriffen ergibt. Er schreibt: "`Wenn der Ausdruck A und der Ausdruck B ganz und gar keinen Sinn haben, so ist nicht ersichtlich, wie es zwischen ihnen einen Sinngegensatz geben kann"' \Citep[siehe][S. 53]{zeichen_merleau}.} 

\vspace{5mm}

\noindent\textbf{$(ii)$ Abgrenzung von der Wortbildtheorie}

\noindent Der Kerngedanke von Merleau-Pontys Sprachphilosophie ist, dass genau diese Trennung von Wort und Sinn fehlgeleitet ist. Im Verlauf dieser Hausarbeit versuche ich zu verdeutlichen, warum die Wortbildtheorie als problematisch anzusehen ist und wie die Konzeption von Sprechen und Sprache in der "`Phänomenologie der Wahrnehmung"' zu verstehen ist. Ich beginne mit einem kleinen Kommentar dazu, was mit der Einheit von Wort und Sinn \emph{nicht} gemeint ist, und beschreibe dann das grundlegende Problem, das Merleau-Ponty in der Wortbildtheorie sieht und ihn zu seiner Theorie der sinnerfüllten Wörter bringt, um eine klare Abgrenzung zwischen den beiden Theorien zu ziehen.

Dass für den Menschen Wort und Sinn eine Einheit bilden, soll in der "`Ph"anomenologie der Wahrnehmung"' natürlich nicht heißen, dass die Person Peter für sich, und sein von mir ausgesprochener oder gedachter Eigenname "`Peter"' ein und das selbe sind. Hier ist Merleau-Ponty natürlich mit der Wortbildtheorie, sowie wohl den meisten sprachphilosophischen Theorien, einer Meinung. Vielmehr wird es aber um die Einheit des sinnlich oder mental wahrgenommenen Wortes "`Peter"', und seiner Bedeutung \emph{für mich}, Peters "`Wesensform"', gehen. 

Merleau-Pontys Hauptkritik an der Wortbildtheorie ist vielmehr das Fehlen eines \emph{sprechenden Subjekts}, denn dort "`gibt [es] nur einen Wortfluß, ohne eine ihn führende Intention des Sprechens"'\footnote{\Cite[Siehe][S. 208]{merleau1966phanomenologie}.}. Sprechen ist für uns doch zumeist eine gezielte, bedeutsame Handlung und kein Reflex, mit dem wir nichts zu tun haben. Wenn ich die Aussage einer anderen Person verneine, dann ist meine Verneinung doch ein mit Sinn durchtränkter, von meinem ganzen Sein erfüllter, intentionaler \emph{Akt}, nicht bloß die mechanische Produktion eines inhaltsleeren Geräusches. Und damit solch ein bewusstes, intentionales Sprechen möglich ist, bedarf es einer Konzeption von Wörtern als einverleibte Werkzeuge für bedeutsame Handlungen. Und dafür \emph{müssen} Wörter gerade einen Sinn in sich tragen, da sonst nicht zu erklären wäre wie, ich mich beim Sprechen bewusst auf etwas beziehen kann. Empirismus und Intellektualismus bieten dies nicht, weshalb in diesen Theorien kein intentional sprechendes Subjekt erklärt werden kann.

Dem Empirismus liegt ein physikalistisches Weltbild zugrunde, aus dem generell kein bewusst handelndes Subjekt entspringen könne, geschweige denn ein bewusst intentional \emph{sprechendes} Subjekt. Vielmehr sei der Mensch entstanden aus mechanistischen Notwendigkeiten, und das Vernehmen von Schallwellenmustern (Wörtern) bringt in seinem Gehirn lediglich neuronale Assoziation mit sich, die mechanisch reproduziert werden, um tierische Bedürfnisse mitzuteilen. Das dies in starkem Widerspruch zu unserem Erleben von Sprechen steht, ist solch eine Theorie für den Phänomenologen Merleau-Ponty nicht haltbar.

In der intellektualistischen Theorie gebe es nur das \emph{denkende} Subjekt, das Wörter als Platzhalter für die dahinter liegenden, sinnbehafteten Gedanken benutzt und zur Kommunikation mit anderen ausspricht. Hier sei Sprache und Sprechen bloß ein Nebenprodukt des Denkens, sodass das Subjekt nur denkt, und die Gedanken dann nur noch in Symbole übersetzt, was eine Konzeption des Denkens als vorsprachlich voraussetzt. Dem entgegnet Merleau-Ponty, dass dies mit Erkenntnissen aus der Aphasieforschung im Widerspruch stünde, denen zufolge ein Verlust der Fähigkeit des Benennens von Dingen immer auch eine allgemeine Unfähigkeit des Erkennens dieser Dinge einhergeht.\footnote{\Cite[Vgl.][S. 208 f.]{merleau1966phanomenologie}.} Gäbe es nur denkende Subjekte, so wäre nicht erklärbar, warum zum Beispiel die Farbennamenamnesie, das heißt die Unfähigkeit, Farben korrekt zu benennen, nicht auch eine rein sprachliche Störung sein, und das Erkennen von Farben ungestört bleiben könnte. Kurzum, die vom Intellektualismus unterstellte Trennung von Sprache und Denken ist mit dem Merleau-Ponty bekannten Stand der Forschung in der Psychologie nicht vereinbar. Des Weiteren steht diese Theorie ebenso im Konflikt mit unserer alltäglichen Erfahrung, den Gedanken \emph{meistens} erst während des Sprechen zu entwickeln.

Um nun auf den Gedanken hinzuleiten, dass Wörter einen Sinn haben, muss zunächst ein Überblick über die Grundzüge der Philosophie Merleau-Pontys gegeben werden.

\vspace{5mm}

\noindent\textbf{$(iii)$ Die Welt und der Leib}

\noindent Spätestens seit Kant besteht die weit verbreitete Auffassung, dass die Welt \emph{an sich} etwas sei, auf das wir aufgrund unserer Endlichkeit keinen unvermittelten Zugriff h"atten, und dass die Welt, in der wir leben, immer nur eine Welt \emph{f"ur uns} sein k"onne, das heißt eine Art unzureichendes Modell der Realit"at, ein \emph{Schein}, ist. Merleau-Ponty schreibt, die \emph{reale} Welt sei "`das best"andige Sein, innerhalb dessen ich all meine Erkenntniskorrekturen vollziehe"'\footnote{\Cite[Siehe][S. 379]{merleau1966phanomenologie}.}. So scheint Merleau-Ponty dieser gängigen Ansicht zu folgen, und meint in der "`Ph"anomenologie der Wahrnehmung"' mit "`Welt"' stets jene \emph{unsere} Welt, in der wir leben, die nur \emph{für uns wirklich} ist und in der wir Dinge und Sinn erfahren - also unsere \emph{affektive Umwelt}.\footnote{\Cite[Vgl.][S. 185]{merleau1966phanomenologie}.} Die Inhalte meiner Welt sind entsprechend immer zugleich die Inhalte meines Bewusstsein.

Um eine Welt zu haben\footnote{"`Eine Welt haben"' ist hier nicht im Sinne eines Besitzverh"altnisses zu verstehen, sondern im Sinne eines \emph{konstituierenden} Bezugs \Citep[vgl.][S. 207]{merleau1966phanomenologie}. Indem ich die Welt "`habe"', \emph{ist} sie erst und \emph{bin} ich als etwas in ihr.}, bedarf es Merleau-Ponty zufolge eines sogenannten \emph{Leibes}. Das Leibkonzept verneint sowohl einen Leib-Seele-Dualismus, als auch monistischen Physikalismus oder Psychismus. Der Leib ist zu verstehen als \emph{inkarnierte Existenz}, er ist das "`Vehikel des Zur-Welt-Seins"'\footnote{\Cite[Siehe][S. 106]{merleau1966phanomenologie}.}. Ich \emph{bin} mein Leib und ich bin durch ihn. Unser Zur-Welt-Sein ist ausgezeichnet durch das Wahrnehmen von \emph{Dingen}. Diese konstituieren sich \emph{als} Dinge gerade erst im Bezug auf den Leib, indem sie auf ihn \emph{einwirken}, und er sie \emph{wahr}nimmt. "`Wahrnehmen ist [...] die Erfahrung des Entspringens eines immanenten Sinnes aus einer Konstellation von Gegebenheiten"'\footnote{\Cite[Siehe][S. 42]{merleau1966phanomenologie}.}. Der Leib ermöglicht mir das Empfinden von Gegebenheiten, und diese wirken auf mich als Phänomene ein, und verändern dadurch meinen Leib, das heißt mein ganzes Sein. Gerade weil ich die Phänomene als voneinander getrennte Dinge wahrnehme, sind sie für mich zwangsläufig immer schon mit Bedeutung erfüllt, ansonsten könnte ich sie nicht unterscheiden. Entsprechend ist f"ur Merleau-Ponty Sinn immer schon der Welt "`einverleibt"'. F"ur ihn bildet das Ding, das wir als solches erkennen, eine \emph{Bedeutungseinheit} - gerade seine Bedeutung macht es "uberhaupt erst zum Ding\footnote{In dieser Hausarbeit ist der Begriff "`Ding"' sehr weit gefasst und kann von physischen Gegenständen, über Verhaltensweisen, bis hin zu abstrakten Konzepten reichen. Es fällt unter ihn alles, was für uns Bewusstseinsinhalt sein kann, weshalb ich ihn auch synonym mit "`Phänomen"' benutze.}. Alles, das uns als \emph{etwas} begegnet, ist daher bedeutsam, tr"agt immer schon Sinn in sich. Ein Ding ist daher immer erstmal nur ein Ding \emph{für uns}.

Unser Bedeutungsverm"ogen, also die Eigenschaft, allem Wahrgenommenen immerzu einen Sinn zu geben, ist daher gerade die Essenz unseres Zur-Welt-Seins. "`Diese Bewegung, in der die Existenz eine faktische Situation sich zu eigen macht und verwandelt, nennen wir die Transzendenz"'\footnote{\Cite[Siehe][S. 202]{merleau1966phanomenologie}.}. Dass ich etwas \emph{als} Ding erfahre hei"st, dass ich mich in es und es in mich hineinlege, dass es mein Sein "andert, dass ich mit dem Ph"anomen in dem Augenblick \emph{koexistiere}\footnote{\Cite[Vgl.][S. 368]{merleau1966phanomenologie}.}. Und so ist sein Sinn nicht \emph{hinter} dem Ding, sondern "`verk"orpert sich in ihm"'\footnote{\Cite[Siehe][S. 370]{merleau1966phanomenologie}.}. Dieses Verh"altnis von Ding und Leib, in welchem beide einander bedingen, stellt f"ur Merleau-Ponty eine "`endg"ultige "Uberwindung der klassischen Entgegensetzung von Subjekt und Objekt"'\footnote{\Cite[Siehe][S. 207]{merleau1966phanomenologie}.} dar. Die Wahrnehmung der Welt ist also nicht vermittelt und interpretiert, sondern unvermittelt und wahrhaftig. Die Dinge \emph{sind} ihre Wirkung auf mich. So ist meine Wahrnehmung von ihnen keine mentale \emph{Repräsentation} von den konkreten Dingen, sondern ihre partikulare Wesenheit, ihr Sein. Wenn ich eine Topfpflanze sehe, so interpretiere ich nicht erst aus dem Gesehenen, dass vor mir eine Topfpflanze steht, sondern ich \emph{sehe die Topfpflanze vor mir stehen}. Erst in der Abstraktion durch den Verstand fange ich an, das Allgemeine aller Topfpflanzen als eigenen abstrakten Begriff vorzustellen. Dieser ist aber keine mentale Repräsentation der vor mir stehenden Topfpflanze, sondern das Wesen der allgemeinen Topfpflanze selbst.

\vspace{5mm}

\noindent\textbf{$(iv)$ Der Ausdruck und die Geste}

\noindent Allgemein ist "`Ausdruck"' in der "`Ph"anomenologie der Wahrnehmung"' zu verstehen als die Manifestation von etwas in der Welt, eine \emph{Verwirklichung}.\footnote{\Cite[Vgl.][S. 217]{merleau1966phanomenologie}.} Die Topfpflanze findet ihren Ausdruck für mich in ihrer Wahrnehmbarkeit durch meine Sinne. Das Einstechen einer Nadel in meinen Arm findet seinen Ausdruck im Schmerz den ich empfinde, und dass ich einen Schmerz empfinde drückt sich wiederum durch mein Aufschreien aus und wird somit sichtbar. Zentral für diesen Gedanken ist, dass der Ausdruck und das Ausgedrückte aber gerade eins sind. Es gibt das Ausgedrückte nicht ohne den Ausdruck und umgekehrt. 

Wenn mir ein trauernder Mensch begegnet, so analysiere ich nicht erst mein Blickfeld, interpretiere die Tr"anen und das zerknitterte Gesicht als Symbole einer \emph{abstrakten} Traurigkeit, sondern ich nehme die Traurigkeit \emph{unvermittelt} wahr, denn die Tr"anen und das zerknitterte Gesicht \emph{sind} die Trauer. Sie sind der Ausdruck der Traurigkeit für mich, und die Person erf"ahrt den Ausdruck der Traurigkeit auf ihre Weise für sich. Dieser Ausdruck des Leibes des anderen und der Eindruck auf meinen Leib \emph{sind} diese Emotion. Es ist "`das Wunder des Ausdrucks: im "Au"seren ein Inneres zu offenbaren"'\footnote{\Cite[Siehe][S. 370]{merleau1966phanomenologie}.}, der Ausdruck ist "`nicht lediglich eine "Ubersetzung, sondern eine Realisierung und Verwirklichung der Bedeutung selbst"'\footnote{\Cite[Siehe][S. 217]{merleau1966phanomenologie}.}. 

Dass ich Schmerz erfahre hei"st, dass sich mein Leib so "andert, dass ich zum einen den Schmerz sinnlich wahrnehme und dass dies nach außen hin sichtbar ist, also dass ich aufschreie und die betroffene Stelle mit den H"anden ber"uhre. Diese Bewegung ist nicht zu trennen von der Erfahrung und umgekehrt. Ein Schauspieler, der Wut auf eine Weise vorspielt, dass wir sie ihm nicht abnehmen, \emph{empfindet sie auch nicht}. Doch wenn wir das Schauspiel einer anderen Schauspielerin abnehmen, so \emph{verstehen} wir ihre Geb"arde gerade weil sie \emph{echt} ist. Es ist nicht blo"ser Zufall, dass Schauspieler*innen zuweilen an ihren Rollen zugrunde gehen. 

Der Leib ist für Merleau-Ponty "`schlechthin das Vermögen natürlichen Ausdrucks"'\footnote{\Cite[Siehe][S. 215]{merleau1966phanomenologie}.}, das heißt intentionalen Ausdrucks. So können Menschen, wie auch viele Tiere, Gesten machen, die ihnen selbst und Anderen etwas bedeuten können. Damit aber Andere meine Geste auch verstehen, müssen ihre individuellen Welten und meine sich in bestimmten Hinsichten "`überschneiden"'. Solch eine Überschneidung wird gemeinhin als "`Kultur"' bezeichnet. Teil einer Kultur zu sein bedeutet, ganz basal betrachtet, dass bestimmte Dinge und Verhaltensweisen, und damit ihre Bedeutungen, f"ur die der Kultur Angehörigen in weiten Teilen dieselben sind. Solche "`Kulturg"uter"' entwerfen eine kulturelle Welt, die Teil unserer Welt ist. So verstehen wir die Gesten und Geb"arden der anderen Menschen in unserer Kultur, weil "`wir die von den beobachteten Zeichen vorgezeichnete Seinsweise uns zu eigen"'\footnote{\Cite[Siehe][S. 370]{merleau1966phanomenologie}.} gemacht haben. Wenn mir ein Polizist durch ein Handzeichen bedeutet stehen zu bleiben, dann ist diese Geste der Ausdruck der Intention des Polizisten, mich zum Stehenbleiben zu bringen. Seine Intention nehme ich in der Geste wahr, da diese Geste ein gemeinsames Kulturgut ist. "`Der Sinn der also `verstandenen' Geste eines Anderen ist nicht hinter ihr gelegen, sondern f"allt zusammen mit der Struktur der von der Geb"arde entworfenen Welt"'\footnote{\Cite[Siehe][S. 220]{merleau1966phanomenologie}.}. Gesten sind also ganz allgemein zu verstehen als einverleibte Werkzeuge der Kommunikation, wie beispielsweise der Tanz, das vorgespieltes Lachen oder das Berühren einer anderen Person.

Doch natürlich gibt es auch Gesten, die uns vollkommen fremd sind. "`Die sexuelle Mimik des Hundes `verstehe' ich nicht, nicht zu reden vom Maik"afer oder der Gottesanbeterin"'\footnote{\Cite[Siehe][S. 219]{merleau1966phanomenologie}.}, denn die Welten der menschlichen Leiber ist jenen derlei anderer Lebewesen zutiefst verschieden. Wenn ein Hase etwas gefährliches wittert, klopft er mehrmals so doll er kann mit den Hinterpfoten auf den Boden, damit die anderen Hasen ihn hören können und sofort verstehen, dass Gefahr besteht. Diese Geste des Klopfens ist, so scheint es jedenfalls, der Ausdruck der Intention der Mitteilung von Gefahr in der Welt der Hasen. Dieses \emph{interpretierte} Verstehen der Geste der Hasen durch den Verstand ist aber fundamental verschieden von meinem "`Verstehen"' des Handzeichens des Polizisten. Gesten sind, wie oben schon zitiert, Teil einer \emph{Seinsweise} von Leibern, die wir unserer inkorporiert haben. Die theoretische Analyse des Phänomens und die darauf folgende Vermutung, dass es sich um eine Warnung handelt, ist gerade ein Zeichen dafür, dass diese Geste nicht Teil der Seinsweise der Wissenschaftler*innen war. Wäre sie Teil der Seinsweise gewesen, hätte man unvermittelt ihren Sinn wahrgenommen. Dass nach Merleau-Ponty gestische Ausdrücke, genau wie alle wahrgenommenen Dinge, nicht vom Verstand interpretiert werden, ist zentral für das Verständnis seiner Sprachphilosophie. 

Es ist natürlich stets möglich, wahrgenommene Phänomene, und so auch Gesten, unterschiedlich zu deuten. Zum einen bildet diese Interpretierbarkeit die Grundlage für die Möglichkeit des Lernens und der Erkenntnis überhaupt, zum anderen weist sie schon auf eine allgemeine, inhärente "`Unschärfe"' oder "`Ambiguität"' in den Dingen hin. So wie wir auch schon den Leib nie in seiner Gänze erleben können\footnote{\Cite[Vgl.][S. ???]{merleau1966phanomenologie}.}, so nehmen wir auch immer nur "`zusammenstimmende Abschattungen"'\footnote{\Cite[Siehe][S. 220]{merleau1966phanomenologie}.} der Dinge wahr. Zum Beispiel sehen wir eine Seite eines physikalischen Gegenstands nicht, oder wir denken während eines Streits nicht an die Freude, die uns die Person normalerweise bringt. So ist die Welt um uns in stetem Wandel, und keine Bedeutung ist in Stein gemeißelt. Der Leib "`ist ein Knotenpunkt lebendiger Bedeutungen, nicht das Gesetz einer bestimmten Anzahl miteinander variabler Koeffizienten"'\footnote{\Cite[Siehe][S. 182]{merleau1966phanomenologie}.}. Das wahrgenommene Phänomen bleibt trotz der Unschärfe als untrennbare Einheit von Ausdruck und Sinn bestehen. Nur weil ich im Streit nicht an die positiven Seiten der Person denke, kann ich doch nach wie vor über die Person als solche nachdenken und nehme sie nicht jeden Moment als neue Begegnung wahr. Weil ich eine \emph{emotionale Essenz} in ihr erkenne\footnote{\Cite[Vgl.][S. 222]{merleau1966phanomenologie}.}, bildet sie weiterhin eine Sinneinheit, nur eben mit immerzu variierend offenbarten oder beachteten Eigenschaften. Ebenso kann auch eine Geste in verschiedenen Situationen aufgrund ihrer Ambiguität unterschiedlich aufgefasst werden. Kommunikation unterliegt daher stets dem Risiko des Missverständnisses. Wenn mein Leib die Geste hingegen versteht, dann aufgrund einer \emph{Wahrnehmungsgewohnheit}, die durch einen vorhergegangen Erwerb einer mit anderen geteilten Welt ermöglicht wurde. Solch ein Erwerb ist nun gerade für den Leib das Erlernen der Fähigkeit, bestimmten Sinn zu erkennen und auszudrücken. Und das "`Lernen"' von einer Kultur ist keine Anhäufung von theoretischem Wissen im Verstand, sondern ein Einfühlen in das Sein in der Welt, die diese Kultur hervorbringt. Dass theoretisches Wissen dem häufig vorausgeht ist unbestreitbar, doch wandelt sich dieses Wissen nach einer Weile in ein Sediment des Zur-Welt-Seins um, sodass ich im Verstehen und Handeln nicht darüber nachdenken muss, sondern es einfach ohne darüber nachzudenken tue. Die Geste wurde Teil meiner Welt. Ist ihr Ausdruck verstanden, ist damit zugleich ihre Bedeutung verstanden, denn beides ist eins.

Gesprochene Wörter sind für Merleau-Ponty nichts anderes als \emph{phonetische Gesten}, Wörter in der Gebärdensprache sind gestische Gebärden und damit ebenso Gesten. Bei natürlichen Sprachen werden beim Sprechen häufig noch Mimiken und Gestiken zur Kommunikationsmodulation verwendet, was weiterhin auf den gestischen Charakter von Sprache hinweist. Es ist also gleich, ob das Wort nun  phonetischer oder gebärdender Natur ist. Es zeigt sich, dass Sprache als eine Art einverleibter "`Werkzeugkasten"' zu verstehen ist, wobei Wörter die Werkzeuge des intentionalen Vorstellens (Denken) und Austauschs (Kommunikation) von Sinn darstellen, denn sie sind die "`äußere Existenz des Sinnes"'\footnote{\Cite[Siehe][S. 216]{merleau1966phanomenologie}.}. Die Werkzeuge bezeichne ich als \emph{einverleibt}, da sie Teil des \emph{Handlungsraumes} des Leibes sind, so wie das Laufen und das Umschauen.

es gibt mehrere zeichen für etwas und mehrere etwasse fürs zeichen

\vspace{5mm}

\noindent\textbf{$(v)$ Denken ist inneres Sprechen}

\noindent Die These, dass Denken vorsprachlich sei, also eine Handlung, die keiner Sprache bedarf, lehnt Merleau-Ponty ab. Vielmehr ist für ihn Denken als \emph{inneres Sprechen} zu verstehen. Sprechen ist der bewusste Akt des Ausdrückens von Sinn, und im Denken ist dieser Akt an einen selbst gerichtet. Es ist eine Form des Ausdrucks, der \emph{Offenbarung}, des "`innersten Seins"' des Leibes\footnote{\Cite[Vgl.][S. 232]{merleau1966phanomenologie}.} und ermöglicht das Thematisieren von Dingen, sowie die "`Stellungnahme des Subjekts in der Welt seiner Bedeutungen"'\footnote{\Cite[Siehe][S. 229]{merleau1966phanomenologie}.}. Sprechen als Akt der Kommunikation mit Anderen wird im nächsten Abschnitt besprochen. 

Wäre Denken vorsprachlich, dann läge es doch nahe, dass Sprache nur im Dialog mit Anderen seine Verwendung f"ande, im eigenen Denkprozess wiederum nicht gebraucht und folglich auch nicht benutzt werden w"urde. Dies steht aber im Widerspruch zu unserer alltäglichen Erfahrung, dass das Denken stets nach Ausdruck sucht.\footnote{\Cite[Vgl.][S. 216]{merleau1966phanomenologie}.} Ein Gedanke kann nicht gefasst werden, wenn er keinen Ausdruck hat - wir haben ihn noch nicht \emph{begriffen}. Ein vorsprachlicher, das heißt unausgedrückter, Gedanke wäre überhaupt nicht da. Damit ich Dinge begreifen, über sie nachdenken und ihren Sinn ausdrücken kann, muss ich sie voneinander unterscheiden können. Solch ein Unterscheiden ist deshalb möglich, weil die Dinge für uns einen Sinn haben, weil wir von ihnen einen Begriff haben. Unser Begriff von etwas ist entsprechend dessen emotionale Essenz für uns, sein Kern. Der gedachte Begriff, das heißt der Gedanke, braucht, wie jedes andere Phänomen, also mentale Repr"asentation, einen Ausdruck, damit er uns "uberhaupt \emph{gegenw"artig}\footnote{\cite[Vgl.][S. 215 f.]{merleau1966phanomenologie}.} ist. Sobald ein Kind das erste Mal einen Zug sieht, wird es "`Was ist das?"' fragen, und sich mit dem Wort "`Zug"' zufrieden geben. Wäre niemand zugegen gewesen zum fragen, m"usste das Kind in Zukunft f"ur den Gedanken an den Zug auf ein mentales Ab\emph{bild} als Ausdruck des Begriffs zur"uckgreifen. Dies wäre aber schwer nach au"sen kommunizierbar, und beim Sprechen müsste es dann mit einem Wort wie "`Dingsbums"' Vorlieb nehmen oder den Zug aufmalen. Aus derlei Hilfestellungen wird möglicherweise auch einmal Sprache entstanden sein. Nicht verwunderlich ist es also, dass Kinder gerne auf Onomatopoesie zurückgreifen und auch die Wörter in den Gebärdensprachen eine gewisse Bildhaftigkeit besitzen. 

Wenn wir uns an die Vergangenheit erinnern, dann "`vergegenw"artigen"' wir sie uns anhand mentaler Ausdr"ucke wie Bilder, Ger"uche oder Emotionen - wir bringen uns das Vergangene durch seine Ausdr"ucke in die Gegenwart. Und genau das ist Denken: Vergegenw"artigung von sinnhaften, mentalen Ausdr"ucken. "`Die wesentliche Leistung des Ausdrucks ist nicht die, da"s er Gedanken [...] zu fixieren gestattet [...], sondern darin, da"s dem gegl"uckten Ausdruck die Bedeutung ein Dasein [...] gleich dem eines Dinges [...] verdankt"'\footnote{\Cite[Siehe][S. 216]{merleau1966phanomenologie}.}. Welche Form die mentalen Ausdrücke haben, ist ebenso nebensächlich wie in der natürlichen Sprache. Wie auch beim Sprechen Gesten und Mimiken hinzu genommen werden können, oder Grammatik missachtet werden kann, oder beim beim Schreiben Emojis eingesetzt werden können, geht die Sprache des Denkens ebenso über den "`\emph{Wort}schatz"' der einzelnen Person hinaus. Wenn ich über etwas nachdenke, so stelle ich es mir vor, wie es mir am natürlichsten erscheint. Für Menschen, die zum Beispiel die gesprochene Sprache beherrschen, liegt es nahe, mentale Wortlaute zum Denken zu benutzen. Das Denken wird bei einer taubstummen Person wahrscheinlich sehr anderen Ausdruck finden, und wieder anders bei einem Kind, das noch nicht gelernt hat zu sprechen. Diese Vielfalt an Ausdrucksformen im Denken können eine Einteilung in "`echtes"' Denken und mentale Repräsentation vortäuschen, gerade weil gewisse Ausdrucksformen so verschieden sind vom einfachen, gesprochenen Wort. Manch einen komplexen Gedanken können wir schon so gewohnt sein, dass es um ihn zu denken keine Worte braucht, da er für uns schon seinen eigenen mentalen Ausdruck gefunden hat.\footnote{\Cite[Vgl.][S. 217]{merleau1966phanomenologie}.} Die verschiedenen Auffassungen und Begriffe, die wir von Dingen haben, sind, wie schon oben festgestellt, stets \emph{unscharf}. Manche Vorstellung ist aber "`höher auflösend"' als andere, und so kommt es zu dem Phänomen, dass besonders klar ausgedrückte Gedanken auch klarer zu begreifen sind. Ich kann eine vage Meinung zu etwas in meinem Kopf herumtragen, doch wenn ich sie beispielsweise detailliert niederschreibe und \emph{ausformuliere}, ist sie für mich sehr viel "`greifbarer"'. Doch das heißt nicht, dass sich die weniger hochauflösende Meinung nicht vorher mental ausgedrückt hat. Die darin vorgekommenen Bewusstseinsinhalte hatten nur einfach eine größere Ambiguität.
%Das reine Denken wenn man so will ist ein Drang (217)

\vspace{5mm}

\noindent\textbf{$(vi)$ Sprechen mit Anderen}

\noindent Es stellt sich nun die Frage, wie beim Kommunizieren mit Anderen der Sinn "`ausgetauscht"' wird. Es ist nicht unüblich Kommunikation so zu verstehen, dass der zu übertragende Gedanke in eine potenziell dechiffrierbare Nachricht verpackt wird, um dann von jemand Anderem entschlüsselt zu werden, die oder der dann den Gedanken möglichst gleich dem Original für sich nachvollzieht. In dieser Vorstellung ist das Verstehen der Nachricht eine intellektuelle Leistung. Demnach müsste ich beim Zuhören eines Gegenübers das selbe tun, wie beim Übersetzen eines Textes aus einer Fremdsprache. Solch ein Verständnis von Kommunikation ist aber weder mit dem Vorhergegangenen, noch mit unserer alltäglichen Erfahrung vereinbar. Entsrpechend 

dass es also eine trennugn gibt zwischen den beiden gedanken, liegt besonders nahe wegen missverständissen

So wie das Denken inneres Sprechen ist, ist auch das Sprechen äußeres Denken. 



Was ist sprache -> die Sprechwelt, der Handlungsraum
Sprache ist die Sprachwelt, in der wir leben, eine abstrakte Sammlung von ‘Regeln’ über das Sprechen. Doch beim sprechen denke ich über diese nicht nach, sie bilden eben die absrakte Struktur meiner Sprachwelt. Von zB ‘der deutschen Sprache’ zu sprechen ist ohnehin schwierig

wie wird sprache erlernt? haben wir im grunde schon, doch nochmal kurz ansprechen 221 f

!!!! wenn ich lang genug mir das verhalten von tieren und mir fremden kulturen allgemein anschaue, dann kann ich auf dauer die bedeutung von den gesten ableiten. dieses interpretieren und zuordnen von mir bekannten bedeutungen zu mir fremden gesten ist fundamental anders als das einssein des zeichens mit der bedeutung. dieses erlernen des fremden ist hier wichtig (in tierkapitel)
	

Die Frage der Herkunft von Kommunizierbarkeit: 220f

Sprache als Sediment des Sprechens 232

Sprache ist nicht tautologisch aufgrund ihrer Unschärfe
-> Kontrast zum wittgentienschen modell
-> versprechen

Sprechen ist das aktive Handlung, die sich der sprache bedient

Wichtig: ursprüngliches sprechen ist gemeint!! siehe zettel

sprechen: "`Der Leib als Ausdruck"': letzten Endes \emph{wird} der Leib das Sprechen und das Denken Vgl 233

wichtig wieder: trennung intellektuelles verstehen des anderen

Das gesprochene oder gebärdende Sprechen ist also nur eine Ausdrucksform, n"amlich eine besondere Art der Geste, die uns eine weitere Erfahrungs- und Mitteilungsdimension gibt.\footnote{\cite[Vgl.][S. 216 f.]{merleau1966phanomenologie}.} Das Wort, gleich einem Ding, wirkt auf uns ein und ver"andert unser Sein. So wie eine Geste als Teil unserer Kulturwelt von Bedeutung erf"ullt ist, so ist es auch das Wort, weil es Teil einer Sprachwelt ist, die es zugleich voraussetzt. Da die Sprachwelt einer uns fremden Kultur nicht Teil unserer Welt ist, vernehmen wir die ihr zugeh"origen W"orter nur als bedeutungslose Laute. Die Muschel ist f"ur mich blo"s ein schön anzusehendes Objekt aus dem Meer, f"ur den Krebs ist es ein Zuhause. 

Schrift, Gebärdenschrift
wie werden alte schriftzeichen entziffert - ägyptologie und so
	-> die sprachen sagen etwas 213


\vspace{5mm}

\noindent\textbf{$(vii)$ Sprechen mit Anderen}

\noindent



\end{onehalfspace}
\nocite{*}
%\bibliography{merleau-ponty-essay}
\printbibliography
\end{document}
