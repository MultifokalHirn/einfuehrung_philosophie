\documentclass[a4paper, emulatestandardclasses]{scrartcl}
\usepackage{graphicx}
\usepackage{fullpage}
%\usepackage{parskip}
\usepackage{color}
\usepackage[ngerman]{babel}
\usepackage[utf8]{inputenc}
\usepackage{hyperref}
\usepackage{calc} 
\usepackage{enumitem}
\usepackage{titlesec}
\usepackage{bussproofs}
\usepackage[export]{adjustbox}
%\pagestyle{headings}

\titleformat{name=\section,numberless}
  {\normalfont\Large\bfseries}
  {}
  {0pt}
  {}
\date{\vspace{-3ex}}
\begin{document}

\title{
    \vspace{-30pt}
	\includegraphics*[width=0.1\textwidth,right]{ErstesSem/images/hu_logo2.png}\\
	\vspace{-10pt}
	Classical Chinese Philosophy\\As Linguistic Analysis (26.04.18)}%}\\\vspace{10pt}\small{Lennard Wolf}}
	\subtitle{Introduction to Chinese Philosophy (Lesegruppe, SS 18)\\
          Aufsicht: Prof. Dr. Beaney\\
          Protokollant: Lennard Wolf}
\maketitle
\vspace{-40pt}

\section*{Was ist Konfuzianismus?}
\textbf{Vorbemerkungen}

\begin{itemize}
  \item Konfuzius war Lehrer am Hof
  \item Kulturrevolution: Alte Doktrin der Unterdrückung
  \item Staatsdoktrin von heute: 30\% von Mao waren falsch, u.A. Konfuzianismus zu verwerfen 
  \item Konfuzianismus als Lehre
\end{itemize}

\noindent \textbf{Dominoeffekt}

\begin{itemize}
  \item Verhalte ich mich korrekt, ist die Familie in Harmonie.
  \item Wenn die Familien in Harmonie sind, ist es auch das Dorf.
  \item Sind die Dörfer in Harmonie, ist es auch die Provinz.
  \item Sind die Provinzen in Harmonie, dann ist es auch das Reich.
  \item Sind die Reiche in Harmonie, dann ist es auch der Kosmos.
\end{itemize}

\section*{Classical Chinese Philosophy As Linguistic Analysis}

\begin{itemize}
  \item Streit zwischen westlichen Sinologen: \emph{Gab es Sprachphilosophie in China?}
  \item Semantische Theorie der Sprache - China: Pragmatische Theorie der Sprache
  \item Westen eher Realismus, In China eher Relativismus
  \item Denken Vom Großen ins Kleine
  \item Überlieferung von alten Texten immer durch die Brille des Neokonfuzianismus
  \item Taoismus war Angriff auf Konfuzianismus, weshalb letzterer den Taoismus immer schlecht darstellte
  \item Kein Wahrheitsprädikat
  \item Chinesische Philosophie ähnlich wie Wittgenstein 2
  \item Kein klarer Unterschied zwischen Deklarativ und Normativ
  \item Kein Leib Seele Dualisms
  \item Kein Propositionaler Glaube, sondern als Disposition zu Handlungen
  \item Es gibt keine feste Satzstruktur
  \item Keine Unterscheidung zwischen Zählnomen und Massennomen
  \item Bedeutung eines Wortes ist sein Gebrauch in der Sprache
  \item Werden Chinesische Philosophie nie verstehen, wenn wir die westlichen Probleme der Philosophie immer reinprojezieren wollen (aber tut er das denn nicht?)
  \item Frage für Chines. Phil nicht nach Wahrheit, sondern: Benutzen wir die Sprache richtig?
  \item Regelutilitarismus (?)
  \item Suche nach einer idealen Form der Sprache, doch es wurde nicht geglaubt, dass eine ideale Sprache konstruiert werden kann (weshalb auch verständlich wäre, dass kein großer Fokus auf Logik existiert)
  \item Idealsprache nicht zielführend, gewisser Grad an Unbestimmtheit in der Sprache sei tatsächlich nützlich
\end{itemize}

\section*{Offene Fragen}

\begin{itemize}
  \item Heart-Mind? Leib? Metaphorisch?
  \item Was ist Correlative Logic?
  \item Wieso sollte Ideenlehre aus solch einer Sprache nicht möglich sein
  \item Wie funktioniert das utilitätsprinzip bei Mu-Tzu
  \item Visualisation of History of Chinese Philosophy
\end{itemize}




%\begin{description}[leftmargin=!,labelwidth=\widthof{\bfseries Unterschied des Bewusstseins}]
%  \item[Bewusstsein] Prozess der Umkehrung/Selbstpr"ufung des Bewusstseins, sowie seines Bewusstseinsgegenstandes, wodurch das Bewusstsein etwas anderes wird. Und diese Ver"anderung ist die Erfahrung. So bestreitet es einen Weg, "uber welchen es zur 	Selbstthematisierung kommt.
%  \item[\emph{An sich f"ur es}] Das Bild das sich das Bewusstsein von etwas macht, sein Wissen. "Uber dieses kommt das Bewusstsein zur Erfahrung. Bild vom Gegenstand entspricht nicht dem, wie er f"ur sich selbst ist (Wahrheit).
%  \item[Wir] Instanz des Autors/Lesers, die dem Bewusstsein "uber die Schulter schaut und "`redupliziert"' die Erfahrung des Bewusstseins im Buch/Lesen.
%  \item[Unterschied des Bewusstseins] Unterschied zwischen dem \emph{an sich f"ur es} (Wissen) und dem, womit das Bewusstsein dieses Wissen vergleicht (Wahrheit).
%  \item[Vollst"andigkeit der Form] Wird erreicht dadurch, dass durch die doppelte bestimmte Negation alles einseitige Wissen "uber den Gegenstand verworfen wird.
%\end{description}


\end{document}
