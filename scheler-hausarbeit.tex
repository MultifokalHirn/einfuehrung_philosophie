\documentclass[a4paper, 12pt]{article}
%\usepackage{CJKutf8} % japanese
\usepackage{graphicx}
\usepackage{hyperref}
\usepackage{fullpage}
%\usepackage{parskip}
\usepackage{color}
\usepackage[ngerman]{babel}
\usepackage{hyperref}
\usepackage{calc} 
\usepackage{enumitem}
\usepackage[utf8]{inputenc}
\usepackage{titlesec}
%\pagestyle{headings}
\usepackage{setspace} %halbzeilig
\usepackage[style=authoryear-ibid,natbib=true]{biblatex}
\usepackage[hang]{footmisc}
\setlength{\footnotemargin}{-0.8em}
%\bibliographystyle{natdin}
\addbibresource{scheler-hausarbeit.bib}
\DeclareDatamodelEntrytypes{standard}
\DeclareDatamodelEntryfields[standard]{type,number}
\DeclareBibliographyDriver{standard}{%
  \usebibmacro{bibindex}%
  \usebibmacro{begentry}%
  \usebibmacro{author}%
  \setunit{\labelnamepunct}\newblock
  \usebibmacro{title}%
  \newunit\newblock
  \printfield{number}%
  \setunit{\addspace}\newblock
  \printfield[parens]{type}%
  \newunit\newblock
  \usebibmacro{location+date}%
  \newunit\newblock
  \iftoggle{bbx:url}
    {\usebibmacro{url+urldate}}
    {}%
  \newunit\newblock
  \usebibmacro{addendum+pubstate}%
  \setunit{\bibpagerefpunct}\newblock
  \usebibmacro{pageref}%
  \newunit\newblock
  \usebibmacro{related}%
  \usebibmacro{finentry}}

%\titleformat{name=\section,numberless}
%  {\normalfont\Large\bfseries}
%  {}
%  {0pt}
%  {}
\date{\vspace{-3ex}}


\begin{document}

\title{\vspace{5ex}
	\includegraphics*[bb=0 0 720 200, width=0.72\textwidth]{ErstesSem/images/hu_logo.png}\\
	\vspace{30pt}
	\scshape\LARGE{Max Schelers Argument gegen den unausgedehnten Geist}\\\vspace{20pt}}
	


\author{Die Erfahrung der Realität durch Widerstand (PS)\\
	\vspace{7pt}
          Dozent: Dr. Matthias Schloßberger\\\vspace{4pt}Lennard Wolf\\
        \small{Matrikelnummer: 583052}\\
        \small{E-Mail: \href{mailto:lennard.wolf@hu-berlin.de}{lennard.wolf@hu-berlin.de}}\\
        \small{Telefonnummer: +49 176 5687 4131}\\
        \small{Studiengang: B.A. Philosophie}\\
        \small{Modul: ?}}

\maketitle

\vspace{\fill}

\begin{minipage}[]{0.92\textwidth}
    \centering
    \onehalfspacing
    \large   
    23. April 2018\\
    Wintersemester 2017/2018

    \vspace{-20mm} 
\end{minipage}%
\thispagestyle{empty}
\newpage
%\clearpage
%\thispagestyle{empty}
%\tableofcontents
%\newpage
\setcounter{page}{1}

\begin{onehalfspace} 

\noindent\textbf{$(o)$ Einleitung}

\noindent Die bis heute einflussreiche, in der Neuzeit besonders durch Ren\'e Descartes populär gemachte Vorstellung, der Geist sei \emph{unausgedehnt}, wird von Max Scheler in seinen Werken immer wieder angesprochen und als \emph{proton pseudos} (falsche Prämisse) der gängigen, modernen Erkenntnistheorie entlarvt. Sie ist tief im westlichen Denken verankert und bildet die Grundlage für die Theorie, dass unsere Wirklichkeit eine Repräsentation der realen Welt sei, "`Repräsentationalismus"' genannt. Dieser Theorie nach ist die Welt, wie wir sie erfahren, also nur eine durch unsere Anschauungen gefärbte Vorstellung, nicht die reale Welt.

Im Repräsentationalismus wird Scheler zufolge eine erkenntnistheoretische \emph{Umkehrung} vollzogen. Er habe erkannt, "`daß in der natürlichen normalen Weltanschauung die \emph{vorwiegende} Täuschungsrichtung ist, wirklich Psychisches vermeintlich für physisch, nicht wirklich Physisches vermeintlich für psychisch zu halten"'\footnote{\Cite[Siehe][S. 257]{scheler-idole}.}. So werden zum Beispiel Gefühle und Emotionen statt als seelische Phänomene als rein physisch, der Raum und die Zeit statt als physische Realität zur psychischen Anschauung umgekehrt. Diese weit verbreiteten, aber an unserer Lebensrealität vorbeigehenden "`Verwirrungen"' des Repräsentationalismus können auf die falsche Prämisse des unausgedehnten Geistes zurückgeführt werden. 

In dieser Hausarbeit stelle ich Max Schelers Argument gegen den Repräsentationalismus vor und zeige, wie dieses notwendig ein Argument gegen den unausgedehnten Geist darstellt. Da im Werk Max Schelers meiner Kenntnis nach diese Argumentation stets nur implizit vorhanden ist, möchte ich sie hiermit systematisch darstellen und somit explizit machen. Zudem gehe ich im Anschluss noch auf Folgen und Probleme des Arguments ein. 

%benutze ich wahrnehmung etc konsistent?
\vspace{3mm}

Ich werde dafür wie folgt vorgehen. In Abschnitt $(i)$ führe ich in die Vorstellung vom unausgedehnten Geist ein und zeige, dass der Repräsentationalismus eine notwendige Konsequenz dieser ist. Es folgt in Abschnitt $(ii)$ eine Darstellung des ersten Teils des Arguments, der sich mit der Möglichkeit einer ausgedehnten Repräsentation auseinandersetzt. Danach beschreibe ich in Abschnitt $(iii)$ den zweiten Teil des Arguments, der sich mit der anderen Möglichkeit, einer unausgedehnten Repräsentation auseinandersetzt. Abschnitt $(iv)$ handelt dann von Schelers eigenen Konzeption von Leib und Umwelt und in Abschnitt $(v)$ fasse ich die Gedanken noch einmal konkludierend zusammen.

%daher struktur der arbeit: ausgangslage klären, dann zeigen dass es zwei möglichkeiten gibt der auflösung, wie der geist sich seine repräsentation schafft: 1. durch auseinandefslten, wo es zu lücke kommt oder 2. dass die repräsentation 0 dimensional bleibt, das bild konsistent bleibt, aber dann doch auf die natürliche erfahrung appeliert wird und gezeigt wird, dass das so nicht sein kann (hier mein eigener input der der zeit (??)

\vspace{5mm}
\noindent\textbf{$(i)$ Der unausgedehnte Geist}

%------------------%
%    Einleitung    %
%------------------%

\noindent Im kartesisch geprägten Bild vom Menschen besteht dieser aus einem ausgedehnten, nicht denkenden Körper (\emph{res extensa}) und einem unausgedehnten, denkenden Geist (\emph{res cogitans}). Der Körper befindet sich in der räumlichen, d.h. ausgedehnten, und zeitlichen Welt, der Geist hat auf (erst einmal) unbestimmte Weise\footnote{Bei Descartes im Speziellen wurde hierfür die Zwirbeldrüse als Tor zur Welt erkannt, andere Theorien legen wiederum anderes nahe. Welcher Natur die kausale Interaktion von \emph{res cogitans} und \emph{res extensa} ist, ist hier aber nicht weiter relevant.} Einblick und Einfluss auf diese Welt, und diese wiederum ebenso auf ihn. Der Geist ist also, als unausgedehntes Ding, nicht \emph{in} der körperlich wahrnehmbaren Welt, hat keine räumliche Form, eine Masse oder gar einen Ort, an dem er sich befindet.\footnote{Da wir spätestens seit Einstein wissen, dass Zeit und Raum nicht voneinander trennbar sind, wäre der Geist also auch "`außerhalb"' der Zeit.} Zur Verdeutlichung denke man zum Beispiel an Zahlen: Diese können zwar im Raum dargestellt werden, doch sind \emph{an sich} keine ausgedehnten Gegenstände - sie sind unausgedehnt. Während die mechanischen Bestandteile des Körpers, wie zum Beispiel das Auge oder das Bein, durch die Naturwissenschaften aufgrund von physikalischen Interaktionen beschreib- und erklärbar sind, können die mentalen Eigenschaften des Geistes, zum Beispiel seine Fähigkeiten zu zweifeln, zu wollen oder zu empfinden, nicht durch die Analyse der Interaktion von ausgedehnten Gegenständen verstanden werden. Zur Veranschaulichung: Der Tisch vor mir ist ein relativ schwerer, mittelgroßer Gegenstand, der zum Beispiel mit dem Mikroskop untersucht werden kann, während meine Wahrnehmung \emph{von ihm} eben kein solcher Gegenstand ist, und auch nicht von einem Mikroskop untersucht werden kann - sie ist nämlich \emph{mental}. 

Diese ontologische Unterscheidung zweier Substanzen \emph{res extensa} und \emph{res cogitans}, die Descartes vertritt, ist für unsere Untersuchung aber nicht weiter relevant. Wir können der Frage, ob der Geist nun tatsächlich aus einer anderen Substanz besteht als der Körper, agnostisch gegenüber bleiben, da auch mit einer monistischen Ontologie der Gedanke des unausgedehnten Geistes vertreten werden kann: So wie es schwierig sein kann, einen aus vielen Vögeln emergierenden Vogelschwarm als einen Gegenstand mit festem Ort, Ausdehnung und Masse zu beschreiben, er aber auf physikalische Mechanismen zurückführbar wäre, wäre es schwierig, den menschlichen Geist als etwas ausgedehntes zu bezeichnen - auch wenn man ihn auf mechanische Interaktionen zurückführen könnte. Wichtig ist für uns statt der \emph{ontologischen} Frage vielmehr die \emph{erkenntnistheoretische} Frage, also wie der Geist die Dinge in der Welt, seinen \emph{eigenen} Körper oder andere Geister in ihren Körpern erkennen kann.

%------------------%
%    Hauptteil     %
%------------------%

Der kartesischen Tradition nach findet der Geist die Dinge in der Welt erst einmal alle auf gleiche Weise vor, nämlich durch \emph{Empfindungen}. Diese können zwar unterschiedlich "`gefärbt"' sein, doch prinzipiell gibt es keine "`inneren"' oder "`äußeren"' Empfindungen\footnote{\Cite[Vgl.][S. ?? (501?)]{scheler-ethik}.}, da der Geist kein "`Innen"' und "`außen"' hat - alle Empfindung ist somit immer schon "`äußere"' Empfindung. Über diese kann das \emph{cogito} dann mithilfe seiner eingeborene Ideen (\emph{ideae innatae}), wie zum Beispiel der Logik, nachdenken und auf Dinge in der Welt, sowie deren Ordnung und Mechanismen, schließen\footnote{Ob es sich immer um bewusstes Nachdenken oder "`unbewusste Kausalschlüße"' handelt, bleibt für uns erst einmal irrelevant.}. Aus den so erkannten Dingen der Außenwelt und ihren Eigenschaften entsteht für den Geist Stück für Stück ein Welt\emph{bild}, eine \emph{Repräsentation}\footnote{Im folgenden auch austauschbar als "`Wirklichkeit"' bezeichnet, als Gegenstück zur \emph{Realität}, oder der Welt an sich.}. Der Tisch der vor mit steht ist also nicht der reale Tisch \emph{an sich}, sondern eine mentale Repräsentation von diesem, meine eigene Interpretation der Empfindungen, "`meine Wirklichkeit"'. Vom Gegenstand wird bei der Repräsentation sowohl abgezogen - ich bin zum Beispiel nicht in der Lage ihn von allen Seiten gleichzeitig zu betrachten - als auch hinzugefügt - ich urteile zum Beispiel über ihn.

Wenn ich das Gesicht eines Freundes erblicke, passiert demnach ungefähr folgendes: Der ausgedehnte Gegenstand, den ich als sein Gesicht interpretieren werde, reflektiert Lichtstrahlen in meine Augen, wo sie wiederum neuronale, elektrische Reize verursachen, die dann im Gehirn nach den komplexen Regeln der neuronalen Ordnungen verarbeitet werden und als neue Informationen zum Geist kommen. Diese Informationen, in denen es nur um die Lichtverhältnisse in meiner Umgebung geht, bringe ich nun in Verbindung zu meinen bisherigen Erfahrungen und ich schlußfolgere, dass die vorliegende Anordnung von Farbtönen im Licht dadurch zustande kommt, dass mein wohl bekannter Freund anwesend ist. Ich \emph{sehe} also meinen Freund nicht direkt, sondern ich mache mir ein Bild der Situation, das eine konstruierte \emph{Repräsentation} des eigentlichen Gesichts meines Freundes beinhaltet und durch physische Reize (Lichtstrahlen/Regungen im Gehirn\footnote{\Cite[Vgl.][S. 270f.]{scheler-idole}. ???}) und meiner Ordnung dieser anhand meiner Anschauungen, eingeborenen Ideen und vorhergehenden Erfahrungen, zustande kam. Es sei erwähnt, dass der der Geist einen unmittelbaren Zugriff auf die Repräsentation der Welt haben müsste, da es ansonsten potenziell zu einem unendlichen Regress von vermittelten Vermittlungen käme.

Ob die Empfindungen, auf denen die Repräsentation basiert, alle nur Täuschungen sind - zum Beispiel von einem Dämonen\footnote{\Cite[Vgl.][S.?]{descartes}.} (?) veranlasst oder ähnlich den Schattenbildern in Platons Höhle, und es die wahrgenommene so Welt gar nicht gibt - das kann der Geist niemals anhand logischer Schlüße erkennen. Alles, was er also sicher wissen kann, ist, dass es ihn selbst gibt.\footnote{Das klassische "`\emph{cogito ergo sum}"' \Citep[Siehe][S.?]{descartes}.} 

Doch warum muss es sich so verhalten? Könnte es nicht auch sein, dass der Geist direkt mit der realen, ausgedehnten Welt im Kontakt steht und keinerlei Repräsentation benötigt? Zum einen ist es natürlich nicht zu bezweifeln, dass der Geist mit der realen Welt im Kontakt stehen muss, da er ansonsten keine Informationen über sie erhalten könnte. Aber gerade weil der Geist nur \emph{Informationen} über die Welt durch Empfindungen erhält, und nicht die Welt an sich (Information über die Beschaffenheit eines Dings ist nicht das Ding selber - der ausgedehnte Tisch kann ja auch nicht in den unausgedehnten Geist "`hineingetan"' werden), hat er es immer nur mit den Empfindungen zu tun, und nicht mit den Dingen. Diese Empfindungen bilden, wie oben beschrieben, die Welt ab: sie ergeben eine Repräsentation. \emph{Repräsentationalismus} - die Theorie, dass die Wirklichkeit eine Repräsentation ist und wir niemals Zugriff auf etwas außerhalb der Repräsentation haben, ist also eine \emph{notwendige} Konsequenz für eine Theorie des unausgedehnten Geistes.


%------------------%
%    Konklusion    %
%------------------%
Diesen Umstand nimmt Scheler sich zum Ausgangspunkt für die nun im Folgenden dargelegte Argumentation, die auf einer Untersuchung von Eigenschaften der Repräsentation beruht. Diese kann nämlich nur entweder $(a)$ selbst ausgedehnter Natur sein oder $(b)$ nicht ausgedehnter Natur sein. Weitere Optionen sind nicht denkbar, da die Optionen die jeweilige Negation voneinander sind. Die folgenden beiden Abschnitte sollen aufzeigen, wieso sie für Scheler beide abzulehnen sind. Wenn beide Optionen abzulehnen sind, wäre der Repräsentationalismus gänzlich abzulehnen, und da dieser eine notwendige Konsequenz der Prämisse des unausgedehnten Geistes ist, wäre folglich die Idee des unausgedehnten Geistes abzulehnen\footnote{\emph{Modus tollens}.}.

\vspace{5mm}
\noindent\textbf{$(ii)$ Die ausgedehnte Repräsentation}

%------------------%
%    Einleitung    %
%------------------%

\noindent Wenden wir uns nun zuerst Option $(a)$, der ausgedehnten Repräsentation der Welt, zu.\footnote{Fragen zum Beispiel dazu, ob dann die gesamte Repräsentation einen eigenen großen Raum "`aufspannt"', und dergleichen lassen wir hier außen vor.} Sie ist wahrscheinlich als die intuitiv plausiblere zu bezeichnen, da der Mensch sich zumeist als ein ausgedehntes Wesen in einer räumlichen Welt sieht. Die Ablehnung dieser Option basiert auf einem Zirkelproblem, das bei der Zuordnung von seelischen Empfindungen zu den physischen, unbelebten Körperdingen aufkommt.

%------------------%
%    Hauptteil     %
%------------------%

Was müsste der Fall sein, wenn die Wirklichkeit des unausgedehnten Geistes ausgedehnt ist? Die Empfindungen, die der (ausgedehnte) Körper macht - also Tastempfindungen, Farbempfindungen, Geräuschempfindungen etc. - würden in einem ersten Schritt in den unausgedehnten Geist "`introjeziert"'\footnote{\Cite[Vgl.][S. ???]{scheler-ethik}.} werden, und können somit zwangsläufig nicht räumlich sein. Basierend auf diesen unausgedehnten Empfindungen wird nun wiederum eine neue, ausgedehnte Repräsentation konstruiert, mit der direkt interagiert werden kann. Da auf einen eigenen Körper erst \emph{a posteriori} geschlossen werden kann,\footnote{Als Einwand könnte aufkommen, dass einen eigenen Körper zu haben möglicherweise \emph{a priori} bekannt ist. Doch selbst wenn dies der Fall wäre, müsste weiterhin geklärt werden, wie der Körper in Abgrenzung von anderen Gegenständen erkannt wird.} muss in der frühen Entwicklungsphase des Menschen ein Zustand existieren, in dem der Geist Empfindungen macht, aber noch keinen eigenen Körper "`identifiziert"' hat. %Da das Wesen der Ausgedehntheit in den Empfindungen nicht mitgegeben ist, muss der Geist schon vorher mit ihm vertraut sein. Das heißt um die unausgedehnten Informationen räumlich vorzustellen, muss die Anschauung der Räumlichkeit \emph{a priori} bekannt sein. 



Befassen wir uns nun zunächst mit der Frage, was ein eigener Körper überhaupt ist. Wenn wir im folgenden vom "`realen Körper"' sprechen, dann ist damit der Körper \emph{an sich} gemeint, und nicht die Vorstellung des eigenen Körpers in der Wirklichkeit - auf diesen wird hier stets bloß mit "`Körper"' referiert. Der reale Körper ermöglicht auf mechanische Art und Weise die Aufnahme und Weiterleitung von Informationen über die Welt um ihn herum - er ist die Schnittstelle des Geistes zur Realität. Mein Körper ist dann die Repräsentation dieses an sich rein mechanischen, toten Gegenstands. Es geht uns nun zuerst um folgende Frage: Ist mein Körper ganz einfach ein Ding wie jedes andere, oder liegt ein qualitativer Unterschied vor? 

Wenn zum Beispiel etwas gegen meinen realen Körper drückt, dann wird mir dies in der Repräsentation durch eine Empfindung am Körper vermittelt. Wenn etwas gegen den realen Tisch vor mir drückt, spüre ich dies \emph{nicht} an meinem Körper. Da es anscheinend ein \emph{am-eigenen-Körper-Spüren} gibt, jedoch kein \emph{am-eigenen-Tisch-Spüren}, scheint es einen Unterschied zwischen meinem Körper und anderen Dingen zu geben. In \emph{Der Formalismus in der Ethik} wirft Scheler daher die Frage auf, was denn aber den eigenen Körper von anderen Gegenständen wie den Tisch unterscheide.\footnote{\Cite[Vgl.][S. 498]{scheler-ethik}. Im Quelltext wird schon von "`Leib"' statt von "`Körper"' gesprochen, doch auf diese Unterscheidung möchte ich erst noch hinaus.} Er schreibt dazu, dass der Körper das "`Zentrum"'\footnote{\Cite[Vgl.][S. 498]{scheler-ethik}. Hier auch weiterhin der Austausch von "`Körper"' und "`Leib"'.} der wahrgenommenen Gegenstände sei, und diese damit \emph{immer} in Relation zu ihm vorgefunden werden. Der Körper hat natürlich etwas dinghaftes an sich, und ich kann zum Beispiel meine Füße auch als einen Gegenstand wie den Tisch betrachten. Aus dieser Doppeldeutigkeit heraus kommt an dieser Stelle die begriffliche Unterscheidung zwischen dem Körper als "`Körper\emph{ding}"', und dem Körper als "`Leib"' zum Tragen. Meinem Körper sind diese beiden Momente stets inne und ich kann das eine nicht ohne das andere haben: Mein Leib ist das stets Präsente, durch das die Gegenstände wahrgenommen werden. Meinen Körper als Ding kann ich aber gleich dem Tisch durch meinen Leib wahrnehmen. Um die Frage nach dem Unterschied nun noch einmal mit dem neuen Begriff zu beantworten: Es gibt einen qualitativen Unterschied zwischen dem Leib und den Gegenständen, aber keinen zwischen dem Körper\emph{ding} und den Gegenständen. Mein Leib ist in der ausgedehnten Wirklichkeit mein ausgedehntes Vehikel - mit einem Teil meines Leibes bin ich unter dem Tisch, mit einem anderen Teil über dem Tisch. 

Da der Leib nicht Teil des Geistes ist\footnote{\Cite[Vgl.][S. 499]{scheler-ethik}.}, sondern Teil der Repräsentation, muss der Geist auf diesen erst \emph{a posteriori}, d.h. durch Erfahrungen aus der Repräsentation, schließen. Hier zeichnet sich nun ein Zirkelschluss ab: Da mir alle Dinge in der Repräsentation immer nur in Relation zum Leib vermittelt sind, so muss bei jeder Erfahrung der Leib schon vorhanden sein. Da ich auf den Leib aber erst durch Erfahrungen schließen muss, wird dies unmöglich: Ich brauche den Leib, um auf den Leib zu schließen.


% übliche erklärung dass eigener körper entdekt wird

%------------------%
%    Konklusion    %
%------------------%



\vspace{5mm}
\noindent\textbf{$(iii)$ Die unausgedehnte Repräsentation}

%------------------%
%    Einleitung    %
%------------------%

\noindent Kommen wir nun zu der zweiten Möglichkeit, nämlich dass die Repräsentation der Welt unausgedehnt ist. Das Argument Schelers ist hier nicht erkenntnistheoretischer Natur, sondern \emph{phänomenologischer}. Scheler appelliert also eher an unsere Erfahrung und 	versucht so aufzuzeigen, wie \emph{unplausibel} diese Option ist.

%------------------%
%    Hauptteil     %
%------------------%

Vergegenwärtigen wir uns zuerst, was dies eigentlich bedeuten würde. Da Räumlichkeit als Weise der Repräsentation der Räumlichkeit wegfällt, habe ich Zugriff auf räumliche Informationen nur durch nichträumliche Abstraktionen von ihnen. Zu wissen, dass sich etwas hinter mir befindet, müsste sich mir dann also auf eine unräumliche Weise darstellen, ebenso wie räumliche Eigenschaften, zum Beispiel Größe oder Oberflächenbeschaffenheit. Ein Schmerz kann sich für mich nicht ausbreiten im räumlichen Sinn, ich kann nur abstrakt wissen, dass der Schmerz eine irgendwie geartete Transformation durchmacht. 

Unsere Idee der Ausgedehntheit kommt aber gerade daher, dass wir meinen, mit ausgedehnten Dingen zu interagieren. Wie verwenden den Begriff nicht auf eine abstrakte Art und Weise in dem Wissen, dass sich dahinter eigentlich etwas fundamental anderes verbirgt. Wir meinen mit dem Begriff das, was er aussagt. Dass sich der Schmerz in unserem Körper ausbreitet ist für uns eben keine rein mentale Information, so wie zum Beispiel eine Zahlenfolge, sondern ist ein räumliches Gefühl des \emph{Ausbreitens}. Scheler appelliert an uns, dass es reiner Irrsinn wäre, anderes zu behaupten. 

Es tut sich aber noch ein weiteres Problem bei dieser Option auf. Es wird nämlich fraglich, woher wir dann überhaupt das Konzept der Ausgedehntheit haben, und was wir damit meinen, wenn wir sie doch eigentlich nie erleben. Ausgedehntheit ist für uns in dieser Konstellation nun einmal nur auf unausgedehnte Weise gegeben, das heißt das tatsächliche Wesen des Konzepts bleibt uns vollkommen fremd. Es wäre für uns nicht fundamental verschieden von jeder anderen Form der Anschauung, die jeweils alle ebenso unausgedehnt sind, und ebendies allesamt gemein haben. Ihr Wesen ist aber so fundamental verschieden von allen anderen Anschauungen die wir hätten, da sie die Unausgedehntheit mit ihnen eben \emph{nicht} gemein haben dürfte. Weil sie für uns dies aber hätte, da wir uns nichts auf nicht unausgedehnte Weise vorstellen könnten, wäre bei einer unausgedehnten Wirklichkeit unser Konzept von Ausgedehntheit unsinnig. Eine Anschauung ist ja gerade eine \emph{eigenständige} Weise der Repräsentation von Sachverhalten, und hier wäre sie dies gerade nicht mehr, sondern eine Repräsentation einer andersartigen Anschauung von Sachverhalten (der eigentlichen Ausgedehntheit). Ausgedehntheit wäre schlichtweg nicht Ausgedehntheit. 

%------------------%
%    Konklusion    %
%------------------%



\vspace{5mm}
\noindent\textbf{$(iv)$ Folgen und Kritik}

%------------------%
%    Einleitung    %
%------------------%

\noindent Wir haben nun gezeigt, dass eine Repräsentation 


Es sei angemerkt, dass eine Theorie vom ausgedehnten Geist zu allererst vor dem hier vorgebrachten Argument geschützt werden müsste. müsste

%------------------%
%    Hauptteil     %
%------------------%

\begin{itemize}
  \item Das Zirkelproblem
\end{itemize}


%------------------%
%    Konklusion    %
%------------------%

\vspace{5mm}
\noindent\textbf{$(v)$ Konklusion}

Rekapitulieren wir noch einmal den Gedankengang. Wenn der Geist unausgedehnt ist, folgt notwendig, dass die Wirklichkeit für ihn eine \emph{Repräsentation} der ausgedehnten Realität ist. Diese Repräsentation kann nur entweder ausgedehnt oder nicht ausgedehnt sein. Sie kann nicht ausgedehnt sein, da die Weise des In-der-Welt-seins und damit die Weise der Gegebenheit nicht .... Unausgedehnt kann sie wiederum auch nicht sein, da .... Da die den gesamten Möglichkeitsraum ausmachenden Optionen abzulehnen sind, ist allgemein die Idee der Repräsentation, beziehungsweise damit auch die gesamte Theorie des \emph{Repräsentationalismus} abzulehnen. Da diese eine notwendige Folge für den unausgedehnten Geistes ist, ist der unausgedehnte Geist schließlich abzulehnen.



\begin{itemize}
  \item Der Leib ermöglicht die Wahrnehmung und damit auch die Wahrnehmung jeglicher Erkenntnis.
  \item Da also, um erkennen zu können, der Leib schon gegeben sein muss, braucht und kann er gar nicht erkannt werden, er ist nur immer schon gegeben.
  \item da weiterhin unklar ist, wie der Leib das Zirkelproblem löst bleibt schlußendlich zu sagen, dass die These eben nur sein kann, dass der Geist nicht ausschließlich unausgedehnt sein kann. Zu sagen dass der Geist damit ausgedehnt sein muss, in Form eines Leibes, so weit zu gehen wäre von diesem Standpunkt aus noch schwierig.
  \item was folgt aus dem Ganzen: zu husserl zurück, dass 
\end{itemize}




%--------------------------------------------------------------------------------
%%---------------MÜLL------------------------------------------------------------
%--------------------------------------------------------------------------------

%Ob Schelers Interpretation des unausgedehnten Geistes hermeneutisch korrekt ist, werde ich außen vor lassen und entsprechend seine eigene Beschreibung übernehmen, an Stellen aber mit Zitaten aus Descartes Meditationen unterfüttern. (??) 

%Abschnitt \emph{Leib und Umwelt} seines Werkes \emph{Der Formalismus in der Ethik}

%klärung begriff des erkennens: durch logischen schluss (ob bewusst oder unbewusst) aus dem gegebenen das Vorliegen von etwas schon bekanntem.

%\vspace{3mm}
%\noindent\textbf{$(iii.$\footnotesize$i$\normalsize$)$ Das Argument}

%  \item wie aber könnte das wesen des vorliegenden beim ersten erkannt werden, damit es beim zweiten mal überhaupt wiedererkannt werden kann?

%In der Erkenntnistheorie geht es zentral um Fragen der Art, wie ein denkendes Subjekt $S$ ein beliebiges Objekt $O$ \emph{erkennen} kann. Ein solcher Erkenntnisgegenstand $O$ kann zum Beispiel ein "`Wissen wie"' ("`Wie mache ich ein Feuer?"') oder ein "`Wissen dass"' ("`Ist es der Fall, dass das Wachs gelb ist?"') sein. Wie also gelangt $S$ zu $O$? 

%noch darauf eingehen, inwiefern dies relevant ist - Besonders auch in computationaler Kognitionswissenschaft 

%  \item Scheler bezieht sich auf Avenarius und Mach (wer sind diese) als Repräsentanten der Theorie: 

% Beispiel: Beim Baby, der Körper gibt signale über ein ding, das Milch gibt. dieses ding gibt auch wärme und schutz und ist somit gut. Romulus und Remus haben ebendies mit der Wölfin erfahren, denn es gibt prinzipiell keinen Unterschied, wer genau aufzieht (bei Tieren sieht man dies auch.) Wären die Bewegungen hin zur Brust reiner Reflex, oder kennt das Baby 

% Die Welt \emph{für uns} ist für den Idealismus zunächst nur Vorstellung, während die \emph{Außen}welt "`real"' und "`wirklich"' sei.\footnote{\Cite[Vgl.][S. 257]{scheler-idole}}

%Hier fokus wohl auch noch auf psychisch - physisch unterscheidung: wenn es wesensunterschied ist dann müssen diese ja auch unterschiedliche Weise wahrgenommen werden (503 ist das wirklich gemeint?)

% der körper bliebe ein immer fremder anderer, der dem geist (dem ich) seine wünsche und nöte kommuniziert (wer "er" auch immer sei), aber allein hilflos wäre diese zu erfüllen. der geist kümmert sich dann darum, doch könnte dies auch lassen - dass seine existenz mit dem Tod des körpers auch beendet wäre, das kann solch ein geist ohne sicherheit nicht sagen (Es ergibt sich auch trivialerweise, dass wenn der Computer, auf dem eine künstliche Intelligenz läuft, zerstört wird, diese Intelligenz auch aufhört zu existieren. ) - solch eine vorstellung liegt auch gerade dem Gedanken der unsterblichen Seele zugrunde. 

% begriff der introjektion: das auseinander des Raumes und der Zeit wird ineinander gefaltet und dann in den geist introjeziert (?). dieser empfindet das dann alles und baut sich durch bewusste oder unbewusste schlußverfahren eine repräsentation der gegebenen empfindung auf, die entweder mental und damit unräumlich bleibt, wobei dann auch kein zeitempfinden erlaubt wäre, oder wieder ausgefaltet wird, wobei fraglich bleibt, wie der geist darauf kommt, eine raumzeitliche repräsentation zu erschaffen, da die information des Wesens der Raumzeit durch die introjektion der unausgedehnten Empfindung ihn nie erreicht hat.

% Da der Geist in sich unabhängig ist von den körperlichen Gegebenheiten, reicht er am Ende auch als eine Unterscheidung des Menschen vom Tier, welches als ungeistig und damit als reiner Automat gesehen werden muss.

%   \item Ich nehme deinen Geist also gar nicht wirklich wahr, ich folgere nur aus irgendeinem Grund aus dem toten Physischen dass sich dahinter irgendetwas verbirgt

%  \item Ein mögliches Gegenargument wäre, dass beide Optionen nicht stimmen, und die Welt, wie sie uns repräsentiert wird, mit ihrer mehrdimensionalität tatsächlich wenigerdimensional ist als der geist in dem sinne, dass die Repräsentation der Empfindungen in Raum und Zeit für den Geist sehr einfach ist und möglicherweise auch nichts mit der "`echten"' Realität zu tun haben - dass also Raum und Zeit in keinster Weise mit der Realität zu tun haben sondern eben nur Anschauungen des Geistes sind.

%--------------------------------------------------------------------------------
%%---------------FRAGEN AN SCHLOßBERGER------------------------------------------
%--------------------------------------------------------------------------------


%wenn aber nun der geist ausgedehnt ist, heißt dies ja nicht dass er ein messbares ding in der physischen welt ist - inwiefern ist belegbar, dass die kartesische vorstellung nicht am ende zustimmen würde? wo ist die klare trennung zu der kartesischen vorstellung?

%ist res extensa denn eine vorstellung des geistes oder nicht?

%Weltbild = Repräsentation?

%--------------------------------------------------------------------------------

\newpage

\end{onehalfspace}
\nocite{*}
\printbibliography
\end{document}
