\documentclass[a4paper, 12pt]{article}
%\usepackage{CJKutf8} % japanese
\usepackage{graphicx}
\usepackage{hyperref}
\usepackage{fullpage}
%\usepackage{parskip}
\usepackage{color}
\usepackage[ngerman]{babel}
\usepackage{hyperref}
\usepackage{calc} 
\usepackage{enumitem}
\usepackage[utf8]{inputenc}
\usepackage{titlesec}
%\pagestyle{headings}
\usepackage{setspace} %halbzeilig
\usepackage[style=authoryear-ibid,natbib=true]{biblatex}
\usepackage[hang]{footmisc}
\setlength{\footnotemargin}{-0.8em}
%\bibliographystyle{natdin}
\addbibresource{descartes-hausarbeit.bib}
\DeclareDatamodelEntrytypes{standard}
\DeclareDatamodelEntryfields[standard]{type,number}
\DeclareBibliographyDriver{standard}{%
  \usebibmacro{bibindex}%
  \usebibmacro{begentry}%
  \usebibmacro{author}%
  \setunit{\labelnamepunct}\newblock
  \usebibmacro{title}%
  \newunit\newblock
  \printfield{number}%
  \setunit{\addspace}\newblock
  \printfield[parens]{type}%
  \newunit\newblock
  \usebibmacro{location+date}%
  \newunit\newblock
  \iftoggle{bbx:url}
    {\usebibmacro{url+urldate}}
    {}%
  \newunit\newblock
  \usebibmacro{addendum+pubstate}%
  \setunit{\bibpagerefpunct}\newblock
  \usebibmacro{pageref}%
  \newunit\newblock
  \usebibmacro{related}%
  \usebibmacro{finentry}}

%\titleformat{name=\section,numberless}
%  {\normalfont\Large\bfseries}
%  {}
%  {0pt}
%  {}
\date{\vspace{-3ex}}


\begin{document}

\title{\vspace{5ex}
	\includegraphics*[bb=0 0 720 200, width=0.72\textwidth]{ErstesSem/images/hu_logo.png}\\
	\vspace{30pt}
	\scshape\LARGE{Körper ohne Körper
}\\\vspace{5pt}\Large{Max Schelers Argumente gegen einen unausgedehnten Geist}\\\vspace{20pt}}
	


\author{Die Erfahrung der Realität durch Widerstand (PS)\\
	\vspace{7pt}
          Dozent: Dr. Matthias Schloßberger\\\vspace{4pt}Lennard Wolf\\
        \small{Matrikelnummer: 583052}\\
        \small{E-Mail: \href{mailto:lennard.wolf@hu-berlin.de}{lennard.wolf@hu-berlin.de}}\\
        \small{Telefonnummer: +49 176 5687 4131}\\
        \small{Studiengang: B.A. Philosophie}\\
        \small{Modul: ?}}

\maketitle

\vspace{\fill}

\begin{minipage}[]{0.92\textwidth}
    \centering
    \onehalfspacing
    \large   
    23. April 2018\\
    Wintersemester 2017/2018

    \vspace{-20mm} 
\end{minipage}%
\thispagestyle{empty}
\newpage
%\clearpage
%\thispagestyle{empty}
%\tableofcontents
%\newpage
\setcounter{page}{1}

\begin{onehalfspace} 

\noindent\textbf{$(o)$ Einleitung}

\noindent Im Abschnitt \emph{Leib und Umwelt} seines Werkes \emph{Der Formalismus in der Ethik} beschreibt Max Scheler die Beziehung der namensgebenden Begriffe und zeigt dafür zwei fundamentale Probleme für die Vorstellung vom unausgedehnten Geistes in der kartesischen Tradition auf.

In dieser Hausarbeit möchte ich diese zwei Argumente Schelers darstellen und anhand von eigenen Gedankenexperimenten veranschaulichen. Ob Schelers Interpretation des unausgedehnten Geistes hermeneutisch korrekt ist, werde ich außen vor lassen und entsprechend seine eigene Beschreibung übernehmen, an Stellen aber mit Zitaten aus Descartes Meditationen unterfüttern. 

Ich werde dafür wie folgt vorgehen. In Abschnitt $(i)$ stelle ich die Theorie vom unausgedehnten Geist im Sinne Schelers dar und gehe kurz darauf ein, inwiefern diese Vorstellungen heute weiterhin relevant sind. Es folgt in Abschnitt $(ii)$ eine Darstellung des ersten und zentralen Arguments, das die Unmöglichkeit des Kennenlernens eines \emph{eigenen} Körpers für einen unausgedehnten Geist beschreibt, sowie das dazugehörige, veranschaulichende Gedankenexperiment. Danach beschreibe ich in Abschnitt $(iii)$ das zweite Argument, in dem Scheler aufzeigt, dass Denken (?) eben doch räumlich ist und untermale dies mit einem Gedankenexperiment. Abschnitt $(iv)$ fasst die Gedanken noch einmal zusammen und bildet die Konklusion.

\vspace{5mm}
\noindent\textbf{$(i)$ Der unausgedehnte Geist}

\begin{itemize}
  \item Ich muss die 8 Punkte alle abarbeiten
\end{itemize}


\vspace{5mm}
\noindent\textbf{$(ii)$ Die Unmöglichkeit des Erlangens eines Körpers}


\newpage

\end{onehalfspace}
\nocite{*}
\printbibliography
\end{document}
