\documentclass[a4paper, 12pt]{article}
%\usepackage{CJKutf8} % japanese
\usepackage{graphicx}
\usepackage{hyperref}
\usepackage{fullpage}
%\usepackage{parskip}
\usepackage{color}
\usepackage[ngerman]{babel}
\usepackage{hyperref}
\usepackage{calc} 
\usepackage{enumitem}
\usepackage[utf8]{inputenc}
\usepackage{titlesec}
%\pagestyle{headings}
\usepackage{setspace} %halbzeilig
\usepackage[style=authoryear-ibid,natbib=true]{biblatex}
\usepackage[hang]{footmisc}
\setlength{\footnotemargin}{-0.8em}
%\bibliographystyle{natdin}
\addbibresource{scheler-hausarbeit.bib}
\DeclareDatamodelEntrytypes{standard}
\DeclareDatamodelEntryfields[standard]{type,number}
\DeclareBibliographyDriver{standard}{%
  \usebibmacro{bibindex}%
  \usebibmacro{begentry}%
  \usebibmacro{author}%
  \setunit{\labelnamepunct}\newblock
  \usebibmacro{title}%
  \newunit\newblock
  \printfield{number}%
  \setunit{\addspace}\newblock
  \printfield[parens]{type}%
  \newunit\newblock
  \usebibmacro{location+date}%
  \newunit\newblock
  \iftoggle{bbx:url}
    {\usebibmacro{url+urldate}}
    {}%
  \newunit\newblock
  \usebibmacro{addendum+pubstate}%
  \setunit{\bibpagerefpunct}\newblock
  \usebibmacro{pageref}%
  \newunit\newblock
  \usebibmacro{related}%
  \usebibmacro{finentry}}

%\titleformat{name=\section,numberless}
%  {\normalfont\Large\bfseries}
%  {}
%  {0pt}
%  {}
\date{\vspace{-3ex}}


\begin{document}

\title{\vspace{5ex}
	\includegraphics*[bb=0 0 720 200, width=0.72\textwidth]{ErstesSem/images/hu_logo.png}\\
	\vspace{30pt}
	\scshape\LARGE{Die Unmöglichkeit des Erkennens\\eines eigenen Körpers
}\\\vspace{5pt}\Large{Max Schelers Argumente gegen einen unausgedehnten Geist}\\\vspace{20pt}}
	


\author{Die Erfahrung der Realität durch Widerstand (PS)\\
	\vspace{7pt}
          Dozent: Dr. Matthias Schloßberger\\\vspace{4pt}Lennard Wolf\\
        \small{Matrikelnummer: 583052}\\
        \small{E-Mail: \href{mailto:lennard.wolf@hu-berlin.de}{lennard.wolf@hu-berlin.de}}\\
        \small{Telefonnummer: +49 176 5687 4131}\\
        \small{Studiengang: B.A. Philosophie}\\
        \small{Modul: ?}}

\maketitle

\vspace{\fill}

\begin{minipage}[]{0.92\textwidth}
    \centering
    \onehalfspacing
    \large   
    23. April 2018\\
    Wintersemester 2017/2018

    \vspace{-20mm} 
\end{minipage}%
\thispagestyle{empty}
\newpage
%\clearpage
%\thispagestyle{empty}
%\tableofcontents
%\newpage
\setcounter{page}{1}

\begin{onehalfspace} 

\noindent\textbf{$(o)$ Einleitung}

\noindent Im Abschnitt \emph{Leib und Umwelt} seines Werkes \emph{Der Formalismus in der Ethik} beschreibt Max Scheler die Beziehung der namensgebenden Begriffe und zeigt dafür zwei fundamentale Probleme für die Vorstellung vom unausgedehnten Geistes in der kartesischen Tradition auf.%noch darauf eingehen, inwiefern dies relevant ist

In dieser Hausarbeit möchte ich diese zwei Argumente Schelers darstellen und anhand von eigenen Gedankenexperimenten veranschaulichen. Ob Schelers Interpretation des unausgedehnten Geistes hermeneutisch korrekt ist, werde ich außen vor lassen und entsprechend seine eigene Beschreibung übernehmen, an Stellen aber mit Zitaten aus Descartes Meditationen unterfüttern. 

Ich werde dafür wie folgt vorgehen. In Abschnitt $(i)$ stelle ich die Theorie vom unausgedehnten Geist im Sinne Schelers dar und gehe kurz darauf ein, inwiefern diese Vorstellungen heute weiterhin relevant sind. Es folgt in Abschnitt $(ii)$ eine Darstellung des ersten und zentralen Arguments, das die Unmöglichkeit des Kennenlernens eines \emph{eigenen} Körpers für einen unausgedehnten Geist beschreibt, sowie das dazugehörige, veranschaulichende Gedankenexperiment. Danach beschreibe ich in Abschnitt $(iii)$ das zweite Argument, in dem Scheler aufzeigt, dass Denken (?) eben doch räumlich ist und untermale dies mit einem Gedankenexperiment. Abschnitt $(iv)$ geht kurz auf Schelers eigene Konzeption von Leib und Umwelt ein und in Abschnitt $(v)$ fasse ich die Gedanken noch einmal konkludierend zusammen.

\vspace{5mm}
\noindent\textbf{$(i)$ Der unausgedehnte Geist}

\begin{itemize}
  \item Unausgedehnt sein heißt, nicht in Raum und Zeit zu sein und damit auch keinen Ort oder eine räumliche Form oder eine Masse zu haben. Der Geist ist kein Körper.  
  \item Besonders auch in computationaler Kognitionswissenschaft 
  \item ähnlich wie im Hölengleichnis
  \item der geist findet ersteinmal alle dinge auf gleichförmige weise vor
  \item es gibt seelische und äußere empfindungen
  \item die äußeren emmpfindungen gelangen über einen unbekannten kanal an den geist
  \item sind emotionen bei denen äußere oder seelische zustände? 
  \item wie wird der geist durch den körper affiziert?
  \item der geist denkt nur über die ihm auf welchen wege auch immer gegebenen empfindungen nach und folgert aus ihnen über die welt - dass es die welt tatsächlich gibt und nicht nur ein dämon die ganzen empfindungen gibt, dessen kann er sich nicht sicher sein
\end{itemize}


\vspace{5mm}
\noindent\textbf{$(ii)$ Die Unmöglichkeit des Erkennens eines eigenen Körpers}

\vspace{3mm}
\noindent\textbf{$(ii.$\footnotesize$i$\normalsize$)$ Das Argument}

\begin{itemize}
  \item erkenntnistheoretisches Problem
  \item Ich muss die 8 Punkte alle abarbeiten
  \item In welchem Akt wird dann zwischen psychisch und physisch unterschieden
  \item "Dass etwas farbe ist", die Washeit oder das Wesen an sich kann nicht beobachtet oder logisch erschloßen werden (ich sehe 2 farben und schließe daraus dass ich farben sehen kann), (schriften aus dem nachlass 395), vielmehr ist das Wesen der Farbe im Wahrnehmen der Farbe mitgegeben, es handelt sich nicht um eine eigene erkenntnisleistung
  \item Der selbe gedanke den descartes selber hat: im Denken ist implizit die eigene Existenz mitgegeben
  \item im fühlen durch den Körper ist analog der Leib auch immer mitgegeben
  \item dass ich mit einem körper fühle müsste als eigene, von außen betrachtete Information wahrgenommen werden, um zu der erkenntnis zu gelangen, dass 
  \item Bedingungen der Möglichkeit, einen eigenen Körper zu erkennen: "dass es einen eigenen Körper gibt" müsste erkannt werden
  \item es müsste eine immer fester werdende zuordnung von seelischen empfindungen zu toten Körperdingen (493) geben
  \item Die Konstanz im miteinander Auftreten bestimmter empfindungen sei die reine Begründung
  \item aber das kann ja nicht ganze argument sein
\end{itemize}

\vspace{3mm}
\noindent\textbf{$(ii.$\footnotesize$ii$\normalsize$)$ Gedankenexperiment}

\begin{itemize}
  \item Menschen denen ein eigenes Körperteil fremd ist
  \item Haare sind auch immer da aber so lang mich dort nichts berührt könnten sie genau so gut weg sein
\end{itemize}


\vspace{5mm}
\noindent\textbf{$(iii)$ Der wegvoltigierte Leib}

\vspace{3mm}
\noindent\textbf{$(iii.$\footnotesize$i$\normalsize$)$ Das Argument}

\begin{itemize}
  \item Wut im Bauch haben
  \item cogitationes sind nicht ortlos!
  \item Gürtelgefühl
  \item Leib als psychophysisches Leben ist nicht vorzufinden und zu erklären , in einem auseinandergerissenen Weltbild - das psychische ist reines Bewusstsein, das physische ist Mechanik (schriften aus dem nachlass III 136)
  \item der körper bliebe ein immer fremder anderer, der dem geist (dem ich) seine wünsche und nöte kommuniziert (wer "er" auch immer sei), aber allein hilflos wäre diese zu erfüllen. der geist kümmert sich dann darum, doch könnte dies auch lassen - dass seine existenz mit dem Tod des körpers auch beendet wäre, das kann solch ein geist ohne sicherheit nicht sagen (Es ergibt sich auch trivialerweise, dass wenn der Computer, auf dem eine künstliche Intelligenz läuft, zerstört wird, diese Intelligenz auch aufhört zu existieren. ) - solch eine vorstellung liegt auch gerade dem Gedanken der unsterblichen Seele zugrunde. 
\end{itemize}

\vspace{3mm}
\noindent\textbf{$(iii.$\footnotesize$ii$\normalsize$)$ Gedankenexperiment}

\begin{itemize}
  \item 
\end{itemize}


\vspace{5mm}
\noindent\textbf{$(iv)$ Leib und Umwelt bei Scheler}

\begin{itemize}
  \item Trennung Leib und Körper
  \item Leibkörper als äußerer Teil des Leibes, Leibseele als innere Gefühle
  \item Leib und Umwelt ist nicht Trennung psychisch physisch
  \item Empfindung = f(Reiz + triebhafter Aufmerksamkeit)
  \item Leib ist zu empfdindung wie form zum gehalt (dass es farbe für uns gibt ist eigenschaft der menschlichen leiblichkeit)
  \item Leib ist Zentrum der Dinge, ich stehe räumlich gesehen immer im mittelpunkt, der ausgangspunkt des koordinatensystems (Schmitz: absoulter und relativer raum)
\end{itemize}

Mögliche Einwände?

Autotopagnosia (wenn teile des körpers nicht als eigene identifiziert werden können -> der körper ist als eigener aber schon erkannt)

\vspace{5mm}
\noindent\textbf{Stichwörter}
\begin{itemize}
  \item symbolische/asymbolische Kenntnis von etwas
  \item Fühlen (asymbolisches Erkennen) -> keine unterteilung psychisch/physisch
  \item Phänomenologie als Wissenschaft exakt dieser reinen, asymbolischen Anschauung
  \item Wie erfahre/fühle ich das objekt, das es zu erkennen gilt? Widerstand
  \item -> um etwas als körper fühlen zu können, muss ich schon einen körper haben, um mit dem objekt zu kollidieren 
  \item Apperzeption? - 
  \item Zöästhesie? - 
  \item Sinnesmodalität - Empfindungskomplexe wie Sehen, Hören, Riechen, Schmecken und Fühlen
  \item propriozeptive Wahrnehmung - Körperinneres wahrnehmen
\end{itemize}



\newpage

\end{onehalfspace}
\nocite{*}
\printbibliography
\end{document}
