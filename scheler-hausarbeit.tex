\documentclass[a4paper, 12pt]{article}
%\usepackage{CJKutf8} % japanese
\usepackage{graphicx}
\usepackage{hyperref}
\usepackage{fullpage}
%\usepackage{parskip}
\usepackage{color}
\usepackage[ngerman]{babel}
\usepackage{hyperref}
\usepackage{calc} 
\usepackage{enumitem}
\usepackage[utf8]{inputenc}
\usepackage{titlesec}
%\pagestyle{headings}
\usepackage{setspace} %halbzeilig
\usepackage[style=authoryear-ibid,natbib=true]{biblatex}
\usepackage[hang]{footmisc}
\setlength{\footnotemargin}{-0.8em}
%\bibliographystyle{natdin}
\addbibresource{scheler-hausarbeit.bib}
\DeclareDatamodelEntrytypes{standard}
\DeclareDatamodelEntryfields[standard]{type,number}
\DeclareBibliographyDriver{standard}{%
  \usebibmacro{bibindex}%
  \usebibmacro{begentry}%
  \usebibmacro{author}%
  \setunit{\labelnamepunct}\newblock
  \usebibmacro{title}%
  \newunit\newblock
  \printfield{number}%
  \setunit{\addspace}\newblock
  \printfield[parens]{type}%
  \newunit\newblock
  \usebibmacro{location+date}%
  \newunit\newblock
  \iftoggle{bbx:url}
    {\usebibmacro{url+urldate}}
    {}%
  \newunit\newblock
  \usebibmacro{addendum+pubstate}%
  \setunit{\bibpagerefpunct}\newblock
  \usebibmacro{pageref}%
  \newunit\newblock
  \usebibmacro{related}%
  \usebibmacro{finentry}}

%\titleformat{name=\section,numberless}
%  {\normalfont\Large\bfseries}
%  {}
%  {0pt}
%  {}
\date{\vspace{-3ex}}


\begin{document}

\title{\vspace{5ex}
	\includegraphics*[bb=0 0 720 200, width=0.72\textwidth]{ErstesSem/images/hu_logo.png}\\
	\vspace{30pt}
	\scshape\LARGE{Der Leib kann nicht erkannt werden
}\\\vspace{5pt}\Large{Max Schelers Argumente gegen einen unausgedehnten Geist}\\\vspace{20pt}}
	


\author{Die Erfahrung der Realität durch Widerstand (PS)\\
	\vspace{7pt}
          Dozent: Dr. Matthias Schloßberger\\\vspace{4pt}Lennard Wolf\\
        \small{Matrikelnummer: 583052}\\
        \small{E-Mail: \href{mailto:lennard.wolf@hu-berlin.de}{lennard.wolf@hu-berlin.de}}\\
        \small{Telefonnummer: +49 176 5687 4131}\\
        \small{Studiengang: B.A. Philosophie}\\
        \small{Modul: ?}}

\maketitle

\vspace{\fill}

\begin{minipage}[]{0.92\textwidth}
    \centering
    \onehalfspacing
    \large   
    23. April 2018\\
    Wintersemester 2017/2018

    \vspace{-20mm} 
\end{minipage}%
\thispagestyle{empty}
\newpage
%\clearpage
%\thispagestyle{empty}
%\tableofcontents
%\newpage
\setcounter{page}{1}

\begin{onehalfspace} 

\noindent\textbf{$(o)$ Einleitung}

\noindent Im Abschnitt \emph{Leib und Umwelt} seines Werkes \emph{Der Formalismus in der Ethik} beschreibt Max Scheler die Beziehung der namensgebenden Begriffe und zeigt dafür zwei fundamentale Probleme für die Vorstellung vom unausgedehnten Geistes in der kartesischen Tradition auf.%noch darauf eingehen, inwiefern dies relevant ist

In dieser Hausarbeit möchte ich diese zwei Argumente Schelers darstellen und anhand von eigenen Gedankenexperimenten veranschaulichen. Ob Schelers Interpretation des unausgedehnten Geistes hermeneutisch korrekt ist, werde ich außen vor lassen und entsprechend seine eigene Beschreibung übernehmen, an Stellen aber mit Zitaten aus Descartes Meditationen unterfüttern. (??) 

%DIeses Thema berührt die Grundlagen der Erkenntnistheorie, da es um die Frage geht, was den Geist das Erkennen ermöglicht.

%es geht auch um die unterscheidung des physischen vom psychischen - wie wird das psyschiche im physischen erkannt, also wie erkenne ich im gegenüber, dass es sich um ein geistiges Wesen handelt? und wie erkannte ich das erste mal, dass es eine solche unterscheidung gibt?

%klärung begriff des erkennens: durch logischen schluss (ob bewusst oder unbewusst) aus dem gegebenen das Vorliegen von etwas schon bekanntem.

% wie aber könnte das wesen des vorliegenden beim ersten erkannt werden, damit es beim zweiten mal überhaupt wiedererkannt werden kann?

%es geht darum, wie mir die welt, andere menschen, meine gefühle und meine identität gegeben sind

Ich werde dafür wie folgt vorgehen. In Abschnitt $(i)$ stelle ich die Theorie vom unausgedehnten Geist im Sinne Schelers dar und gehe kurz darauf ein, inwiefern diese Vorstellungen heute weiterhin relevant sind. Es folgt in Abschnitt $(ii)$ eine Darstellung des ersten und zentralen Arguments, das die Unmöglichkeit des Kennenlernens eines \emph{eigenen} Körpers für einen unausgedehnten Geist beschreibt, sowie das dazugehörige, veranschaulichende Gedankenexperiment. Danach beschreibe ich in Abschnitt $(iii)$ das zweite Argument, in dem Scheler aufzeigt, dass Denken (?) eben doch räumlich ist und untermale dies mit einem Gedankenexperiment. Abschnitt $(iv)$ geht kurz auf Schelers eigene Konzeption von Leib und Umwelt ein und in Abschnitt $(v)$ fasse ich die Gedanken noch einmal konkludierend zusammen. %möglicherweise nochmal schauen ob die einteilung sinn macht - beim schreiben am besten

\vspace{5mm}
\noindent\textbf{$(i)$ Der unausgedehnte Geist}

%In der Erkenntnistheorie geht es zentral um Fragen der Art, wie ein denkendes Subjekt $S$ ein beliebiges Objekt $O$ \emph{erkennen} kann. Ein solcher Erkenntnisgegenstand $O$ kann zum Beispiel ein "`Wissen wie"' ("`Wie mache ich ein Feuer?"') oder ein "`Wissen dass"' ("`Ist es der Fall, dass das Wachs gelb ist?"') sein. Wie also gelangt $S$ zu $O$? 

In der kartesischen Tradition stellt man sich Erkenntnis als etwas Mentales (?) vor. Das Mentale ist 

\begin{itemize}
  \item Unausgedehnt sein heißt, nicht in Raum und Zeit zu sein und damit auch keinen Ort oder eine räumliche Form oder eine Masse zu haben. Der Geist ist kein Körper.  
  \item Scheler bezieht sich auf Avenarius und Mach (wer sind diese) als Repräsentanten der Theorie
  \item Besonders auch in computationaler Kognitionswissenschaft 
  \item ähnlich wie im Hölengleichnis
  \item der geist findet ersteinmal alle dinge auf gleichförmige weise vor
  \item es gibt seelische und äußere empfindungen
  \item die äußeren emmpfindungen gelangen über einen unbekannten kanal an den geist
  \item sind emotionen bei denen äußere oder seelische zustände? 
  \item wie wird der geist durch den körper affiziert?
  \item der geist denkt nur über die ihm auf welchen wege auch immer gegebenen empfindungen nach und folgert aus ihnen über die welt - dass es die welt tatsächlich gibt und nicht nur ein dämon die ganzen empfindungen gibt, dessen kann er sich nicht sicher sein
  \item Ob dieses Folgern bewusst abläuft oder ob es sich um "`unbewusste Kausalschlüße"' handelt (welchergestalt diese auch immer wären) ist für die generelle Theorie ersteinmal irrelevant.
  \item metacognition
  \item innere und äußere empfindungen fallen ineinander, da es ja nur äußere Empfindungen geben kann, da der geist durch deine unausgedehntheit kein "inneres" hat (501)
  \item Selbst wenn es einen Leib gäbe, dann würde dieser immer noch in letzter Instanz die Information in den unräumlichen Geist introjezieren.
  \item Beispiel: Beim Baby, der Körper gibt signale über ein ding, das Milch gibt. dieses ding gibt auch wärme und schutz und ist somit gut. Romulus und Remus haben ebendies mit der Wölfin erfahren, denn es gibt prinzipiell keinen Unterschied, wer genau aufzieht (bei Tieren sieht man dies auch.) Wären die Bewegungen hin zur Brust reiner Reflex, oder kennt das Baby 
  \item Wie geht die Theorie mit Reflexen um? Und zwar nicht bloß muskulären Reflexen, sondern zB im Sport, "muscle memory"?
\end{itemize}


\vspace{5mm}
\noindent\textbf{$(ii)$ Die Unmöglichkeit des Erkennens eines eigenen Leibes}

\vspace{3mm}
\noindent\textbf{$(ii.$\footnotesize$i$\normalsize$)$ Das Argument}

304? 

\begin{itemize}
  \item erkenntnistheoretisches Problem
  \item Ich muss die 8 Punkte alle abarbeiten (?) -> am Ende
  \item Scheler geht auf bestimmte theorien ein, ich versuche das argument auf eine allgemeinere weise darzustellen
  \item In welchem Akt wird dann zwischen psychisch und physisch unterschieden
  \item "Dass etwas farbe ist", die Washeit oder das Wesen an sich kann nicht beobachtet oder logisch erschloßen werden (ich sehe 2 farben und schließe daraus dass ich farben sehen kann), (schriften aus dem nachlass 395), vielmehr ist das Wesen der Farbe im Wahrnehmen der Farbe mitgegeben, es handelt sich nicht um eine eigene erkenntnisleistung
  \item Der selbe gedanke den descartes selber hat: im Denken ist implizit die eigene Existenz mitgegeben
  \item im fühlen durch den Körper ist analog der Leib auch immer mitgegeben
  \item dass ich mit einem körper fühle müsste als eigene, von außen betrachtete Information wahrgenommen werden, um zu der erkenntnis zu gelangen, dass 
  \item Bedingungen der Möglichkeit, einen eigenen Körper zu erkennen: "dass es einen eigenen Körper gibt" müsste erkannt werden
  \item es müsste eine immer fester werdende zuordnung von seelischen empfindungen zu toten Körperdingen (493) geben
  \item Die Konstanz im miteinander Auftreten bestimmter empfindungen sei die reine Begründung
  \item aber das kann ja nicht ganze argument sein
  \item Das Argument ist auf Seite 500:
  \item Es wird Leib und sein Wesen Leiblichkeit in eins gesetzt - Leiblichkeit müsste erkannt werden können doch diese ist gerade die Gegebenheit der Dinge und kann nicht induktiv erkannt werden.
\end{itemize}

\vspace{3mm}
\noindent\textbf{$(ii.$\footnotesize$ii$\normalsize$)$ Gedankenexperiment}

\begin{itemize}
  \item Menschen denen ein eigenes Körperteil fremd ist
  \item Haare sind auch immer da aber so lang mich dort nichts berührt könnten sie genau so gut weg sein
  \item 
\end{itemize}


\vspace{5mm}
\noindent\textbf{$(iii)$ Der wegvoltigierte Leib}

\vspace{3mm}
\noindent\textbf{$(iii.$\footnotesize$i$\normalsize$)$ Das Argument}

\begin{itemize}
  \item Wut im Bauch haben
  \item Hier fokus wohl auf psychisch - physisch unterscheidung: wenn es wesensunterschied ist dann müssen diese ja auch unterschiedliche Weise wahrgenommen werden (503 ist das wirklich gemeint?)
  \item cogitationes sind nicht ortlos!
  \item Gürtelgefühl
  \item eingeborene Ideen (ideae innatae)
  \item Leib als psychophysisches Leben ist nicht vorzufinden und zu erklären , in einem auseinandergerissenen Weltbild - das psychische ist reines Bewusstsein, das physische ist Mechanik (schriften aus dem nachlass III 136)
  \item der körper bliebe ein immer fremder anderer, der dem geist (dem ich) seine wünsche und nöte kommuniziert (wer "er" auch immer sei), aber allein hilflos wäre diese zu erfüllen. der geist kümmert sich dann darum, doch könnte dies auch lassen - dass seine existenz mit dem Tod des körpers auch beendet wäre, das kann solch ein geist ohne sicherheit nicht sagen (Es ergibt sich auch trivialerweise, dass wenn der Computer, auf dem eine künstliche Intelligenz läuft, zerstört wird, diese Intelligenz auch aufhört zu existieren. ) - solch eine vorstellung liegt auch gerade dem Gedanken der unsterblichen Seele zugrunde. 
  \item Ich nehme deinen Geist also gar nicht wirklich wahr, ich folgere nur aus irgendeinem Grund aus dem toten Physischen dass sich dahinter irgendetwas verbirgt
\end{itemize}

\vspace{3mm}
\noindent\textbf{$(iii.$\footnotesize$ii$\normalsize$)$ Gedankenexperiment}

\begin{itemize}
  \item 
\end{itemize}


\vspace{5mm}
\noindent\textbf{$(iv)$ Leib und Umwelt bei Scheler}

\begin{itemize}
  \item Trennung Leib und Körper
  \item der leib muss ein inneres haben, ansonsten wäre er nicht ausgedehnt
  \item fokus auf reihenfolge der erkenntnisleistung
  \item Leibkörper als äußerer Teil des Leibes, Leibseele als innere Gefühle
  \item Leib und Umwelt ist nicht Trennung psychisch physisch
  \item Empfindung = f(Reiz + triebhafter Aufmerksamkeit)
  \item Leib ist zu empfdindung wie form zum gehalt (dass es farbe für uns gibt ist eigenschaft der menschlichen leiblichkeit)
  \item Leib ist Zentrum der Dinge, ich stehe räumlich gesehen immer im mittelpunkt, der ausgangspunkt des koordinatensystems (Schmitz: absoulter und relativer raum)
  \item wie kann der leib den anderen leib erkennen und als von totem gestein unterscheiden?
  \item die inneren und äußeren Empfindungen, die jeweils ineinander schon verknüpft sind (und ein gesamtbild der innen-, bzw. außenwelt geben), ergeben sie gemeinsam die Gesamte Erfahrung des Lebewesens - von sich selbst und der Umwelt. (506 ff?) "`Leibeinheit"'
  \item beispiel dass es klare trennugn gibt zwischen innerem und äußeren: äußeres ist in seinen bestimmtheiten gegeben zb als rau, glatt, fest, weich, dunkel, hell, während innere empfindungen, also schmerz kitzel etc eben keine solche gegenstände mit physischen bestimmtheiten sind (506) letztere können gar nicht undeutlich sein, im gegensatz zu ersteren
  \item 
\end{itemize}

Mögliche Einwände?

Autotopagnosia (wenn teile des körpers nicht als eigene identifiziert werden können -> der körper ist als eigener aber schon erkannt)

\vspace{5mm}
\noindent\textbf{$(v)$ Konklusion}

\begin{itemize}
  \item Der Leib ermöglicht die Wahrnehmung und damit auch die Wahrnehmung jeglicherErkenntnis.
  \item Da also, um erkennen zu können, der Leib schon gegeben sein muss, braucht und kann er gar nicht erkannt werden, er ist nur immer schon gegeben.
\end{itemize}


\vspace{5mm}
\noindent\textbf{Stichwörter}
\begin{itemize}
  \item symbolische/asymbolische Kenntnis von etwas
  \item Fühlen (asymbolisches Erkennen) -> keine unterteilung psychisch/physisch
  \item Phänomenologie als Wissenschaft exakt dieser reinen, asymbolischen Anschauung
  \item Wie erfahre/fühle ich das objekt, das es zu erkennen gilt? Widerstand
  \item -> um etwas als körper fühlen zu können, muss ich schon einen körper haben, um mit dem objekt zu kollidieren 
  \item Apperzeption? - 
  \item Zöästhesie? - 
  \item Sinnesmodalität - Empfindungskomplexe wie Sehen, Hören, Riechen, Schmecken und Fühlen
  \item propriozeptive Wahrnehmung - Körperinneres wahrnehmen
  \item Projektionstheorie der Empfndungen (wird von avenarius korrekterweise zurückgewiesen (504))
  \item Synästhesie
\end{itemize}



\newpage

\end{onehalfspace}
\nocite{*}
\printbibliography
\end{document}
