\documentclass[a4paper, 12pt]{article}
%\usepackage{CJKutf8} % japanese
\usepackage{graphicx}
\usepackage{hyperref}
\usepackage{fullpage}
%\usepackage{parskip}
\usepackage{color}
\usepackage[ngerman]{babel}
\usepackage{hyperref}
\usepackage{calc} 
\usepackage{enumitem}
\usepackage[utf8]{inputenc}
\usepackage{titlesec}
\usepackage{stmaryrd} %lightning symbol in formalisierung
%\pagestyle{headings}
\usepackage{setspace} %halbzeilig
\usepackage[style=authoryear-ibid,natbib=true]{biblatex}
\usepackage[hang]{footmisc}
\setlength{\footnotemargin}{-0.8em}
%\bibliographystyle{natdin}
\addbibresource{scheler-hausarbeit.bib}
\DeclareDatamodelEntrytypes{standard}
\DeclareDatamodelEntryfields[standard]{type,number}
\DeclareBibliographyDriver{standard}{%
  \usebibmacro{bibindex}%
  \usebibmacro{begentry}%
  \usebibmacro{author}%
  \setunit{\labelnamepunct}\newblock
  \usebibmacro{title}%
  \newunit\newblock
  \printfield{number}%
  \setunit{\addspace}\newblock
  \printfield[parens]{type}%
  \newunit\newblock
  \usebibmacro{location+date}%
  \newunit\newblock
  \iftoggle{bbx:url}
    {\usebibmacro{url+urldate}}
    {}%
  \newunit\newblock
  \usebibmacro{addendum+pubstate}%
  \setunit{\bibpagerefpunct}\newblock
  \usebibmacro{pageref}%
  \newunit\newblock
  \usebibmacro{related}%
  \usebibmacro{finentry}}

%\titleformat{name=\section,numberless}
%  {\normalfont\Large\bfseries}
%  {}
%  {0pt}
%  {}
\date{\vspace{-3ex}}


\begin{document}

\title{\vspace{5ex}
	\includegraphics*[bb=0 0 720 200, width=0.72\textwidth]{ErstesSem/images/hu_logo.png}\\
	\vspace{30pt}
	\scshape\LARGE{Max Schelers Argument gegen den unausgedehnten Geist}\\\vspace{20pt}}
	


\author{Die Erfahrung der Realität durch Widerstand (PS)\\
	\vspace{7pt}
          Dozent: Dr. Matthias Schloßberger\\\vspace{4pt}Lennard Wolf\\
        \small{Matrikelnummer: 583052}\\
        \small{E-Mail: \href{mailto:lennard.wolf@hu-berlin.de}{lennard.wolf@hu-berlin.de}}\\
        \small{Telefonnummer: +49 176 5687 4131}\\
        \small{Studiengang: B.A. Philosophie}\\
        \small{Modul: ?}}

\maketitle

\vspace{\fill}

\begin{minipage}[]{0.92\textwidth}
    \centering
    \onehalfspacing
    \large   
    23. April 2018\\
    Wintersemester 2017/2018

    \vspace{-20mm} 
\end{minipage}%
\thispagestyle{empty}
\newpage
%\clearpage
%\thispagestyle{empty}
%\tableofcontents
%\newpage
\setcounter{page}{1}

\begin{onehalfspace} 

\noindent\textbf{$(o)$ Einleitung}

\noindent Die bis heute einflussreiche, in der Neuzeit besonders durch Ren\'e Descartes populär gemachte Vorstellung, der Geist sei \emph{unausgedehnt}, wird von Max Scheler in seinen Werken immer wieder angesprochen und als \emph{proton pseudos} (falsche Prämisse) der gängigen, modernen Erkenntnistheorie entlarvt. Sie ist tief im westlichen Denken verankert und bildet die Grundlage für die Theorie, dass unsere Wirklichkeit eine Repräsentation der realen Welt sei, "`Repräsentationalismus"' genannt. Dieser Theorie nach ist die Welt, wie wir sie erfahren, also nur eine durch unsere Anschauungen gefärbte Vorstellung, nicht die reale Welt.

Im Repräsentationalismus wird Scheler zufolge eine erkenntnistheoretische \emph{Umkehrung} vollzogen. Er habe erkannt, "`daß in der natürlichen normalen Weltanschauung die \emph{vorwiegende} Täuschungsrichtung ist, wirklich Psychisches vermeintlich für physisch, nicht wirklich Physisches vermeintlich für psychisch zu halten"'\footnote{\Cite[Siehe][S. 257]{scheler-idole}.}. So werden zum Beispiel Gefühle und Emotionen statt als seelische Phänomene als rein physisch, der Raum und die Zeit statt als physische Realität zur psychischen Anschauung umgekehrt. Diese weit verbreiteten, aber an unserer Lebensrealität vorbeigehenden "`Verwirrungen"' des Repräsentationalismus können auf die falsche Prämisse des unausgedehnten Geistes zurück\-geführt werden. 

In dieser Hausarbeit stelle ich Max Schelers Argument gegen den Repräsentationalismus vor und zeige, wie dieses notwendig ein Argument gegen den unausgedehnten Geist darstellt. Da im Werk Max Schelers meiner Kenntnis nach diese Argumentation stets nur implizit vorhanden ist, möchte ich sie hiermit systematisch darstellen und somit explizit machen. Zudem gehe ich im Anschluss noch auf Folgen und Probleme des Arguments ein. 

%benutze ich wahrnehmung etc konsistent?
\vspace{3mm}

Ich werde dafür wie folgt vorgehen. In Abschnitt $(i)$ führe ich in die Vorstellung vom unausgedehnten Geist ein und zeige, dass der Repräsentationalismus eine notwendige Konsequenz dieser ist. Es folgt in Abschnitt $(ii)$ eine Darstellung des ersten Teils des Arguments, der sich mit der Möglichkeit einer ausgedehnten Repräsentation auseinandersetzt. Danach beschreibe ich in Abschnitt $(iii)$ den zweiten Teil des Arguments, der sich mit der anderen Möglichkeit, einer unausgedehnten Repräsentation auseinandersetzt. Abschnitt $(iv)$ handelt dann von Schelers eigenen Konzeption von Leib und Umwelt und in Abschnitt $(v)$ fasse ich den Gedankengang noch einmal abschließend zusammen.

%daher struktur der arbeit: ausgangslage klären, dann zeigen dass es zwei möglichkeiten gibt der auflösung, wie der geist sich seine repräsentation schafft: 1. durch auseinandefslten, wo es zu lücke kommt oder 2. dass die repräsentation 0 dimensional bleibt, das bild konsistent bleibt, aber dann doch auf die natürliche erfahrung appeliert wird und gezeigt wird, dass das so nicht sein kann (hier mein eigener input der der zeit (??)

\vspace{5mm}
\noindent\textbf{$(i)$ Der unausgedehnte Geist}

%------------------%
%    Einleitung    %
%------------------%

\noindent Im kartesisch geprägten Bild vom Menschen besteht dieser aus einem ausgedehnten, nicht denkenden Körper (\emph{res extensa}) und einem unausgedehnten, denkenden Geist (\emph{res cogitans}). Der Körper befindet sich in der räumlichen, d.h. ausgedehnten\footnote{Die Begriffe "`räumlich"' und "`ausgedehnt"' werden in diesem Text sehr ähnlich verwendet, und haben nur eine Unterscheidung im Abstraktionsgrad. Die Bewegung eines ausgedehnten Gegenstandes zum Beispiel wäre dem Autor nach besser als "`räumlich"' zu bezeichnen, da sie schwerer \emph{greifbar} ist als der als "`ausgedehnt"' bezeichnete Gegenstand selbst. "`Räumlichkeit"' ist also abstrakter zu fassen und unterstellt nicht automatisch Gegenständlichkeit.}, und zeitlichen Welt, der Geist hat auf (erst einmal) unbestimmte Weise\footnote{Von Descartes wurde hierfür zum Beispiel die Zwirbeldrüse im Gehirn als Schnittstelle von Körper und Geist "`identifiziert"'. Welcher Natur die (kausale) Interaktion von \emph{res cogitans} und \emph{res extensa} ist, sei hier aber weiter nicht relevant. Es sei jedoch erwähnt, dass diese Frage ein zentrales Problem für die substanzdualistische Theorie des Leib-Seele-Problems darstellt.} Einblick und Einfluss auf diese Welt, und diese wiederum ebenso auf ihn. Der Geist ist also, als unausgedehntes Etwas, nicht \emph{in} der körperlich wahrnehmbaren Welt, hat keine räumliche Form, eine Masse oder einen Ort, an dem er sich befindet.\footnote{Da wir spätestens seit Einstein wissen, dass Zeit und Raum nicht voneinander trennbar sind, müsste daher der Geist eigentlich auch "`außerhalb"' der Zeit sein.} Zur Verdeutlichung denke man zum Beispiel an Zahlen: Diese können zwar im Raum dargestellt werden, doch sind \emph{an sich} keine ausgedehnten Gegenstände - sie sind unausgedehnt, geistig. Während die mechanischen Bestandteile des Körpers, wie zum Beispiel das Auge oder das Bein, durch die Naturwissenschaften aufgrund von physikalischen Interaktionen beschreib- und erklärbar sind, können die mentalen Eigenschaften des Geistes, beispielsweise seine Fähigkeiten zu zweifeln, zu wollen oder zu empfinden, nicht durch die Analyse der Interaktion von ausgedehnten Gegenständen verstanden werden. Zur Veranschaulichung: Der Tisch vor mir ist ein relativ schwerer, so und so großer Gegenstand, der mit dem Mikroskop untersucht werden kann, während meine Wahrnehmung \emph{von ihm} eben kein solcher Gegenstand ist, und auch nicht von einem Mikroskop untersucht werden kann - sie ist nämlich \emph{mental}. 

Diese ontologische Unterscheidung zweier Substanzen \emph{res extensa} und \emph{res cogitans}, die Descartes vertritt, ist für unsere Untersuchung aber nicht weiter relevant. Wir können der Frage, ob der Geist nun tatsächlich aus einer anderen Substanz besteht als der Körper, agnostisch gegenüber bleiben, da auch mit einer monistischen Ontologie der Gedanke des unausgedehnten Geistes vertreten werden kann: Auf die gleiche Art wie es schwierig sein mag, einen aus vielen Vögeln emergierenden Vogelschwarm als einen Gegenstand mit festem Ort, Ausdehnung und Masse zu beschreiben, wäre es schwierig, den menschlichen Geist als etwas ausgedehntes zu bezeichnen - auch wenn man beide auf mechanische Interaktionen zurückführen könnte. Wichtig ist für uns statt der \emph{ontologischen} Frage vielmehr die \emph{erkenntnistheoretische} Frage, also wie der Geist die Gegenstände, seinen \emph{eigenen} Körper oder andere geistige Wesen in der Welt erkennen kann.

%------------------%
%    Hauptteil     %
%------------------%

Der kartesischen Tradition nach findet der Geist die Dinge in der Welt erst einmal alle auf gleiche Weise vor, nämlich durch \emph{Empfindungen}. Diese können zwar unterschiedlich "`gefärbt"' sein, doch prinzipiell gibt es keine "`inneren"' oder "`äußeren"' Empfindungen\footnote{\Cite[Vgl.][S. ?? (501?)]{scheler-ethik}.}, da der Geist kein "`Innen"' und "`Außen"' hat - jede Empfindung kann somit immer schon als "`äußere"' Empfindung bezeichnet werden. Über diese kann das \emph{cogito} dann mithilfe seiner eingeborene Ideen (\emph{ideae innatae}), wie zum Beispiel der Logik, nachdenken und auf Dinge in der Welt, sowie deren Anordnung und Mechanismen, schließen\footnote{Ob es sich immer um bewusstes Nachdenken oder "`unbewusste Kausalschlüße"' handelt, bleibt für uns erst einmal irrelevant.}. Aus den so erkannten Dingen der Außenwelt und ihren Eigenschaften entsteht für den Geist Stück für Stück ein Welt\emph{bild}, eine \emph{Repräsentation}\footnote{Im Folgenden auch austauschbar als "`\emph{Wirk}lichkeit"' bezeichnet, als Gegenstück zur \emph{Realität}, oder der Welt an sich.}. Der Tisch, der vor mit steht, ist also nicht der reale Tisch \emph{an sich}, sondern eine mentale Repräsentation von diesem, meine eigene Interpretation der Empfindungen, meine \emph{Wirklichkeit}. Vom Gegenstand an sich wird bei seiner Repräsentation sowohl abgezogen - ich bin zum Beispiel nicht in der Lage, ihn von allen Seiten gleichzeitig zu betrachten - als auch hinzugefügt - ich urteile zum Beispiel über ihn, wodurch er einen Schimmer verschiedenster Bedeutungen bekommt.

Wenn ich das Gesicht eines Freundes erblicke, passiert demnach ungefähr folgendes: Der ausgedehnte Gegenstand, den ich als sein Gesicht interpretieren werde, reflektiert Lichtstrahlen in meine Augen, wo sie wiederum elektrische Reize verursachen, die dann im Gehirn nach den komplexen Regeln der neuronalen Verknüpfungen verarbeitet werden und als neue Empfindungen an den Geist gegeben werden. Diese Informationen, in denen es nur um die Lichtverhältnisse in meiner Umgebung geht, bringe ich nun in Verbindung zu meinen bisherigen Erfahrungen und ich schlußfolgere, dass die vorliegende Anordnung von Farbtönen im Licht dadurch zustande kommt, dass mein Freund wohl anwesend ist. Ich \emph{sehe} also meinen Freund nicht direkt, sondern ich mache mir ein Bild der Situation, das eine konstruierte \emph{Repräsentation} des eigentlichen Gesichts meines Freundes beinhaltet und durch physische Reize (Lichtstrahlen/Regungen im Gehirn\footnote{\Cite[Vgl.][S. 270f.]{scheler-idole}. ???}) und meiner Ordnung dieser anhand meiner Anschauungen, eingeborenen Ideen und vorhergehenden Erfahrungen, zustande kam. Hier ist wichtig zu bemerken, dass der Geist einen unvermittelten Zugriff auf die Repräsentation der Welt haben müsste, da es ansonsten potenziell zu einem unendlichen Regress von vermittelten Vermittlungen käme.

Ob die Empfindungen, auf denen wir unsere Repräsentation aufbauen, alle nur Täu\-schungen sind - zum Beispiel wie bei Descartes von einem Dämonen veranlasst und es eine physische Realität gar nicht gibt, oder sich alles den Schattenbildern in Platons Höhle ähnlich verhält - das kann der Geist niemals anhand logischer Schlüsse erkennen, da er neben ihnen keine weiteren Informationen erhält. Alles, was er also \emph{sicher} nur wissen kann, ist, dass es ihn selbst, das denkende und empfindende Wesen, gibt. Die Welt \emph{für ihn} ist bloß ein vorgestelltes Bild davon, wie es "`da draußen"' in der realen Welt möglicherweise so "`aussieht"'.

Doch warum muss es sich so verhalten? Könnte es nicht auch sein, dass der Geist direkt mit der realen, ausgedehnten Welt im Kontakt steht und keinerlei Repräsentation benötigt? Zum einen ist es natürlich nicht zu bezweifeln, dass der Geist mit der realen Welt im Kontakt stehen muss, da er ansonsten keine Informationen über sie erhalten könnte. Aber gerade weil der Geist nur \emph{Informationen} über die Welt durch Empfindungen erhält, und nicht die Welt an sich (Information über die Beschaffenheit eines Dings ist nicht das Ding selber - der ausgedehnte Tisch kann ja auch nicht in den unausgedehnten Geist "`hineingetan"' werden), hat er es immer nur mit den Empfindungen zu tun, und nicht mit den Dingen. Diese Empfindungen bilden, wie oben beschrieben, die Welt ab: sie ergeben eine Repräsentation. \emph{Repräsentationalismus} - die Theorie, dass die Wirklichkeit eine Repräsentation ist und wir niemals Zugriff auf etwas außerhalb der Repräsentation haben - ist also eine \emph{notwendige} Konsequenz für eine Theorie des unausgedehnten Geistes.

% irgendwo, warum es plausibel ist: wie haben alle verschiedene erfahrungen der selben Dinge: Begriff qualia

%------------------%
%    Konklusion    %
%------------------%
Diesen Umstand nimmt Scheler sich zum Ausgangspunkt für die nun im Folgenden dargelegte Argumentation, die auf einer Untersuchung der Art der Gegebenheit der repräsentierten Welt beruht. Die Dinge der Welt können mir nämlich nur entweder $(a)$ auf ausgedehnte Weise oder $(b)$ nicht ausgedehnte Weise gegeben sein. Weitere Optionen sind nicht denkbar, da die Optionen die jeweilige Negation voneinander sind. Die folgenden beiden Abschnitte sollen aufzeigen, wieso sie für Scheler \emph{beide} abzulehnen sind. Wenn beide Optionen abzulehnen sind, wäre der Repräsentationalismus gänzlich abzulehnen, und da dieser eine notwendige Konsequenz der Prämisse des unausgedehnten Geistes ist, wäre folglich\footnote{\emph{Modus tollens}.} die Idee des unausgedehnten Geistes abzulehnen.

\vspace{5mm}
\noindent\textbf{$(ii)$ Option (a): Die Dinge sind für mich ausgedehnt}

%------------------%
%    Einleitung    %
%------------------%

\noindent Wenden wir uns nun zuerst Option $(a)$, dass die Dinge für mich ausgedehnt sind, zu.\footnote{Fragen zum Beispiel dazu, ob dann die gesamte Repräsentation einen eigenen großen Raum "`aufspannt"', und dergleichen lassen wir hier außen vor.} Sie ist wahrscheinlich als die intuitiv plausiblere zu bezeichnen, da die meisten der Meinung sind, dass die Dinge um sie herum eine Größe, eine Tiefe etc. haben. Schelers Ablehnung dieser Option basiert auf einem Zirkelproblem, das bei der Frage nach dem Wesen des \emph{eigenen Körpers} aufkommt.

%------------------%
%    Hauptteil     %
%------------------%

Wir beginnen mit einer ganz basalen und zentralen Frage für die Erkenntnistheorie: \emph{Was ist die} Reihenfolge der Erkenntnisleistungen, durch die sich mir die Dinge auf ausgedehnte Weise erscheinen lassen? Diese Frage wollen wir nun versuchen anhand der Konsequenzen eines unausgedehnten Geistes, der Gegenstände ausgedehnt erfahren kann, zu beantworten. 

Die Reize, die dem (ausgedehnten, \emph{realen}) Körper widerfahren - also Tastreize, Farbreize, Geräuschreize etc. - würden in einem ersten Schritt in den unausgedehnten Geist "`introjeziert"'\footnote{\Cite[Vgl.][S. ???]{scheler-ethik}.} werden. Da die Reize so zu Inhalten des Bewusstseins gemacht wurden, sind sie nun selber unausgedehnt und haben damit jegliche Räumlichkeit verloren. Basierend auf diesen unausgedehnten "`Empfindungen"' wird nun wiederum eine neue, ausgedehnte Repräsentation konstruiert, mit der \emph{direkt}\footnote{Wie oben schon angesprochen, käme es bei einem vermittelten Zugriff auf die Repräsentation womöglich zu einem unendlichen Regress. Wie es möglich sein kann, dass der unausgedehnte Geist unvermittelten Zugriff auf eine ausgedehnte Repräsentation haben kann, während dies bei der ausgedehnten Realität \emph{gerade wegen ihrer Ausgedehntheit} nicht möglich war, überlassen wir der Kontemplation des geneigten Lesers.} interagiert werden kann. In dieser Repräsentation befinden sich alle ausgedehnten Gegenstände, sowohl der Tisch, als auch der eigene Körper. Sowohl auf den Tisch, als auch auf einen eigenen Körper wird also erst \emph{a posteriori}, das heißt durch Erkenntnisse aus den Empfindungen geschlossen.\footnote{Man könnte einwänden, dass einen eigenen Körper zu haben möglicherweise \emph{a priori} als eingeborene Idee bekannt ist. Doch selbst wenn dies der Fall wäre, müsste weiterhin geklärt werden, wie der Körper aus den Empfindungen heraus in Abgrenzung zu anderen Gegenständen identifiziert werden kann.} Folglich muss in der frühen Entwicklungsphase des Menschen ein Zustand existieren, in dem der Geist Empfindungen macht, aber noch keinen eigenen Körper "`identifiziert"' hat.  

Befassen wir uns nun zunächst mit der Frage, was ein eigener Körper überhaupt ist. Wenn wir im Folgenden vom "`realen Körper"' sprechen, dann ist damit der Körper \emph{an sich} gemeint, und nicht die Vorstellung des eigenen Körpers in der Repräsentation - auf diesen wird hier stets bloß mit "`Körper"' referiert. Der reale Körper ermöglicht auf mechanische Art und Weise die Aufnahme und Weiterleitung von Informationen über die Welt um ihn herum - er ist die Schnittstelle des Geistes zur Realität. Mein Körper ist dann die Repräsentation dieses an sich rein mechanischen, toten Gegenstands. Es geht uns nun zuerst um folgende Frage: Ist mein Körper ganz einfach ein Ding wie jedes andere, oder liegt ein qualitativer Unterschied zu diesen vor? 

Wenn zum Beispiel etwas gegen meinen realen Körper drückt, dann wird mir dies in der Repräsentation durch eine Empfindung \emph{am} Körper vermittelt. Wenn etwas gegen den realen Tisch vor mir drückt, spüre ich dies nicht \emph{an} meinem Körper\footnote{Natürlich nehme ich es \emph{mit} dem Körper, zum Beispiel mit den Augen, wahr. Es geht hier aber um die Tastempfindung, die durch direkte Berührung von Gegenstand und Körper ermöglicht wird.}. Da es anscheinend ein \emph{am-eigenen-Körper-Spüren} gibt, jedoch kein \emph{am-eigenen-Tisch-Spüren}, scheint es einen Unterschied zwischen meinem Körper und anderen Dingen zu geben: Tastempfindungen . In \emph{Der Formalismus in der Ethik} wirft Scheler daher die Frage auf, was denn aber den eigenen Körper von anderen Gegenständen wie den Tisch unterscheide.\footnote{\Cite[Vgl.][S. 498]{scheler-ethik}. Im Quelltext wird (schon) von "`Leib"', statt von "`Körper"' gesprochen.} Er schreibt dazu, dass der Körper das "`Zentrum"'\footnote{\Cite[Vgl.][S. 498]{scheler-ethik}. Hier auch weiterhin der Austausch von "`Körper"' und "`Leib"'.} der wahrgenommenen Gegenstände sei, und diese damit \emph{immer} in Relation zu ihm vorgefunden werden, da "`wie nach Kants treffendem Satze das »Ich« alle unsere Erlebnisse (seelischer Art) begleiten können muß, so auch der Leib alle Organempfindungen"'\footnote{\Cite[Siehe][S. 495]{scheler-ethik}.}. Unser Körper hat natürlich auch etwas Dinghaftes an sich, und so kann ich zum Beispiel meinen Fuß, gleich dem Tisch, als einen Gegenstand betrachten. Aus dieser Doppeldeutigkeit heraus kommt an dieser Stelle die Notwendigkeit der begrifflichem Unterscheidung zwischen dem Körper als "`Körper\emph{ding}"', und dem Körper als "`Körper\emph{leib}"' zum Vorschein. Meinem Körper sind diese beiden Momente stets inne und ich kann das eine nicht ohne das andere haben: Mein Leib ist bei jeder Erfahrung präsent, die ausgedehnte Wirklichkeit wird räumlich durch den ausgedehnten Leib vermittelt. Meinen Körper \emph{als Ding} kann ich andererseits, gleich dem Tisch, zum Beispiel durch Sehen und Tasten \emph{leiblich} wahrnehmen. Diese sinnliche Wahrnehmung ist ohne den Leib undenkbar, da jener gerade ihr Medium ist. Um die Frage nach dem Unterschied nun noch einmal mit dem neuen Begriff zu beantworten: Es gibt einen qualitativen Unterschied zwischen dem Leib und den Gegenständen, aber keinen zwischen dem Körper\emph{ding} und den Gegenständen: "`Es ist irrig, der \emph{Körperleib} würde genau so wie \emph{andere} Körper ursprünglich vorgefunden"'\footnote{\Cite[Siehe][S. 494]{scheler-ethik}.}. Mein Leib ist in der ausgedehnten Wirklichkeit mein ausgedehntes Vehikel - normalerweise im Accord mit dem gegenständlichen Körper bin ich mit einem Teil meines Leibes unter dem Tisch, mit einem anderen Teil über dem Tisch. 

%------------------%
%    Konklusion    %
%------------------%

Da der Leib nicht Teil des Geistes ist\footnote{\Cite[Vgl.][S. 499]{scheler-ethik}.}, sondern Teil der Repräsentation, muss der Geist auf diesen erst \emph{a posteriori}, d.h. durch Erfahrungen aus der Repräsentation, schließen. Hier zeichnet sich nun ein Zirkelschluss ab: Da mir alle Dinge in der Repräsentation immer nur anhand des Leibes vermittelt sind, so muss bei jeder Erfahrung der Leib schon vorhanden sein. Da ich auf den Leib aber erst durch Erfahrungen schließen muss, wird dies unmöglich: \emph{Ich brauche den Leib, um auf den Leib zu schließen}. Die Vorstellung, dass mein Leib, mein Wahrnehmungsvehikel, ursprünglich unausgedehnt gewesen sei, lässt aus diesem Grund keine Erklärung zu, durch welche Erkenntnisleistungen oder sonst welcher Handlungen der Leib seine Ausdehnung erhält. Wenn der Leib, als Repräsentation des verkörperten Seins in der Realität, 

Hier möchte ich noch einmal kurz auf zwei Gedanken hinweisen, die Scheler mit seiner Argumentation ablehnen möchte. Scheler stellt die Vorstellung des kartesischen Repräsentationalismus so dar, dass nach diesem der eigene Körper dadurch als etwas Besonderes erkannt wird, dass er als Gegenstand immerzu vorhanden ist. Dies verwirft er nicht nur durch das Aufweisen des Zirkelschlusses, sondern auch, indem er darauf aufmerksam macht, dass eine Person, die sich ihr Leben lang in einer Gefängniszelle befindet, diese trotzdem nicht plötzlich mit dem eigenen Körper verwechseln würde.\footnote{\Cite[Vgl.][S. 494]{scheler-ethik}.} Dass der neugeborene Mensch erst noch lernen muss, seinen Körper richtig zu \emph{koordinieren}, ist desweiteren auch kein gültiges Gegenargument, da es vielmehr eher nur aufzeigt, dass die \emph{Möglichkeit} des Bewegens durch "`die Unterscheidung der Sphären »Leib« und »Außenwelt« [...] hierbei längst \emph{vorausgesetzt}"'\footnote{\Cite[Siehe][S. 496]{scheler-ethik}.} sein muss.\footnote{\Cite[Vgl. zudem auch][S. 492]{scheler-ethik}.} 


\vspace{5mm}
\noindent\textbf{$(iii)$ Option (b): Die Dinge sind für mich unausgedehnt}

%------------------%
%    Einleitung    %
%------------------%

\noindent Kommen wir nun zu Option (b), in der dem unausgedehnten Geist die Gegenstände der Welt unausgedehnt erscheinen. Das Argument Schelers zeigt in diesem Fall nicht, wie im Vorhergegangenen, ein \emph{logisches} Problem auf, sondern wie Fern diese Option von unserer alltäglichen Erfahrung ist. Er will also veranschaulichen, wie zutiefst \emph{unplausibel} sie ist.

%------------------%
%    Hauptteil     %
%------------------%

Vergegenwärtigen wir uns zuerst, was eine Wirklichkeit der unausgedehnten Dinge eigentlich bedeuten würde. Dafür müssen wir aber zuerst festhalten, dass der Begriff "`Räumlichkeit"' trotzdem etwas für uns bedeutet. Da \emph{tatsächliche} Räumlichkeit als Weise der Repräsentation der realen räumlichen Anordnung (wenn es diese denn gibt)\footnote{Dazu noch mehr in Abschnitt $(iv)$.} aber wegfällt, habe ich Zugriff auf Informationen zu dieser Anordnung nur durch nichträumliche Abstraktionen oder Abbildungen von ihnen. Dass sich etwas hinter mir befindet, müsste sich mir daher auf eine "`unräumliche"' Weise darstellen, ebenso wie räumliche Eigenschaften von Gegenständen, zum Beispiel Größe oder Oberflächenbeschaffenheit. Ein Schmerz kann sich für mich auch nicht im räumlichen Sinne \emph{ausbreiten}, ich kann vielmehr nur auf \emph{abstrahierte} Weise erfahren, dass der Schmerz eine besonders geartete Transformation durchmacht, die "`Ausbreitung"' genannt wird. Dass dies nichts mit "`echter"', räumlicher Ausbreitung zu tun haben kann, ergibt sich. Da Räumlichkeit für uns in dieser Option nur auf unausgedehnte Weise gegeben ist, bleibt uns ihr tatsächliches Wesen vollkommen unbekannt. Sie wäre für uns nicht fundamental verschieden von jeder anderen Form der Repräsentation, da diese jeweils alle ebenso unausgedehnt sind, und ebendies allesamt gemein haben. Bei Option (b) wäre Ausgedehntheit für uns also überhaupt keine Ausgedehntheit. Sie wäre keine eigenständige Weise des Erlebens und Vorstellens, sondern vielmehr eine Repräsentation einer fundamental andersartigen Anordnung von Dingen (der eigentlichen Ausgedehntheit).

Scheler appelliert nun an unsere alltägliche Erfahrung: Das Wesen der Räumlichkeit, der Ausgedehntheit, ist doch aber gerade so fundamental verschieden von allen anderen Anschauungen die wir haben, da sie die Unausgedehntheit mit ihnen eben \emph{nicht} gemein hat. Dass der Tisch \emph{vor mir} steht, ist zwar als abstraktes Faktum zu denken, und dafür bedarf es auch keiner räumlichen Erfahrung, aber die Erfahrung selber, und auch die leibliche Vorstellung dieses Zustands, \emph{ist} räumlich. Wäre dem nicht so, kämen wir gar nicht auf die Unterscheidung zwischen ausgedehnt und unausgedehnt. Unsere Vorstellung von Ausgedehntheit kommt doch gerade daher, dass wir meinen, mit ausgedehnten Dingen zu interagieren und Empfindungen \emph{verorten} zu können. Wie verwenden den Begriff nicht auf eine vermittelnde Art und Weise in dem Wissen, dass sich dahinter eigentlich etwas fundamental anderes verbirgt.\footnote{Auch wenn uns manche dies weismachen wollen.} Wir meinen mit ihm das, was er aussagt. Dass sich der Schmerz in unserem Körper ausbreitet, ist für uns keine abstrakte Information, so wie zum Beispiel eine Zahlenfolge, sondern ist uns als \emph{räumliches Erlebnis des Ausbreitens} gegeben. Scheler betont, dass es irrig wäre, anderes zu behaupten.\footnote{"`Es ist irrig, die \emph{Leibsensationen} seien »seelische Phänomene«."' \Citep[Siehe][S. 494]{scheler-ethik}.} 

%------------------%
%    Konklusion    %
%------------------%

Wie es in der Phänomenologie üblich ist, basiert dieses Argument zuallererst auf der Rück\-be\-sinnung auf die Phänomene, die wir erleben, beziehungsweise, wie wir sie erleben. Anstatt eine Ansammlung abstraktem Wissens, das aus Ergebnissen logischer (Fehl-)Schlüsse zusammengesetzt ist, als Ausgangspunkt zu nehmen und dann bei Widersprüchen mit unserer Alltagserfahrung einfach letztere als offensichtliche Täuschung hinzustellen, solle von dem ausgegangen werden, von dem wir uns sicher sein können, dass es existiert - das erlebte Phänomen. Die Oberflächen von Gegenständen, unsere eigene Körpergröße im Verhältnis zu anderen Dingen, Schmerzen, die Position meiner Hand im Verhältnis zum meinem Kopf - all dies ist mir räumlich gegeben, ich weiß von ihnen nicht erst in einem zweiten Schritt als abstrakte Fakten, nachdem ich unausgedehnte Daten interpretiert habe. Dieses Argument ist zum einen elegant in seiner Einfachheit, doch daher auch auf den ersten Blick leichter angreifbar als jenes aus dem vorigen Abschnitt. Eine Gegenposition könnte aber nur schwer die Erfahrung, auf die sich Scheler stützt, von der Hand weisen. Sie müsste daher den Widerspruch zwischen unserer Erfahrung und einer unausgedehnten Repräsentation auflösen können, was wohl alles andere als trivial wäre. 

\vspace{5mm}
\noindent\textbf{$(iv)$ Mögliche Einwände und Konsequenzen} 

%------------------%
%    Einleitung    %
%------------------%

\noindent Wir haben nun anhand Schelers Gedanken gezeigt, dass unter der Prämisse des unausgedehnten Geistes eine Repräsentation mit ausgedehnten Dingen zu Zirkularität führt, und eine Repräsentation mit unausgedehnten Dingen unplausibel ist. Ich werde mich nun im Folgenden zuerst mit ein paar wenigen, möglichen Einwänden zu diesem Argument befassen und dann näher auf Schelers alternative Vorstellung von der Reihenfolge der Erkenntnisleistungen eingehen. Dadurch werde ich einschlägige Konsequenzen aufzeigen, die seine Erkenntnistheorie scharf von jenen, die auf einem Repräsentationalismus basieren, trennen.

%------------------%
%    Hauptteil     %
%------------------%

\begin{itemize}
  \item Reihenfolge der Erkenntnisleistung beschreiben?
  \item Einwand dass der Leib schon besteht aber erweitert wird: Selbst wenn der Leib schon ausgedehnt ist und nur kleiner, löst sich der Zirkel dadurch nicht auf. Ein "`Anwachsen"' eines Leibes bliebe also weiter unsinnig.
  \item Noch einmal auf die umkehrung des physischen und psychischen eingehen 
  \item Doppelempfindugen 493 ?
  \item auf das schon angedeutete Problem, dass Räumlichkeit möglicherweise gar nicht \emph{real} existiert, eingegangen werden. Denn wenn Räumlichkeit tatsächlich nur eine 
\end{itemize}

Es sei nun angemerkt, dass eine Theorie vom ausgedehnten Geist zu allererst vor dem hier vorgebrachten Argument geschützt werden müsste. 


%------------------%
%    Konklusion    %
%------------------%



\vspace{5mm}
\noindent\textbf{$(v)$ Konklusion}

\noindent Rekapitulieren wir noch einmal den Gedankengang\footnote{Im Anhang befindet sich eine formalisierte Skizze der Argumentstruktur.}. Wenn der Geist unausgedehnt ist, folgt notwendig, dass die Wirklichkeit für ihn eine \emph{Repräsentation} der ausgedehnten Realität ist. Diese Repräsentation kann nur entweder ausgedehnt oder nicht ausgedehnt sein. Sie kann nicht ausgedehnt sein, da der Leib, das ausgedehnte Vehikel der Wahrnehmung in der Wirklichkeit, nicht erkannt werden kann, ohne dass er schon vorausgesetzt wäre. Dass die Repräsentation unausgedehnt sei, ist zutiefst unplausibel, da dies nicht mit unserer alltäglichen Erfahrung übereinstimmt und unser Gerede von Ausgedehntheit unsinnig macht. Da die den gesamten Möglichkeitsraum ausmachenden Optionen abzulehnen sind, ist allgemein die Idee der Repräsentation, beziehungsweise damit auch die gesamte Theorie des \emph{Repräsentationalismus} abzulehnen. Da diese eine notwendige Folge für den unausgedehnten Geistes ist, ist der unausgedehnte Geist schließlich abzulehnen.



%--------------------------------------------------------------------------------
%%---------------MÜLL------------------------------------------------------------
%--------------------------------------------------------------------------------

%Ob Schelers Interpretation des unausgedehnten Geistes hermeneutisch korrekt ist, werde ich außen vor lassen und entsprechend seine eigene Beschreibung übernehmen, an Stellen aber mit Zitaten aus Descartes Meditationen unterfüttern. (??) 

%Abschnitt \emph{Leib und Umwelt} seines Werkes \emph{Der Formalismus in der Ethik}

%klärung begriff des erkennens: durch logischen schluss (ob bewusst oder unbewusst) aus dem gegebenen das Vorliegen von etwas schon bekanntem.

%\vspace{3mm}
%\noindent\textbf{$(iii.$\footnotesize$i$\normalsize$)$ Das Argument}

%  \item wie aber könnte das wesen des vorliegenden beim ersten erkannt werden, damit es beim zweiten mal überhaupt wiedererkannt werden kann?

%In der Erkenntnistheorie geht es zentral um Fragen der Art, wie ein denkendes Subjekt $S$ ein beliebiges Objekt $O$ \emph{erkennen} kann. Ein solcher Erkenntnisgegenstand $O$ kann zum Beispiel ein "`Wissen wie"' ("`Wie mache ich ein Feuer?"') oder ein "`Wissen dass"' ("`Ist es der Fall, dass das Wachs gelb ist?"') sein. Wie also gelangt $S$ zu $O$? 

%noch darauf eingehen, inwiefern dies relevant ist - Besonders auch in computationaler Kognitionswissenschaft 

%  \item Scheler bezieht sich auf Avenarius und Mach (wer sind diese) als Repräsentanten der Theorie: 

% Beispiel: Beim Baby, der Körper gibt signale über ein ding, das Milch gibt. dieses ding gibt auch wärme und schutz und ist somit gut. Romulus und Remus haben ebendies mit der Wölfin erfahren, denn es gibt prinzipiell keinen Unterschied, wer genau aufzieht (bei Tieren sieht man dies auch.) Wären die Bewegungen hin zur Brust reiner Reflex, oder kennt das Baby 

% Die Welt \emph{für uns} ist für den Idealismus zunächst nur Vorstellung, während die \emph{Außen}welt "`real"' und "`wirklich"' sei.\footnote{\Cite[Vgl.][S. 257]{scheler-idole}}

%Hier fokus wohl auch noch auf psychisch - physisch unterscheidung: wenn es wesensunterschied ist dann müssen diese ja auch unterschiedliche Weise wahrgenommen werden (503 ist das wirklich gemeint?)

% der körper bliebe ein immer fremder anderer, der dem geist (dem ich) seine wünsche und nöte kommuniziert (wer "er" auch immer sei), aber allein hilflos wäre diese zu erfüllen. der geist kümmert sich dann darum, doch könnte dies auch lassen - dass seine existenz mit dem Tod des körpers auch beendet wäre, das kann solch ein geist ohne sicherheit nicht sagen (Es ergibt sich auch trivialerweise, dass wenn der Computer, auf dem eine künstliche Intelligenz läuft, zerstört wird, diese Intelligenz auch aufhört zu existieren. ) - solch eine vorstellung liegt auch gerade dem Gedanken der unsterblichen Seele zugrunde. 

% begriff der introjektion: das auseinander des Raumes und der Zeit wird ineinander gefaltet und dann in den geist introjeziert (?). dieser empfindet das dann alles und baut sich durch bewusste oder unbewusste schlußverfahren eine repräsentation der gegebenen empfindung auf, die entweder mental und damit unräumlich bleibt, wobei dann auch kein zeitempfinden erlaubt wäre, oder wieder ausgefaltet wird, wobei fraglich bleibt, wie der geist darauf kommt, eine raumzeitliche repräsentation zu erschaffen, da die information des Wesens der Raumzeit durch die introjektion der unausgedehnten Empfindung ihn nie erreicht hat.

% Da der Geist in sich unabhängig ist von den körperlichen Gegebenheiten, reicht er am Ende auch als eine Unterscheidung des Menschen vom Tier, welches als ungeistig und damit als reiner Automat gesehen werden muss.

%   \item Ich nehme deinen Geist also gar nicht wirklich wahr, ich folgere nur aus irgendeinem Grund aus dem toten Physischen dass sich dahinter irgendetwas verbirgt

%  \item Ein mögliches Gegenargument wäre, dass beide Optionen nicht stimmen, und die Welt, wie sie uns repräsentiert wird, mit ihrer mehrdimensionalität tatsächlich wenigerdimensional ist als der geist in dem sinne, dass die Repräsentation der Empfindungen in Raum und Zeit für den Geist sehr einfach ist und möglicherweise auch nichts mit der "`echten"' Realität zu tun haben - dass also Raum und Zeit in keinster Weise mit der Realität zu tun haben sondern eben nur Anschauungen des Geistes sind.

%Da das Wesen der Ausgedehntheit in den Empfindungen nicht mitgegeben ist, muss der Geist schon vorher mit ihm vertraut sein. Das heißt um die unausgedehnten Informationen räumlich vorzustellen, muss die Anschauung der Räumlichkeit \emph{a priori} bekannt sein.

%--------------------------------------------------------------------------------
%%---------------FRAGEN AN SCHLOßBERGER------------------------------------------
%--------------------------------------------------------------------------------


%wenn aber nun der geist ausgedehnt ist, heißt dies ja nicht dass er ein messbares ding in der physischen welt ist - inwiefern ist belegbar, dass die kartesische vorstellung nicht am ende zustimmen würde? wo ist die klare trennung zu der kartesischen vorstellung?

%ist res extensa denn eine vorstellung des geistes oder nicht?

%Weltbild = Repräsentation?

%--------------------------------------------------------------------------------

\newpage

\end{onehalfspace}
\nocite{*}
\printbibliography

\newpage
\noindent\textbf{Anhang}
\vspace{6pt}

\noindent Das Folgende ist eine einfache Formalisierung des allgemeinen Gedankengangs, der zur Ablehnung des unausgedehnten Geistes führt. Hier wird die \emph{reductio ad absurdum} gekürzt, die bei deutlicherer Ausführung, wie sie im Text oben vorzufinden ist, im Rückschluss zwei Instanzen von \emph{modus tollens} beinhaltet. 

\vspace{14pt}

\noindent Zu zeigen, dass der Geist nicht unausgedehnt ist:

\vspace{4pt}
($\circ$) \hspace*{1em} $\neg U$

\vspace{4pt}
\noindent Angenommen zur \emph{reductio}, der Geist sei unausgedehnt:

\vspace{4pt}
(1) \hspace*{1em} $U$

\vspace{4pt}
\noindent Aus dem unausgedehnten Geist folgt notwendig der Repräsentationalismus:  

\vspace{4pt}
(2) \hspace*{1em}  $U \rightarrow R$

\vspace{4pt}
\noindent \emph{Modus Ponens} aus (1) und (2):

\vspace{4pt}
(3) \hspace*{1em} $R$

\vspace{4pt}
\noindent Aus dem Repräsentationalismus folgt notwendig eine Kontravalenz aus ausgedehnter Rep. und unausgedehnter Rep.:

\vspace{4pt}
(4) \hspace*{1em} $R \rightarrow (R1 \oplus R2)$


\vspace{4pt}
\noindent \emph{Modus Ponens} aus (3) und (4):

\vspace{4pt}
(5) \hspace*{1em} $R1 \oplus R2$

\vspace{4pt}

\noindent Es gilt aus der Definition der Kontravalenz: 

\vspace{4pt}
(6) \hspace*{1em}  $(\neg R1 \land \neg R2)  \rightarrow \neg (R1 \oplus R2)$
\vspace{4pt}

\noindent Wir haben argumentativ gezeigt (in der Formalisierung ausgelassen), dass mit der Prämisse $U$ sowohl ausgedehnte Rep., als auch unausgedehnte Rep. abzulehnen ist: 

\vspace{4pt}
(7) \hspace*{1em}  $\neg R1 \land \neg R2$

\vspace{4pt}
\noindent \emph{Modus Ponens} aus (6) und (7):

\vspace{4pt}
(8) \hspace*{1em} $\neg (R1 \oplus R2)$


\vspace{14pt}
\noindent (5) $\lightning$ (8).  Da die Annahme zur \emph{reductio} in einen Widerspruch führt, ist ($\circ$) gezeigt.



\end{document}
