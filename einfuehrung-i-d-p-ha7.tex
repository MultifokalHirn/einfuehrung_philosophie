\documentclass[a4paper]{article}
\usepackage{graphicx}
\usepackage{fullpage}
%\usepackage{parskip}
\usepackage{color}
\usepackage[ngerman]{babel}
\usepackage{hyperref}
\usepackage{calc} 
\usepackage{enumitem}
\usepackage{titlesec}
\usepackage[export]{adjustbox}
%\pagestyle{headings}

\titleformat{name=\section,numberless}
  {\normalfont\Large\bfseries}
  {}
  {0pt}
  {}
\date{\vspace{-3ex}}
\begin{document}

\title{
    \vspace{-30pt}
	\includegraphics*[width=0.1\textwidth,left]{images/hu_logo2.png}\\
	\vspace{-10pt}
	Einf"uhrung in die Philosophie}
\author{Lennard Wolf\\
        \small{\href{mailto:lennard.wolf@student.hu-berlin.de}{lennard.wolf@student.hu-berlin.de}}}
\maketitle
\vspace{0pt}

\section*{AB 7: Genaue Textlekt"ure}
\large

%\vspace{10pt}
\noindent\textbf{(ii)}

\begin{enumerate}
  \item Sein Vater ist der Ehemann seiner Mutter.
  \item Darth Vader ist der Vater von Luke Skywalker.
  \item Der CEO von Tesla ist der Gr"under von SpaceX.
\end{enumerate}

Mit "`Beziehung"' ist hier wahrscheinlich etwas wie Relation im mathematischen Sinne gemeint, also dass die beiden in der Relation stehenden Objekte in einem bestimmten Zusammenhang stehen, sei es durch gemeinsame Eigenschaften, Verhalten etc. Eine Beziehungsfunktion k"onnte in die Menge aller Tupel, die in dieser Relation stehen, schauen und immer einen Wahrheitswert ausgeben, ob diese Beziehung besteht oder nicht. In dem Sinne betrachtet w"urde man Freges Frage auch mit "`Ja"' beantworten.\\

\begin{itemize}
  \item Synthetisch wahr: Der CEO von Tesla ist der Gr"under von SpaceX.
  \item Analytisch wahr: Die Frau die ihn gezeugt hat ist seine Mutter.
\end{itemize}

Die Zeichen die wir benutzen sind nicht immer nur arbitr"are Namen ("`Peter"'), sondern h"aufig auch zusammengesetzte Begriffe ("`der Sohn von Peter"'). Diese zusammengesetzten Begriffe beinhalten Informationen, die ein einfacher Name nicht enth"alt, daher liegt es f"ur Frege nahe, diese Ebene des "`Sinns"' auch zu betrachten.



\end{document}
