\documentclass[a4paper, 12pt]{article}
%\usepackage{CJKutf8} % japanese
\usepackage{graphicx}
\usepackage{hyperref}
\usepackage{fullpage}
%\usepackage{parskip}
\usepackage{color}
\usepackage[ngerman]{babel}
\usepackage{hyperref}
\usepackage{calc} 
\usepackage{enumitem}
\usepackage[utf8]{inputenc}
\usepackage{titlesec}
%\pagestyle{headings}
\usepackage{setspace} %halbzeilig
\usepackage[style=authoryear-ibid,natbib=true]{biblatex}
\usepackage[hang]{footmisc}
\setlength{\footnotemargin}{-0.8em}
%\bibliographystyle{natdin}
\addbibresource{debatten-ha5.bib}
\DeclareDatamodelEntrytypes{standard}
\DeclareDatamodelEntryfields[standard]{type,number}
\DeclareBibliographyDriver{standard}{%
  \usebibmacro{bibindex}%
  \usebibmacro{begentry}%
  \usebibmacro{author}%
  \setunit{\labelnamepunct}\newblock
  \usebibmacro{title}%
  \newunit\newblock
  \printfield{number}%
  \setunit{\addspace}\newblock
  \printfield[parens]{type}%
  \newunit\newblock
  \usebibmacro{location+date}%
  \newunit\newblock
  \iftoggle{bbx:url}
    {\usebibmacro{url+urldate}}
    {}%
  \newunit\newblock
  \usebibmacro{addendum+pubstate}%
  \setunit{\bibpagerefpunct}\newblock
  \usebibmacro{pageref}%
  \newunit\newblock
  \usebibmacro{related}%
  \usebibmacro{finentry}}

%\titleformat{name=\section,numberless}
%  {\normalfont\Large\bfseries}
%  {}
%  {0pt}
%  {}
\date{\vspace{-3ex}}
\begin{document}

\title{\vspace{5ex}
	\includegraphics*[bb=0 0 720 200, width=0.72\textwidth]{ErstesSem/images/hu_logo.png}\\
	\vspace{30pt}
	\scshape\LARGE{Zusammenfassung V}\\\Large{Die letzte Welle}\vspace{20pt}}
	


\author{Regionalwissenschaftliche Debatten\\
	\vspace{7pt}
          Dozent: Prof. Dr. phil. Vincent Houben\\\vspace{4pt}Lennard Wolf\\
        \small{Matrikelnummer: 583052}\\
        \small{E-Mail: lennard.wolf@hu-berlin.de}}

        %\href{mailto:lennard.wolf@student.hu-berlin.de}{lennard.wolf@student.hu-berlin.de}}}      

\maketitle

\vspace{\fill}

\begin{minipage}[]{0.92\textwidth}
    \centering
    \onehalfspacing
    \large   
    29. Januar 2018\\
    Wintersemester 17/18

    \vspace{-20mm} 
\end{minipage}%
\thispagestyle{empty}
\newpage
%\clearpage
%\thispagestyle{empty}
%\tableofcontents
%\newpage
\setcounter{page}{1}

\begin{onehalfspace} 

%\noindent\textbf{Zusammenfassung}

% Überlegen Sie sich Zwischenüberschriften zu den einzelnen Abschnitten des Textes!

% Erarbeiten Sie sich aus jedem Abschnitt zwei Kernaussagen!

% Versuchen Sie, eine Definition des Konzeptes "Orientalismus" zu formulieren!

\noindent 
"`Die letzte Welle"' ist ein Kapitel aus dem erstmals 1983 herausgebrachten Buch \emph{Die Erfindung der Nation} von Benedict Anderson, in dem die sogenannte "`letzte Welle"' an Nationalstaatsentstehungen nach dem Zweiten Weltkrieg thematisiert wird.

Mit dem Ersten Weltkrieg ging die Zeit der Dynastien zu Ende und nach dem Zweiten Weltkrieg kam die letzte Welle der Nationalstaaten. Grundlage für diese neuen Strukturen waren der Nationalismus der Kolonialmächte, sowie die jeweiligen Propaganda- und Bildungssysteme. Die Grenzziehung verlief entlang den einstigen Verwaltungseinheiten der Kolonialreiche und das jeweilige Nationalgefühl wurde durch Reisen und die Zweisprachichkeit der jungen Menschen der Region bestärkt. Das Selbstverständnis von Nationen wird meist historisch begründet, zum Teil auch mit Übertreibungen und Hinzudichtungen. Anderson unterscheidet zwischen dem europäischen \emph{Volksnationalismus} und dem amerikanischen \emph{Offiziellen Nationalismus}.

Besonders die neuen Formen der Bildungssysteme, die zentral und hierarchisch von den neuen Staaten organisiert wurden, spielten auf zweierlei Weise eine Kernrolle, was der Autor am Beispiel Indochinas (europäisches Modell) verdeutlicht. Erstens wurde das alte Bildungssystem abgeschafft und so die bestehende philosophische, regionale und politische Selbsteinordnung erneuert und zweitens sollte eine indigene Elite hervorgebracht werden, die die Bürokratie leiten würde. 

Diese der Russifizierung ähnelnden Entwicklungen durch die Übernahme der Kolonialmachtskulturen und Bürokratiestrukturen, die Staatsbürgerschaft und die "`Züchtung"' der indigenen Elite für die Vorstandszimmer passten die Regionen dem globalen Kapitalismus an, wodurch sie sich in diesen einordnen, und an ihm teilhaben konnten.





%Sie bleibt ja aber offensichtlich vollkommen in der Moderne hängen (was ja nicht schlimm ist)
%Moderne ist ein edlritch das über sich selbst nachdenkt, aber ultimative angst vor seinem tod hat


\end{onehalfspace}
\nocite{*}
%\bibliography{merleau-ponty-essay}
\printbibliography
\end{document}
