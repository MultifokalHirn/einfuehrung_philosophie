\documentclass[a4paper, emulatestandardclasses, 12pt]{scrartcl}
\usepackage{graphicx}
\usepackage{hyperref}
\usepackage{fullpage}
%\usepackage{parskip}
\usepackage{color}
\usepackage[ngerman]{babel}
\usepackage{hyperref}
\usepackage{calc} 
\usepackage{enumitem}
\usepackage{titlesec}
%\pagestyle{headings}
\usepackage{setspace} %halbzeilig
\usepackage[authoryear,round]{natbib}
\bibliographystyle{natdin}

%\titleformat{name=\section,numberless}
%  {\normalfont\Large\bfseries}
%  {}
%  {0pt}
%  {}
\date{\vspace{-3ex}}
\begin{document}

\title{\vspace{5ex}
	\includegraphics*[width=0.72\textwidth]{ErstesSem/images/hu_logo.png}\\
	\vspace{30pt}
	\scshape\LARGE{Das Wort selbst ist Träger des Sinnes\\Zu Merleau-Pontys Verständnis der Sprache}}
	
	\subtitle{\vspace{20pt}Tutorium zu Aristoteles: Nikomachische Ethik (VEV)\\
	\vspace{6pt}
          Tutorin: Larissa Gniffke}


\author{\vspace{-4pt}Lennard Wolf\\
        \small{\href{mailto:lennard.wolf@student.hu-berlin.de}{lennard.wolf@student.hu-berlin.de}}}      

\maketitle

\vspace{\fill}

\begin{minipage}[b]{\textwidth}
    \centering
    \onehalfspacing
    \large   
    11. Juli 2017\\
    Sommersemester 2016

    \vspace{-20mm} 
\end{minipage}%
\thispagestyle{empty}
\newpage
%\clearpage
%\thispagestyle{empty}
%\tableofcontents
%\newpage
\setcounter{page}{1}

\begin{onehalfspace} 



\noindent\textbf{$(o)$ Einleitung}

\noindent 

\vspace{5mm}

\noindent\textbf{$(i)$ "`Schwierigkeit, wie wir gut werden k"onnen, ohne es schon zu sein"'}

\noindent 

\vspace{5mm}
\noindent\textbf{$(ii)$ Tugendhafte Handlung}	

\noindent 


\vspace{5mm}
\noindent\textbf{$(iii)$ Versuch einer L"osung}	

\noindent 


\vspace{5mm}
%in II 2 wird die Gewöhnung erklärt

%\vspace{3mm}
%\begin{addmargin}[.065\linewidth]{.065\linewidth}% indent 0pt left, .5\linewidth right
%\footnotesize
%
%\noindent \emph{"`What relation must one fact (such as a sentence) have to another in order to be capable of being a symbol for that other? This last is a logical question, and is the one with which Mr Wittgenstein is concerned. He is concerned with the conditions for accurate Symbolism, i.e. for Symbolism in which a sentence `means' something quite definite."'} -- Bertrand Russell \cite[S. 7]{wittgenstein1922tractatus}
%
%\end{addmargin}
%\normalsize
\vspace{3mm}

\end{onehalfspace}
\nocite{*}
\bibliography{aristoteles-essay}

\end{document}
