\documentclass[a4paper, 12pt]{article}
\usepackage{CJKutf8} % japanese
\usepackage{graphicx}
\usepackage{hyperref}
\usepackage{fullpage}
%\usepackage{parskip}
\usepackage{color}
\usepackage[ngerman]{babel}
\usepackage{hyperref}
\usepackage{calc} 
\usepackage{enumitem}
\usepackage[utf8]{inputenc}
\usepackage{titlesec}
%\pagestyle{headings}
\usepackage{setspace} %halbzeilig
\usepackage[style=authoryear-ibid,natbib=true]{biblatex}
%\bibliographystyle{natdin}
\addbibresource{merleau-ponty-essay.bib}
\DeclareDatamodelEntrytypes{standard}
\DeclareDatamodelEntryfields[standard]{type,number}
\DeclareBibliographyDriver{standard}{%
  \usebibmacro{bibindex}%
  \usebibmacro{begentry}%
  \usebibmacro{author}%
  \setunit{\labelnamepunct}\newblock
  \usebibmacro{title}%
  \newunit\newblock
  \printfield{number}%
  \setunit{\addspace}\newblock
  \printfield[parens]{type}%
  \newunit\newblock
  \usebibmacro{location+date}%
  \newunit\newblock
  \iftoggle{bbx:url}
    {\usebibmacro{url+urldate}}
    {}%
  \newunit\newblock
  \usebibmacro{addendum+pubstate}%
  \setunit{\bibpagerefpunct}\newblock
  \usebibmacro{pageref}%
  \newunit\newblock
  \usebibmacro{related}%
  \usebibmacro{finentry}}


%\titleformat{name=\section,numberless}
%  {\normalfont\Large\bfseries}
%  {}
%  {0pt}
%  {}
\date{\vspace{-3ex}}
\begin{document}

\title{\vspace{5ex}
	\includegraphics*[bb=0 0 720 200, width=0.72\textwidth]{ErstesSem/images/hu_logo.png}\\
	\vspace{30pt}
	\scshape\LARGE{"`Das Wort selbst ist Tr"ager des Sinnes"'}\\\Large{Zu Merleau-Pontys Verst"andnis von Sprache}\\\vspace{20pt}}
	


\author{Merleau-Ponty: Ph"anomenologie der Wahrnehmung (PS)\\
	\vspace{6pt}
          Dozent: Dr. Matthias Schlo"sberger\\\vspace{4pt}Lennard Wolf\\
        \small{\href{mailto:lennard.wolf@student.hu-berlin.de}{lennard.wolf@student.hu-berlin.de}}}      

\maketitle

\vspace{\fill}

\begin{minipage}[b]{\textwidth}
    \centering
    \onehalfspacing
    \large   
    21. Juli 2017\\
    Sommersemester 2017

    \vspace{-20mm} 
\end{minipage}%
\thispagestyle{empty}
\newpage
%\clearpage
%\thispagestyle{empty}
%\tableofcontents
%\newpage
\setcounter{page}{1}

\begin{onehalfspace} 

\noindent\textbf{$(o)$ Einleitung}

\noindent Klassische Sprachtheorien unterscheiden zwischen den Sprachsymbolen an sich, also den W"ortern, und ihren "`dahinter liegenden"' Bedeutungen. Freges ber"uhmtes Beispiel von Morgenstern und Abendstern\footnote{\cite{sinnundbedeutung}} veranschaulicht dies: So gibt es zum einen die Venus als Ding in der Welt, und zum anderen die Begriffe "`Morgenstern"' und "`Abendstern"', die jeweils Symbole in der Sprache f"ur diesen real existierenden Planeten sind. Da es also zwei verschiedene Begriffe mit der selben Bedeutung\footnote{Es sei angemerkt, dass der \emph{Sinn} f"ur Frege aber jeweils ein anderer ist, n"amlich entsprechend jener Stern, den man zu der bestimmten Tageszeit sieht.} gibt, l"asst sich eine Kontingenz in der Symbolwahl erkennen. Das Wort "`Abendstern"' ist folglich nur ein leeres Symbol, das erst interpretiert werden muss, da es "`nur"' auf ein au"ser ihm liegendes \emph{anderes} verweist - das lediglich \emph{verweisende} Symbol h"atte genauso gut "`\begin{CJK}{UTF8}{min}一つ星\end{CJK}"'\footnote{"`\begin{CJK}{UTF8}{min}一つ星\end{CJK}"' ist die japanische "Ubersetzung von "`Morgenstern"' und "`Abendstern"'.} sein k"onnen.

Solch ein Verst"andnis von Sprache ist nach der Ansicht von Marcel Merleau-Ponty jedoch unzul"anglich. Ihm zufolge liegt die Bedeutung nicht "`hinter"' den W"ortern, sondern \emph{in} ihnen. In seinem Buch "`Ph"anomenologie der Wahrnehmung"' \citep{merleau1966phanomenologie} f"uhrt er unter anderem in sein Verst"andnis von Sprache ein, welches ich in diesem Essay thematisieren m"ochte. Im Besonderen werde ich versuchen, seine eher ungew"ohnlich erscheinende These "`Das Wort selbst ist Tr"ager des Sinnes"' verst"andlich zu machen.\newline

Daf"ur werde ich wie folgt vorgehen. In Abschnitt $(i)$ gebe ich einen groben Einblick in Merleau-Pontys Konzeption des menschlichen Bedeutungsverm"ogens, anhand dessen ich in Abschnitt $(ii)$ die Bedeutsamkeit von Ausdr"ucken im Allgemeinen, und Gesten im Besonderen herausarbeite. Daraufhin rekonstruiere ich in Abschnitt $(iii)$ Merleau-Pontys Argumentation daf"ur, dass die Sprache auch "`nur"' eine Form des Ausdrucks ist und somit eine gleichartige Bedeutsamkeit besitzt. Danach fasse ich die gewonnenen Erkenntnisse in Abschnitt $(iv)$ zusammen, um auf die Ausgangsthese abschliessend hinzuf"uhren.

\vspace{5mm}

\noindent\textbf{$(i)$ Die bedeutsame Welt}

\noindent Es ist eine weit verbreitete Auffassung, dass die Welt \emph{an sich} etwas sei, auf das wir keinen unmittelbaren Zugriff h"atten, und dass die Welt, in der wir leben, immer nur eine Welt \emph{f"ur uns} sein k"onne, d.h. eine Art unzureichendes Modell der Realit"at. Merleau-Ponty schreibt, die reale Welt sei "`das best"andige Sein, innerhalb dessen ich all meine Erkenntniskorrekturen vollziehe"'\footnote{\cite[S. 379]{merleau1966phanomenologie}}. Die Welt \emph{f"ur uns} ist also die Welt, in der wir leben und die \emph{uns wahrhaftig} ist. F"ur Merleau-Ponty ist das Ding, das wir als solches erkennen, eine \emph{Bedeutungseinheit} - gerade seine Bedeutung macht es "uberhaupt erst zum Ding. Alles, das uns begegnet, ist daher bedeutsam, hat immer schon Sinn.

Es ist gerade die Essenz des in-der-Welt-Seins, allem Wahrgenommenen immerzu einen Sinn zu geben. "`Diese Bewegung, in der die Existenz eine faktische Situation sich zu eigen macht und verwandelt, nennen wir die Transzendenz"'\footnote{\cite[S. 202]{merleau1966phanomenologie}}. Dass ich etwas \emph{als} Ding erfahre hei"st, dass ich es mir zu eigen mache, dass es mein Sein "andert, dass ich mit dem Ph"anomen in dem Augenblick \emph{koexistiere}\footnote{\cite[vgl.][S. 368]{merleau1966phanomenologie}}. Und so ist sein Sinn nicht \emph{hinter} dem Ding, sondern "`verk"orpert sich in ihm"'\footnote{\cite[S. 370]{merleau1966phanomenologie}}. Diesen letzten Punkt werde ich nun im folgenden Abschnitt weiter zu erhellen versuchen. 

\vspace{5mm}

\noindent\textbf{$(ii)$ Ausdruck}

\noindent "`Ein Verhalten zeichnet eine bestimmte Weise der Weltbegegnung ab."'\footnote{\cite[S. 370]{merleau1966phanomenologie}}. Es ist "`das Wunder des Ausdrucks: im "Au"seren ein Inneres zu offenbaren"'\footnote{\cite[S. 370]{merleau1966phanomenologie}}.

wir machen uns die durch die zeichen vorgezeichnete seinsweise zu eigen (370), denn etwas als etwas erkennen hei"st es in meine welt zu \emph{inkorporieren}.


\begin{itemize}
  \item Aphasie ist sprachstörung bei der...
  \item Sprachstörungen werden getrennt von sprechstörungen
  \item MP stellt in frage, dass worte nur leere symbole sind, die wir gedanklich übersetzen in die 'echten gedanken'.
  \item so muss doch aber denken vor der sprache existieren, aber diese wäre unbewusst (211) (musik ohne töne, 225)
  \item ausdruck ist verwirklichung der bedeutung selbst (217)
  \item geste trägt sinn in sich, die gebärde IST der zorn, ich übersetze ihn nicht erst (219)
  \item denken ist eine wandlung des seins, bedeutung ist wandlung des seins, so auch innerer ausdruck und äusserer ausdruck wandlung des seins ist
  \item wenn wir ein wort verstehen, dann dadurch dass wir die gesamte sprache als gegeben vorraussetzen (xxx)
  \item konklusion: wenn sprache und denken einander umschliessen, entsteht sinn (216)
\end{itemize}

----------

\begin{itemize}
  \item Gesten als Ausdruck des Leibes (Korelation Innerer und "Au"serer Ausdruck)
  \item Denken als Ausdruck
  \item sprachliche Geste (zeichen sind nicht konvention) (bezug Sprechakttheorie)
  \item benennung ist kategorisierung, kategorisierung ist nur durch inneren ausdruck möglich
  \item das symbol erst befähigt zum denken, nicht das denken zum symbol
  \item symbol ist denken
  \item wort ist nicht übersetzung fertiger gedanken!! (211)
  \item die welt findet in der sprache ausdruck (222) 
\end{itemize}



anekdoten

\begin{itemize}
  \item ich möchte argumentation rekonstruieren um den satz zu deuten
  \item die sprache HAT einen sinn -> was bedeutet HAT? -> haben nicht im sinne von besitz, sondern im sinne des durch Bezug sich konstituieren -> stärkere Bedeutung 
  \item Intentionales sprechen im gegensatz zu rein motorischem, automatischem Sprechen -> Kranke kann sich nicht vorrat bedienen (208)
  \item dingsbums, was ist das? bei kindern, um es gedanklich zu thematisieren muss es benannt werden
  \item der leib ist es der zeigt und spricht 233
  \item ein gedanke ist ein mangel, der sich auszufullen sucht (218)
  \item bildsprache/zeichen 
  \item betonung von worten gibt sprache eine weitere dimension, durch sie soll der ausdruck verdeutlicht werden
\end{itemize}


\vspace{5mm}
\noindent\textbf{$(ii)$ Tugendhafte Handlung}	

\noindent 


\vspace{5mm}
\noindent\textbf{$(iii)$ Versuch einer L"osung}	

\noindent 


\vspace{5mm}

\vspace{3mm}

\end{onehalfspace}
\nocite{*}
%\bibliography{merleau-ponty-essay}
\printbibliography
\end{document}
