\documentclass[a4paper, 12pt]{article}
\usepackage{CJKutf8} % japanese
\usepackage{graphicx}
\usepackage{hyperref}
\usepackage{fullpage}
%\usepackage{parskip}
\usepackage{color}
\usepackage[ngerman]{babel}
\usepackage{hyperref}
\usepackage{calc} 
\usepackage{enumitem}
\usepackage[utf8]{inputenc}
\usepackage{titlesec}
%\pagestyle{headings}
\usepackage{setspace} %halbzeilig
\usepackage[style=authoryear-ibid,natbib=true]{biblatex}
\usepackage[hang]{footmisc}
\setlength{\footnotemargin}{-0.8em}
%\bibliographystyle{natdin}
\addbibresource{merleau-ponty-essay.bib}
\DeclareDatamodelEntrytypes{standard}
\DeclareDatamodelEntryfields[standard]{type,number}
\DeclareBibliographyDriver{standard}{%
  \usebibmacro{bibindex}%
  \usebibmacro{begentry}%
  \usebibmacro{author}%
  \setunit{\labelnamepunct}\newblock
  \usebibmacro{title}%
  \newunit\newblock
  \printfield{number}%
  \setunit{\addspace}\newblock
  \printfield[parens]{type}%
  \newunit\newblock
  \usebibmacro{location+date}%
  \newunit\newblock
  \iftoggle{bbx:url}
    {\usebibmacro{url+urldate}}
    {}%
  \newunit\newblock
  \usebibmacro{addendum+pubstate}%
  \setunit{\bibpagerefpunct}\newblock
  \usebibmacro{pageref}%
  \newunit\newblock
  \usebibmacro{related}%
  \usebibmacro{finentry}}

%\titleformat{name=\section,numberless}
%  {\normalfont\Large\bfseries}
%  {}
%  {0pt}
%  {}
\date{\vspace{-3ex}}
\begin{document}

\title{\vspace{5ex}
	\includegraphics*[bb=0 0 720 200, width=0.72\textwidth]{ErstesSem/images/hu_logo.png}\\
	\vspace{30pt}
	\scshape\LARGE{"`Das Wort selbst ist Tr"ager des Sinnes"'}\\\Large{Sinn, Ausdruck und Sprache bei Merleau-Ponty}\\\vspace{20pt}}
	


\author{Merleau-Ponty: Ph"anomenologie der Wahrnehmung (PS)\\
	\vspace{6pt}
          Dozent: Dr. Matthias Schlo"sberger\\\vspace{4pt}Lennard Wolf\\
        \small{\href{mailto:lennard.wolf@student.hu-berlin.de}{lennard.wolf@student.hu-berlin.de}}}      

\maketitle

\vspace{\fill}

\begin{minipage}[]{0.92\textwidth}
    \centering
    \onehalfspacing
    \large   
    21. Juli 2017\\
    Sommersemester 2017

    \vspace{-20mm} 
\end{minipage}%
\thispagestyle{empty}
\newpage
%\clearpage
%\thispagestyle{empty}
%\tableofcontents
%\newpage
\setcounter{page}{1}

\begin{onehalfspace} 

\noindent\textbf{$(o)$ Einleitung}

\noindent Klassische Sprachtheorien unterscheiden zwischen den Sprachsymbolen an sich, also den W"ortern, und ihren "`dahinter liegenden"' Bedeutungen. Freges ber"uhmtes Beispiel von Morgenstern und Abendstern\footnote{\cite{sinnundbedeutung}.} veranschaulicht dies: So gibt es zum einen die Venus als Ding in der Welt, und zum anderen die Begriffe "`Morgenstern"' und "`Abendstern"', die jeweils Symbole in der Sprache f"ur diesen real existierenden Planeten sind. Dass es zwei verschiedene Begriffe mit der selben Bedeutung\footnote{Es sei angemerkt, dass der \emph{Sinn} f"ur Frege aber jeweils ein anderer ist, n"amlich entsprechend jener Stern, den man zu der bestimmten Tageszeit sieht.} gibt, l"asst eine Kontingenz in der Zuordnung von Symbol und Symbolisiertem erkennen. Das Wort "`Abendstern"' ist folglich nur ein \emph{leeres} Zeichen, das erst interpretiert werden muss, da es "`nur"' auf ein au"ser ihm liegendes \emph{anderes} hindeutet - das also lediglich \emph{verweisende} Wort h"atte genauso gut "`\begin{CJK}{UTF8}{min}一つ星\end{CJK}"'\footnote{"`\begin{CJK}{UTF8}{min}一つ星\end{CJK}"' ist die japanische "Ubersetzung von "`Morgenstern"' und "`Abendstern"'.} sein k"onnen.

Solch ein Verst"andnis von Sprache ist Marcel Merleau-Pontys Ansicht jedoch unzul"anglich. Ihm zufolge liegt die Bedeutung nicht "`hinter"' den W"ortern, sondern \emph{in} ihnen. In seinem Buch "`Ph"anomenologie der Wahrnehmung"' \citep{merleau1966phanomenologie} gibt er unter anderem Einblick in sein Verst"andnis von Sprache, zu welchem ich in diesem Essay hinf"uhren m"ochte. Im Besonderen werde ich versuchen, seine eher ungew"ohnlich erscheinende These "`das Wort selbst ist Tr"ager des Sinnes"'\footnote{\cite[S. 211]{merleau1966phanomenologie}.} verst"andlich zu machen.\newline

Daf"ur werde ich wie folgt vorgehen. In Abschnitt $(i)$ gebe ich einen groben Einblick in Merleau-Pontys Konzeption des menschlichen Bedeutungsverm"ogens, anhand dessen ich in Abschnitt $(ii)$ durch die Kl"arung des Leibbegriffs die Bedeutsamkeit von Ausdr"ucken herausarbeite. Daraufhin rekonstruiere ich in Abschnitt $(iii)$ Merleau-Pontys Begr"undungen daf"ur, dass die Sprache eine besondere Art des Ausdrucks ist und somit eine gleichartige Bedeutsamkeit besitzt. Danach fasse ich die gewonnenen Erkenntnisse in Abschnitt $(iv)$ zusammen, um auf die Ausgangsthese abschliessend hinzuf"uhren.

\vspace{5mm}

\noindent\textbf{$(i)$ Die bedeutsame Welt}

\noindent Es ist eine weit verbreitete Auffassung, dass die Welt \emph{an sich} etwas sei, auf das wir keinen unmittelbaren Zugriff h"atten, und dass die Welt, in der wir leben, immer nur eine Welt \emph{f"ur uns} sein k"onne, d.h. eine Art unzureichendes Modell der Realit"at. Merleau-Ponty schreibt, die reale Welt sei "`das best"andige Sein, innerhalb dessen ich all meine Erkenntniskorrekturen vollziehe"'\footnote{\Cite[S. 379]{merleau1966phanomenologie}.}. \emph{Unsere} Welt ist einfach die Welt, in der wir leben, die \emph{uns wahrhaftig} ist und in der wir Dinge erfahren. F"ur Merleau-Ponty bildet das Ding, das wir \emph{als solches} erkennen, eine \emph{Bedeutungseinheit} - gerade seine Bedeutung macht es "uberhaupt erst zum Ding. Alles, das uns \emph{als} etwas begegnet, ist daher bedeutsam, tr"agt immer schon Sinn in sich.

Unser Bedeutungsverm"ogen, also die Eigenschaft, allem Wahrgenommenen immerzu einen Sinn zu geben, ist gerade die Essenz unseres In-Der-Welt-Seins. "`Diese Bewegung, in der die Existenz eine faktische Situation sich zu eigen macht und verwandelt, nennen wir die Transzendenz"'\footnote{\Cite[S. 202]{merleau1966phanomenologie}.}. Dass ich etwas \emph{als} Ding erfahre hei"st, dass ich mich in es und es in mich hineinlege, dass es mein Sein "andert, dass ich mit dem Ph"anomen in dem Augenblick \emph{koexistiere}\footnote{\Cite[Vgl.][S. 368]{merleau1966phanomenologie}.}. Und so ist sein Sinn nicht \emph{hinter} dem Ding, sondern "`verk"orpert sich in ihm"'\footnote{\Cite[S. 370]{merleau1966phanomenologie}.}. Diesen letzten Punkt werde ich nun im folgenden Abschnitt durch eine Beschreibung des Leibkonzeptes weiter zu erhellen versuchen. 

\vspace{5mm}

\noindent\textbf{$(ii)$ Leib und Ausdruck}

\noindent Zentral in der "`Ph"anomenologie der Wahrnehmung"' ist das Konzept des Leibes, der die \emph{inkarnierte Existenz} ist, das "`Vehikel des Zur-Welt-Seins"'\footnote{\Cite[S. 106]{merleau1966phanomenologie}.}. Der Leib ist unser Mittel "uberhaupt eine Welt zu haben.\footnote{Anm.: Dieser Satz entstammt meinen eigenen Notizen, doch ist m"oglicherweise ein direktes Zitat aus der "`Ph"anomenologie der Wahrnehmung"', das ich jedoch nicht wiederfinden konnte.} Und dieses "`haben"' der Welt ist hier nicht im Sinne eines Besitzverh"altnisses zu verstehen, sondern im Sinne eines konstituierenden Bezugs\footnote{\Cite[Vgl.][S. 207]{merleau1966phanomenologie}.}. Entsprechend konstituieren sich die Dinge auch erst im Bezug auf den Leib, indem sie auf ihn \emph{einwirken}, und er sie \emph{wahr}nimmt\footnote{"`Wahrnehmen ist [...] die Erfahrung des Entspringens eines immanenten Sinnes aus einer Konstellation von Gegebenheiten"' \citep[S. 42]{merleau1966phanomenologie}.}. F"ur Merleau-Ponty ist Sinn dadurch immer \emph{verk"orperter} Sinn, er ist der Welt "`einverleibt"'.

Dieses Verh"altnis von Ding und Leib, in welchem beide einander bedingen, stellt f"ur Merleau-Ponty eine "`endg"ultige "Uberwindung der klassischen Entgegensetzung von Subjekt und Objekt"'\footnote{\Cite[S. 207]{merleau1966phanomenologie}.} dar. Die Wahrnehmung der Welt ist also nicht vermittelt und interpretiert, sondern unmittelbar und wahrhaftig. Der Ausdruck der Welt \emph{ist} ihre Wirkung auf mich, und offenbart mir ihren Sinn. Wenn mir ein trauernder Mensch begegnet, so analysiere ich nicht erst mein Blickfeld, interpretiere die Tr"anen und das zerknitterte Gesicht als Symbole einer \emph{abstrakten} Traurigkeit, sondern ich nehme die Traurigkeit \emph{direkt} wahr, denn die Tr"anen und das zerknitterte Gesicht \emph{sind} die Trauer. Sie sind der Ausdruck der Traurigkeit nach au"sen, und die andere Person erf"ahrt den Ausdruck der Traurigkeit nach innen. Gerade dieser Ausdruck aus dem Leib und der Ausdruck in den Leib \emph{sind} diese Emotion. Es ist "`das Wunder des Ausdrucks: im "Au"seren ein Inneres zu offenbaren"'\footnote{\Cite[S. 370]{merleau1966phanomenologie}.}, der Ausdruck ist "`nicht lediglich eine "Ubersetzung, sondern eine Realisierung und Verwirklichung der Bedeutung selbst"'\footnote{\Cite[S. 217]{merleau1966phanomenologie}.}. 

Dass ich zum Beispiel einen Schmerz erfahre hei"st, dass sich mein Leib so "andert, dass ich zum einen den Schmerz sinnlich wahrnehme und dass ich zum anderen eine \emph{Geste mache}, also beispielsweise aufschreie und die betroffene Stelle mit den H"anden ber"uhre. Die Geste ist nicht zu trennen von der Erfahrung und umgekehrt. Ein Schauspieler, der Wut auf eine Weise vorspielt, dass wir sie ihm nicht abnehmen, \emph{empfindet sie auch nicht}. Doch wenn wir es einer anderen Schauspielerin abnehmen, so \emph{verstehen} wir ihr Geb"arde gerade weil sie \emph{wahr} ist. Es ist nicht blo"ser Zufall, dass Schauspieler*innen zuweilen an ihren Rollen zugrunde gehen. 

Innerhalb einer Kultur leben hei"st, dass die Dinge, und damit ihre Bedeutungen, f"ur uns in weiten Teilen dieselben sind, wie f"ur die anderen in der Kultur. Diese "`Kulturg"uter"' entwerfen eine Welt, die Teil unserer Welt ist. So verstehen wir die Gesten und Geb"arden der anderen Menschen in unserer Kultur, weil "`wir die von den beobachteten Zeichen vorgezeichnete Seinsweise uns zu eigen machen"'\footnote{\Cite[S. 370]{merleau1966phanomenologie}.}. "`Die sexuelle Mimik des Hundes `verstehe' ich nicht, nicht zu reden vom Maik"afer oder der Gottesanbeterin"'\footnote{\Cite[S. 219]{merleau1966phanomenologie}.}, denn die Welten der menschlichen Leiber ist jenen derlei anderer Lebewesen zutiefst verschieden. "`Der Sinn der also `verstandenen' Geste eines Anderen ist nicht hinter ihr gelegen, sondern f"allt zusammen mit der Struktur der von der Geb"arde entworfenen Welt"'\footnote{\Cite[S. 220]{merleau1966phanomenologie}.}.


\vspace{5mm}

\noindent\textbf{$(iii)$ Sprache und Denken}

\noindent Sprache ist zumindest in den modernen menschlichen Kulturen ein zentrales Gut. Es l"asst sich entsprechend schon erahnen, dass sie in Merleau-Pontys Weltbild eine ihr inh"arente Bedeutung hat. Doch vorher ist noch zu kl"aren, in welcher Beziehung sie zum Denken steht, denn der in Abschnitt \emph{(o)} angesprochenen, g"angigen Auffassung von Sprache zufolge "`ist es das Denken, das einen Sinn hat, das Wort hingegen ist nur dessen leere H"ulle"'\footnote{\Cite[S. 210]{merleau1966phanomenologie}.}. Der "`hinter"' dem Wort liegende Sinn sei also im \emph{Gedanken} zu finden, was eine Konzeption des Denkens als vorsprachlich voraussetzt. Weshalb gerade dies Merleau-Ponty folgend nicht der Fall sein kann, und wie die Beziehung zwischen Sprache und Denken stattdessen aussehen muss, ist das Thema dieses Abschnitts. 

Wenn Denken vorsprachlich ist, dann w"are es doch einleuchtend, dass Sprache nur im Dialog mit anderen seine Verwendung f"ande, doch im eigenen Denkprozess nicht gebraucht und folglich auch nicht benutzt werden w"urde. Dies ist jedoch nicht der Fall, vielmehr sucht das Denken stets nach Ausdruck\footnote{\cite[Vgl.][S. 216]{merleau1966phanomenologie}.}. Ein Gedanke erscheint uns noch nicht \emph{gefasst}, wenn er nicht in Worte gefasst ist - wir haben ihn noch nicht \emph{begriffen}. 

Grundvoraussetzung f"ur das Denken ist die Kategorisierung von Dingen. Damit ich "uber ein Ding nachdenken kann, muss ich es als Ding verstehen und unterscheiden k"onnen von anderem. Diese Unterscheidung geht gerade deshalb, weil wir den Dingen einen Sinn, einen Kern zuordnen k"onnen. Der Gedanke von etwas \emph{ist} entsprechend gerade dessen Kernsinn, dessen "`Idee"'. Sobald ein Kind das erste Mal einen Zug sieht, wird es reflexartig "`Was ist das?"' fragen, und sich mit dem Wort "`Zug"' zufrieden geben. W"usste die gefragte Person keine Antwort auf die Frage, m"usste das Kind in Zukunft f"ur den Gedanken an den Zug, seinen Kernsinn, auf ein mentales Ab\emph{bild} zur"uckgreifen, das schwer nach au"sen kommunizierbar w"are, und beim Sprechen mit einem Wort wie "`Dingsbums"' Vorlieb nehmen. Das mentale Bild vom Zug \emph{ist} f"ur das Kind aber gerade die Idee, der Gedanke des Zuges - zur Kommunikation w"urde es ihn dann einfach malen. Nicht verwunderlich ist es also, dass in einigen Sprachen, wie beispielsweise dem Japanischen, die geschriebenen Symbole eine \emph{Bildhaftigkeit} besitzen. Das japanische Wort f"ur "`Person"' wird `hito' (`ひと') ausgesprochen und hat das zugeh"orige Kanji-Schriftzeichen "`人"'\footnote{Es sei angemerkt, dass die japanischen Kanji-Schriftzeichen aus dem Chinesischen "ubernommen wurden und dort entsprechend "ahnliche Verwendung finden.}, das einer laufenden Person "ahnelt. Dass "`木"' die Entsprechung zu "`Baum"' ist, leuchtet ebenso gut ein. Was dies veranschaulichen soll, ist, dass das, was das Kind in dem Beispiel getan h"atte, wenn es kein Wort gehabt h"atte, genau das ist, was die Kulturen in der Entwicklung der Schriftzeichen getan haben. Der Gedanke \emph{braucht} mentale Repr"asentation, einen Ausdruck, damit er uns als solcher "uberhaupt \emph{gegenw"artig}\footnote{\cite[Vgl.][S. 215 f.]{merleau1966phanomenologie}.} ist. Wenn wir uns an die Vergangenheit erinnern, dann "`vergegenw"artigen"' wir sie uns anhand mentaler Ausdr"ucke wie Bilder, Ger"uche oder Emotionen - wir bringen uns das Vergangene durch seine Ausdr"ucke in die Gegenwart. Und genau das ist Denken: Vergegenw"artigung von Sinn durch Ausdr"ucke. "`Die wesentliche Leistung des Ausdrucks ist nicht die, da"s er Gedanken [...] zu fixieren gestattet [...], sondern darin, da"s dem gegl"uckten Ausdruck die Bedeutung ein Dasein [...] gleich dem eines Dinges [...] verdankt"'\footnote{\Cite[S. 216]{merleau1966phanomenologie}.}. 

Die Sprache ist also "`nur"' eine Ausdrucks\emph{form}, n"amlich eine besondere Art der Geb"arde, und sie gibt uns eine neue Erfahrungsdimension.\footnote{\cite[Vgl.][S. 216 f.]{merleau1966phanomenologie}.} Das Wort, gleich einem Ding, wirkt auf uns ein und ver"andert unser Sein. So wie eine Geb"arde aufgrund unserer Kulturwelt von Bedeutung erf"ullt ist, ist es auch das Wort, das Teil einer Sprachwelt ist, die es zugleich voraussetzt. Da die Sprachwelt einer uns fremden Kultur nicht Teil unserer Welt ist, vernehmen wir die ihr zugeh"origen W"orter nur als bedeutungslose Laute. Die Muschel ist f"ur mich blo"s ein Objekt aus dem Meer, f"ur den Krebs ist es ein Zuhause. Damit der fremde Ausdruck f"ur mich Bedeutung haben kann, muss die Welt, die er und die ihn konstituiert, erst ein Teil meiner Welt sein. 

\vspace{5mm}

\noindent\textbf{$(iv)$ Konklusion}

\noindent In den vorangegangenen Abschnitten habe ich versucht, einen "Uberblick "uber Merleau-Pontys Vorstellungen bez"uglich Sinn, Ding und Ausdruck zu geben, sowie deren Anwendung auf die Sprache zu plausibilisieren. 

Ein Ding ist, da wir es als solches wahrnehmen, immer schon Tr"ager von Sinn. Ein Ausdruck ist unsere Erfahrung von Sinn, er wirkt auf unseren Leib ein und offenbart das Innere des Wahrgenommenen. Der Ausdruck des Dings ist f"ur mich seine sinnliche Wahrnehmung und offenbart mir seinen inneren Sinn. Sinn und Ausdruck sind zwei Seiten der selben M"unze. Damit ein Gedanke gedacht werden kann, muss er, f"ur seine Gegenwart f"ur mich, einen Ausdruck haben. Der Ausdruck des Gedankens, das mentale Symbol, das Wort, \emph{ist} daher die "au"sere Form des inneren Sinns. \emph{Das Wort selbst ist Tr"ager des Sinnes}.

%am anfnag war das wort
%was ist versprechen? 


% die gegenposition ist keine wirkliche, vllt sogar falsch: MP würde nicht bestreiten, dass die worte auf konvention beruhen
% nicht direkt die traurigkeit wahrnehmen, sondern unvermittelt
% sprache und denken: verschiedene formen von denken: nicht sprechendes kind nimmt natürlich wahr
% sprache des ausdrucks steht dem kind offen

% im verstehen des ausdrucks verstehen wir den gedanken

% das konventionelle betonen: ursprungstheorien, onomatopetisch, ähnlich wie bildsprache

% versuchen: sauber an den begriffen arbeiten: repräsentationalismus; mehr im kontrast zu anderem verstehen

% wittgenstein, anfang, ähnlich

%sprache hat problem: etwas kann sich ausdrücken , oder nicht, aber das ist ja nicht der fall, und sprache hat auch was aktives, das dürfen wir nicht mitdenken, denn ausdruck hat nichts aktives 

% besonders klar machen, wie ausdruck zu verstehen, 

% als handwerksarbeit verstehen, 

\end{onehalfspace}
\nocite{*}
%\bibliography{merleau-ponty-essay}
\printbibliography
\end{document}
