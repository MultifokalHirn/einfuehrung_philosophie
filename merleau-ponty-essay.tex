\documentclass[a4paper, 12pt]{article}
\usepackage{CJKutf8} % japanese
\usepackage{graphicx}
\usepackage{hyperref}
\usepackage{fullpage}
%\usepackage{parskip}
\usepackage{color}
\usepackage[ngerman]{babel}
\usepackage{hyperref}
\usepackage{calc} 
\usepackage{enumitem}
\usepackage[utf8]{inputenc}
\usepackage{titlesec}
%\pagestyle{headings}
\usepackage{setspace} %halbzeilig
\usepackage[style=authoryear-ibid,natbib=true]{biblatex}
%\bibliographystyle{natdin}
\addbibresource{merleau-ponty-essay.bib}
\DeclareDatamodelEntrytypes{standard}
\DeclareDatamodelEntryfields[standard]{type,number}
\DeclareBibliographyDriver{standard}{%
  \usebibmacro{bibindex}%
  \usebibmacro{begentry}%
  \usebibmacro{author}%
  \setunit{\labelnamepunct}\newblock
  \usebibmacro{title}%
  \newunit\newblock
  \printfield{number}%
  \setunit{\addspace}\newblock
  \printfield[parens]{type}%
  \newunit\newblock
  \usebibmacro{location+date}%
  \newunit\newblock
  \iftoggle{bbx:url}
    {\usebibmacro{url+urldate}}
    {}%
  \newunit\newblock
  \usebibmacro{addendum+pubstate}%
  \setunit{\bibpagerefpunct}\newblock
  \usebibmacro{pageref}%
  \newunit\newblock
  \usebibmacro{related}%
  \usebibmacro{finentry}}


%\titleformat{name=\section,numberless}
%  {\normalfont\Large\bfseries}
%  {}
%  {0pt}
%  {}
\date{\vspace{-3ex}}
\begin{document}

\title{\vspace{5ex}
	\includegraphics*[bb=0 0 720 200, width=0.72\textwidth]{ErstesSem/images/hu_logo.png}\\
	\vspace{30pt}
	\scshape\LARGE{"`Das Wort selbst ist Tr"ager des Sinnes"'}\\\Large{Sinn, Ausdruck und Sprache bei Merleau-Ponty}\\\vspace{20pt}}
	


\author{Merleau-Ponty: Ph"anomenologie der Wahrnehmung (PS)\\
	\vspace{6pt}
          Dozent: Dr. Matthias Schlo"sberger\\\vspace{4pt}Lennard Wolf\\
        \small{\href{mailto:lennard.wolf@student.hu-berlin.de}{lennard.wolf@student.hu-berlin.de}}}      

\maketitle

\vspace{\fill}

\begin{minipage}[]{0.92\textwidth}
    \centering
    \onehalfspacing
    \large   
    21. Juli 2017\\
    Sommersemester 2017

    \vspace{-20mm} 
\end{minipage}%
\thispagestyle{empty}
\newpage
%\clearpage
%\thispagestyle{empty}
%\tableofcontents
%\newpage
\setcounter{page}{1}

\begin{onehalfspace} 

\noindent\textbf{$(o)$ Einleitung}

\noindent Klassische Sprachtheorien unterscheiden zwischen den Sprachsymbolen an sich, also den W"ortern, und ihren "`dahinter liegenden"' Bedeutungen. Freges ber"uhmtes Beispiel von Morgenstern und Abendstern\footnote{\cite{sinnundbedeutung}.} veranschaulicht dies: So gibt es zum einen die Venus als Ding in der Welt, und zum anderen die Begriffe "`Morgenstern"' und "`Abendstern"', die jeweils Symbole in der Sprache f"ur diesen real existierenden Planeten sind. Dass es zwei verschiedene Begriffe mit der selben Bedeutung\footnote{Es sei angemerkt, dass der \emph{Sinn} f"ur Frege aber jeweils ein anderer ist, n"amlich entsprechend jener Stern, den man zu der bestimmten Tageszeit sieht.} gibt, l"asst eine Kontingenz in der Zuordnung von Symbol und Symbolisiertem erkennen. Das Wort "`Abendstern"' ist folglich nur ein \emph{leeres} Zeichen, das erst interpretiert werden muss, da es "`nur"' auf ein au"ser ihm liegendes \emph{anderes} hindeutet - das also lediglich \emph{verweisende} Wort h"atte genauso gut "`\begin{CJK}{UTF8}{min}一つ星\end{CJK}"'\footnote{"`\begin{CJK}{UTF8}{min}一つ星\end{CJK}"' ist die japanische "Ubersetzung von "`Morgenstern"' und "`Abendstern"'.} sein k"onnen.

Solch ein Verst"andnis von Sprache ist nach der Ansicht von Marcel Merleau-Ponty jedoch unzul"anglich. Ihm zufolge liegt die Bedeutung nicht "`hinter"' den W"ortern, sondern \emph{in} ihnen. In seinem Buch "`Ph"anomenologie der Wahrnehmung"' \citep{merleau1966phanomenologie} f"uhrt er unter anderem in sein Verst"andnis von Sprache ein, welches ich in diesem Essay thematisieren m"ochte. Im Besonderen werde ich versuchen, seine eher ungew"ohnlich erscheinende These "`Das Wort selbst ist Tr"ager des Sinnes"' verst"andlich zu machen.\newline

Daf"ur werde ich wie folgt vorgehen. In Abschnitt $(i)$ gebe ich einen groben Einblick in Merleau-Pontys Konzeption des menschlichen Bedeutungsverm"ogens, anhand dessen ich in Abschnitt $(ii)$ durch die Kl"arung des Leibbegriffs die Bedeutsamkeit von Ausdr"ucken herausarbeite. Daraufhin rekonstruiere ich in Abschnitt $(iii)$ Merleau-Pontys Argumentation daf"ur, dass die Sprache auch "`nur"' eine Form des Ausdrucks ist und somit eine gleichartige Bedeutsamkeit besitzt. Danach fasse ich die gewonnenen Erkenntnisse in Abschnitt $(iv)$ zusammen, um auf die Ausgangsthese abschliessend hinzuf"uhren.

\vspace{5mm}

\noindent\textbf{$(i)$ Die bedeutsame Welt}

\noindent Es ist eine weit verbreitete Auffassung, dass die Welt \emph{an sich} etwas sei, auf das wir keinen unmittelbaren Zugriff h"atten, und dass die Welt, in der wir leben, immer nur eine Welt \emph{f"ur uns} sein k"onne, d.h. eine Art unzureichendes Modell der Realit"at. Merleau-Ponty schreibt, die reale Welt sei "`das best"andige Sein, innerhalb dessen ich all meine Erkenntniskorrekturen vollziehe"'\footnote{\cite[S. 379]{merleau1966phanomenologie}.}. \emph{Unsere} Welt ist einfach die Welt, in der wir leben, die \emph{uns wahrhaftig} ist und in der wir Dinge erfahren. F"ur Merleau-Ponty bildet das Ding, das wir \emph{als solches} erkennen, eine \emph{Bedeutungseinheit} - gerade seine Bedeutung macht es "uberhaupt erst zum Ding. Alles, das uns als etwas begegnet, ist daher bedeutsam, hat immer schon Sinn.

Unser Bedeutungsverm"ogen, also die Eigenschaft, allem Wahrgenommenen immerzu einen Sinn zu geben, ist gerade die Essenz unseres In-Der-Welt-Seins. "`Diese Bewegung, in der die Existenz eine faktische Situation sich zu eigen macht und verwandelt, nennen wir die Transzendenz"'\footnote{\cite[S. 202]{merleau1966phanomenologie}.}. Dass ich etwas \emph{als} Ding erfahre hei"st, dass ich mich in es und es in mich hineinlege, dass es mein Sein "andert, dass ich mit dem Ph"anomen in dem Augenblick \emph{koexistiere}\footnote{\cite[Vgl.][S. 368]{merleau1966phanomenologie}.}. Und so ist sein Sinn nicht \emph{hinter} dem Ding, sondern "`verk"orpert sich in ihm"'\footnote{\cite[S. 370]{merleau1966phanomenologie}.}. Diesen letzten Punkt werde ich nun im folgenden Abschnitt durch eine Beschreibung des Leibkonzeptes weiter zu erhellen versuchen. 

\vspace{5mm}

\noindent\textbf{$(ii)$ Leib und Ausdruck}

\noindent Zentral in der "`Ph"anomenologie der Wahrnehmung"' ist das Konzept des Leibes, der die \emph{inkarnierte Existenz} ist, das "`Vehikel des Zur-Welt-Seins"'\footnote{\cite[S. 106]{merleau1966phanomenologie}.}. Der Leib ist unser Mittel "uberhaupt eine Welt zu haben.\footnote{Anm.: Dieser Satz entstammt meinen eigenen Notizen, doch ist m"oglicherweise ein direktes Zitat aus der "`Ph"anomenologie der Wahrnehmung"', das ich jedoch nicht wiederfinden konnte.} Und dieses "`haben"' der Welt ist hier nicht im Sinne eines Besitzverh"altnisses zu verstehen, sondern im Sinne eines konstituierenden Bezugs\footnote{\cite[Vgl.][S. 207]{merleau1966phanomenologie}.}. Entsprechend konstituieren sich die Dinge auch erst im Bezug auf den Leib, indem sie auf ihn \emph{einwirken}, und er sie \emph{wahr}nimmt\footnote{"`Wahrnehmen ist [...] die Erfahrung des Entspringens eines immanenten Sinnes aus einer Konstellation von Gegebenheiten"' \citep[S. 42]{merleau1966phanomenologie}.}. F"ur Merleau-Ponty ist Sinn dadurch immer \emph{verk"orperter} Sinn, er ist der Welt "`einverleibt"'.

Dieses Verh"altnis von Ding und Leib, in welchem beide einander bedingen, stellt f"ur Merleau-Ponty eine "`endg"ultige "Uberwindung der klassischen Entgegensetzung von Subjekt und Objekt"'\footnote{\cite[S. 207]{merleau1966phanomenologie}.} dar. Die Wahrnehmung der Welt ist also nicht vermittelt und interpretiert, sondern unmittelbar und wahrhaftig. Der Ausdruck der Welt \emph{ist} der Eindruck auf mich. Wenn mir ein trauernder Mensch begegnet, so analysiere ich nicht erst mein Blickfeld, interpretiere die Tr"anen und das zerknitterte Gesicht als Symbole einer \emph{abstrakten} Traurigkeit, sondern ich nehme die Traurigkeit \emph{direkt} wahr, denn die Tr"anen und das zerknitterte Gesicht \emph{sind} die Trauer. Sie sind der Ausdruck der Traurigkeit nach au"sen, und die andere Person erf"ahrt den Ausdruck der Traurigkeit nach innen. Gerade dieser Ausdruck aus dem Leib und der Ausdruck in den Leib \emph{sind} diese Emotion. Es ist "`das Wunder des Ausdrucks: im "Au"seren ein Inneres zu offenbaren"'\footnote{\cite[S. 370]{merleau1966phanomenologie}.}, der Ausdruck ist "`nicht lediglich eine "Ubersetzung, sondern eine Realisierung und Verwirklichung der Bedeutung selbst"'\footnote{\cite[S. 217]{merleau1966phanomenologie}.}.

Dass ich zum Beispiel einen Schmerz erfahre hei"st, dass sich mein Leib so "andert, dass ich zum einen den Schmerz sinnlich wahrnehme und dass ich zum anderen eine \emph{Geste mache}, also beispielsweise aufschreie und die entsprechende Stelle mit den H"anden ber"uhre. Die Geste ist nicht zu trennen von der Erfahrung und umgekehrt. Ein Schauspieler, der Wut auf eine Weise vorspielt, dass wir sie ihm nicht abnehmen, \emph{empfindet sie auch nicht}. Doch wenn wir es einer anderen Schauspielerin abnehmen, so \emph{verstehen} wir ihr Geb"arde gerade weil sie \emph{wahr} ist. Es ist nicht blo"ser Zufall, dass Schauspieler*innen zuweilen an ihren Rollen zugrunde gehen. 

Innerhalb einer Kultur leben hei"st, dass die Dinge, und damit ihre Bedeutungen, f"ur uns in weiten Teilen die selben sind, wie f"ur die anderen in der Kultur. Diese "`Kulturg"uter"' entwerfen eine Welt, die Teil unserer Welt ist. So verstehen wir die Gesten und Geb"arden der anderen Menschen in unserer Kultur, weil "`wir die von den beobachteten Zeichen vorgezeichnete Seinsweise uns zu eigen machen"'\footnote{\cite[S. 370]{merleau1966phanomenologie}.}. Die "`sexuelle Mimik des Hundes `verstehe' ich nicht, nicht zu reden vom Maik"afer oder der Gottesanbeterin"'\footnote{\cite[S. 219]{merleau1966phanomenologie}.}, denn die Welten der menschlichen Leiber ist jenen derlei anderer Lebewesen zutiefst verschieden. "`Der Sinn der also `verstandenen' Geste eines Anderen ist nicht hinter ihr gelegen, sondern f"allt zusammen mit der Struktur der von der Geb"arde entworfenen Welt"'\footnote{\cite[S. 220]{merleau1966phanomenologie}.}.


\vspace{5mm}

\noindent\textbf{$(iii)$ Sprache und Denken}

\vspace{5mm}

\noindent\textbf{$(ii)$ "`Das Wort selbst ist Tr"ager des Sinnes"'}





\begin{itemize}
  \item Aphasie ist sprachstörung bei der...
  \item Sprachstörungen werden getrennt von sprechstörungen
  \item so muss doch aber denken vor der sprache existieren, aber diese wäre unbewusst (211) (musik ohne töne, 225)
  \item wenn wir ein wort verstehen, dann dadurch dass wir die gesamte sprache als gegeben vorraussetzen (xxx)
  \item konklusion: wenn sprache und denken einander umschliessen, entsteht sinn (216)
  \item Gesten als Ausdruck des Leibes (Korelation Innerer und "Au"serer Ausdruck)
  \item sprachliche Geste (zeichen sind nicht konvention) (bezug Sprechakttheorie)
  \item benennung ist kategorisierung, kategorisierung ist nur durch inneren ausdruck möglich
  \item das symbol erst befähigt zum denken, nicht das denken zum symbol
  \item symbol ist denken
  \item wort ist nicht übersetzung fertiger gedanken!! (211)
  \item Intentionales sprechen im gegensatz zu rein motorischem, automatischem Sprechen -> Kranke kann sich nicht vorrat bedienen (208)
  \item dingsbums, was ist das? bei kindern, um es gedanklich zu thematisieren muss es benannt werden
  \item der leib ist es der zeigt und spricht 233
  \item ein gedanke ist ein mangel, der sich auszufullen sucht (218)
  \item bildsprache/zeichen 
  \item betonung von worten gibt sprache eine weitere dimension, durch sie soll der ausdruck verdeutlicht werden
  \item die welt findet in der sprache ausdruck (222) 
\end{itemize}



\end{onehalfspace}
\nocite{*}
%\bibliography{merleau-ponty-essay}
\printbibliography
\end{document}
