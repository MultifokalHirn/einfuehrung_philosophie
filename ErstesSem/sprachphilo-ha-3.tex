\documentclass[a4paper, emulatestandardclasses, 12pt]{scrartcl}
\usepackage{graphicx}
\usepackage{fullpage}
%\usepackage{parskip}
\usepackage{color}
\usepackage[ngerman]{babel}
\usepackage{hyperref}
\usepackage{calc} 
\usepackage{enumitem}
\usepackage{titlesec}
%\pagestyle{headings}
\usepackage{setspace} %halbzeilig
\usepackage[authoryear,round]{natbib}
\bibliographystyle{natdin}

%\titleformat{name=\section,numberless}
%  {\normalfont\Large\bfseries}
%  {}
%  {0pt}
%  {}
\date{\vspace{-3ex}}
\begin{document}

\title{\vspace{5ex}
	\includegraphics*[width=0.72\textwidth]{images/hu_logo.png}\\
	\vspace{30pt}
	\scshape\LARGE{Kurzessay zur\\Gebrauchstheorie der Sprache}}
	
	\subtitle{\vspace{20pt}Einf"uhrung in die Sprachphilosophie\\
          \vspace{6pt}
          Tutorium Benjamin\\}


\author{\vspace{-4pt}Lennard Wolf\\
        \small{\href{mailto:lennard.wolf@student.hu-berlin.de}{lennard.wolf@student.hu-berlin.de}}}      

\maketitle

\vspace{\fill}

\begin{minipage}[b]{\textwidth}
    \centering
    \onehalfspacing
    \large   
    01. Februar 2017\\
    Wintersemester 2016/2017

    \vspace{-20mm} 
\end{minipage}%
\thispagestyle{empty}
\newpage
\clearpage
\setcounter{page}{1}

\begin{onehalfspace} 

\noindent\textbf{$(o)$ Einleitung}

\noindent In dieser Arbeit m"ochte ich erl"autern, weshalb eine Gebrauchstheorie wie sie Ludwig Wittgenstein vorgestellt hat unumg"anglich ist. Desweiteren versuche ich aufzuzeigen, warum diese jedoch systematische Bedeutungstheorien nicht abl"ost, sondern einen Platz zu ihrer Seite einnimmt. 

Dazu werde ich wie folgt vorgehen. In Abschnitt $(i)$ erl"autere ich ein Problem mit der klassischen Bedeutungstheorie von W"ortern. Es folgt in $(ii)$ eine Darstellung der alternativen Gebrauchstheorie nach Wittgenstein. In $(iii)$ zeige ich, dass sein antisystematischer Ansatz jedoch nicht mit dem typischen Erwerb von Fremdsprachenkenntnissen vereinbar ist. Daher pl"adiere ich in $(iv)$ f"ur eine Gebrauchstheorie, die auch systematische Ans"atze erlaubt. 
\vspace{5mm}

\noindent\textbf{$(i)$ Klassische Bedeutungstheorie}

\noindent In seinem Buch "`Philosophische Untersuchungen"' \citep{wittgenstein1963tractatus} stellt Wittgenstein zu Beginn eine klassische\footnote{Diese Bezeichnung stammt vom Autor.} Bedeutungstheorie vor, wie sie die \emph{Philosophie der idealen Sprache} vorsieht, und auch zum Beispiel von Augustinus vertreten wurde. Diese besagt, dass ein jedes Wort eine einheitliche Definition habe, die den Sprecher*innen einer Sprache bekannt ist.\footnote{"`Jedes Wort hat eine Bedeutung. Diese Bedeutung ist dem Wort zugeordnet. Sie ist der Gegenstand, f"ur welchen das Wort steht"' \citep[PU \S 1]{wittgenstein1963tractatus}.} Anhand des Beispiels von "`Gem"usesalat"' m"ochte ich nun zeigen, dass dies eine unzureichende Theorie f"ur Wortbedeutung ist.

Das Wort "`Gem"usesalat"' h"atte nach einer klassischen Bedeutungstheorie eine eindeutig definierte Bedeutung, n"amlich die der Kategorie jener Salate, die zum Gro"steil aus Gem"use zusammengestellt werden. Bei "`Fruchtsalat"' w"urde es sich analog verhalten. Jetzt wollen wir den Tomatensalat einer dieser beiden Kategorien zuordnen: Gingen wir von der genannten Kategoriendefinition aus, so m"ussten wir sagen, dass der Tomatensalat ein Fruchtsalat und kein Gem"usesalat ist. Dies ergibt sich daraus, dass die Bedeutung von "`Fruchtsalat"' eine Kategorie ist, die den Tomatensalat einschlie"st, w"ahrend die Bedeutung von "`Gem"usesalat"' diesen ausschlie"st. Problematisch hieran ist, dass diese Kategorisierung jedoch f"ur die meisten Sprecher*innen der deutschen Sprache unintuitiv ist, von ihnen vielleicht sogar als falsch angesehen wird.\footnote{Es ist eine g"angige, jedoch falsche Auffassung, dass Tomaten als Gem"use zu z"ahlen sind. Es handelt sich bei diesen n"amlich um Fr"uchte.} 

Es zeigt sich also, dass eine feste Verbindung von Wort und Bedeutung durch eine einheitliche Definition nicht immer eine ad"aquate Darstellung des Wortverst"andnisses im allt"aglichen Sprachgebrauch ist. Aus diesem Grund entwickelte Ludwig Wittgenstein seine \emph{Gebrauchstheorie}, die den Grundstein f"ur die \emph{Philosophie der normalen Sprache} darstellt.

\vspace{5mm}
\noindent\textbf{$(ii)$ Antisystematische Gebrauchstheorie nach Wittgenstein}	

\noindent Im Gegensatz zu der oben beschriebenen klassischen Bedeutungstheorie wird die Bedeutung eines Wortes in der Gebrauchstheorie nicht durch eine feste Definition festgelegt, sondern durch ihren Gebrauch in der Sprachgemeinschaft.\footnote{"`Die Bedeutung eines Wortes ist sein Gebrauch in der Sprache."' \citep[PU \S 43]{wittgenstein1963tractatus}.}

Dies hat unter anderem den Vorteil, den im allt"aglichen Sprachgebrauch vorkommenden Sprachwandel als zwangsl"aufigen Teil einer Sprache in die Theorie zu integrieren. Desweiteren l"ost sie das oben vorgestellte Problem der zu rigiden Definition von "`Gem"usesalat"' indem gesagt wird, dass sich die Bedeutung des Wortes aus dessen regelm"a"sigem Gebrauch \emph{zeigt}. Wenn es also typisch ist zu sagen, dass der Tomatensalat ein Gem"usesalat ist, dann folgt daraus, dass die Bedeutung von "`Gem"usesalat"' eine Kategorie ist, unter die der Tomatensalat f"allt. 

F"ur Wittgenstein ergibt sich aus der nat"urlichen Vielfalt der Gebrauchsformen zudem, dass eine Bedeutungstheorie nicht systematisierbar sei. Beispielhaft daf"ur sei das Wort "`Spiel"', f"ur das es unm"oglich zu sein scheint, eine Definition zu finden, die alle als Spiel anzusehende Aktivit"aten einschlie"st.\footnote{Vgl. \cite[PU \S 66]{wittgenstein1963tractatus}.} Jeder Versuch, eine Bedeutungstheorie f"ur eine Sprache durch Definitionen von Wortbedeutungen samt einer Aufbautheorie f"ur Satzbedeutungen zu erhalten, sei also zum Scheitern verurteilt. 

%auch das antisystematische hier erläutern

\vspace{5mm}
\noindent\textbf{$(iii)$ Systematischer Spracherwerb}	

\noindent Wenn ich nun eine weitere Sprache zu meiner Muttersprache dazulernen m"ochte, w"are ich nach Wittgensteins Ansatz also erst dann dazu in der Lage, wenn ich den allt"aglichen Gebrauch nachverfolge und mir einpr"age. 

Aus Erfahrung wei"s ich, dass das zum Beispiel mit besonders rituellen "Au"serungen wie "`Hallo"' gut funktioniert. Wenn ich mir daraufhin aber den Rest einer Unterhaltung anh"ore, k"ame ich nicht sonderlich weit. Ich w"are am besten beraten, mir ein Anf"angerbuch f"ur die zu lernende Sprache zu besorgen und dieses durchzuarbeiten. Dort werden mir dann zum Beispiel Bilder von Gegenst"anden mit der entsprechenden Vokabel darunter gezeigt.\footnote{Ein Extremfall daf"ur sind die ersten Stufen des Konzepts der \emph{Rosetta Stone} Sprachlernsoftware. Ein Vergleich mit der primitiven Sprache aus \cite[PU \S 2]{wittgenstein1963tractatus} liegt zu solch einem fr"uhen Stadium auch nahe.} Danach lerne ich die Grundregeln der Grammatik und baue dann meine ersten validen S"atze zusammen. Im weiteren Verlauf solch eines Programms w"urde ich immer mehr Vokabeln und alle Regeln der Grammatik lernen, und bei einem guten Verst"andnis der Sprache ankommen. 

Innerhalb dieses Prozesses ist es f"ur mich tats"achlich nicht von N"oten, je eine echte Unterhaltung in der Sprache geh"ort zu haben, dank eines stabilen und verl"asslichen \emph{Kerns von Satzaufbauregeln und Wortbedeutungen} innerhalb der Sprache. Es zeigt sich also, dass es wohl doch m"oglich und hilfreich zu sein scheint, Sprachen und ihre Bedeutungstheorien zu systematisieren, selbst wenn es nicht auf komplette Weise gelingen mag. Die oben genannte Ansicht Wittgensteins, Bedeutungstheorien f"ur Sprachen seien aufgrund der Vielfalt des Wort- und Grammatikgebrauchs nicht systematisierbar, kann daher nicht vollst"andig stimmen.

\vspace{5mm}
\noindent\textbf{$(iv)$ Synthese}	

\noindent Aus diesem Grund halte ich es f"ur sinnvoll, eine Gebrauchstheorie zu erarbeiten, die sowohl die systematisierbaren, als auch die nichtsystematisierbaren Aspekte der Sprachbedeutung in sich einschlie"st. Es muss anerkannt werden, dass gro"se Abschnitte der Grammatik sowie des Vokabulars einer Sprache und deren Zusammenh"ange mit Bedeutung systematisierbar sind und \emph{sein m"ussen}, da ansonsten das Lernen von Sprachen unm"oglich w"are. 

Wittgenstein lag vollkommen richtig in der Erkenntnis, dass Sprachen so komplex sind, dass eine \emph{vollst"andige} vereinheitlichende Bedeutungstheorie unm"oglich ist. Nichtsdestotrotz hei"st dies noch lange nicht, dass Linguist*innen und Sprachphilosoph*innen deswegen keine Fortschritte in der systematischen Erarbeitung eines tieferen Verst"andnisses von Sprache erlangen k"onnen.\footnote{Diese zeigte sich zum Beispiel schon wenige Jahrzehnte nach der Ver"offentlichung der Philosophischen Untersuchungen an den Entwicklungen der Sprechakttheorie.} 

\end{onehalfspace}

\bibliography{sprachphilo-ha-3.bib}

\end{document}
