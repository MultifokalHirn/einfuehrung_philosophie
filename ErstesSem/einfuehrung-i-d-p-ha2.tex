\documentclass[a4paper]{article}
\usepackage{graphicx}
\usepackage{fullpage}
%\usepackage{parskip}
\usepackage{color}
\usepackage[ngerman]{babel}
\usepackage{hyperref}
\usepackage{calc} 
\usepackage{enumitem}
\usepackage{titlesec}
%\pagestyle{headings}

\titleformat{name=\section,numberless}
  {\normalfont\Large\bfseries}
  {}
  {0pt}
  {}
\date{\vspace{-3ex}}
\begin{document}

\title{
	\includegraphics*[width=0.52\textwidth]{images/hu_logo.png}\\
	\vspace{10pt}
	Einf"uhrung in die Philosophie}
\author{Lennard Wolf\\
        \href{mailto:lennard.wolf@student.hu-berlin.de}{lennard.wolf@student.hu-berlin.de}}
\maketitle


\section*{AB 2: Das Schiff des Theseus}
\large
\textbf{(1)}
\begin{description}[leftmargin=!,labelwidth=\widthof{\bfseries P22}]
  \item[P1] $T$ bedeutet$_{F}$ genau \emph{ein} Schiff.
  \item[P2] $X$ und $Y$ bedeuten$_{F}$ \emph{zwei} einzelne Schiffe.
  \item[P3] Die Bedeutungen$_{F}$ von $X$ und $Y$ sind nicht identisch.
  \item[K] $X$ und $Y$ k"onnen nicht beide gleichzeitig Schiff $T$ bedeuten$_{F}$\footnote{Mit `bedeuten$_{F}$' ist in dieser Hausarbeit `bedeuten \emph{im Sinne Freges}' gemeint, also dass die Bedeutung eines Eigennamens ein definitives Objekt in der objektiven Realit"at ist. }.
\end{description}
\vspace{8pt}
\textbf{(2)}

Im Folgenden werde ich zeigen, dass das von Rosenberg angef"uhrte Gegenargument zu der These, dass weder $X$ noch $Y$ Schiff $T$ bedeuten$_{F}$, nicht schl"ussig ist. 

Welches das Schiff des Theseus ist, h"angt von Theseus selbst ab, denn `Schiff des Theseus' ist nur ein Name f"ur jenes Schiff, welches Theseus f"ur sein eigenes h"alt. Dieser Name kann f"ur eine gewisse Zeit nichts bedeuten$_{F}$, da z.B. Theseus sich in dieser Situation selbst nicht sicher sein k"onnte, welches der beiden denn nun \emph{sein} Schiff ist. In solch einem Moment w"urden tats"achlich weder $X$ noch $Y$ Schiff $T$ bedeuten$_{F}$, und Schiff $T$ w"are nicht \emph{verschwunden}, sondern h"atte tempor"ar einfach nur keine Bedeutung$_{F}$.

Entsprechend ist das von Rosenberg angef"uhrte \emph{R"atsel} nur eine sprachliche Verwirrung und der Vorschlag, dass weder $X$ noch $Y$ Schiff $T$ bedeuten$_{F}$, eine valide M"oglichkeit.

\end{document}
