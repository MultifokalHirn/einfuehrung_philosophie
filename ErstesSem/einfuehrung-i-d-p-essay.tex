\documentclass[a4paper, emulatestandardclasses, 12pt]{scrartcl}
\usepackage{graphicx}
\usepackage{fullpage}
%\usepackage{parskip}
\usepackage{color}
\usepackage[ngerman]{babel}
\usepackage{hyperref}
\usepackage{calc} 
\usepackage{enumitem}
\usepackage{titlesec}
%\pagestyle{headings}
\usepackage{setspace} %halbzeilig
\usepackage[authoryear,round]{natbib}
\bibliographystyle{natdin}

\date{\vspace{-3ex}}
\begin{document}

\title{%\vspace{5ex}
	\includegraphics*[width=0.1\textwidth]{images/hu_logo2.png}\\
	\vspace{50pt}
	\scshape\LARGE{Einw"ande zu Frank Jacksons "`Epiphenomenal Qualia"'}}
	
\author{Lennard Wolf\\
        \small{\href{mailto:lennard.wolf@student.hu-berlin.de}{lennard.wolf@student.hu-berlin.de}}}      
%Abstract?

\maketitle

\vspace{\fill}

\begin{minipage}[b]{\textwidth}
    \centering
    \onehalfspacing
    \large   
    %30. November 2016\\
    Einf"uhrung in die Philosophie\\
    Wintersemester 2016/2017

    \vspace{-20mm} 
\end{minipage}%
\thispagestyle{empty}
\newpage
\clearpage
\setcounter{page}{1}

\begin{onehalfspace} 


\noindent\textbf{$(i)$ Einleitung}

\noindent In dieser Arbeit m"ochte ich zwei Einw"ande zu Frank Jacksons Argumentation gegen den Physikalismus darlegen. Der erste bezieht sich auf das Wesen der Information, die eine Person erlangt, wenn sie das erste Mal Farben erblickt. Der zweite auf die von Jackson angef"uhrte Schlussfolgerung, dass solche Information ein Beweis f"ur das Existieren von Nichtphysikalischem ist. Mein Ziel ist es hierbei also nicht, eine "`wahre"' Antwort zum Leib-Seele-Problem zu erarbeiten, sondern vielmehr zu zeigen, dass der dualistische Standpunkt von Jackson in seiner Argumentation zu vorschnell erreicht wird.
\vspace{5mm}

\noindent\textbf{$(ii)$ Jacksons Argumentation}

\noindent In seinem Essay "`Epiphenomenal Qualia"' \citep{jackson1982epiphenomenal} argumentiert Frank Jackson gegen den \emph{Physikalismus}, der besagt, dass es nichts gibt, das nicht physikalisch sei. Jackson stellt zu Beginn fest, dass aus dem Physikalismus folgen m"usse, dass alle \emph{richtigen} Informationen "uber die Welt sich ausschlie"slich auf etwas physikalisches beziehen muss. Durch ein Gedankenexperiment versucht er daraufhin zu zeigen, dass es jedoch richtige Informationen gibt, die sich \emph{nicht} auf etwas physikalisches beziehen, wodurch "uber den Modus tollens gezeigt w"are, dass der Physikalismus nicht stimmen kann. 

Das Gedankenexperiment verl"auft folgenderma"sen: Mary ist eine Wissenschaftlerin die sich auf Farbforschung spezialisiert hat, jedoch in ihrem Leben noch nie tats"achlich Farben gesehen hat. Durch ihre lange und gr"undliche Forschung konnte sie \emph{alle} physikalischen Informationen "uber Farben erlangen. Doch wenn Mary zum ersten Mal Farben selber sieht, so Jackson, lerne sie doch noch etwas dazu. Diese neue und richtige Information k"onne sich entsprechend nicht mehr auf etwas physikalisches beziehen, wodurch der Physikalismus als falsch zu betrachten sei. Indem er also sagt, dass eine monistische Sicht auf das Leib-Seele-Problem unm"oglich sei, folgt f"ur ihn daraus eine Form des Substanzdualismus, n"amlich der \emph{Epiph"anomenalismus}.


\vspace{5mm}
\noindent\textbf{$(iii)$ Erster Einwand}

\noindent Jackson irrt sich meiner Meinung nach in der Annahme, dass Mary beim ersten Erblicken von Farben tats"achlich etwas \emph{"uber Farben} lernt. Vielmehr denke ich, dass sie etwas \emph{"uber sich selbst} lernt, n"amlich wie f"ur sie die subjektive Erfahrung ist, wenn bestimmte mit Farben in Verbindung gebrachte Wellenl"angen von Licht auf ihre Retina treffen. Dies m"ochte ich veranschaulichen, indem ich das Gedankenexperiment um Mary auf das Feuer "ubertrage. 

Feuer hat aus bestimmten Gr"unden an bestimmten Stellen eine bestimmte Temperatur. Dies ist eine der vielen physikalischen Information "uber das Feuer, die Mary nat"urlich kennen w"urde. Wenn sie nun aber das erste Mal ihre Hand ins Feuer legt und den Schmerz sp"urt, dann lernt sie nicht, dass das Feuer hei"s \emph{ist}, sondern dass es sich hei"s \emph{anf"uhlt}, denn "`hei"s"' ist ein relativer Begriff, der nur in Beziehung zu etwas anderem eine Bedeutung hat.

Man muss daher, so scheint es mir, unterscheiden zwischen Informationen "uber solche Eigenschaften, die das Feuer wie die Farben objektiv, das hei"st messbar, haben und jene, die man ihnen aufgrund von sinnlicher Erfahrung zuschreibt. Informationen letzterer Art beziehen sich meiner Meinung nach aber nicht mehr nur auf das betreffende Ding als solches, sondern darauf, \emph{wie etwas ist}. Es er"offnet sich hier nun die Frage, ob denn eine Information dieser Art, zum Beispiel dass Feuer hei"s ist, auch eine \emph{richtige} Information ist, was sie ja sein muss, um f"ur Jackson argumentativ relevant zu sein. Um \emph{richtig} sein zu k"onnen, muss eine Information aber \emph{propositionalen} Charakter haben, und bei erlerntem Wissen dar"uber, \emph{wie etwas ist}, ist dies gerade nicht der Fall. Und daher, dass Mary eben nur neue Informationen dieser Art erlangt, hat sie keine neuen \emph{Fakten} "uber Feuer und Farbe gelernt. 

Sollte Jackson gemeint haben, dass die neuen Informationen gar nicht nur Fakten "uber die Eigenschaften von Farben sein sollten, dann w"are sein Argument nicht mehr schl"ussig, da dann angenommen werden m"usste, dass zwangsl"aufig alles sinnlich wahrgenommene nichtphysikalische Informationen erzeugt. Er w"urde dann also nichtphysikalische Informationen beweisen, indem er sie annimmt.

\vspace{5mm}
\noindent\textbf{$(iv)$ Zweiter Einwand}

\noindent Desweiteren bin ich nicht der Auffassung, dass sich aus Informationen der Art, wie Jackson sie sich vorstellt, schlie"sen l"asst, dass es Dinge gibt, die nicht physikalischer Natur sind.

Damit diese Schlussfolgerung m"oglich ist, muss es \emph{ausgeschlossen} sein, dass diese neuen Informationen sich auf etwas physikalisches beziehen, da sie ansonsten aus der Luft gegriffen ist. Doch hier sto"sen wir an eine Wand: Es ist, so scheint es mir, derzeit weder zeigbar, dass es ausgeschlossen ist, noch dass es der Fall sein k"onnte. Vielmehr wird hier meist mit einer ideologischen Voreingenommenheit argumentiert. Wie Thomas Nagel schon in seinem weitreichend rezipierten Essay "`What Is it Like to Be a Bat?"' \citep{nagel1974like} feststellte, ist es f"ur uns unm"oglich, unsere eigene Subjektivit"at zu verlassen, um eine andere Subjektivit"at zu erfahren (wie zum Beispiel die einer Fledermaus), geschweige denn um einen objektiven Standpunkt einzunehmen. Entsprechend k"onnen wir weder erfahren wie es ist, etwas anderes als wir selbst zu sein, noch k"onnen wir mit den objektiven Beschreibungsmethoden der Naturwissenschaften eine subjektive Erfahrung vermitteln. 

Die gesamte Debatte um das Leib-Seele-Problem kreist um die Frage nach dem Zusammenhang von Materie und subjektiver Erfahrung, doch genau diese Frage ist, wie Nagel zeigt, derzeit nur mit reiner Spekulation zu beantworten. Die oben unterstellte Annahme Jacksons, dass es ausgeschlossen sei, dass die subjektive Erfahrung von Farbe in physikalischen Dingen und Prozessen verankert ist, ist eine solche reine Spekulation. Ich k"onnte das Gegenteil genauso wenig begr"unden, weshalb ich mich davor h"ute. Es zeigt sich daher, dass f"ur die Schlussfolgerung von der Andersartigkeit der subjektiven Information von der objektiven (messbaren) Information auf die Existenz nichtphysikalischer Dinge ebendiese angenommen werden m"ussen. 


\vspace{5mm}
\noindent\textbf{$(v)$ Konklusion}

\noindent Ich habe mit meinen beiden Einw"anden versucht zu zeigen, dass Jackson zu voreilig und durch reine Spekulation auf einen Substanzdualismus schlie"st. Die gesamte Debatte erscheint mir wie die nach Gott: Entweder man nimmt die Existenz nichtphysikalischer Dinge an, oder man unterstellt ihre Nichtexistenz. Beide Standpunkte sind von ihrer Natur her nicht beweisbar, da es unm"oglich ist, nichtphysikalische Dinge zu zeigen, wie auch zu zeigen, dass es sie nicht gibt -- also analog zu der Frage nach der Existenz Gottes. 

Es scheint mir daher so, dass Jackson hier einen Beitrag zu einer Scheindebatte geliefert hat, in der auch die stringenteste Argumentation nicht zu einem gewinnbringenden Ergebnis f"uhren kann. Mein pers"onliches Fazit ist folglich, mich mit dieser Debatte nicht weiter auseinanderzusetzen.

%Es stellt sich f"ur mich daher die Frage, weshalb bis heute Philosophen "uber dieses Thema streiten, wenn doch ihre noch so gut verkleideten Argumentationen am Ende wieder nur auf pers"onlichen Meinungen fu"sen, die selbst nur auf Spekulationen aufgebaut sind. Ich bezweifle dass in diesen Diskussionen irgendwer jemals irgendwen "uberzeugen kann, und wenn dann nur weil die Meinung als besonders "`logisch"' daherkommt.



\end{onehalfspace}
%\nocite{*}
\bibliography{einfuehrung-i-d-p-essay}

\end{document}

%“Dennett even composes his own weird science fiction scenario in order to reinforce his point. He proposes that, when Mary leaves her room, somebody tries to trick her by showing her a banana that is painted blue. The trickster hopes that, since Mary knows from her readings that bananas are supposed to be yellow, she will mistake the qualitative feel of blue for that of yellow. But Dennett insists that Mary cannot be fooled, because she already knows, “in exquisite detail, exactly what physical impression a yellow object or a blue object (or a green object, etc) would make on [her] nervous system”.”

%Excerpt From: Steven Shaviro. “Discognition.” iBooks. 


%\vspace{2mm}
%\noindent Mein erster Einwand richtet sich gegen Annahmen, die dem Gedankenexperiment zugrundeliegen und der zweite gegen die Schlussfolgerung von dem Szenario auf die Existenz nichtphysikalischer Dinge. 

%Bluetooth ist ein standardisiertes Funkverfahren im UHF-Frequenzbereich. Mit unseren derzeitigen sinnlichen M"oglichkeiten sind wir nicht in der Lage, Bluetooth Verbindungen wahrzunehmen. Da diese Technologie von Menschen entwickelt wurde k"onnen wir uns aber sicher sein, dass uns alle physikalischen Informationen "uber Bluetooth zug"anglich sind: Wie die Funkwellen erzeugt werden, welche Frequenz sie haben, wie das Kommunikationsprotokoll aussieht und so weiter. Man stelle sich nun vor, dass ein Bluetooth-Empf"anger entwickelt wird, der Menschen eingepflanzt werden kann und sich mit ihrem Nervensystem verbindet, um Bluetooth \emph{erfahrbar} zu machen. 
