\documentclass[twoside,twocolumn]{book}
\usepackage{graphicx}
\usepackage{color}
\usepackage{hyperref}
\usepackage{german}
\usepackage{calc} 
\usepackage{enumitem}

\usepackage[utf8]{inputenc}
\usepackage{dici}
%\pagestyle{headings}

\newcommand{\todo}[1]{\textcolor{red}{TODO: #1}\PackageWarning{TODO:}{#1!}}

\begin{document}

\title{
   \vspace{-70pt}
	\includegraphics*[width=0.75\textwidth]{images/hu_logo.png}\\
	\vspace{24pt}
	Philosophisches W"orterbuch}
\author{Lennard Wolf\\
        \href{mailto:lennard.wolf@student.hu-berlin.de}{lennard.wolf@student.hu-berlin.de}}
\maketitle

\textbf{Anmerkung des Autors:} 

Definitionen können zum Teil oder komplett aus Internetquellen (zumeist der Wikipedia) direkt oder editiert übernommen worden sein. Daher erhebe ich in diesem Werk \emph{keinen} Anspruch auf originär auktorale Arbeit. Es handelt sich ausschließlich um eine Begleitarbeit zum Studium. Der Übersicht halber werden Quellenangaben auch unterlassen. 

Sollte sich die Person, von der ein sich in diesem Text befindlichen Textstück ursprünglich verfasst wurde, daran st"oren, bitte ich diese Person mich über die auf der ersten Seite angegebene e-mail Adresse zu kontaktieren.

\begin{dictionary}
\bigletter{A}
\term{Arbeit (philosophische Kategorie)}{Erfasst alle Prozesse der bewussten schöpferischen Auseinandersetzung des Menschen mit der Natur und der Gesellschaft. Sinngeber dieser Prozesse sind die selbstbestimmt und eigenverantwortlich handelnden Menschen mit ihren individuellen Bedürfnissen, Fähigkeiten und Anschauungen im Rahmen der aktuellen Naturgegebenheiten und gesellschaftlichen Arbeitsbedingungen.}

\bigletter{B}

\bigletter{C}
\term{Cartesianismus}{Die Philosophie nach Ren\'{e} Descartes (1596-1650), welche rationalistisches Denken progapagiert}
\term{Cartesianischer Dualismus}{Nach Ren\'{e} Descartes (1596-1650); Lehrt die Existenz zweier miteinander wechselwirkender, voneinander verschiedener \emph{Substanzen} – Geist und Materie. Im Gegensatz dazu siehe \emph{Monismus} und \emph{Newtons dualistische Naturphilosophie}.}
\term{Ceteris paribus}{ "`unter sonst gleichen Bedingungen"'. Es spielt eine große Rolle bei Experimenten. Will man herausfinden, wie eine erste Variable eine zweite beeinflusst, schaut man sich mehrere Situationen an, in denen beide auftauchen.}

\bigletter{D}
\term{\emph{de dicto}-Lesart}{"`Hans glaubt, dass der Morgenstern ein Planet ist."' $\rightarrow$ Hans glaubt etwas, das sich so sagen l"asst: Der Morgenstern ist ein Planet.}
\term{\emph{de re}-Lesart}{"`Hans glaubt, dass der Morgenstern ein Planet ist."' $\rightarrow$ Hans glaubt \emph{von dem Morgenstern} etwas, das sich so sagen l"asst: \emph{er} ist ein Planet.}
\term{Deontologische Ethik}{}
\term{Dilemma}{Lemma mit zwei H"ornern. Siehe \emph{Lemma}}
\term{Dualismus}{Position, dass sich alle Phänomene der Welt auf zwei einander ausschließende Grundprinzipien/Entitäten/Substanzen zurückführen lassen. Beispiele sind \emph{Newtons dualistische Naturphilosophie}, sowie \emph{Cartesianischer Dualismus}. Im Gegensatz dazu siehe \emph{Monismus} und \emph{Pluralismus}.}

\bigletter{E}
\term{Empirismus}{Erkenntnistheoretische Richtung, die als Quelle der Erkenntnis allein die Sinneserfahrung, die Beobachtung, das Experiment gelten l"asst.}
\term{Entit"at}{}
\term{Ethik}{}
\term{Extension (Semantik)}{Die Gesamtheit der Dinge, auf die er sich erstreckt (die unter ihn fallen, die er umfasst). Demnach ist die Extension des Begriffes "`Mensch"' die Gesamtheit aller Menschen. Im Gegensatz dazu siehe \emph{Intension (Semantik)}}

\bigletter{F}
\term{Falsifikation}{Der Nachweis der Ungültigkeit einer Aussage, Methode, These, Hypothese oder Theorie. Im Gegensatz dazu siehe \emph{Positivismus} mit der Verifizierung.}
\term{Faust}{}

\bigletter{G}
\term{Genealogie}{Eine historische Methode in den Geisteswissenschaften, welche die geschichtliche Entwicklung verschiedener Sachverhalte der Gegenwart untersucht.}

\bigletter{H}
\term{Hamlet (Shakespeare)}{Claudius, der Bruder K"onig Hamlets, ermordet den Herrscher, rei"st die Krone an sich und heiratet Gertrude, die Witwe des K"onigs. Prinz Hamlet strebt danach, seinen Vater zu r"achen, und st"urzt dabei alle Beteiligten ins Ungl"uck.}
\term{Hiob}{}
\term{Hiobsbotschaft}{}

\bigletter{I}
\term{Intension (Semantik)}{Die Intension eines Begriffes besteht aus der Gesamtheit der Merkmale oder Eigenschaften, die den Dingen, die er umfasst, faktisch gemeinsam sind. Demnach enth"alt die Intension des Begriffes "`Mensch"' die Merkmale belebt, sterblich, auf zwei Beinen gehend, ungefiedert, vernunftbegabt, Werkzeuge produzierend etc. Im Gegensatz dazu siehe \emph{Extension (Semantik)}}
\term{Induktionsproblem}{Aus einer endlichen Menge an Beobachtungen kann man auf keine Allaussagen schlie"sen.}

\bigletter{K}
\term{Konsequentialismus}{Besagt: Eine Handlung $H$ ist genau dann moralisch richtig, wenn sie die im Vergleich mit allen Alternativen besten Konsequenzen nach sich zieht.}
\term{Kontingenz}{Etwas ist kontingent genau dann, wenn es nicht notwendig ist. }
\term{Kontextprinzip (Frege)}{Das Prinzip besagt dass Begriffe nur im Zusammenhang eines Satzes etwas bedeuten. So erh"alt etwa der Begriff "`Stein"' erst eine Bedeutung, wenn er im Elementarsatz "`$x$ ist ein Stein"' auftritt.}
\term{Kritischer Rationalismus}{Eine von Karl Popper begründete philosophische Denkrichtung. Popper beschreibt ihn als Lebenseinstellung, „die zugibt, dass ich mich irren kann, dass du recht haben kannst und dass wir zusammen vielleicht der Wahrheit auf die Spur kommen werden“. Kennzeichnend ist ein vorsichtig optimistischer Blickwinkel auf Leben und Dinge.}

\bigletter{L}
\term{Lemma}{Hilfssatz (wahrer, bewiesener Satz, der aber kein Axiom ist) | Wird meist als Pr"amisse genommen, um die Beweisf"uhrung zu verk"urzen}
\term{Litotes}{Stilfigur der doppelten Verneinung (z. B. "`nicht unüblich"') oder der Verneinung des Gegenteils (z. B. "`nicht selten"'). Damit kann zum Beispiel eine Behauptung vorsichtig ausgedrückt oder eine Aussage abgeschwächt werden (Untertreibung). Aber auch eine Hervorhebung kann indirekt bewirkt werden. }

\bigletter{M}
\term{Metaphysik}{Metaphysische Systementwürfe behandeln in ihren klassischen Formen die zentralen Probleme der theoretischen Philosophie, nämlich die Beschreibung der Fundamente, Voraussetzungen, Ursachen oder „ersten Begründungen“, der allgemeinsten Strukturen, Gesetzlichkeiten und Prinzipien sowie von Sinn und Zweck der gesamten Realität bzw. allen Seins.}
\term{Monismus}{Position, dass sich alle Phänomene der Welt auf ein Grundprinzip zurückführen lassen (steht dem Pantheismus nah). Im Gegensatz dazu siehe \emph{Dualismus} und \emph{Pluralismus}}
\term{Moral}{}

\bigletter{N}
\term{Natural Kind Term}{(`Nat"urliche Art Ausdr"ucke') Ausdr"ucke f"ur Substanzen oder Arten von Gegenst"anden, wie sie in unserer nat"urlichen Umwelt vorkommen und von den Naturwissenschaften kategorisiert werden.}
\term{Newtons dualistische Naturphilosophie}{Nach Sir Isaac Newton (1643-1727); Lehrt die Existenz der Wechselwirkung aktiver immaterieller \emph{Kräfte der Natur} mit der absolut passiven Materie. Im Gegensatz dazu siehe \emph{Cartesianischer Dualismus} und \emph{Monismus}.}

\bigletter{O}
\term{Ontologie}{Die Lehre vom Seienden: Was gibt es? Woraus besteht/Was ist die Welt?}
\term{Ontologische Verpflichtung (Quine)}{Man geht bei jeder philosophischen Aussage eine O.V. ein, dass man glaubt, dass es alles von unserer Aussage ber"uhrtem geben muss. [\emph{ontological commitment}]}

\bigletter{P}
\term{Pluralismus}{Position, dass sich alle Phänomene der Welt auf eine Vielzahl einander ausschließender Grundprinzipien/Entitäten/Substanzen zurückführen lassen. Es existieren viele verschiedene Formen des Pluralismus Im Gegensatz dazu siehe \emph{Monismus} und \emph{Pluralismus}.}
\term{"`\emph{Proofs Too Much}"'-Argument}{dfghjk}

\term{Prop"adeutik}{Einführung in die Sprache und Methodik einer Wissenschaft}

\term{Positivismus}{Eine Richtung in der Philosophie, die fordert, Erkenntnis auf die Interpretation von „positiven“ Befunden, Mathematik oder Logik zu beschränken, also solchen, die im Experiment unter vorab definierten Bedingungen einen erwarteten Nachweis erbringen. (\emph{Verifikationsmethode}). Im Gegensatz dazu siehe \emph{Falsifikation}}
\term{Positivismusstreit}{Auseinandersetzung über Methoden und Werturteile in den Sozialwissenschaften, 60er Jahre im deutschen Sprachraum. Kernpositionen: Adorno vertrat das Konzept der Totalität, Popper den Ansatz des Kritischen Rationalismus}

\bigletter{S}
\term{Sollen}{Sollen impliziert Können}
\term{Sprachakt}{Eine Handlung, die durch Sprache vollzogen wird. ("`Hiermit erkl"are ich sie zu Mann und Frau."')}
\term{Suum Cuique}{}


\bigletter{T}
\term{Tatsache (Wittgenstein)}{Das Bestehen von Sachverhalten. Etwas das der Fall ist. Die Welt ist alles was der Fall ist.}
\term{Tugendethik}{ [\emph{virtue ethics}]}

\bigletter{U}
\term{Urteil (Logik)}{Form einer Feststellung, die in der sprachlichen Form eines Satzes ausgedrückt wird. Dabei wird das Urteil mit dem Vorgang der Bildung der Feststellung, ihrem propositionalen Gehalt oder der Bewertung dieses Gehalts identifiziert.}

\bigletter{V}
\term{Verfremdung (Hegel)}{}

\bigletter{W}
\term{Wissenschaftstheorie}{[auch: \emph{Wissenschaftsphilosophie}]}
\term{Warum-Frage}{Fragt nacht Erkl"arungen oder Rechtfertigungen (Context of Discovery/Context of Justification)}

\bigletter{X}
\term{}{}

\bigletter{Y}
\term{}{}

\bigletter{Z}
\term{}{}

\end{dictionary}
\end{document}
