\documentclass[a4paper]{article}
\usepackage{graphicx}
\usepackage{fullpage}
\usepackage{parskip}
\usepackage{color}
\usepackage[ngerman]{babel}
\usepackage{hyperref}
\usepackage{calc} 
\usepackage{enumitem}
\usepackage{titlesec}
%\pagestyle{headings}

\titleformat{name=\section,numberless}
  {\normalfont\Large\bfseries}
  {}
  {0pt}
  {}

\begin{document}

\title{
	\includegraphics*[width=0.6\textwidth]{images/hu_logo.png}\\
	\vspace{24pt}
	Einf"uhrung in die Philosophie}
%\subtitle{Vorlesung WS 16/17\\
%          Prof. Template\\
%          Philosophisches Institut I \\ 
%          Humboldt Universit"at zu Berlin}
\author{Lennard Wolf\\
        \href{mailto:lennard.wolf@student.hu-berlin.de}{lennard.wolf@student.hu-berlin.de}}
\maketitle


\section*{AB 1b: Argumentrekonstruktion}
\large
Im Folgenden werde ich zuerst aufzeigen, was das Kernproblem mit dem ontologischen Argument des Anselm von Canterbury ist, und anschlie\ss end werde ich darauf aufbauend die drei gegebenen Rekonstruktionen in ihrer Form und "Uberzeugungskraft vergleichen.

Rekonstruktion $B$ deutet durch ihren Minimalismus schon den fragw"urdigen Kern des Arguments an: Im Grunde wird gesagt, \emph{Etwas ist nur vollkommen, wenn es existiert}, und da Gott per Definition vollkommen ist, existiert er daraus folgend. So ausgedr"uckt erkennt man schnell, dass Anselms Argument nichts weiter aussagt, dass die Existenz Gottes durch seine Definition zu beweisen ist, und somit wurde kein wirklicher Beweis betrieben, sondern nur eine Tautologie verschleiert.

Der Unterschied zwischen den drei Rekonstruktionen des Arguments ist, wie stark die jeweilige eben diese Verschleierung betrieben hat. $B$ ist in Form eines direkten Beweises und ist somit zum einen die am einfachsten verst"andliche, aber eben auch am leichtesten zu entlarven. (1') und (2') lassen sich ganz leicht zu \emph{Etwas ist nur vollkommen, wenn es (notwendigerweise) existiert} zusammenfassen, und das daraus folgende (3') ist, wenn man die Pr"amissen f"ur g"ultig h"alt, zwangsl"aufig wahr. Das dies nicht sonderlich "uberzeugend ist wurde oben gezeigt.

$A$ und $C$ sind in Form von indirekten Beweisen, da sie das Gegenteil (\emph{Gott existiert nicht}) zu widerlegen versuchen. $C$ ist an sich identisch zu $A$, denn (2''), (3'') und (4'') sind zusammengefasst inhaltlich gleich mit (2), nur sind sie genauer und k"onnten als lesefreundlicher bezeichnet werden. 

Da jene Rekonstruktion, die am st"arksten verschleiert im Grunde als die \emph{"uberzeugendste} gewertet werden kann, w"are $A$ hier der klare Favorit. (2) ist nicht nur unangenehm formuliert, es ist durch seine Verschleierung auch so krude, dass es der lesenden Person schwer macht, auch nur dar"uber nachzudenken. Daher steht meiner Meinung nach das Argument in dieser Rekonstruktion am besten dar. 


\end{document}
