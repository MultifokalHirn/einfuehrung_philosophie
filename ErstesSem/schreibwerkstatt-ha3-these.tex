\documentclass[a4paper]{article}
\usepackage{graphicx}
\usepackage{fullpage}
%\usepackage{parskip}
\usepackage{color}
\usepackage[ngerman]{babel}
\usepackage{hyperref}
\usepackage{calc} 
\usepackage{enumitem}
\usepackage{titlesec}
\usepackage[export]{adjustbox}
%\pagestyle{headings}

\titleformat{name=\section,numberless}
  {\normalfont\Large\bfseries}
  {}
  {0pt}
  {}
\date{\vspace{-3ex}}
\begin{document}

\title{
    \vspace{-30pt}
	\includegraphics*[width=0.1\textwidth,left]{images/hu_logo2.png}\\
	\vspace{-10pt}
	Philosophische Schreibwerkstatt}
\author{Lennard Wolf\\
        \small{\href{mailto:lennard.wolf@student.hu-berlin.de}{lennard.wolf@student.hu-berlin.de}}}
\maketitle
\vspace{0pt}

\section*{Essaythema}
\large
\vspace{2pt}
\textbf{Textgrundlage}

Frankfurt, Harry G.: Alternative Handlungsm"oglichkeiten und moralische Verantwortung. In: \emph{Freiheit und Selbstbestimmung: ausgew"ahlte Texte}. Walter de Gruyter GmbH \& Co KG, 2001.\\

\noindent \textbf{Inhalt}

In \emph{Alternative Handlungsm"oglichkeiten und moralische Verantwortung} versucht Frankfurt das \emph{principle of alternative possibilities} (PAP), nach welchem Handlungsalternativen Vorraussetzung f"ur moralische Verantwortlichkeit sind, durch ein gedankenexperimentelles Gegenbeispiel in Frage zu stellen. Er verfolgt damit das Ziel, das klassische inkompatiblistische Argument, dass es keine moralische Verantwortung in einer determinierten Welt gibt, zu entkr"aften, da es PAP als Pr"amisse hat.\\

\noindent \textbf{Essay}

Ich habe vor zu diskutieren, ob Frankfurts Argument die Indeterminiertheit der Welt implizit vorraussetzt. Darauf aufbauend plane ich eine kontextualistische Gegen"uberstellung der verschiedenen Interpretationen der Kernbegriffe der Diskussion um Kompatibil"at von Willensfreiheit und Determinismus um zu zeigen, dass Texte wie Frankfurts m"oglicherweise auf einer ganz anderen gedanklichen Ebene verlaufen als es die Inkompatibilisten verstehen (wollen).

\end{document}
