\documentclass[a4paper]{article}
\usepackage{graphicx}
\usepackage{fullpage}
%\usepackage{parskip}
\usepackage{color}
\usepackage[ngerman]{babel}
\usepackage{hyperref}
\usepackage{calc} 
\usepackage{enumitem}
\usepackage{titlesec}
\usepackage{bussproofs}
\usepackage[export]{adjustbox}
%\pagestyle{headings}

\titleformat{name=\section,numberless}
  {\normalfont\Large\bfseries}
  {}
  {0pt}
  {}
\date{\vspace{-3ex}}
\begin{document}

\title{
    \vspace{-30pt}
	\includegraphics*[width=0.1\textwidth,left]{images/hu_logo2.png}\\
	\vspace{-10pt}
	Einf"uhrung in die Philosophie}
\author{Lennard Wolf\\
        \small{\href{mailto:lennard.wolf@student.hu-berlin.de}{lennard.wolf@student.hu-berlin.de}}}
\maketitle
\vspace{0pt}

\section*{AB 9: Englische Texte lesen / Argumente rekonstruieren}
\large

%\vspace{10pt}
\noindent\textbf{(ii)}\\
\noindent Jackson zufolge besagt der Physikalismus, dass es nichts gibt, das nicht physikalisch ist.\\

\noindent\textbf{(iii)}\\
\begin{description}[leftmargin=!,labelwidth=\widthof{\bfseries M $\rightarrow \neg$P1}]
  \item[P] Es gibt nichts, das nicht physikalisch ist. (Physikalismus)
  \item[P1] Alle m"oglichen richtigen Informationen k"onnen sich nur auf das Physikalische beziehen. 
  \item[P $\rightarrow$ P1] Wenn es nichts gibt, das nicht physikalisch ist, dann k"onnen sich alle m"oglichen richtigen Informationen nur auf das Physikalische beziehen.
  \item[M] Wenn Mary, die alles (physikalische) was es "uber Farben zu wissen gibt wei"s, das erste Mal Farben sieht, lernt sie etwas dazu.
  \item[M $\rightarrow \neg$P1] Wenn Mary etwas dazu lernt, dann gibt es Informationen, die sich nicht auf das Physikalische beziehen.
  \item[$\neg$P1] Es gibt Informationen, die sich nicht auf das Physikalische beziehen. 
  \item[K] Es gibt Dinge, die nicht physikalisch sind. (\emph{$\neg$P, modus tollens})
\end{description}

\begin{prooftree}
  \AxiomC{$P \rightarrow P1$}
  \AxiomC{$M$}
  \AxiomC{$M \rightarrow \neg P1$}
  \BinaryInfC{$\neg P1$}
  \BinaryInfC{$\neg P$}
\end{prooftree}


\noindent\textbf{(iv)}\\
\noindent Ich halte diese Argumentation aus folgendem Grund f"ur nicht "uberzeugend: Nur weil wir eine Information, die sich auf das Qualia bezieht, nicht \emph{ausdr"ucken k"onnen}, hei"st das noch lange nicht, dass sie sich nicht am Ende trotzdem auf etwas Physikalisches bezieht. 

Von der Meinung, dass Informationen dar"uber, \emph{wie etwas ist}, sich nicht am Ende trotzdem auf etwas Physikalisches \emph{bezieht}, bin ich nicht "uberzeugt. Vielmehr unterstelle ich, dass unsere eingeschr"ankten Sinne und Denkf"ahigkeiten uns nur daran hindern, so etwas zu "`denken"' und auszudr"ucken. Ich unterstelle daher, dass Mary eben nicht \emph{alle} physikalischen Informationen "uber Farben kannte, geschweige denn "uberhaupt kennen \emph{konnte}.

\noindent\textbf{(iv)}\\
\noindent a wenn mary beispiel korrekt ist, dann haben alle sachen die wir nicht wahrnehmen können nichtphysikalische eigenschfaften
bluetooth hat keine nichtphysikalischen eigenschaften (wie will mir jemand von nichtphysikalischen reden ohne es mir zeigen zu können -> esoterik)
mary beispiel zieht nicht


\end{document}
