\documentclass[]{scrartcl}
\usepackage{graphicx}
\usepackage{color}
\usepackage{german}
\usepackage{hyperref}
\usepackage{calc} 
\usepackage{enumitem}
%\pagestyle{headings}

% customize dictum format:
\usepackage[T1]{fontenc}
\setkomafont{dictumtext}{\itshape\small}
\setkomafont{dictumauthor}{\normalfont}
\renewcommand*\dictumwidth{\linewidth}
\renewcommand*\dictumauthorformat[1]{--- #1}
\renewcommand*\dictumrule{}
\newcommand{\todo}[1]{\textcolor{red}{TODO: #1}\PackageWarning{TODO:}{#1!}}

\begin{document}

\title{
	\includegraphics*[width=0.75\textwidth]{images/hu_logo.png}\\
	\vspace{24pt}
	Studienbeginn und Orientierung}
\subtitle{Block WS 16/17\\
          Humboldt Universit"at zu Berlin}
\author{Lennard Wolf\\
        \href{mailto:lennard.wolf@student.hu-berlin.de}{lennard.wolf@student.hu-berlin.de}}
\maketitle
\begin{abstract}

Studienprobleme? Ich habe doch gerade erst mit dem Studium angefangen. Aber nachdem die erste Euphorie verflogen ist, kann es wichtig sein zu wissen, welche Lösungen es gibt, wenn es doch mal nicht so l"auft, wie man will. Was mache ich, wenn ich durch den dritten Pr"ufungsversuch falle oder mein Studienfach nicht meinen Erwartungen entspricht? An diesem Tag laden wir Sie zu folgenden Veranstaltungen ein, die Sie auf ein erfolgreiches Studium vorbereiten:

Informationen zum Studium, Studienaufbau von Compass-Tutor -- Erfolgreich studieren - aber wie? -- F"uhrung Campus Mitte -- Vortrag \emph{Stressbew"altigung - Stress erkennen, vermeiden, bew"altigen}

\end{abstract}
\newpage

\tableofcontents
\newpage

\section{Studienorganisation \& CMS}

\subsection{Agnes}
\url{hu.berlin/agnes}
\newline

Vor Semesterbeginn sollte in dem Vorlesungsverzeichnis Agnes eine Belegung ausgetestet werden indem die Kurse \emph{vorgemerkt} werden, dann der Stundenplan angeschaut wird und dann auf \emph{belegen} und dann auf \emph{speichern} geklickt wird.

Leistungsnachweise sind hier einzusehen sowie der resultierende Notenspiegel. Adresse kann hier geändert werden.

\subsection{Moodle}
\url{hu.berlin/moodle}
\newline

Das Moodle ist f"ur Kursmaterialen und -dateien da, d.h. f"ur das Austeilen von Aufgaben. Zudem gibt es kursspezifische Foren. Da "uber das Moodle Mails verschickt werden, sollte die e-mail Adresse hier spezifiziert sein! Kurse werden im Moodle am besten über die Kursnummern gefunden.

Anmelden geht mit dem allgemeinen HU-Account. 

\subsection{Amor}
\url{hu.berlin/amor}
\newline

Oder auch: die allgemeinen Internetdienste (eduroam, VPN etc.).
Mit dem VPN lassen sich jstor, wiley, springerlink etc. erreichen.

\subsection{Unibibliotheken}

Der Studierendenausweis ist gleichzeitig aus der Benutzerausweis f"ur die Bibliotheken. Die Benutzernummer steht entsprechend auf dem Studierendenausweis (HUHS...). Das vorl"aufige Passwort ist die PLZ mit dem ersten Buchstaben des Strassennamens. Lizenzen f"ur die Suche in professionellen DBs nur "uber eduroam oder VPN.

Man sollte sich wenn man online einen Text gefunden hat den man sich ausleihen will, die Signatur aufschreiben bevor man an den Ausleihautomaten in der Bib geht. Deutschlandweite Fernleihe ist m"oglich, mit KOBV berlinweit. Schließf"acher mit Mensacard. Gruppenarbeitsr"aume sind kostenlos reservierbar.
\newpage


\section{Erfolgreich im Studium}

33\% der Studierenden brechen ihr Studium ab. 

\subsection{10 Tipps}

\begin{enumerate}
  \item Wissen
  \item Glück
  \item Freude
\end{enumerate}

\section{Stressbew"altigung}

\begin{itemize}
    \item Thema 1
    \item Thema 2
  \end{itemize}


\section{Neue Begriffe}

\begin{description}[leftmargin=!,labelwidth=\widthof{\bfseries Blockveranstaltung}]
  \item[Blockveranstaltung] Veranstaltung die nur ein bis wenige Male stattfindet und dann in einem Block, zB Samstag von 8 bis 19 Uhr
  \item[ct] \emph{con temporare} -- 15 min sp"ater
  \item[CMS] ???
  \item[MAP] \textbf{M}odul \textbf{A}bschluss \textbf{P}r"ufung
  \item[jstor] ???
  \item[Wiley] ???
  \item[Springerlink] ???
  \item[KOBV] ???
\end{description}



\begin{figure}[h]
	\centering
	\includegraphics[width=0.82\textwidth]{images/studienorga/stress.jpg}
	\caption{Kurs Stressbew"altigung}
	\label{fig:stress}
\end{figure}



\section{Notes to Self}

\begin{itemize}
    \item Mehr als 30 Punkte pro Semester planen
    \item Matrikelnummer lernen
    \item Erstitage: Fragen wann/wie Pr"ufungsanmeldung zu passieren hat
    \item Hauptfach geht immer vor
    \item TANs sind für Pr"ufungsanmeldungen (1 pro 1)
    \item vllt mit Endnote etc. vertraut machen
    \item Erasmusb"uro?
  \end{itemize}



\end{document}
