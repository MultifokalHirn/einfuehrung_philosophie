\documentclass[emulatestandardclasses]{scrartcl}
\usepackage{graphicx}
\usepackage{CJKutf8} % japanese
\usepackage{color}
\usepackage[ngerman]{babel}
\usepackage{hyperref}
\usepackage{fullpage}
\usepackage[utf8]{inputenc}
\usepackage{calc} 
\usepackage{enumitem}
\usepackage{titlesec}
\date{\vspace{-3ex}}
\begin{document}

\title{
	\includegraphics*[bb=0 0 720 200, width=0.72\textwidth]{ErstesSem/images/hu_logo.png}\\
	\vspace{25pt}
	Introduction to Chinese Philosophy}
\subtitle{\vspace{10pt}
			Prof. Dr. Michael Beaney\\
			Proseminar SS 2018\\
          Institut für Philosophie\\ 
          Humboldt Universit"at zu Berlin}
\author{Lennard Wolf\\
        \small{\href{mailto:lennard.wolf@student.hu-berlin.de}{lennard.wolf@student.hu-berlin.de}}}
\maketitle
\begin{abstract}

This course is intended as an introduction to ancient Chinese philosophy, offering an overview of its main schools and exploring some of its main themes. One of our aims will be to demonstrate the relevance of an understanding of ancient Chinese philosophy to contemporary Western philosophy. The lectures will be given in English, but questions can be asked in German.

\end{abstract}
\newpage

\tableofcontents
%\listoffigures
\newpage


\section{What is Chinese Philosophy?\\(18.04.18)}

\begin{itemize}
  \item Moodle-PW: dao
  \item If you do Philosophy of language, you should think about other languages too
  \item The Master said: “Learning without thinking is bewildering; thinking without learning is dangerous.”
  \item Beaney will stir the thinking in the right direction
  \item The relationship between learning and thinking, reading lectures is a theme in this lecture, as well as in Chinese philosophy itself
  \item 12 topics, might change over the course, break in the middle
  \item Big question: How to properly integrate Chinese philosophy into our philocophical practice?
\end{itemize}


\subsection{Why study Chinese philosophy?}

\begin{itemize}
  \item Chinese civilization one of the longest in human history, with a rich cultural and literary tradition
  \item philosophy an important part of this
  \item ancient China far away from our modern Western world, spatio-temporally and culturally (both linguistically and conceptually)
  \item offers fundamental challenges to Western ways of thinking
  \item Chinese philosophy ought to be offered in any philosophy curriculum
  \item not just in "`area studies"'
  \item should be taken seriously a sphilosophy
  \item understanding other cultures as important as understanding "`the world"'
  \item we are now in a better position than ever to explore Chinese philosophy
  \item Primary Text Just came out: Bryan van ? : Taking back philosophy (?)
  \item Author wrote in Blog: If philosophy taught in the west is only about Western philosophy, it should be called "`Western philosophy"'
  \item Chinese philosophy is "`ghettoized"' to area studies: its not to be taken seriously \emph{as} philosophy, but just as some aspect of a foreign culture
  \item Heidegger, Derrida: "`There is no Chinese philosophy, just Chinese thought"'
\end{itemize}

\subsection{Beaneys background to Chinese philosophy}

\begin{itemize}
  \item When he took over journal, half of the journals article were on just 7 philosophers; He asks himself, how to bring history of foreign philosophy
  \item Summer School in PKU for Western philosophy; very influential
  \item Only recently they have opened up to analytic philosophy
  \item Analytic philosophy has been used by both western and chinese scholars to analyze ancient Chinese texts
  \item Big new question for him: translation
  \item Thing that irritates him: Most philosophers take translation for granted
  \item Translation is a great tecnique for properly understanding a text
  \item Huge project to translate Western texts into Chinese, making up new words on the go
  \item Book that opened his eyes : Chad Hansens  \emph{A Daoist Theory of Chinese Thought}
  \item key idea: The dao is basically a rule following practice (rule following is essential to Wittgentseins later philosophy) he reads the whole of Chinese philosophy as a response to the \emph{rule following paradox}
  \item Plan: Break the caricature of Chinese philosophy (that its mostly just sociopolitical talk (confucianism) or some mysticism (daoism))
  \item Last 50 years: work on forgotten, suppressed texts, "`Mohist"' philosophy
  \item Now we have a Neomohist tradition! "`There was analytic philosophy in China, long before Frege \& Wittgenstein."'
  \item How to create experts in Chinese philosophy? Western and Eastern work together
\end{itemize}


\subsection{What is CHinese philosophy}

\begin{itemize}
  \item Confucianism is still very influential, like Christianity in the West
  \item Xi JInping sometimes uses quotes of cinfucius
  \item Daoism and Buddhism (7th century) also important
  \item No distinction between things and concepts (subject - predicate structure)
  \item School of Names: Philosophy of language \& logic
  \item 6 main schools: ()PASTE()
  \item Big 3 confucians (Kongzi, Mengzi, Xunzi) "`zi"' is Master
  \item First utilitarian: mozi (?)
  \item 
  \item 
\end{itemize}

\textbf{Later Traditions}

\begin{itemize}
  \item Chinese Buddhism:
  \item 
  \item Neo Confucianism:
  \item 
\end{itemize}

\subsection{History of China}

\begin{itemize}
  \item 16th c.: Jesuits came to convert, but also translated a lot
  \item Christian Wolff: Lost his job for what he said about Chinese philosophy; according to Wolff you could be a perfectly moral person, even if you didnt believe in god $\rightarrow$ Chinese philosophy is a good example for that | famous lecture in 1721 for which he was suspended as professor (sometimes seen as father of enlightenment because of that)
  \item Hegel: said rather nasty things about the Chinese
  \item Took over the Term "`philosophy"' from tetsugaku
  \item 2 stances "`\emph{Chinese logic is inferior}"' vs. "`\emph{Chinese logic is different}"'
  \item They are now creating a tradition, the history is written, with western categories in mind
  \item Cua: The Emergence of the History of Chinese Philosophy
  \item Hu Shi AN Outline of the History of Chinese Philosphy (Inspired by pragmatism)
  \item Fung Yu-Ian, History of Chonese PHilosophy
  \item Lao Si-guang History of CHinese Philosophy: Criticies other books, as they are too influenced by west;  "`what china lacks, is analytical skills"' but you need analytical tools to understand and construct a history; "`China lacks a modern philosophy"'
\end{itemize}


\subsection{Fung Yu-Ians History of Cionese PHilosophy}

\begin{itemize}
  \item 1. What is nature of Chinese philosophy?
  \item just metaphysics and ethics, rather than logic and epistemology
  \item aim: sagelyness within, and kingliness without
  \item 2. Does Chinese philosophy lack a system?
  \item sd
  \item 3. Is there no growth in Chinese philosophy?
  \item 
\end{itemize}


\end{document}
