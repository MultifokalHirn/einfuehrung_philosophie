\documentclass[emulatestandardclasses]{scrartcl}
\usepackage{graphicx}
\usepackage{CJKutf8} % japanese
\usepackage{color}
\usepackage[ngerman]{babel}
\usepackage{hyperref}
\usepackage{fullpage}
\usepackage[utf8]{inputenc}
\usepackage{calc} 
\usepackage{enumitem}
\usepackage{titlesec}
\date{\vspace{-3ex}}
\begin{document}

\title{
	\includegraphics*[bb=0 0 720 200, width=0.72\textwidth]{ErstesSem/images/hu_logo.png}\\
	\vspace{25pt}
	Introduction to Chinese Philosophy}
\subtitle{\vspace{10pt}
			Prof. Dr. Michael Beaney\\
			Proseminar SS 2018\\
          Institut für Philosophie\\ 
          Humboldt Universit"at zu Berlin}
\author{Lennard Wolf\\
        \small{\href{mailto:lennard.wolf@student.hu-berlin.de}{lennard.wolf@student.hu-berlin.de}}}
\maketitle
\begin{abstract}

This course is intended as an introduction to ancient Chinese philosophy, offering an overview of its main schools and exploring some of its main themes. One of our aims will be to demonstrate the relevance of an understanding of ancient Chinese philosophy to contemporary Western philosophy. The lectures will be given in English, but questions can be asked in German.

\end{abstract}
\newpage

\tableofcontents
%\listoffigures
\newpage


\section{What is Chinese Philosophy?\\(18.04.18)}

\begin{itemize}
  \item Moodle-PW: dao
  \item If you do Philosophy of language, you should think about other languages too
  \item The Master said: “Learning without thinking is bewildering; thinking without learning is dangerous.”
  \item Beaney will stir the thinking in the right direction
  \item The relationship between learning and thinking, reading lectures is a theme in this lecture, as well as in Chinese philosophy itself
  \item 12 topics, might change over the course, break in the middle
  \item Big question: How to properly integrate Chinese philosophy into our philocophical practice?
\end{itemize}


\subsection{Why study Chinese philosophy?}

\begin{itemize}
  \item Chinese civilization one of the longest in human history, with a rich cultural and literary tradition
  \item philosophy an important part of this
  \item ancient China far away from our modern Western world, spatio-temporally and culturally (both linguistically and conceptually)
  \item offers fundamental challenges to Western ways of thinking
  \item Chinese philosophy ought to be offered in any philosophy curriculum
  \item not just in "`area studies"'
  \item should be taken seriously a sphilosophy
  \item understanding other cultures as important as understanding "`the world"'
  \item we are now in a better position than ever to explore Chinese philosophy
  \item Primary Text Just came out: Bryan van ? : Taking back philosophy (?)
  \item Author wrote in Blog: If philosophy taught in the west is only about Western philosophy, it should be called "`Western philosophy"'
  \item Chinese philosophy is "`ghettoized"' to area studies: its not to be taken seriously \emph{as} philosophy, but just as some aspect of a foreign culture
  \item Heidegger, Derrida: "`There is no Chinese philosophy, just Chinese thought"'
\end{itemize}

\subsection{Beaneys background to Chinese philosophy}

\begin{itemize}
  \item When he took over journal, half of the journals article were on just 7 philosophers; He asks himself, how to bring history of foreign philosophy
  \item Summer School in PKU for Western philosophy; very influential
  \item Only recently they have opened up to analytic philosophy
  \item Analytic philosophy has been used by both western and chinese scholars to analyze ancient Chinese texts
  \item Big new question for him: translation
  \item Thing that irritates him: Most philosophers take translation for granted
  \item Translation is a great tecnique for properly understanding a text
  \item Huge project to translate Western texts into Chinese, making up new words on the go
  \item Book that opened his eyes : Chad Hansens  \emph{A Daoist Theory of Chinese Thought}
  \item key idea: The dao is basically a rule following practice (rule following is essential to Wittgentseins later philosophy) he reads the whole of Chinese philosophy as a response to the \emph{rule following paradox}
  \item Plan: Break the caricature of Chinese philosophy (that its mostly just sociopolitical talk (confucianism) or some mysticism (daoism))
  \item Last 50 years: work on forgotten, suppressed texts, "`Mohist"' philosophy
  \item Now we have a Neomohist tradition! "`There was analytic philosophy in China, long before Frege \& Wittgenstein."'
  \item How to create experts in Chinese philosophy? Western and Eastern work together
\end{itemize}


\subsection{What is Chinese philosophy}

\begin{itemize}
  \item Confucianism is still very influential, like Christianity in the West
  \item Xi JInping sometimes uses quotes of cinfucius
  \item Daoism and Buddhism (7th century) also important
  \item No distinction between things and concepts (subject - predicate structure)
  \item School of Names: Philosophy of language \& logic
  \item 6 main schools: ()PASTE()
  \item Big 3 confucians (Kongzi, Mengzi, Xunzi) "`zi"' is Master
  \item First utilitarian: mozi (?)
\end{itemize}

\textbf{Later Traditions}

\begin{itemize}
  \item Chinese Buddhism:
  \item |
  \item Neo Confucianism:
  \item |
\end{itemize}

\subsection{History of China}

\begin{itemize}
  \item 16th c.: Jesuits came to convert, but also translated a lot
  \item Christian Wolff: Lost his job for what he said about Chinese philosophy; according to Wolff you could be a perfectly moral person, even if you didnt believe in god $\rightarrow$ Chinese philosophy is a good example for that | famous lecture in 1721 for which he was suspended as professor (sometimes seen as father of enlightenment because of that)
  \item Hegel: said rather nasty things about the Chinese
  \item Took over the Term "`philosophy"' from tetsugaku
  \item 2 stances "`\emph{Chinese logic is inferior}"' vs. "`\emph{Chinese logic is different}"'
  \item They are now creating a tradition, the history is written, with western categories in mind
  \item Cua: The Emergence of the History of Chinese Philosophy
  \item Hu Shi AN Outline of the History of Chinese Philosphy (Inspired by pragmatism)
  \item Fung Yu-Ian, History of Chonese PHilosophy
  \item Lao Si-guang History of CHinese Philosophy: Criticies other books, as they are too influenced by west;  "`what china lacks, is analytical skills"' but you need analytical tools to understand and construct a history; "`China lacks a modern philosophy"'
\end{itemize}


\subsection{Fung Yu-Ians History of Chinese Philosophy}

\begin{itemize}
  \item 1. What is nature of Chinese philosophy?
  \item just metaphysics and ethics, rather than logic and epistemology
  \item aim: sagelyness within, and kingliness without
  \item 2. Does Chinese philosophy lack a system?
  \item 3. Is there no growth in Chinese philosophy?
\end{itemize}


\section{Chinese Philosophy and the Chinese Language\\(25.04.18)}

\subsection{Introduction}

\begin{itemize}
  \item How influential were those texts, that Beaney is talking about?
  \item Another Goal: Understand the differences between Western and Chinese People
  \item zhuangzi: philosopher who was (but not only) concerned with language
\end{itemize}

\subsection{Chinese language}

\begin{itemize}
  \item 5 types of characters: pictograms, simple ideograms, compound ideograms, (phonetic) loan characters, semantic-phonetic compounds (cf. Van Norden 2011, pp. 236–42)
  \item characters have an internal structure that can reveal aspects of its meaning (e.g. semantic content, associations) and pronunciation
\end{itemize}


\subsection{Analytic Philosophy and Chinese Philosophy}

\begin{itemize}
  \item white (form) is element of predicates that make up a ??
\end{itemize}


\subsection{Rule-following and dao}

\begin{itemize}
  \item CHad Hansen: Daoist Theory of Chinese Philosophy
\end{itemize}


\subsection{Being and existence}

\begin{itemize}
  \item in English, `is' is used in various senses, Chinese does not have such an all encompassing word
  \item are there different concepts involved?
  \item Difference between presicate is and is of existence (Essence and Existence)
  \item es gibt etwas tut geben
  \item language can only do so much
\end{itemize}


\section{Top 20 Concepts in Chinese Philosophy\\(02.05.18)}

\begin{itemize}
  \item Effortless Action
  \item Zhang:  Key Concepts in Chinese Phulosophy
  \item 天 (tian): heaven, Heaven, Nature (Ist es wirklich Himmel oder Natur?)
  \item 気 (qi): vapour, nust, breath, air, energy, spirit
  \item %陰陽 / 阴阳 
(Yin \& Yang): feminine / masculine
  \item %辩
(bian): dispute, distinguish
  \item 名 (ming): name, word (school of names 名家) | rectification of names: from names come duties; what does it mean to use language correctly? If everything was named correctly and the names
  \item 心 (xin): heart, mind, heart-mind - center of human thought and feeling
  \item 思 (si): reflect on, think, what allows us to realize the goodness of human nature
  \item zhi: wisdom, knowledge distinguisish between knowing and wisdom()
  \item -子 junzi: gentleman noble, morally superior person
  \item (sheng): sage, manifesting the greatest possible cirtue, perfect junzi (S.C. Angle: Sagehood)
  \item (qing): feeling, genuine emotion, reality feedback
  \item (xing): human nature, essence, character, quality
  \item 理 (li): principle, pattern, reason (not reasoning) (ultimate metaphysical principle)
  \item (fa): law, standard, method (central concept in Legalism)
  \item (yi): right, righteousness, right conduct, moral, appropriate - central issue in Confucianism (more important than life)
  \item (de): virtue, goodness, power; junzi has moral charisma
  \item (li): rites, ritual, ceremony, etiquette (fundamental to Confucianism)
  \item (ren): benevolence, humaneness, humanity, goodness, compassion
  \item (dao): path, way (of doing things) - discourse, way, doctrine, guide, guide-speak, guiding discourse, - a way that can be followed is not a constant 
  \item (wuwei): non-action, effortless action (自)
\end{itemize}


\section{Confucius\\(09.05.18)}

Very representative of Confucius: \emph{The Master said, "`Both keeping past teachings alive and understanding the present— someone able to do this is worthy of being a teacher.}"' (Analects, 2.11)



\subsection{Confucius: introduction}

\begin{itemize}
  \item Confucius (Kongzi), 551–479 BCE (Buddha (died c. 400 BCE); Socrates (c. 470–399 BCE))
  \item born in the state of Lu, wandered from state to state hoping to find a ruler to advise, acquired various followers
  \item member of the Ru (literati) | hence the term `Ruist' for Confucians
  \item the Analects (Lunyu) the primary source of his sayings | 20 books, 3–9 the earliest, 16–20 written later
  \item Confucius did not write anything, his disciples did
  \item Confucius’ ideas developed by Mencius (Mengzi) and Xunzi
  \item Slingerland: Confucius Analects
  \item Good Introduction to Analects: in van Norderns collection

\end{itemize}

\subsection{Learning and Teaching}

\begin{itemize}
  \item Always keep in mind: teaching has to be adjusted to the student  
  \item Cultivation of self (to a gentleman), until you do everything effortlessly | through learning from people of the past (positively as well as negatively), and people of the present
  \item learning and thinking require one another
  \item learning through observation and emulation (fake it till you make it)
  \item teaching must be adjusted to the person being taught
  \item Learning must always be directed (coming to a teacher/book without a question is meaningless)
\end{itemize}


\section{Mengzi\\(23.05.18)}

\subsection{Introduction}

\begin{itemize}
  \item Ethics start at home - this enables us to make moral judgments
  \item Norden: Mengzi Hackett 2008
  \item D. C. Lau Mencius, Penguin (older translation - interesting Appendix: on the method of analogy)
  \item the \emph{Mengzi} is primary source of his thought
  \item Second confucian
  \item develops and offers reasons for confucius' ideas
  \item less concerned with rituals
\end{itemize}

\subsection{Argumentation}

\begin{itemize}
  \item Assumption, that we are basically the same, so that 
  \item analogies are very important
  \item problem with analogies: always have to think about how far the analogy can be taken
  \item "`climbing a tree in search of a fish"'
  \item To not have a ruler is to be an animal
\end{itemize}


\subsection{The Core Virtues}

\begin{itemize}
  \item serve family (benevolence), older brother (righteousness)
  \item wisdom is to know this
  \item ritual is to regulate and adorn (schmücken) these two
  \item music is to delight in them
  \item to delight in them is to grow them
  \item to grow them is to one day live in full harmony
  \item 
\end{itemize}




\subsection{Holding to the middle}

\begin{itemize}
  \item not dogmatically! how we must act must be responsive to the particular case
\end{itemize}


\subsection{Human nature}

\begin{itemize}
  \item human nature does not need to be violated to make people benevolent and righteous
  \item human nature is like water
  \item argumentation as a battle of analogies
  \item like sprouts that have to be cultivated - h n is something that develops
  \item cultivation in confucianism is definetely 
\end{itemize}



\subsection{Reflection}

\begin{itemize}
  \item reflection (si) is what allows us to realize the goodness of human nature
  \item reflection is to extend our virtues (e.g. extend compassion from humans to animals)
  \item reflecting on and taking delight in the ‘sprouts’ of the virtues helps them develop
  \item you refine your behaviour by thinking through similar sitations
  \item Mengzi’s philosophy can then be seen as itself exemplifying reflective self-cultivation, the expression of which helps others to realize their human nature
\end{itemize}


\section{Yang Zhu, Laozi and Early Daoism\\(30.05.18)}

\subsection{Yang Zhu}

\begin{itemize}
  \item put debate on human nature on the agenda (thats why he was important)
  \item earlier than mengzi
  \item human nature is stable and self interested (Yang Zhu) vs. human nature is malleable (for better or worse)
  \item daoists: human nature in general is stable (virtuous) but has to be found
  \item Have to act for ourself
  \item psychological egoism: its our nature
  \item ethical egoism: we should act in a self interested way
  \item mencius arguments against it: people will kill you if you teach egoism: those that hate it and who accept it
\end{itemize}

\subsection{Daoism: an introduction}

\begin{itemize}
  \item Criticising confucians and mengzi
  \item two key texts: Doedejing (Laozi), Zhungzi
  \item "`Daoism did not exist in ancient China"' - was not conceptualized until later
  \item Laozi may not have existed - and certainly no single author of Laozi
  \item 2 books: dao and de - one of the three most frequently translated works (with Bible and Bhaggavat Gita)
  \item "`The Classic of the Way and Virtue"'
  \item Winnie the Pooh is banned in China (Xi Zhinping)
  \item When following the way they dont need 
  \item Dao is not a thing, it is not discrimintaed from other things.
  \item against selfconsciousness?
  \item Act, but through nonaction
  \item Be active, but have no activities
  \item dont make discriminations, valuations, judgments
\end{itemize}

\section{?}


\subsection{White Horse Reading notes}

\begin{itemize}
  \item 2
  \item Combining Argument: does not take into account subgroupings/classification
  \item a horse does not have color, a white horse does?
  \item 3
  \item that which is uncombined is changed by combination
\end{itemize}


\section{Zhuangzi\\(13.06.18)}

\emph{"`A trap is for fish: when you’ve got the fish, you can forget the trap. A snare is for rabbits: when you’ve got the rabbit, you can forget the snare. Words are for meaning: when you’ve got the meaning, you can forget the words. Where can I find someone who’s forgotten words so I can have a word with him?"'}


\subsection{Introduction}


\begin{itemize}
  \item "`Never knew"' he was a `Daoist' (that term was given to the tradition later on)
  \item Three parts: Inner chapter (1-7, attributed to Zhuangzi), Outer Chapters (8-21), Misc. Chapters (22-33)
  \item Outer Chapters are work of four different groups (Zhuangzis followers, ethical egoists - followers of Yang Zhu, primitivists and syncretists (eclectics))
  \item His Daoism is arguably a Daoism of the immanent, not of the transcendent
  \item Pluralist rather than monist
  \item Uses dao as a concept of guidance, rather than a reality concept
  \item Schleiermacher: (translate in terms known to the target audience, "`domesticating translation"') - vs. bring the reader to the author (enriching the own language through transliteration, "`foreignizing"')
  \item Western texts when translated to Chinese, they were domesticated and the Zhuangzi was the key resource for the language, that was used (it is very idiosyncratic and individualistic, like Western texts to the Chinese)
  \item Therefore its not just a key text of daoism, but also plays a key role in the reception of Western texts!
  \item "`What I am saying is basically what Confucius said"' instead of argumentation $\rightarrow$ "`What the foreign text is saying is basically what Zhuangzi said"'
\end{itemize}

\subsection{Perspectivism}


\begin{itemize}
  \item judgment is based on perspective and thus relative
  \item any `transperspectival' perspective is also just a perspective
  \item How to solve this problem?
  \item The premier philosopher of perspective (Hansen)
  \item 
\end{itemize}


\subsection{Evening Things}


\begin{itemize}
  \item ch 2, `On equalizing things'The sorting which evens things out or `The adjustment of controversies'
  \item The title can be interpreted differently
  \item shi-fei attitude: attitude towards right and wrong
  \item there may be indefinetely many daos of language - whose dao is right?
  \item Mohists and Confucians disagree - we mustnt judge which one is right, but see them both in their light
  \item There is nothing that can not be looked at that way, there is nothing that cannot be looked at this way. But that is not the way I see things; Only as I know things myself do I know them.
\end{itemize}

\subsection{Scepticism}


\begin{itemize}
  \item How do I know that my perspective is not limited?
  \item equalizing the myriad things in seeing that they are all in the act of transformation
\end{itemize}



\section{Yijing\\(13.06.18)}

\begin{itemize}
  \item Is on Korean Flag: yin - yang sign; Heaven, Fire, Water, Earth
  \item Foundational cosmological text for all of chinese 
  \item Definition of Yi: Change as Arche / / Yi is Lizard, Chameleon - Imaging the shape
  \item Fundamental Assumption of Change
  \item Divinations as first purpose of book
  \item Oracle Bones: Earliest Strata - later got more and more sophisticated
  \item Yan Yang cosmology
  \item hermeneutics of the trigrams
  \item Confucius's reverence for the Yijing
\end{itemize}


\section{Xunzi: Human nature is bad // Ritual\\(28.06.18)}

\subsection{Lektürenotizen Xunzi}

\begin{itemize}
  \item Goodness only through effort under the guidance of teachers
  \item Argument in beginning: People are bad because they are bad
  \item Metaphor: Like crooked wood (but there is no effort in the wood?)
  \item Das was verändert werden kann, das ist von effort betroffen. nature kann nicht durch effort verändert werden!
  \item How are standards of righteosness prouced? thorugh effort!
  \item Things that need del. effort vs. things that give immediate satisfaction
  \item The sage differs from the masses because of his efforts
  \item People desire to be good, because they do not have it in them. (Because you seek what you dont have)
  \item If people were good, why would we need the sage kings? (....)
  \item Pottery is not in the potters nature, so rituals are not in peoples nature.
  \item You value the gentlman BECAUSE he has 'transformed his nature‘ (bas translation??) should probably be overcome
  \item You cannot be made to become a sage
  \item Nur weil man es nicht kann, heißt es nicht, dass man nicht können \emph{kann}.
  \item Therefore petty man can become sage and vice versa.
  \item Fragen: 
  \item Aus welcher Sicht ist nature bad? Aus der Sicht der schon `erleuchteten'?
  \item Analogie dass Vasen hergestellt werden durch effort - es ist nicht in der nature. Aber was ist mit anderen abstrakten, aber scheinbar schlechten sachen?
  \item Why would I seek everything I dont have?
  \item Streit zwischen mengzi und xunzi wie Rousseau gegen Hobbes
  \item Wichtig: Erst wenn righteousness established ist, kann darüber reflektiert werden?
\end{itemize}

\subsection{Lektürenotizen Ritual}

\begin{itemize}
  \item Li is ritual
  \item kongzi: li can be changed, if yi (righteosness) is used in style
\end{itemize}


\section{Legalism\\(28.06.18)}

\begin{itemize}
  \item Stand ist nicht zwangsläufig der beste Marker für Macht, besonders in einem Machtgeflecht, das meist nicht nach seinen nach außen hin existierenden standards/strukturen erläuft
  \item Meritokratie vs. rectified utopia
  \item “For this reason, the sage does not expect to follow the ways of the ancients or model his behavior on an unchanging standard of what is acceptable. 
  \item “Therefore, an enlightened ruler uses the people’s strength and does not listen to their words; he rewards their achievements and completely prohibits useless activities. 
  \item an sich sinnvoll, würde heute nicht mehr gelten
\end{itemize}


\subsection{Legalism\\Protokoll (05.07.18)}

\begin{itemize}
  \item Spannende Stelle:
  \item Schönes Argument Power of Position:
  \item PoP heißt, eine besondere stelle kann mir besindere Macht geben, Bsp: Lehrer: wenn er mir sagt ich soll leise sein dann bin ich leise, weil er eine machtposition hat im Klassenzimmer
  \item Klassicher Ansatz: Power of Position vuktioneirt nicht wiel, wenn es gelten, würde, dann würde es keinen sinn machen, jmd besonders qualifizierten auf eine stelle zu setzen,  
  \item shezi: mag sein dass man das so hanhabt, aber macht kann nicht autoorität verleihen , weil bsp hofwagenreiter vs Bauer, der 
  \item Prämisse ist aber falsch weil ich bei einer position nicht immer nur die wahl habe zwischen dem besten und den schlechtesten - man muss sich auch damit zufriedengeben können, eine mittlere wahl zu treffen/ 2. Punkt: es funktioniert so nicht, dass wir die position nur an worthy men vergeben, weil sonst hätten wir chaos: es komtm nur alle 1000 jahre ein worthy man // wenn ich nur 90\% sättigen kann, warte ich nicht, bis ich 100\% sättigen kann
  \item Kapitel durchgehen: 
  \item Verwirrend: ZU viel nutzen mit zu wenig aufwand wirst du bestraft, aber wenn du auch - nur überspitzte darstellung von ansicht, immer alles perfekt auszuführen
  \item wenn ich es belohne, wenn jmd seine position verlässt, dann schaffe ich chaos
  \item Beispiel Mantel- und Hutminister: Mantelminister ist müde (?), der Hutminister gibt einen Hut, und wird bestraft, weil er seine position verlässt
  \item Wie viel eigenverantwortung kann ich den Leuten denn übertragen? du darfst nicht übererfüllen aber auch nicht untererfüllen
  \item Modes of government, bestrafen wenn unter- und übererfüllt wird: versch. Schwierigkeiten: wie ist entwicklung dann noch möglich (mach mal so gut wie geht, dann kann ich nicht bestrafen wenn zu viel gemacht wird, aber dann hat man wieder das problem, dass leute sich durch besonders viel hervor tun wollen)
  \item Unterkomplexe Herrschaftslösung für ein doch eigentlich komplexes Land/?
  \item Ruler darf untergebene nicht wissen lassen, was er will - ansonsten kommt es zu machtspielen
  \item Ruler würde als "dumm" dargestellt
  \item Two handles (Ch 7): Punishment and Favour
  \item Brauche beide und muss beide 100\% konsequent anwenden - worthy man muss bestraft werden wie unworthy man
  \item Du darfst die Handles niemals aus der Hand geben 
  \item Gibt es Raum für spontanes Handeln (Algorithmus im Gesetz?)? Was passiert wenn es eine Hungernot gibt?
  \item Konstantin hat Film gesehen: Kriegsminister sagt allen Offizieren unterschiedliche Taktiken und lässt so alle im Dunkeln
  \item Ist Legalism und Machiavellian thought vergleichbar?
  \item Das Law muss konsequent durchgesetzt werden - auch der Ruler muss das Law befolgen
  \item Was passiert wenn ein schlechter Herrscher kommt?
  \item Es ist nicht moralisch schlecht, jemanden zu manipulieren - dadurch bleibt es möglich, einen bösen Herrscher loszuwerden, und vllt auch Innovation?
  \item Ist der fähige Mensch bei ihm automatisch ein rechtschaffener Mensch? Wahrscheinlich nicht.
  \item Law: If the Law can not be trusted, then the status of the ruler is in danger.
  \item For strictly regulating the offices and overawing the people, thwarting licentiousness and idleness, and stopping trickery and falsehood, nothing is better than punishments.
  \item Wir haben 2 wichtige Denkschulen die als gleichwertig eingeshcätzt werden, oder leute schreiben sich beide auf die fahne, aber das geht ja nicht weil sie sich in ihren prämissen unterscheiden -> kritik an jenen leuten die beides liken 
\end{itemize}



\section{Legalism Lecture\\(11.07.18)}

\subsection{Introduction}

\begin{itemize}
  \item The Way lies in not being seen, this is the Way of the Ruler. Cover your tracks, conceal your starting points, and your subordinates will not be able to see where you are coming from. Get rid of wisdom, dispense with ability, and your subordinates will not be able to guess your intentions. Hold on to what people have said before, and look to see if they match it with results.
  \item legalism: fa: laws, standards, methods
  \item Han Feizi key representative
  \item three key doctrines: shi(power of posiiton), shu (術, methods), fa(法, standards)
  \item Importance of having standards: for effective government we need clear objective standards, just like a carpenter needs rules for construction
  \item the law of standard applies to everyone
  \item ruler can not be tricked because he does not rely on his own judgment (but he becomes preticable??)
\end{itemize}

\subsection{The Way of the Ruler}

\begin{itemize}
  \item Ruler does not reveal his intentions
  \item 2 handles: reward and punish
  \item Two models of government: ch 43
\end{itemize}

\subsection{The Power of Position}

\begin{itemize}
  \item Topic of ch. 8
  \item amoralized version of Confucian conception of moral charisma (dé)
  \item status and position alone generate power
  \item worthiness and wisom are never sufficient
\end{itemize}

\subsection{The Way of the Ruler}

\begin{itemize}
  \item you need position and moral worthiness
  \item but there are not enough 
\end{itemize}
\subsection{The Difficulty of Persuasion}

\begin{itemize}
  \item demonstrates real psychological insight
  \item know how to highlight those qualities of which the person being persuaded is proud, while eliminating those of which they are ashamed
  \item difficulty is to know what is in the heart of the person who is to be persuaded
\end{itemize}

\subsection{Critique of Confucianism}

\begin{itemize}
  \item Six main critisms (chs. 49-50):
  \item 1) ciritique 
\end{itemize}


\section{Q \& A Lecture\\(11.07.18)}

\begin{itemize}
  \item How can a system of rituals and standards be flexible?
  \item 
\end{itemize}


\end{document}
