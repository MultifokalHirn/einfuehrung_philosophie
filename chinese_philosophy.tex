\documentclass[emulatestandardclasses]{scrartcl}
\usepackage{graphicx}
\usepackage{CJKutf8} % japanese
\usepackage{color}
\usepackage[ngerman]{babel}
\usepackage{hyperref}
\usepackage{fullpage}
\usepackage[utf8]{inputenc}
\usepackage{calc} 
\usepackage{enumitem}
\usepackage{titlesec}
\date{\vspace{-3ex}}
\begin{document}

\title{
	\includegraphics*[bb=0 0 720 200, width=0.72\textwidth]{ErstesSem/images/hu_logo.png}\\
	\vspace{25pt}
	Introduction to Chinese Philosophy}
\subtitle{\vspace{10pt}
			Prof. Dr. Michael Beaney\\
			Proseminar SS 2018\\
          Institut für Philosophie\\ 
          Humboldt Universit"at zu Berlin}
\author{Lennard Wolf\\
        \small{\href{mailto:lennard.wolf@student.hu-berlin.de}{lennard.wolf@student.hu-berlin.de}}}
\maketitle
\begin{abstract}

This course is intended as an introduction to ancient Chinese philosophy, offering an overview of its main schools and exploring some of its main themes. One of our aims will be to demonstrate the relevance of an understanding of ancient Chinese philosophy to contemporary Western philosophy. The lectures will be given in English, but questions can be asked in German.

\end{abstract}
\newpage

\tableofcontents
%\listoffigures
\newpage


\section{What is Chinese Philosophy?\\(18.04.18)}

\begin{itemize}
  \item Moodle-PW: dao
  \item If you do Philosophy of language, you should think about other languages too
  \item The Master said: “Learning without thinking is bewildering; thinking without learning is dangerous.”
  \item Beaney will stir the thinking in the right direction
  \item The relationship between learning and thinking, reading lectures is a theme in this lecture, as well as in Chinese philosophy itself
  \item 12 topics, might change over the course, break in the middle
  \item Big question: How to properly integrate Chinese philosophy into our philocophical practice?
\end{itemize}


\subsection{Why study Chinese philosophy?}

\begin{itemize}
  \item Chinese civilization one of the longest in human history, with a rich cultural and literary tradition
  \item philosophy an important part of this
  \item ancient China far away from our modern Western world, spatio-temporally and culturally (both linguistically and conceptually)
  \item offers fundamental challenges to Western ways of thinking
  \item Chinese philosophy ought to be offered in any philosophy curriculum
  \item not just in "`area studies"'
  \item should be taken seriously a sphilosophy
  \item understanding other cultures as important as understanding "`the world"'
  \item we are now in a better position than ever to explore Chinese philosophy
  \item Primary Text Just came out: Bryan van ? : Taking back philosophy (?)
  \item Author wrote in Blog: If philosophy taught in the west is only about Western philosophy, it should be called "`Western philosophy"'
  \item Chinese philosophy is "`ghettoized"' to area studies: its not to be taken seriously \emph{as} philosophy, but just as some aspect of a foreign culture
  \item Heidegger, Derrida: "`There is no Chinese philosophy, just Chinese thought"'
\end{itemize}

\subsection{Beaneys background to Chinese philosophy}

\begin{itemize}
  \item When he took over journal, half of the journals article were on just 7 philosophers; He asks himself, how to bring history of foreign philosophy
  \item Summer School in PKU for Western philosophy; very influential
  \item Only recently they have opened up to analytic philosophy
  \item Analytic philosophy has been used by both western and chinese scholars to analyze ancient Chinese texts
  \item Big new question for him: translation
  \item Thing that irritates him: Most philosophers take translation for granted
  \item Translation is a great tecnique for properly understanding a text
  \item Huge project to translate Western texts into Chinese, making up new words on the go
  \item Book that opened his eyes : Chad Hansens  \emph{A Daoist Theory of Chinese Thought}
  \item key idea: The dao is basically a rule following practice (rule following is essential to Wittgentseins later philosophy) he reads the whole of Chinese philosophy as a response to the \emph{rule following paradox}
  \item Plan: Break the caricature of Chinese philosophy (that its mostly just sociopolitical talk (confucianism) or some mysticism (daoism))
  \item Last 50 years: work on forgotten, suppressed texts, "`Mohist"' philosophy
  \item Now we have a Neomohist tradition! "`There was analytic philosophy in China, long before Frege \& Wittgenstein."'
  \item How to create experts in Chinese philosophy? Western and Eastern work together
\end{itemize}


\subsection{What is Chinese philosophy}

\begin{itemize}
  \item Confucianism is still very influential, like Christianity in the West
  \item Xi JInping sometimes uses quotes of cinfucius
  \item Daoism and Buddhism (7th century) also important
  \item No distinction between things and concepts (subject - predicate structure)
  \item School of Names: Philosophy of language \& logic
  \item 6 main schools: ()PASTE()
  \item Big 3 confucians (Kongzi, Mengzi, Xunzi) "`zi"' is Master
  \item First utilitarian: mozi (?)
\end{itemize}

\textbf{Later Traditions}

\begin{itemize}
  \item Chinese Buddhism:
  \item |
  \item Neo Confucianism:
  \item |
\end{itemize}

\subsection{History of China}

\begin{itemize}
  \item 16th c.: Jesuits came to convert, but also translated a lot
  \item Christian Wolff: Lost his job for what he said about Chinese philosophy; according to Wolff you could be a perfectly moral person, even if you didnt believe in god $\rightarrow$ Chinese philosophy is a good example for that | famous lecture in 1721 for which he was suspended as professor (sometimes seen as father of enlightenment because of that)
  \item Hegel: said rather nasty things about the Chinese
  \item Took over the Term "`philosophy"' from tetsugaku
  \item 2 stances "`\emph{Chinese logic is inferior}"' vs. "`\emph{Chinese logic is different}"'
  \item They are now creating a tradition, the history is written, with western categories in mind
  \item Cua: The Emergence of the History of Chinese Philosophy
  \item Hu Shi AN Outline of the History of Chinese Philosphy (Inspired by pragmatism)
  \item Fung Yu-Ian, History of Chonese PHilosophy
  \item Lao Si-guang History of CHinese Philosophy: Criticies other books, as they are too influenced by west;  "`what china lacks, is analytical skills"' but you need analytical tools to understand and construct a history; "`China lacks a modern philosophy"'
\end{itemize}


\subsection{Fung Yu-Ians History of Chinese Philosophy}

\begin{itemize}
  \item 1. What is nature of Chinese philosophy?
  \item just metaphysics and ethics, rather than logic and epistemology
  \item aim: sagelyness within, and kingliness without
  \item 2. Does Chinese philosophy lack a system?
  \item 3. Is there no growth in Chinese philosophy?
\end{itemize}


\section{Chinese Philosophy and the Chinese Language\\(25.04.18)}

\subsection{Introduction}

\begin{itemize}
  \item How influential were those texts, that Beaney is talking about?
  \item Another Goal: Understand the differences between Western and Chinese People
  \item zhuangzi: philosopher who was (but not only) concerned with language
\end{itemize}

\subsection{Chinese language}

\begin{itemize}
  \item 5 types of characters: pictograms, simple ideograms, compound ideograms, (phonetic) loan characters, semantic-phonetic compounds (cf. Van Norden 2011, pp. 236–42)
  \item characters have an internal structure that can reveal aspects of its meaning (e.g. semantic content, associations) and pronunciation
\end{itemize}


\subsection{Analytic Philosophy and Chinese Philosophy}

\begin{itemize}
  \item white (form) is element of predicates that make up a ??
\end{itemize}


\subsection{Rule-following and dao}

\begin{itemize}
  \item CHad Hansen: Daoist Theory of Chinese Philosophy
\end{itemize}


\subsection{Being and existence}

\begin{itemize}
  \item in English, `is' is used in various senses, Chinese does not have such an all encompassing word
  \item are there different concepts involved?
  \item Difference between presicate is and is of existence (Essence and Existence)
  \item es gibt etwas tut geben
  \item language can only do so much
\end{itemize}


\section{Top 20 Concepts in Chinese Philosophy\\(02.05.18)}

\begin{itemize}
  \item Effortless Action
  \item Zhang:  Key Concepts in Chinese Phulosophy
  \item 天 (tian): heaven, Heaven, Nature (Ist es wirklich Himmel oder Natur?)
  \item 気 (qi): vapour, nust, breath, air, energy, spirit
  \item %陰陽 / 阴阳 
(Yin \& Yang): feminine / masculine
  \item %辩
(bian): dispute, distinguish
  \item 名 (ming): name, word (school of names 名家) | rectification of names: from names come duties; what does it mean to use language correctly? If everything was named correctly and the names
  \item 心 (xin): heart, mind, heart-mind - center of human thought and feeling
  \item 思 (si): reflect on, think, what allows us to realize the goodness of human nature
  \item zhi: wisdom, knowledge distinguisish between knowing and wisdom()
  \item -子 junzi: gentleman noble, morally superior person
  \item (sheng): sage, manifesting the greatest possible cirtue, perfect junzi (S.C. Angle: Sagehood)
  \item (qing): feeling, genuine emotion, reality feedback
  \item (xing): human nature, essence, character, quality
  \item 理 (li): principle, pattern, reason (not reasoning) (ultimate metaphysical principle)
  \item (fa): law, standard, method (central concept in Legalism)
  \item (yi): right, righteousness, right conduct, moral, appropriate - central issue in Confucianism (more important than life)
  \item (de): virtue, goodness, power; junzi has moral charisma
  \item (li): rites, ritual, ceremony, etiquette (fundamental to Confucianism)
  \item (ren): benevolence, humaneness, humanity, goodness, compassion
  \item (dao): path, way (of doing things) - discourse, way, doctrine, guide, guide-speak, guiding discourse, - a way that can be followed is not a constant 
  \item (wuwei): non-action, effortless action (自)
\end{itemize}


\section{Confucius\\(09.05.18)}

Very representative of Confucius: \emph{The Master said, "`Both keeping past teachings alive and understanding the present— someone able to do this is worthy of being a teacher.}"' (Analects, 2.11)



\subsection{Confucius: introduction}

\begin{itemize}
  \item Confucius (Kongzi), 551–479 BCE (Buddha (died c. 400 BCE); Socrates (c. 470–399 BCE))
  \item born in the state of Lu, wandered from state to state hoping to find a ruler to advise, acquired various followers
  \item member of the Ru (literati) | hence the term `Ruist' for Confucians
  \item the Analects (Lunyu) the primary source of his sayings | 20 books, 3–9 the earliest, 16–20 written later
  \item Confucius did not write anything, his disciples did
  \item Confucius’ ideas developed by Mencius (Mengzi) and Xunzi
  \item Slingerland: Confucius Analects
  \item Good Introduction to Analects: in van Norderns collection

\end{itemize}

\subsection{Learning and Teaching}

\begin{itemize}
  \item Always keep in mind: teaching has to be adjusted to the student  
  \item Cultivation of self (to a gentleman), until you do everything effortlessly | through learning from people of the past (positively as well as negatively), and people of the present
  \item learning and thinking require one another
  \item learning through observation and emulation (fake it till you make it)
  \item teaching must be adjusted to the person being taught
  \item Learning must always be directed (coming to a teacher/book without a question is meaningless)
\end{itemize}


\section{Mengzi\\(23.05.18)}

\subsection{Introduction}

\begin{itemize}
  \item Ethics start at home - this enables us to make moral judgments
  \item Norden: Mengzi Hackett 2008
  \item D. C. Lau Mencius, Penguin (older translation - interesting Appendix: on the method of analogy)
  \item the \emph{Mengzi} is primary source of his thought
  \item Second confucian
  \item develops and offers reasons for confucius' ideas
  \item less concerned with rituals
\end{itemize}

\subsection{Argumentation}

\begin{itemize}
  \item Assumption, that we are basically the same, so that 
  \item analogies are very important
  \item problem with analogies: always have to think about how far the analogy can be taken
  \item "`climbing a tree in search of a fish"'
  \item To not have a ruler is to be an animal
\end{itemize}


\subsection{The Core Virtues}

\begin{itemize}
  \item serve family (benevolence), older brother (righteousness)
  \item wisdom is to know this
  \item ritual is to regulate and adorn (schmücken) these two
  \item music is to delight in them
  \item to delight in them is to grow them
  \item to grow them is to one day live in full harmony
  \item 
\end{itemize}




\subsection{Holding to the middle}

\begin{itemize}
  \item not dogmatically! how we must act must be responsive to the particular case
\end{itemize}


\subsection{Human nature}

\begin{itemize}
  \item human nature does not need to be violated to make people benevolent and righteous
  \item human nature is like water
  \item argumentation as a battle of analogies
  \item like sprouts that have to be cultivated - h n is something that develops
  \item cultivation in confucianism is definetely 
\end{itemize}



\subsection{Reflection}

\begin{itemize}
  \item reflection (si) is what allows us to realize the goodness of human nature
  \item reflection is to extend our virtues (e.g. extend compassion from humans to animals)
  \item reflecting on and taking delight in the ‘sprouts’ of the virtues helps them develop
  \item you refine your behaviour by thinking through similar sitations
  \item Mengzi’s philosophy can then be seen as itself exemplifying reflective self-cultivation, the expression of which helps others to realize their human nature
\end{itemize}


\section{Yang Zhu, Laozi and Early Daoism\\(30.05.18)}

\subsection{Yang Zhu}

\begin{itemize}
  \item put debate on human nature on the agenda (thats why he was important)
  \item earlier than mengzi
  \item human nature is stable and self interested (Yang Zhu) vs. human nature is malleable (for better or worse)
  \item daoists: human nature in general is stable (virtuous) but has to be found
  \item Have to act for ourself
  \item psychological egoism: its our nature
  \item ethical egoism: we should act in a self interested way
  \item mencius arguments against it: people will kill you if you teach egoism: those that hate it and who accept it
\end{itemize}

\subsection{Daoism: an introduction}

\begin{itemize}
  \item Criticising confucians and mengzi
  \item two key texts: Doedejing (Laozi), Zhungzi
  \item "`Daoism did not exist in ancient China"' - was not conceptualized until later
  \item Laozi may not have existed - and certainly no single author of Laozi
  \item 2 books: dao and de - one of the three most frequently translated works (with Bible and Bhaggavat Gita)
  \item "`The Classic of the Way and Virtue"'
  \item Winnie the Pooh is banned in China (Xi Zhinping)
  \item When following the way they dont need 
  \item Dao is not a thing, it is not discrimintaed from other things.
  \item against selfconsciousness?
  \item Act, but through nonaction
  \item Be active, but have no activities
  \item dont make discriminations, valuations, judgments
\end{itemize}

\subsection{Reflection}

\begin{itemize}
  \item 
\end{itemize}

\subsection{Reflection}

\begin{itemize}
  \item 
\end{itemize}

\end{document}
