\documentclass[]{scrartcl}
\usepackage{graphicx}
\usepackage{color}
\usepackage[ngerman]{babel}
\usepackage{hyperref}
\usepackage{calc} 
\usepackage{enumitem}
\usepackage{titlesec}
%\pagestyle{headings}
%\pagestyle{headings}

\begin{document}

\title{
	\includegraphics*[width=0.75\textwidth]{images/hu_logo.png}\\
	\vspace{24pt}
	Einf"uhrung in die \\ Architektur der Moderne\\ Konzeptionen, Imaginationen,\\ mediale Rezeption}
\subtitle{Proseminar WS 16/17\\
          Prof. Dr. Kai Kappel\\
          Institut f"ur Kunst- und Bildgeschichte \\ 
          Humboldt Universit"at zu Berlin}
\author{Lennard Wolf\\
        \href{mailto:lennard.wolf@student.hu-berlin.de}{lennard.wolf@student.hu-berlin.de}}
\maketitle
\begin{abstract}

Das einf"uhrende Seminar er"offnet kritische Zug"ange zur Begriffsbildung, Theorie und den wichtigsten Str"omungen der modernen Architektur. 

\end{abstract}
\newpage

\tableofcontents

\listoffigures
\newpage


\section{Einf"uhrende Sitzung\\(18.10.16)}
\subsection{Organisatorisches}

\begin{itemize}
    \item Erste beide Semester sind Grundlagenmodule und zu belegen
    \item Klausur wird geschrieben, erste und einzige in 1. Sem
    \item Struktur: "Uberblicksvorlesung + Proseminar + Tutorium
    \item MAP: Referat mit Team (ganzer Monat; MAP wird bei Agnes angemeldet)
    \item Moodle pw: \emph{Stahltr"ager}
    \item Pflichtsprechstunde vor dem Vortrag
\end{itemize}

Vortrag: \emph{Le Corbusiers Konzept einer Ville Contemporaine}; 20min, 1 Handout A4
Aufgabe dabei: Sagen, was die Bauten verk"orpern, die Ideengeschichte dahinter. Ob erst geb"aude und dann idee oder andersherum h"angt vom Vortrag ab.

\subsection{Einf"uhrung}
\subsubsection{Definitionen der Moderne?}
\begin{itemize}
    \item Haeufig sagt man: nach dem ersten Weltkrieg 1918 (Bauhaus)
    \item Aber: Jugendstil, Reformarchitektur, Eisen \& Stahlkonstruktionen etc..?
    \item Baustoffe, Schmuck etc.
    \item Seminar setzt im 19.Jh an
\end{itemize}

\subsubsection{Wann endet die Moderne?}

\begin{itemize}
    \item Postmoderne, Zweite Moderne etc.?
    \item \emph{Modern is not contemporary}
\end{itemize}

\section{Wie beschreibe und analysiere ich moderne Architektur?\\(25.10.16)}

Treffen an der Ged"achtniskirche


\section{Zum Modernebegriff (nicht nur) in der Architektur\\(01.11.16)}

\subsection{Vornotizen zur Sitzung}
\subsubsection{Definitionen und Abgrenzungen von Moderne}

\begin{description}[leftmargin=!,labelwidth=\widthof{\bfseries P2}]
  \item[Greenberg (Arch.)] \emph{Moderne Architektur bedeutet -- vereinacht gesagt -- funktionelle, geometrische Strenge und das Vermeiden von Dekoration und Ornament.} | Kap. `Modern und postmodern', Die Essenz der Moderne 
  \item[Greenberg (Kunst)] Bezeichnet K"unstler der Moderne als \emph{unbeugsamen Helden, der sich auf keinerlei Kompromisse mit dem Bestehenden einl"a\ss t und allen Verlockungen der eleganten Welt edelm"utig widersteht}; \emph{Bedeutende Werke wird ein K"unstler nur zu Wege bringen, wenn er alle au\ss erk"unstlcihen Anspr"uche und Erwartungen abweist, um seine Arbeit ganz aus sich sekbst heraus zu begr"unden.} $\rightarrow$ Das Publikum wurde dem Autor nach immer verst"andnisloser. | \emph{..selbstkritische Besinnung der Kunst auf ihre ureigenen und spezifischen M"oglichkeiten ist, Greenberg zufolge, das Charakterisikum des `Modernismus' }
  
  \item[Kretschmer] Latein: \emph{der Gegenwart angeh"orig} | im 19.Jh. als Gegensatz zu \emph{vergangen} | `Siegeszug' erst nach WWII | \emph{Die Moderne ist generell eine von technischem Positivismus gekennzeichnete Emanzipationsbewegung zur Befreiung aus der Fremdbestimmung.} | \emph{Es ging um eine m"oglichst gute Versorgung der neuen Massengesellschaft.} 
\end{description}



\section{Stahl und Glaskonstruktionen: Weltausstellungen des 19. Jahrhunderts\\(08.11.16)}

\section{Stahl und Glaskonstruktionen: Hochhausbauten\\(15.11.16)}

\section{Buildings to live in, not to look at. Absagen an die Gliederungen des Historismus\\(22.11.16)}




\newpage
\section{"Uber den Professor}
Prof. Dr. Kai Kappel
%\begin{figure}[h]
%	\centering
%	\includegraphics[width=0.5\textwidth]{images/template.png}
%	\caption{Template Bild}
%	\label{fig:template}
%\end{figure}


\end{document}
