\documentclass[]{scrartcl}
\usepackage{graphicx}
\usepackage{color}
\usepackage{german}
\usepackage{hyperref}
\usepackage{calc} 
\usepackage{enumitem}
%\pagestyle{headings}

% customize dictum format:
\usepackage[T1]{fontenc}
\setkomafont{dictumtext}{\itshape\small}
\setkomafont{dictumauthor}{\normalfont}
\renewcommand*\dictumwidth{\linewidth}
\renewcommand*\dictumauthorformat[1]{--- #1}
\renewcommand*\dictumrule{}
\newcommand{\todo}[1]{\textcolor{red}{TODO: #1}\PackageWarning{TODO:}{#1!}}

\begin{document}

\title{
	\includegraphics*[width=0.75\textwidth]{images/hu_logo.png}\\
	\vspace{24pt}
	Einf"uhrung in die \\ Architektur der Moderne\\ Konzeptionen, Imaginationen,\\ mediale Rezeption}
\subtitle{Proseminar WS 16/17\\
          Prof. Dr. Kai Kappel\\
          Institut f"ur Kunst- und Bildgeschichte \\ 
          Humboldt Universit"at zu Berlin}
\author{Lennard Wolf\\
        \href{mailto:lennard.wolf@student.hu-berlin.de}{lennard.wolf@student.hu-berlin.de}}
\maketitle
\begin{abstract}

Das einf"uhrende Seminar er"offnet kritische Zug"ange zur Begriffsbildung, Theorie und den wichtigsten Str"omungen der modernen Architektur. 

\end{abstract}
\newpage

\tableofcontents
\newpage

\listoffigures
\newpage


\section{Einf"uhrende Sitzung\\(18.10.16)}

\begin{itemize}
    \item erste beide semester sind grundlagenmodule zu belgeen
    \item Klausuren werden geschrieben, erste und einzige in 1. Sem
    \item struktur: "Uberblicksvorlesung + proseminar + tutorium ()
    \item MAP: Referat mit Team (ganzer Monat; MAP wird bei Agnes angemeldet)
    \item moodle pw: Stahltr"ager
    \item Pflichtsprechstunde vor dem Vortrag
    \item \textbf{----------}
    \item \textbf{definitionen der moderne?} 
    \item haeufig sagt man: nach dem ersten Weltkrieg 1918 (Bauhaus)
    \item Aber: Jugendstil, Reformarchitektur, Eisen \& STahlkonsturktionen etc..?
    \item Baustoffe, Schmuck etc.
    \item seminar setzt im 19.jhd an
    \item \textbf{Wann endet die moderne?}
    \item Postmoderne, Zweite Moderne etc?
    \item \emph{Modern is not contemporary}
    \item 
    \item referate: sagen, was die bauten verk"orpern, die Ideengeschichte dahinter
    \item ob erst geb"aude und dann idee oder andersherum h"angt vom vortrag ab
\end{itemize}

Vortrag: Le Corbousiers


\section{Wie beschreibe und analysiere ich moderne Architektur?\\(25.10.16)}

Treffen an der Ged"achtniskirche


\section{Zum Modernebegriff (nicht nur) in der Architektur\\(01.11.16)}

%\begin{figure}[h]
%	\centering
%	\includegraphics[width=0.5\textwidth]{images/template.png}
%	\caption{Template Bild}
%	\label{fig:template}
%\end{figure}


\section{Stahl und Glaskonstruktionen: Weltausstellungen des 19. Jahrhunderts\\(08.11.16)}

\section{Stahl und Glaskonstruktionen: Hochhausbauten\\(15.11.16)}

\section{Buildings to live in, not to look at. Absagen an die Gliederungen des Historismus\\(22.11.16)}




\newpage
\section{"Uber den Professor}
Prof. Dr. Kai Kappel


\end{document}
