\documentclass[emulatestandardclasses]{scrartcl}
\usepackage{graphicx}
\usepackage{color}
\usepackage[ngerman]{babel}
\usepackage{hyperref}
\usepackage{fullpage}
\usepackage{calc} 
\usepackage{enumitem}
\usepackage{titlesec}
\newcommand{\todo}[1]{\textcolor{red}{TODO: #1}\PackageWarning{TODO:}{#1!}}
\date{\vspace{-3ex}}
\begin{document}

\title{
	\includegraphics*[width=0.75\textwidth]{images/hu_logo.png}\\
	\vspace{24pt}
	Einf"uhrung in die \\ Architektur der Moderne\\ Konzeptionen, Imaginationen,\\ mediale Rezeption}
\subtitle{Proseminar WS 16/17\\
          Prof. Dr. Kai Kappel\\
          Institut f"ur Kunst- und Bildgeschichte \\ 
          Humboldt Universit"at zu Berlin}
\author{Lennard Wolf\\
        \small{\href{mailto:lennard.wolf@student.hu-berlin.de}{lennard.wolf@student.hu-berlin.de}}}
\maketitle
\begin{abstract}

Das einf"uhrende Seminar er"offnet kritische Zug"ange zur Begriffsbildung, Theorie und den wichtigsten Str"omungen der modernen Architektur. 

\end{abstract}
\newpage

\tableofcontents

\listoffigures
\newpage


\section{Einf"uhrende Sitzung\\(18.10.16)}
\subsection{Organisatorisches}

\subsubsection{Grunds"atzliches}

\begin{itemize}
    \item Erste beide Semester sind Grundlagenmodule und zu belegen
    \item Klausur wird geschrieben, erste und einzige in 1. Sem
    \item Struktur: "Uberblicksvorlesung + Proseminar + Tutorium
    \item MAP: Referat mit Team (ganzer Monat; MAP wird bei Agnes angemeldet)
    \item Moodle pw: \emph{Stahltr"ager}
    \item Pflichtsprechstunde vor dem Vortrag
\end{itemize}

\subsubsection{Vortrag}

\begin{itemize}
    \item \emph{Le Corbusiers Konzept einer Ville Contemporaine}; 20min, 1 Handout A4
    \item Aufgabe dabei: Sagen, was die Bauten verk"orpern, die Ideengeschichte dahinter. Ob erst geb"aude und dann idee oder andersherum h"angt vom Vortrag ab
    \item Auch Kritik an m"oglichen Erkl"arungen aus Geschichtsb"uchern "uben (wenn angebracht)
    \item Mindestens 5 -- 10 Quellenangaben auf dem Handout. Auch bei Kubikat und Art historica (?) schauen! Nicht nur Primus Bücher nennen. Handout bei Moodle hochladen, \emph{am besten eine Woche vorher}! 
    \item Folien immer als \texttt{.ppt}! Mit extrem hochaufl"osenden Bildern.
    \item Diskussionsfrage parat haben!
    \item Beim Vortrag stehen!
\end{itemize}



\subsubsection{Klausur}
Handouts der Vortr"age sind Lernquellen.
Benotet??

\subsection{Einf"uhrung}
\subsubsection{Definitionen der Moderne?}
\begin{itemize}
    \item Haeufig sagt man: nach dem ersten Weltkrieg 1918 (Bauhaus)
    \item Aber: Jugendstil, Reformarchitektur, Eisen \& Stahlkonstruktionen etc..?
    \item Baustoffe, Schmuck etc.
    \item Seminar setzt im 19.Jh. an
\end{itemize}

\subsubsection{Wann endet die Moderne?}

\begin{itemize}
    \item Postmoderne, Zweite Moderne etc.?
    \item \emph{Modern is not contemporary}
\end{itemize}

\section{Wie beschreibe und analysiere ich moderne Architektur?\\(25.10.16)}

Treffen an der Ged"achtniskirche


\section{Zum Modernebegriff (nicht nur) in der Architektur\\(01.11.16)}

\subsection{Vornotizen}
\textbf{Definitionen und Abgrenzungen von Moderne}

\begin{description}[leftmargin=!,labelwidth=\widthof{\bfseries P2}]
  \item[Greenberg (Arch.)] \emph{Moderne Architektur bedeutet -- vereinacht gesagt -- funktionelle, geometrische Strenge und das Vermeiden von Dekoration und Ornament.} | Kap. `Modern und postmodern', Die Essenz der Moderne 
  \item[Greenberg (Kunst)] Bezeichnet K"unstler der Moderne als \emph{unbeugsamen Helden, der sich auf keinerlei Kompromisse mit dem Bestehenden einl"a\ss t und allen Verlockungen der eleganten Welt edelm"utig widersteht}; \emph{Bedeutende Werke wird ein K"unstler nur zu Wege bringen, wenn er alle au\ss erk"unstlcihen Anspr"uche und Erwartungen abweist, um seine Arbeit ganz aus sich sekbst heraus zu begr"unden.} $\rightarrow$ Das Publikum wurde dem Autor nach immer verst"andnisloser. | \emph{..selbstkritische Besinnung der Kunst auf ihre ureigenen und spezifischen M"oglichkeiten ist, Greenberg zufolge, das Charakterisikum des `Modernismus' }
  
  \item[Kretschmer] Latein: \emph{der Gegenwart angeh"orig} | im 19.Jh. als Gegensatz zu \emph{vergangen} | `Siegeszug' erst nach WWII | \emph{Die Moderne ist generell eine von technischem Positivismus gekennzeichnete Emanzipationsbewegung zur Befreiung aus der Fremdbestimmung.} | \emph{Es ging um eine m"oglichst gute Versorgung der neuen Massengesellschaft.} 
\end{description}

\textbf{Besprechung der Notizen}

\begin{itemize}
  \item Probleme zum einen der Datierung und der begrifflichen Festsetzung
  \item Es ist schwer w"ahrend einer Zeit "uber diese Zeit selber zu reden
  \item Freigang: Es gibt keine gro\ss e Erz"ahlung! Es besteht aus vielerlei Nebenwegen.
  \item Es ist kein Stil, kein formaler Rahmen, keine feste Zeit
\end{itemize}


\subsection{Sitzungsvortrag}

\begin{itemize}
  \item Seit Sp"atantike (5.Jh.): Die noch selbst erlebt Zeit
  \item Neuzeit: Kunst/Architektur die nicht dezidiert von der antiken Bautradition ableitbar ist (Schon Hauch von Avantgarde)
  \item Moderne in Geisteswissenschaften: industrielle Revolution, AUfkl"arung und S"akularisierung | Moderne in Philosophie: Aufkl"arung
  \item Beginn zwischen sp"aten 18. und mittleren 19. Jh. $\rightarrow$ Zeit des "Ubergangs von einem feudalistischen zu einem b"urgerlichen Gesellschaftsmodell
  \item Erste Erw"ahung 1886 von Eugen Wolff im Kontext der Literatur
  \item Moderne ist kein Stil- oder Epochenbegriff, sondern die Bezeichnung einer intellektuell wie formal (materiell wie gestalterisch) au\ss erordentlich vielf"altigen Str"omung.
  \item Bedingt durch die z.T. neuen, z.T. nun industriell herstellbaren Materialien Eisen/Stahl, Glas und sp"ater Beton: Industiebauten, Verkehrsbauten, Messebauten des 19. und 20. Jh.s (Paxton, Fr\`{e}res, Eiffel: \emph{Historische Moderne}) 
  \item Hochhausbauten des sp"aten 19. und erszen Drittel des 20. Jh.s in den USA; Arts and Crafts Architektur; Reformarchitektur (Jugendstil bzw. Wiener Sezession, Gartenstadtbewegung, Deutscher Werkbund, Darmst"adter Mathildenh"ohe, heimatbezofene Architektur)
  \item Klassiche Moderne 1910-1933 (Avantgardistische Architektur, Expressionistische, Organische Architektur, Traditionalismus) und viele mehr...
  \item Metabolismus, Cities:Moving
\end{itemize}


\section{Stahl und Glaskonstruktionen: Weltausstellungen des 19. Jahrhunderts\\(15.11.16)}

\subsection{Vortrag: Crystal Palace}

\begin{itemize}
  \item Crystal Palace. Jospeh Pacton 1851 Weltausstellungen London Hyde Park
  \item Warum ist er Modern und warum ist er Schl"usselwerk?
  \item These: 
\end{itemize}

\subsubsection{Hintergrund}

\begin{itemize}
  \item Nationalismus etc.
  \item Weltausstellungen Ziele: Plattform Vergleich von G"utern, Absatz f"ordern, Nation vereinen, statt gewalt in Industrie konkurrieren
  \item Schnell, Bau sollte demontierbar sein und B"aume sollten
  \item keiner der eingereichten Bau wird angenommen, paxton nach 5 Monaten mit Fox anderson \& co fertig gestellt
  \item wurde sp"ater nochmal aufgebaut; ist 1936 abgebrannt
  \item Vorl"aufer: Great Conservatory, 1941 von Paxton
  \item Bau ohne Ger"uste, Arbeiter verschraubten auf Leitern und dann hielt sich das Konstrukt von allein
\end{itemize}

\subsubsection{Beschreibung}

\begin{itemize}
  \item Flache D"acher (???) viele kleine Satteld"acher, Wasser kann gut ablaufen
  \item Besteht aus ganz vielen kleinen Modulen
  \item Ebenerdig, von B"aumen umringt
  \item Fassade besteht aus aneinandergereiten Glass...
  \item Grundriss: 563m lang 142m breit, 24m maximal h"ohe
  \item Querschiff war erst nicht geplant, doch 2 Ulmen mussten in den Bau inkludiert werden
  \item Haupteingang im S"uden
  \item Machinenraum im kleinen Glashaus
  \item Glasplatten waren gr"osser nicht herstellbar, und so musste sie die gesamte Formation des Baus nach deren Gr"osse ausrichten (\emph{kleinste Einheit})
  \item Komplett unverkleidet $\rightarrow$ Struktur des Baus ist gut erkennbar von innen wie von au"sen
\end{itemize}

\subsubsection{Wirkung und Wirken}
\begin{itemize}
  \item \emph{Wahrnehmungsschock}; \emph{aufl"sen der Raumgrenzen durch das Fehlen von W"anden}; \emph{Desorientierung} durch Licht das durch das Glasdach str"omt; \emph{Zweifel an Sicherheit}; Nutzen gibt Orientierung, Menschen werden; Semper: Keine Masse und Dekoratives, keine Feierlichkeit $\rightarrow$ \textbf{keine Architektur} sondern nur Nutzbau
  \item Druch die L"ange von fast 600m hatte der Bau eine fast unendliche Wirkung 
  \item Heute: symbol f"ur technischen Fortschritt, Wegbereiter zur Moderne
  \item Industriell vorfabrizierte Einzelteile
  \item Nutz statt Representativbau
  \item Rechenprozess ist Grundlage, nicht k"unstlerische Idee
  \item War gut f"ur neue industrielle Bauten die flexibel sein mussten (z.B. Kaufh"auser, Markthallen)
  \item neue Licht- und Luftgestaltung
  \item `Wendepunkt f"ur Entwicklung der Baugeschichte' | etwas "ubertrieben, eher: Bauten mit der selben \emph{technisierten} und effizienzorientierten Haltung
  \item Komplett ohne steinernem St"utzbau (im Gegensatz zu anderen Bauten wie Bahnh"ofen in Paris und London)
\end{itemize}


\subsection{Vortrag: Der Eiffelturm}

\subsubsection{Hintergrund}

\begin{itemize}
  \item Grund: Weltausstellung 1889 in Paris
  \item Ziel: Beeindrucken der Besucher
  \item Handelsminister: Ideenwettbewerb
  \item Eiffel hatte schon l"anger einen solchen Bau zum Ziel der riesig gro"s ist
  \item Eiffel hatte Apartment oben auf der Spitze
  \item 300m hoch
  \item wurde sp"ater zu einem Rundfunkturm
  \item Skelettbau der Freiheitsstatue
  \item 
\end{itemize}


\subsubsection{Beschreibung}

\begin{itemize}
  \item Quadratische Grundfl"ache 125m
  \item Steilaufragende Pyramide
  \item 3 Etagen, unterste mit verziertem Torbogen
  \item Stahlbeton
  \item Puddelverfahren: Stahl aus Roheisen satt Holzkohle
  \item Wie h"alt er sich: das Material! wiegt insgesamt 7000 Tonnen
  \item Einfaches Muster wirkt durch starke Wiederholung komplex
\end{itemize}


\subsubsection{Wirkung und Wirken}

\begin{itemize}
  \item Filigran und monstr"os zugleich
  \item L"oste Turmbauwelle aus
  \item Kritiker: K"unstlerisches Verbrechen gegen die Baukunst; Giraffenk"afig; Sinnlos
  \item Architekten fanden es unfein dass pl"otzlich Ingenieure bauen
  \item Nur weil er noch einen Bogen und ein paar Ornamente rangeklatscht bekommen hat ist es gleich Baukunst? 
  \item Modern weil: In die H"ohe gebaut, Ingenieursdenken, 
\end{itemize}


\section{Stahl und Glaskonstruktionen: Hochhausbauten\\(15.11.16)}


Was kann man nicht mehr mit stein machen?

\section{Buildings to live in, not to look at. Absagen an die Gliederungen des Historismus\\(22.11.16)}

Bei Postmoderne: McMansions!

%
%\newpage
%\section{"Uber den Professor}
%Prof. Dr. Kai Kappel
%\begin{figure}[h]
%	\centering
%	\includegraphics[width=0.5\textwidth]{images/template.png}
%	\caption{Template Bild}
%	\label{fig:template}
%\end{figure}


\end{document}
