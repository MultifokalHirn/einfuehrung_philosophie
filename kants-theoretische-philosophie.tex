\documentclass[emulatestandardclasses]{scrartcl}
\usepackage{graphicx}
\usepackage{color}
\usepackage[ngerman]{babel}
\usepackage{hyperref}
\usepackage{fullpage}
\usepackage[dvipsnames]{xcolor}
\usepackage{calc} 
\usepackage{enumitem}
\usepackage{titlesec}
\newcommand{\todo}[1]{\textcolor{red}{TODO: #1}\PackageWarning{TODO:}{#1!}}
\date{\vspace{-3ex}}
\begin{document}

\title{
	\includegraphics*[width=0.75\textwidth]{ErstesSem/images/hu_logo.png}\\
	\vspace{24pt}
	Einf"uhrung in Kants theoretische Philosophie}
\subtitle{VEV SS 17\\
          Prof. Dr. Tobias Rosefeldt\\
          Philosophisches Institut I \\ 
          Humboldt Universit"at zu Berlin}
\author{Lennard Wolf\\
        \small{\href{mailto:lennard.wolf@student.hu-berlin.de}{lennard.wolf@student.hu-berlin.de}}}
\maketitle
\begin{abstract}

In dieser Vorlesung soll ein "Uberblick "uber die wichtigsten Themen von Kants theoretischer Philosophie gegeben werden. Im Zentrum stehen wird seine ber"uhmte These, dass alle Dinge in Raum und Zeit nur "`Erscheinungen"' sind und wir nur solche Erscheinungen, nicht aber „Dinge an sich“ erkennen k"onnen. Wir wollen untersuchen, welche philosophischen Probleme Kant zu dieser These bewogen haben, was die These eigentlich genau bedeutet und welche Rolle sie f"ur die beiden wichtigsten Aufgaben von Kants kritischer Philosophie spielt, d.h. f"ur die Erkl"arung der M"oglichkeit einer bestimmten Form apriorischen, d.h. erfahrungsunabh"angigen Wissens, und f"ur die Fundamentalkritik an den Erkenntnisanspr"uchen der traditionellen Metaphysik.

\end{abstract}
\newpage

\tableofcontents
\listoffigures
\newpage


\section{Einf"uhrung / Ausgangspunkt 1: Probleme der Metaphysik\\(20.04.17)}

\subsection{Einf"uhrung}

\subsubsection{Organisatorisches}

\begin{itemize}
  \item Selbstorganisierte Lesegruppe (optional) 
  \item Tutorientermin Do 12-14
  \item Ziel der Vorlesung: Verstehen was Kant in der KdrV haben will.
\end{itemize}

\subsubsection{Kritik der reinen Vernunft}

\begin{description}[leftmargin=!,labelwidth=\widthof{\bfseries Transzendentaler Idealismus (TI)}]
  \item[Transzendentaler Idealismus (TI)] Unterscheidung zwischen Erscheinungen und den Dingen an sich (Erstmals in der Inauguraldissertation vertreten)
  \item[Erscheinungen] Alles sinnlich wahrgenommene, Dinge die uns \emph{affizieren} und auf diese Weise uns Kenntnis von ihnen geben | Vorstellungen | Beispiele: Tische, B"aume, wir selber | Gegensatz zu Dingen an sich (auch \emph{Phaenomena}) 
  \item[Dinge an sich] Die Dinge wie sie objektiv \emph{sind}, nicht wie sie wahrgenommen werden.
  \item[Eigenschaften] Wahrgenommenes der Erscheinungen
  \item[Subjektivit"at von Raum \& Zeit] Raum und Zeit sind Bestimmungen von Erscheinungen $\rightarrow$ Dinge an sich sind nicht in Raum und Zeit sondern erscheinen uns nur so weil wir eine Anschauung in Raum und Zeit haben 
\end{description}

\subsubsection{Fragen der Vorlesung}

\begin{enumerate}
  \item {\color{NavyBlue}Was genau bedeutet die Rede von Erscheinungen und Dingen an sich?}\\
{\color{ForestGreen} Zu kl"aren.}
  \item {\color{NavyBlue} Was ist Kants philosophische Motivationslage vor 1770, die einen verstehen l"asst, weshalb er sich zu so einer extremen Position wie TI gedr"angt zu fühlen meint?}\\
{\color{ForestGreen} Zu kl"aren.}
    \item {\color{NavyBlue} Was sind Kants Gr"unde dafür TI anzunehmen?}\\
{\color{ForestGreen} Zu kl"aren.}
    \item {\color{NavyBlue} Was k"onnen wir laut Kant jeweils über Erscheinungen und
über die Dinge an sich wissen?}\\
{\color{ForestGreen} Zu kl"aren.}
    \item {\color{NavyBlue} Welche Probleme ergeben sich aus TI?}\\
{\color{ForestGreen} Zu kl"aren.}
    \item {\color{NavyBlue} Wie unterscheidet sich die Theorie der Inauguraldissertation
eigentlich von der in der Kritik der reinen Vernunft (1781), wenn doch beide TI beinhalten?}\\
{\color{ForestGreen} Zu kl"aren.}
\end{enumerate}

\subsection{Ausgangspunkt 1: Probleme der Metaphysik}

\subsubsection{Philosophische Frustration}

\begin{itemize}
  \item Problem: Philosophie besteht nur aus unkl"arbaren Streitigkeiten
  \item Kant: Das liegt daran dass es sich um metaphysische Fragen handelte und diese sind f"ur den menschlichen Verstand nicht kl"arbar
\end{itemize}


\subsubsection{Antinomien der reinen Vernunft}

\textbf{Antinomie der Teilung}

\begin{description}[leftmargin=!,labelwidth=\widthof{\bfseries Contra-Argument}]
  \item[Frage] Bestehen materielle K"orper aus einfach Teilen oder nicht?
  \item[Pro-Argument] Kompoisitionsbeziehungen bestehen immer nur kontingenterweise $\rightarrow$ Was w"are wenn alle Beziehungen aufgehoben sind? $\rightarrow$ Es m"ussen kleinste Teile existieren, denn sonst w"urde alles aus nichts bestehen
  \item[Contra-Argument] Raum besteht nicht aus Punkten und l"asst sich daher immer teilen. Damit Gegenst"ande den Raum ausf"ullen k"onnen m"ussen sie ebenso nicht punktuell sein.
\end{description}
\vspace{9pt}
\noindent \textbf{Antinomie von Freiheit und Determinismus}

\begin{description}[leftmargin=!,labelwidth=\widthof{\bfseries "Uberzeugung 2}]
  \item["Uberzeugung 1] Manchmal sind wir frei in dem, was wir tun. Wir tun
etwas, weil wir uns aus freien St"ucken dazu entschieden haben, es zu tun.
  \item["Uberzeugung 2] Alles, was in der Welt geschieht, also auch jede unserer Handlungen und Entscheidungen, ist kausal durch das determiniert, was vorher in der Welt geschehen ist.
  \item[Frage] K"onnen "Uberzeugung 1 und 2 beide wahr sein? Wenn nicht, welche ist falsch?
\end{description}


\section{Ausgangspunkt 2: Synthetische Urteile a priori\\(27.04.17)}

\subsection{Lekt"urenotizen}

\begin{itemize}
  \item Erst Erfahrung, dann Erkenntnis
  \item doch nicht alle Erkenntnis entspringt aus Erfahrung
  \item Gibt es von der Erfahrung und selbst von allen Eindrücken der Sinne unabh"angiges Erkenntniß? $\rightarrow$ \emph{a priori} $\rightarrow$ Ich kann eine Erkenntnis haben ohne die daf"ur n"otige Erfahrung zu machen (Einfallen des Hauses bei Untergraben)
  \item a priori:
  \item a posteriori: empirische Erkenntnisse/ Erfahrungserkenntnisse
  \item analytisch: im Begriff selber
  \item synthetisch: \emph{irgendwie} erschlossen | WIE?
\end{itemize}

\subsection{Kants Hauptthese (Transzendentaler Idealismus)}

\begin{description}[leftmargin=!,labelwidth=\widthof{\bfseries ii}]
  \item[i] Wir m"ussen unterscheiden zwischen subjektabh"angigen Erscheinungen, d.h. den Gegenst"anden unserer sinnlichen Vorstellungen, und den Dingen, so wie sie an sich selbst beschaffen sind.
  \item[ii] Raum und Zeit sind blo"s Bestimmungen von Erscheinungen, d.h. Dinge sind nicht an sich selbst und unabh"angig von uns in Raum und Zeit.
\end{description}

\subsection{Motivation}

\begin{description}[leftmargin=!,labelwidth=\widthof{\bfseries Erkenntnistheorie}]
  \item[Metaphysik] Was zu zeigen ist: TI –- und nur TI –- kann erkl"aren, wie die widersprechenden Intuitionen zustande kommen und kompatibel gemacht werden k"onnen, und so den Streit schlichten.
  \item[Erkenntnistheorie] Was zu zeigen ist: TI –- und nur TI –- kann erkl"aren, wie diese Art von Wissen m"oglich ist. Es geht um Wissen, das in sogenannten \emph{synthetischen Urteilen a priori} zum Ausdruck kommt.
\end{description}


\subsection{Begriffe}

\begin{description}[leftmargin=!,labelwidth=\widthof{\bfseries ii}]
  \item[Was hei"st es, dass der Begriff des Unverheiratetseins in dem
des Junggesellen enthalten ist?] Alle K"orper sind ausgedehnt‘ [...] ist ein analytisch Urtheil. Denn ich darf nicht "uber den Begriff, den ich mit dem Wort K"orper verbinde, hinausgehen, um die Ausdehnung als mit demselben verkn"upft zu finden, sondern jenen Begriff nur zergliedern, d.i. des Mannigfaltigen, welches ich jederzeit in ihm denke, mir nur bewu"st werden, um dieses Pr"adicat darin anzutreffen
  \item[] 
\end{description}

\section{Raum (und Zeit) als Formen der Anschauung\\(04.05.17)}

\subsection{Raum als Anschauung a priori}

\subsubsection{Das Argument}

\begin{description}[leftmargin=!,labelwidth=\widthof{\bfseries P2}]
  \item[P1] In der Geometrie gibt es synthetische Urteile a priori.
  \item[P2] (P1) kann nur dann wahr sein, wenn wir eine Anschauung a priori vom Raum haben.
  \item[P3] Es ist nur dann m"oglich, eine Anschauung a priori vom Raum zu haben, wenn sich die R"aumlichkeit alles Angeschauten allein der Weise verdankt, wie wir das "au"serlich Angeschaute repr"asentieren. (Geist ordnet Empfindungen auf r"aumliche Art und Weise)
  \item[K] Die R"aumlichkeit alles Angeschauten verdankt sich allein der Weise, wie wir das "außerlich Angeschaute repr"asentieren.
\end{description}

\subsubsection{M"oglicher Einwand}

Ok, wenn wir eine Anschauung a priori des Raumes h"atten, dann würde das erkl"aren, wie es respektable synthetische Urteile a priori in der Geometrie geben kann.

ABER: Die Annahme, dass wir eine Anschauung a priori vom Raum haben (und ihre Implikation, dass der Raum nur die Form unserer "außeren Anschauung ist), ist ausgesprochen merkw"urdig, und deswegen ist die Erkl"arung nicht "uberzeugend.

(vgl. Heidi-Fall: "`Ich kann deswegen alle Menschen durch mein Mantra gl"ucklich machen, weil menschliche Seelen letztlich nur Knoten in einem Netz kosmischer Energie sind, das ich durch das Mantra in die richtige Schwingung bringe!"')

\subsection{Die Vier Raumargumente}

\subsubsection{Das erste Raumargument}

"`Der Raum \emph{ist kein empirischer Begriff, der von "außeren Erfahrungen abgezogen (abstrahiert) worden. Denn damit gewisse Empfindungen auf etwas au"ser mich bezogen werden, (d.i. auf etwas in einem anderen Orte des Raumes, als darinnen ich mich befinde,) imgleichen, damit ich sie als außer und neben einander, mithin nicht blo"s verschieden, sondern als in verschiedenen Orten vorstellen k"onne, dazu mu"s die Vorstellung des Raumes schon zum Grunde liegen. Demnach kann} die Vorstellung des Raumes \emph{nicht aus den Verh"altnissen der "außeren Erscheinungen durch Erfahrung erborgt sein, sondern diese "au"sere Erfahrung ist selbst nur durch gedachte Vorstellung allererst m"oglich.}"' (KrV A 23/B 38)

\begin{description}[leftmargin=!,labelwidth=\widthof{\bfseries P2}]
  \item[P1] Um verschiedene Gegenst"ande als au"ser mir und außer und nebeneinander seiend zu repr"asentieren, muß ich die Vorstellung vom Raum bereits haben.
  \item[P2] Um die Vorstellung vom Raum aus der Erfahrung gewinnen zu k"onnen, müßte ich zuerst verschiedene Gegenst"ande als au"ser mir und außer und nebeneinander seiendseiend repr"asentieren.
  \item[K] Also kann ich die Vorstellung vom Raum nicht aus der Erfahrung gewinnen.
\end{description}

A und B als nebeneinander seiend repr"asentieren impliziert, den Raum um A und B herum zu repr"asentieren.	

\subsection{Die Idealit"at des Raumes}

\begin{description}[leftmargin=!,labelwidth=\widthof{\bfseries P2}]
  \item[P1] Um verschiedene Gegenst"ande als au"ser mir und außer und nebeneinander seiend zu repr"asentieren, muß ich die Vorstellung vom Raum bereits haben.
  \item[P2] Um die Vorstellung vom Raum aus der Erfahrung gewinnen zu k"onnen, müßte ich zuerst verschiedene Gegenst"ande als au"ser mir und außer und nebeneinander seiendseiend repr"asentieren.
  \item[K] Also kann ich die Vorstellung vom Raum nicht aus der Erfahrung gewinnen.
\end{description}

\section{Die Unterscheidung zwischen Erscheinungen und Dingen an sich\\(11.05.17)}

\subsection{Einleitung}

\begin{itemize}
  \item "`Brille"' ist \emph{vor} der Erfahrung
  \item Wie ist Trendelenburgscher L"ucke zu l"osen? 
  \item Es stellen sich 2 Fragen:
\end{itemize}

\begin{description}[leftmargin=!,labelwidth=\widthof{\bfseries P2}]
  \item[Was genau sind Erscheinungen, was genau Dinge an sich?] 
  \item[In welchem Sinne sind Erscheinungen von uns und unserem Geist anh"angig?]
\end{description}

\subsection{Die ph"anomenalistische Zwei-Welten-Interpretation}

\subsubsection{Die Idee}

\begin{description}[leftmargin=!,labelwidth=\widthof{\bfseries M}]
  \item[Was genau sind Erscheinungen, was genau Dinge an sich?] Erscheinungen sind Vorstellungen, d.h. mentale Zust"ande von Subjekten, deren Existenz und Beschaffenheit vollst"andig von diesen abh"angen. Dinge an sich sind au"sergeistige Entit"aten.
  \item[In welchem Sinne sind Erscheinungen von uns und unserem Geist anh"angig?] Erscheinungen sind in genau dem Sinne subjektiv, in dem auch Vorstellungen (bzw. deren interne Gegenst"ande) subjektiv sind. Ihre Existenz und Beschaffenheit h"angt vollst"andig von der eines bestimmten Subjekts ab.
\end{description}

Die idealistische Interpretation impliziert, daß es sich bei Erscheinungen und Dingen an sich um zwei verschiedene Arten von Dingen handelt: Erscheinungen sind unsere eigenen Vorstellungen. Dinge an sich sind von uns verschiedene und ontologisch von uns unabh"angige Gegenst"ande in der Welt.

\subsubsection{Probleme}

\begin{description}[leftmargin=!,labelwidth=\widthof{\bfseries M}]
  \item[Innerer und "Au"serer Sinn] Unsere eigenen mentalen Zust"ande sind laut Kant ausschließ- lich Gegenst"ande unseres inneren Sinns und sind nur zeitlich, nicht r"aumlich geordnet. Gegenst"ande im Raum sind aber Gegenst"ande des "au"seren Sinns und stehen in r"aumlichen Verh"altnissen zueinander. W"aren Erscheinungen wie K"orper Vorstellungen, dürften sie ebenfalls nur Gegenst"ande des inneren Sinns sein.
  \item[Fehlende Intersubjektivit"at] Laut Kant ist unsere Erkenntnis der Seele und ihrer Zust"ande deswegen defizit"ar, weil jeder nur seine eigene Seele anschauen kann. Dagegen meint er, dass verschiedene Subjekte dieselben Gegenst"anden im Raum wahrnehmen k"onnen. Wie aber k"onnten rein intentionale Gegenst"ande (wie Traumobjekte) intersubjektiv zug"anglich sein?
  \item[Kants Anticartesianismus] Die Interpretation setzt voraus, daß wir einen besseren/ unmittelbareren Zugang zu unserem eigenen Geist haben als zu au"sergeistigen Gegenst"anden. Kant nimmt aber an, da"s wir auch unsere eigene Seele und ihre Zust"ande (wie Vorstellungen) nur als Erscheinungen erkennen, nicht so wie sie an sich selbst sind. Wenn "`K"orper sind Erscheinungen"' soviel hei"sen w"urde wie „was wir für K"orper halten, sind nur Vorstellungen von K"orpern"', w"urde "`unsere eigenen Vorstellungen sind Erscheinungen"' hei"sen "`was wir f"ur unsere eigenen Vorstellungen halten, sind nur Vorstellungen von Vorstellungen"'. usw. \emph{ad infinitum}.
  \item[Die Identit"at von Erscheinungen und D.a.s.] Die idealistische Interpretation impliziert, dass es sich bei Erscheinungen und Dingen an sich um zwei verschiedene Arten von Dingen handelt. Kant betont aber wiederholt (besonders in der 2. Auflage der Kritik der reinen Vernunft, dass es ihm bei der Unterscheidung zwischen Erscheinungen und Dingen an sich nicht um die Unterscheidung zwischen zwei Arten von Gegenst"anden geht, sondern darum, zwischen der Weise zu unterscheiden, auf die uns Gegenst"ande erscheinen, und der Weise, wie eben diese Gegenst"ande an sich selbst beschaffen sind.
\end{description}

\subsection{Die Zwei-Aspekte Interpretation}

\subsubsection{Die Idee}

\begin{description}[leftmargin=!,labelwidth=\widthof{\bfseries M}]
  \item[Was genau sind Erscheinungen, was genau Dinge an sich?] Erscheinungen sind Gegenst"ande als Tr"ager von Eigenschaften, die wesentlich damit zu tun haben, wie uns diese Gegen- st"ande erscheinen. Dinge an sich sind Gegenst"ande als Tr"ager von Eigenschaften, die ihnen unabh"angig davon zukommen, wie sie uns erscheinen.
  \item[In welchem Sinne sind Erscheinungen von uns und unserem Geist anh"angig?] Eigenschaften, die wesentlich damit zu tun haben, wie uns diese Gegenst"ande erscheinen, sind in dem Sinne subjekt- abh"angig, dass Gegenst"ande sie verlieren würden, wenn unser Geist auf bestimmte Weise anders beschaffen w"are (z.B. wenn wir andere Formen der Anschauung h"atten).
\end{description}

%\newpage
%\section{"Uber den Professor}
%Prof. Mustermann ist..


%\begin{figure}[h]
%	\centering
%	\includegraphics[width=0.5\textwidth]{images/template.png}
%	\caption{Template Bild}
%	\label{fig:template}
%\end{figure}

\end{document}


