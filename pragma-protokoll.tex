\documentclass[a4paper, emulatestandardclasses]{scrartcl}
\usepackage{graphicx}
\usepackage{fullpage}
%\usepackage{parskip}
\usepackage{color}
\usepackage[ngerman]{babel}
\usepackage[utf8]{inputenc}
\usepackage{hyperref}
\usepackage[normalem]{ulem}
\usepackage{calc} 
\usepackage{enumitem}
\usepackage{titlesec}
\usepackage{bussproofs}
\usepackage[export]{adjustbox}
%\pagestyle{headings}

\titleformat{name=\section,numberless}
  {\normalfont\Large\bfseries}
  {}
  {0pt}
  {}
\date{\vspace{-3ex}}
\begin{document}

\title{
    \vspace{-30pt}
	\includegraphics*[width=0.1\textwidth,right]{ErstesSem/images/hu_logo2.png}\\
	\vspace{-10pt}
	Rorty II (25.01.18)}%}\\\vspace{10pt}\small{Lennard Wolf}}
	\subtitle{Pragmatismus (Lesegruppe, WS 17/18)\\
          %Dozent: Prof. Michael Beaney\\
          Protokollant: Lennard Wolf}
\maketitle
\vspace{-40pt}

\section*{Rortys Text}
%\textbf{Korrekturen von Missverständnissen}

\textbf{Herangehensweise von Rorty} 

Viele Philosophen von seiner Zeit anschauen und zeigen, dass alle auf das selbe hinzielen, nämlich: Transzendentalen Charakter der Epistemologie aufheben (Putnam, Davidson). Zudem: Anschein aufheben, dass wir mit unserer Sprache wirklich, dass die Natur Kriterien hergibt, wie sie zu beschreiben wäre (Putnams realism $\rightarrow$ not that language mirrors the world, but that our mirror of nature is set of diagrams that \emph{cope} with nature:

 "`\emph{It is the difference between, roughly, science as a Mirror of Nature, and as a set of working diagrams for coping with nature. To say that we are coping, by our lights, pretty well is true but trivial. To say that we are mirroring correctly is "only a picture," and one which we have never been able to make.}"')\newline

\noindent\textbf{Davidsons Thesen}

1. Philosophie der Sprache und die Frage danach, was es überhaupt gibt, sind getrennt! Analyse der Sprache ist eben nur das, keine Analyse der Welt! 

2. Dass Wahrheit und Bedeutung einer Sprache nicht zu trennen sind. Schwer zu sagen, ob eine Sprache, die komplett andere conceptual schemes haben auch \emph{wahr} sind. Wahrheit ist immer nur in einer Sprache.

\section*{Diskussion}


\begin{itemize}
  \item Realtivismus: Vermengung von "`für wahr halten"' und "`Wahrheit"': \emph{Weil viele anderes für wahr halten, gibt es keine Wahrheit.}

  \item Metaepistemologie: Was sind unsere Kriterien für Wahrheit? 

  \item Welche Kriterien kann Natur uns geben für Sprache? Die mit der besseren Sprache überleben besser. 

  \item Es liegt an uns zu sagen, ob die Einteilung die wir machen richtig ist oder nicht (Pragmatismus)

  \item Wir stützen uns immer nur auf unsere schon vorher bestehenden Überzeugungen und rationalisieren nur (bei Diskussionen über Metaphysik und so) |  Diese Diskussionen werden aber niemals irgendwo ankommen. "`Worüber man nicht sprechen kann davon soll man schweigen"'

  \item Die Philosophen haben ihre Diskussionen sich selbst geschaffen. Descartes et al.: Künstlicher Zweifel

  \item Wir können die Probleme die wir uns selbst gestellt haben nicht mit unserer Sprache lösen.
\end{itemize}

Davidsons Sprachholismus: "`If sentences depend for their meaning on their structure, and we understand the meaning of each item in the structure only as an abstraction from the totality of sentences in which it features, then we can give the meaning of any sentence (or word) only by giving the meaning of every sentence (and word) in the language."' (S. 302)\newline


\noindent\textbf{Teil VI: Truth Goodness und Relativism}

\begin{itemize}
  \item Alltägliche Bedeutung von "`Gut"' ist in der Sprache und die philosophische Bedeutung von "`Gut"' ist der Versuch, das aus der Sprache heraus zunehmen. 
  \item Bernard Williams, \emph{Wahrheit und Wahrhaftigkeit}: Genealogie der Wahrhaftigkeit | All diejenigen, die gegen dem Wahrheitsanspruch philosophischer Theorien argumentieren, arbeiten am selben Projekt!
  \item Apel: Der performative Selbstwiderspruch, bei Vorwürfen, es gäbe keine Wahrheit.
  \item Jede Sprache braucht eine Menge von Wahrheiten als Grundlage
  \item Quine: Ontologische Verpflichtungen; Naturalismuskritik ist meist impulsiv
\end{itemize}

\noindent\textbf{Fazit der Gruppe}

Gebrauchstheorie der Wahrheit und Kontextualismus sind von allen gut nachvollziehbar


\section*{Offenes}

\begin{itemize}
  \item \sout{Vorschlag: Wir sollten Putnam lesen}
\end{itemize}

\end{document}
