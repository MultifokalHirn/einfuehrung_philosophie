\documentclass[a4paper, 12pt]{article}
%\usepackage{CJKutf8} % japanese
\usepackage{graphicx}
\usepackage{hyperref}
\usepackage{fullpage}
%\usepackage{parskip}
\usepackage{color}
\usepackage[ngerman]{babel}
\usepackage{hyperref}
\usepackage{calc} 
\usepackage{enumitem}
\usepackage[utf8]{inputenc}
\usepackage{titlesec}
%\pagestyle{headings}
\usepackage{setspace} %halbzeilig
\usepackage[style=authoryear-ibid,natbib=true]{biblatex}
\usepackage[hang]{footmisc}
\setlength{\footnotemargin}{-0.8em}
%\bibliographystyle{natdin}
\addbibresource{wissenschaftl-arbeiten-ha1.bib}
\DeclareDatamodelEntrytypes{standard}
\DeclareDatamodelEntryfields[standard]{type,number}
\DeclareBibliographyDriver{standard}{%
  \usebibmacro{bibindex}%
  \usebibmacro{begentry}%
  \usebibmacro{author}%
  \setunit{\labelnamepunct}\newblock
  \usebibmacro{title}%
  \newunit\newblock
  \printfield{number}%
  \setunit{\addspace}\newblock
  \printfield[parens]{type}%
  \newunit\newblock
  \usebibmacro{location+date}%
  \newunit\newblock
  \iftoggle{bbx:url}
    {\usebibmacro{url+urldate}}
    {}%
  \newunit\newblock
  \usebibmacro{addendum+pubstate}%
  \setunit{\bibpagerefpunct}\newblock
  \usebibmacro{pageref}%
  \newunit\newblock
  \usebibmacro{related}%
  \usebibmacro{finentry}}

%\titleformat{name=\section,numberless}
%  {\normalfont\Large\bfseries}
%  {}
%  {0pt}
%  {}
\date{\vspace{-3ex}}
\begin{document}

\title{\vspace{5ex}
	\includegraphics*[bb=0 0 720 200, width=0.72\textwidth]{ErstesSem/images/hu_logo.png}\\
	\vspace{30pt}
	\scshape\LARGE{Zusammenfassung I}\\\Large{Piraterie vor den afrikanischen\\Küsten und ihre Ursachen}\\\vspace{20pt}}
	


\author{Einführung in das wissenschaftliche Arbeiten\\
	\vspace{7pt}
          Dozent: Prof. Dr. Michael Mann\\\vspace{4pt}Lennard Wolf\\
        \small{Matrikelnummer: 583052}\\
        \small{E-Mail: lennard.wolf@hu-berlin.de}}

        %\href{mailto:lennard.wolf@student.hu-berlin.de}{lennard.wolf@student.hu-berlin.de}}}      

\maketitle

\vspace{\fill}

\begin{minipage}[]{0.92\textwidth}
    \centering
    \onehalfspacing
    \large   
    28. November 2017\\
    Wintersemester 17/18

    \vspace{-20mm} 
\end{minipage}%
\thispagestyle{empty}
\newpage
\null
\thispagestyle{empty}
%\clearpage
%\thispagestyle{empty}
%\tableofcontents
\newpage
\setcounter{page}{1}

\begin{onehalfspace} 

%Einführung

\noindent In ihrem 2009 veröffentlichten Essay \emph{Piraterie vor den afrikanischen Küsten und ihre Ursachen} befassen sich Edward A. Ceska und Michael Ashkenazi mit der Frage nach den Ursachen der zunehmenden Piraterie in Somalia und Nigeria. Anhand der daraus gewonnenen Erkenntnisse kritisieren sie die bisherigen Gegenmaßnahmen und stellen alternative Empfehlungen vor.

%Somalia

Seit dem Ende des Barre Regimes 1991 leidet Somalia immer wieder unter Stammeskriegen, Hungersnöten und politischer Unruhe. Die zu der Zeit der Veröffentlichung des Quelltextes bestehende Übergangsregierung ist nicht in der Lage, diese Probleme zu bewältigen. Die Küsten und küstennahen Gebiete sind wegen der daraus entstehenden Quasiautonomie als "`Niemandsmeer"' zu bezeichnen. Die 20 000 Schiffe, die dort jährlich verkehren, sind zunehmend von Piraterie betroffen, die wie andere illegale Handlungen aufgrund mangelnder Kontrolle zunehmen. Clankriege verschlimmern die Situation zudem, wodurch junge Männer in die Piraterie geführt werden, um Vergeltung üben zu können. Die Piraten verfügen große Mengen an Waffen und sind in der Lage, komplexe Operationen mit Mutterschiffen zu organisieren. So kann Piraterie ein gutes Leben ermöglichen in einem Land, das von Elend geplagt ist. 

%Nigeria

Nigeria ist seit der Entdeckung von Ölreserven im Nigerdelta 1958 in religiösen, wirtschaftlichen und politischen Konflikten verwickelt. Besonders die Bewohner des besagten Deltas sind von den Aktivitäten der Ölkonzerne betroffen. Diese geben zwar Lizenzgebühren an die Regierung ab, doch davon komm nur bedingt etwas bei den Betroffenen an. Folglich entstanden militante Bewegungen, die zum einen die Umstände der Anwohner verbessern, zum anderen aber auch Vergeltung üben wollen, aufgrund der Naturverschmutzung und der Bohrung nach dem vermeintlich ihnen zustehenden Öl. Die Tätigkeiten dieser Piratengruppen, das heißt Ölraub, Entführungen zu Lösegeldzwecken und Anschläge sind also letztendlich auf Rache zurückzuführen, womit auch das aggressivere Vorgehen zu erklären ist. Leichter Zugang zu Waffen, sowie tendenzielle Zunahme an Unmut im Delta aufgrund fehlender Veränderung der Situation, deuten nur auf Zuwachs der militanten Aktivitäten hin. 

%Ähnlichkeiten, Unterschiede, Lösungen und Zukunftsaussichten

Da es sowohl in Somalia, wie auch in Nigeria an Kontrolle über den Küstenraum mangelt, und die soziale Situation jeweils mangelhaft ist, wird Piraterie von den Gesamtsituationen nur immer mehr befördert. Desillusionierte junge Menschen erhoffen sich durch sie eine Chance auf ein besseres Leben für sich und ihre Familie. Ein Unterschied zwischen den Ländern besteht jedoch in den Motivationen der Piraten, welche in Nigeria auch zu sinnloser Gewalt führen können, in Somalia wiederum nicht. Die Somalis sind besser organisiert in ihrem Streben nach finanziellen Mitteln und Rohstoffen, die Nigerianer bleiben gefangen in ihrem "`Ressourcenfluch"'.

%Empfehlungen

Anstatt die Gegenmaßnahmen der internationalen Gemeinde wie bisher auf den Schutz von Schiffen zu beschränken, sollte laut den Autoren gesichert werden, dass in Somalia die Übergangsregierung unterstützt wird darin, die Küstenregionen zu schützen, und in Nigeria, dass die Lizenzgelder der Ölkonzerne auch bei den Bewohnern des Nigerdelta ankommt. 




\end{onehalfspace}
\nocite{*}
%\bibliography{merleau-ponty-essay}
\printbibliography
\end{document}
