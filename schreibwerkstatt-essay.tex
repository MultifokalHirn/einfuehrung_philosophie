\documentclass[a4paper, emulatestandardclasses, 12pt]{scrartcl}
\usepackage{graphicx}
\usepackage{fullpage}
%\usepackage{parskip}
\usepackage{color}
\usepackage[ngerman]{babel}
\usepackage{hyperref}
\usepackage{calc} 
\usepackage{enumitem}
\usepackage{titlesec}
%\pagestyle{headings}
\usepackage{setspace} %halbzeilig
\usepackage[authoryear,round]{natbib}
\bibliographystyle{natdin}

\date{\vspace{-3ex}}
\begin{document}

\title{\vspace{5ex}
	\includegraphics*[width=0.72\textwidth]{images/hu_logo.png}\\
	\vspace{30pt}
	\scshape\LARGE{Warum Searles chinesischer\\Raum ein Bewusstsein hat [AT]}}
	
	\subtitle{\vspace{20pt}"Ubung Schreiben und Argumentieren\\
          %\vspace{6pt}
          %Essay\\
          }

\author{\vspace{-4pt}Lennard Wolf\\
        \small{\href{mailto:lennard.wolf@student.hu-berlin.de}{lennard.wolf@student.hu-berlin.de}}}      
%Abstract?

\maketitle

\vspace{\fill}

\begin{minipage}[b]{\textwidth}
    \centering
    \onehalfspacing
    \large   
    %30. November 2016\\
    Wintersemester 2016/2017

    \vspace{-20mm} 
\end{minipage}%
\thispagestyle{empty}
\newpage
\clearpage
\setcounter{page}{1}

\begin{onehalfspace} 

%\noindent\textbf{Fragestellung:}\\
%\indent Kann man mit Hilfe von Russells Theorie der Kennzeichnungen Freges R"atsel l"osen, ohne (wie Frege) die Existenz von Sinnen anzunehmen?\\\indent Wenn ja: Wie? Wenn nein: Warum nicht?
%
%\begin{center}
%\vspace{-9pt}\line(1,0){450}
%\end{center}


\noindent In dieser Arbeit m"ochte ich Gr"unde daf"ur anf"uhren, weshalb Searles Gedankenexperiment des chinesischen Raums die M"oglichkeit von maschinellem, beziehungsweise Software-getriebenem Bewusstsein nicht widerlegt.

Dazu werde ich wie folgt vorgehen. In $(i)$ umrei"se ich den diskursiven Kontext (?), in welchem das Gedankenexperiment entstanden ist und in $(ii)$ beschreibe ich es. %Es folgt in $(v)$ eine Gegen"uberstellung der beiden L"osungen, in der gezeigt wird, warum Russells Modell auf Freges Sinne verzichten kann.% In $(vi)$ schliesse ich den Kurzessay ab. 
\vspace{5mm}

\noindent\textbf{$(i)$ Der chinesische Raum}

\begin{itemize}
  \item Kontext (Turing etc)
  \item Der chinesische Raum
  \item Searles Argumentation
  \item Searles Pr"amissen
  \item Das menschliche Bewusstsein
  \item Konklusion
\end{itemize}



\vspace{3mm}

%\noindent\textbf{$(v)$ Gegen"uberstellung}



\end{onehalfspace}

\bibliography{schreibwerkstatt-essay}

\end{document}
