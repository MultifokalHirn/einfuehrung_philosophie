\documentclass[a4paper, emulatestandardclasses, 12pt]{scrartcl}
\usepackage{graphicx}
\usepackage{fullpage}
%\usepackage{parskip}
\usepackage{color}
\usepackage[ngerman]{babel}
\usepackage{hyperref}
\usepackage{calc} 
\usepackage{enumitem}
\usepackage{titlesec}
%\pagestyle{headings}
\usepackage{setspace} %halbzeilig
\usepackage[authoryear,round]{natbib}
\bibliographystyle{natdin}

\date{\vspace{-3ex}}
\begin{document}

\title{%\vspace{5ex}
	\includegraphics*[width=0.1\textwidth]{images/hu_logo2.png}\\
	\vspace{50pt}
	\scshape\LARGE{Wissen, Denken \& Verst"andnis}}
	
	\subtitle{"Uberlegungen zu Searles chinesischem Zimmer}
\author{Lennard Wolf\\
        \small{\href{mailto:lennard.wolf@student.hu-berlin.de}{lennard.wolf@student.hu-berlin.de}}}      
%Abstract?

\maketitle

\vspace{\fill}

\begin{minipage}[b]{\textwidth}
    \centering
    \onehalfspacing
    \large   
    %30. November 2016\\
    "Ubung Schreiben und Argumentieren\\
    Wintersemester 2016/2017

    \vspace{-20mm} 
\end{minipage}%
\thispagestyle{empty}
\newpage
\clearpage
\setcounter{page}{1}

\begin{onehalfspace} 

%\noindent\textbf{Fragestellung:}\\
%\indent Kann man mit Hilfe von Russells Theorie der Kennzeichnungen Freges R"atsel l"osen, ohne (wie Frege) die Existenz von Sinnen anzunehmen?\\\indent Wenn ja: Wie? Wenn nein: Warum nicht?
%
%\begin{center}
%\vspace{-9pt}\line(1,0){450}
%\end{center}


\noindent In dieser Arbeit f"uhre ich Gr"unde daf"ur an, weshalb Searles Gedankenexperiment des chinesischen Zimmers die M"oglichkeit von maschinellem Verst"andnis von Sachverhalten nicht widerlegt. Ich werde zeigen, dass sich hinter der von Searle angef"uhrten, ihm zufolge den Menschen vorbehaltene Bef"ahigung zu "`Verst"andnis"' und "`Intentionalit"at"' nichts weiter verbirgt, als eine vom physischen K"orper getrennte "`Seele"'.

Dazu werde ich wie folgt vorgehen. In $(i)$ beschreibe ich den Turing Test, welcher dem in $(ii)$ beschriebenen Gedankenexperiment des chinesischen Zimmers zugrunde liegt. Es folgt in  $(iii)$ eine Erkl"arung von Searles Begriff des "`Verst"andnis"' und in $(iv)$ das typische Gegenargument, n"amlich das der \emph{Systemtheorie}. Daraufhin f"uhre ich in $(v)$ meine "Uberlegungen zu der Austauschbarkeit von \emph{expliziten} und \emph{impliziten} Informationen an, welche ich in $(vi)$ auf das Wissen und Denken des Menschen "ubertrage. In $(vii)$ folgt die Konklusion aus dem Zuvorgegangenen.
\vspace{5mm}


\noindent\textbf{$(i)$ Der Turing Test}

\noindent In seinem einflussreichen Essay "`Computing machinery and intelligence"' \citep{turing1950computing} stellte Alan Turing das erste Mal seine Variation des \emph{Imitation Game} vor, heute besser bekannt als \emph{Turing Test}. Dieser Test, der zugleich als Gedankenexperiment, wie auch als tats"achlich ausf"uhrbar anzusehen ist, geht der Frage nach, ob Maschinen denken k"onnen. 

Es handelt sich dabei um ein Verh"ohr, in dem eine Person zwei Individuen unterschiedlichste Fragen stellt um herauszufinden, welche von beiden ein \emph{Mensch} und welche eine \emph{Maschine} ist. Zentral ist, dass die Kommunikation nur auf schriftlichem Wege stattzufinden hat, denn Turing zufolge sei es nicht zielf"uhrend, das menschliche Denken auf "Au"serlichkeiten zu reduzieren.\footnote{Dies l"asst sich zum Beispiel an den folgenden Textstellen erkennen: "`[...] we should feel there was little point in trying to make a `thinking machine' more human by dressing it up in [...] artificial flesh. [...] We do not wish to penalise the machine for its inability to shine in beauty competitions, [...]"'. \citep[S. 434]{turing1950computing}} Turing betont also die Trennung des Denkens von seinem physischen Ursprung, oder auch "`K"orper"'. Eine Maschine hat genau dann den Turing Test bestanden, wenn die verh"ohrende Person es nicht schafft herauszufinden, welche der befragten Individuen mit Sicherheit der "`echte Mensch"' ist. Solch eine Maschine w"are dann entsprechend in der Lage so zu denken wie es Menschen tun. Mit der Existenz solch einer Maschine w"are die urspr"ungliche Frage, ob Maschinen denken k"onnen, beantwortet.

%Es sollte angemerkt werden, dass Turing, nach der Darstellung des \emph{Imitation Game}, eine Reihe von m"oglichen Einw"anden gegen die Aussagekraft des Experiments vorstellt und jeweils anf"uhrt, warum sie seiner Meinung nach nicht gelten. %Ich m"ochte hier schon erw"ahnt wissen, dass der nun folgende Einwand von Searle mir nur wie eine Kombination aus den von Turing schon besprochenen Einw"anden ist.

%\newpage
\vspace{5mm}
\noindent\textbf{$(ii)$ Das chinesische Zimmer}

\noindent Das Gedankenexperiment des chinesischen Zimmers stellte John Searle in seinem Essay "`Minds, brains, and programs"' \citep{searle1980minds} vor. In diesem wird ein Turing Test auf chinesisch vollzogen, und die "`Maschine"' besteht aus einem Zimmer, in dem sich ein Mensch ohne jedwede Kenntnisse der chinesischen Sprache befindet, sowie eine Menge an Zetteln auf denen Paare von chinesischen Schriftzeichenketten stehen. Wenn die (chinesisch sprechende) verh"ohrende Person einen Zettel mit einer Schriftzeichenkette hineinreicht, sucht die Person in dem Zimmer nach ebendieser in seinem Zettelstapel, schreibt die damit gepaarte Schriftzeichenkette (die "`Antwort"' auf die "`Frage"') auf einen Zettel und reicht diesen wieder nach drau"sen. 

Auf die fragende Person wirke es nun so, dass die Person in dem Zimmer chinesisch versteht, was ja aber bekannterweise nicht der Fall ist. Damit m"ochte Searle zeigen, dass eine Maschine, die den Turing Test besteht, die Verh"ohrfragen nicht zwangsl"aufig \emph{verstehen} muss, und genauso wenig die Ausgaben, die sie produziert. Vielmehr wurde nur die F"ahigkeit pr"asentiert, syntaktische Anweisungen zu befolgen und kein semantisches Verst"andnis, was jedoch ein Kernmerkmal des menschlichen Denkens sei. Daher, so Searle, k"onne man aus dem Turing Test weder auf Intelligenz, noch auf Verst"andnis, oder gar Bewusstsein schlie"sen. Einer Turing Test bestehenden Maschine zu unterstellen, sie bes"a"se ein Denkverm"ogen, das mit dem des Menschen vergleichbar ist, so folgert Searle, w"are also nicht g"ultig.\footnote{vgl. \cite[S.424]{searle1980minds}.}

\vspace{5mm}
\noindent\textbf{$(iii)$ Verst"andnis}

\noindent Da Searle davon spricht, dass eine Maschine kein Verst"andis (\emph{understanding}) von ihren "`Ein- und Ausgaben"' vergleichbar zu einem Menschen habe, ist es wichtig, seine Deutung des Begriffs noch einmal genauer zu betrachten.

Searle f"uhrt als Beispiel an, dass es einen Unterschied gibt zwischen seinem Verst"andnis von der Englischen Sprache und dem Verst"andnis einer automatischen T"ur von ihren Instruktionen. Dieses "`Verst"andnis"' das die T"ur von den Signalen ihres Sensors hat ist  Searle zufolge gleichzusetzen mit dem eines Taschenrechners von der Mathematik, und dem eines Autos vom Fahren: N"amlich in dem Sinne, dass es sich bei all diesen \emph{gar nicht um Verst"andnis handelt}, sondern nur um das Ausf"uhren von maschinellen Instruktionen.\footnote{vgl. \cite[S.419]{searle1980minds}.}

Im Gegensatz dazu h"atten Menschen eine \emph{Intentionalit"at}. Dieser Begriff ist wesentlicher Bestandteil des kontempor"aren Diskurses der Philosophie des Geistes und ist zu verstehen als eine Art "`Gerichtetheit"' eines geistigen Wesens gegen"uber einem Bezugsobjekt.\footnote{vgl. \cite{sep-intentionality}.} Menschen seien also in der Lage, ihren Geist gezielt auf etwas zu richten, Maschinen nicht. Daraus scheint f"ur Searle zu folgen, dass es eben dieser "`Geist"' ist, der die Menschen von den Maschinen unterscheidet, die n"amlich nur \emph{Symbolmanipulation}, also Datenauswertung, betreiben. 

Doch woher genau dieser "`Geist"' kommt, beantwortet Searle nur insoweit als dass er sagt, dass aus einem ihm unbekannten Grund der menschliche Organismus eine bio-chemische Struktur aufweist, die unter bestimmten Umst"anden f"ahig ist zu Perzeption, Verst"andnis, Lernen, Intentionalit"at und dergleichen.\footnote{vgl. \cite[S.422]{searle1980minds}.} Daraus folgert er, dass es theoretisch denkbar w"are, einen k"unstlichen Organismus gleich einem Menschen zu kreieren. Dieser k"onne dann potenziell auch Intentionalit"at, und damit Verst"andnis sowie Denkverm"ogen aufweisen. Doch eine Maschine, die nur aus elektronischen und mechanischen Einzelteilen besteht, sei dazu niemals in der Lage. 

Dies unterstreicht er mit dem Argument, dass in dem chinesischen Zimmer weder der Mensch, noch die Zettel, also keines der Bestandteile des Zimmers Chinesisch versteht. So verhalte es sich auch mit allen anderen Maschinen mit denen man den Turing Test durchf"uhren w"urde. Und wenn keines der Bestandteile ein Verst"andnis davon hat, was es gerade tut, dann habe auch die gesamte Maschine kein Verst"andnis davon.

\vspace{5mm}
\noindent\textbf{$(iv)$ Die systemtheoretische Antwort}

\noindent Eine klassische Entgegnung, die Searle auch selbst in seinem Essay bespricht, ist die systemtheoretische Antwort.\footnote{Original: \emph{systems reply}, \cite[S.419]{searle1980minds} --  "Ubersetzung des Autors.} Ihr liegen Gedanken aus der Komplexit"atsforschung zugrunde, denen zufolge sich das Verhalten komplexer Systeme nicht sinnvoll durch Betrachtung der Einzelteile verstehen l"asst. Dies liege daran, dass dieses Verhalten \emph{emergent} sei, was hei"st, dass es aus der Komplexit"at heraus entstehe und damit das Verhalten der Subsysteme transzendiere. 

Ein klassisches Beispiel f"ur Emergenz ist der Vogelschwarm: Die einzelnen V"ogel haben ein relativ simples Verhalten, demzufolge sie ihre Flugrichtung und Tempo an den anliegenden V"ogeln orientieren. Dies allein betrachtet weist wenig Komplexit"at auf, doch wenn man den gesamten Schwarm betrachtet, lassen sich stets wandelnde, schwer fassbare Muster erkennen. In der Systemtheorie w"urde man nun sagen, dass ein \emph{dem Schwarm eigenes Verhalten} emergiert.

Entsprechend w"are die systemtheoretische Antwort zu Searles Gedanken, dass das chinesische Zimmer als Gesamtsystem betrachtet eben doch Chinesisch verst"unde, denn aus dem Zusammenspiel aus Zetteln und Mensch entsteht eine neue Entit"at, die "uber die Summe ihrer Einzelteile hinausgeht. Und diese habe, wie an ihrem Verhalten erkennbar ist, ein "`Verst"andnis"' von der chinesischen Sprache.

"Ubertragen auf den Turing Test und die Frage, ob Maschinen denken k"onnen, hie"se das, dass es m"oglich sein k"onnte, menschliches Denken aus einer komplexen Kombination von bestimmten einzelnen Subsystemen als emergentes Verhalten zu erzeugen. Gleichzeitig kann man diesen Ansatz auch als Erkl"arung unseres Bewusstseins betrachten: Die einzelnen Neuronen in unserem Gehirn haben wahrscheinlich f"ur sich betrachtet weder Intentionalit"at noch Verst"andnis, doch aus ihrer schieren Menge, der komplizierten Vernetzung und den darauf basierenden vielschichtigen Interaktionen k"onnte das menschliche Bewusstsein emergieren.

\vspace{5mm}
\noindent\textbf{$(v)$ Informationen und Prozesse}

\noindent Auff"allig ist an diesem Argument, dass es, wie schon jenes von Searle "uber das "`Unverst"andnis"' von Maschinen, eher eine \emph{intuitive Einsch"atzung} und keine klar empirisch oder logisch entwickelte Konklusion ist. Daher m"ochte ich nun von diesen beiden Gedanken Abstand nehmen und eine allgemeinere Betrachtung von Denkprozessen skizzieren.

Die \emph{Von-Neumann-Architektur} f"ur Computersysteme, das Referenzmodell f"ur den Bau heutiger Rechner, macht eine meiner Meinung nach einleuchtende Einteilung f"ur das Verarbeiten von Eingaben und das Erzeugen von Ausgaben: Computer sind grob in einen Speicher, in dem bekannte Informationen abgelegt und ausgelesen werden k"onnen, und ein Rechenwerk zu unterteilen.\footnote{vgl. \cite{von1993first}.} Diese Einteilung ging aus der Erkenntnis hervor, dass das Denken betrachtet werden kann als Zusammenspiel von \emph{Informationen} und \emph{Prozessen}. Wenn mich jemand nach meinem Namen fragt, dann startet in meinem Gehirn ein Prozess, der in meinem "`Erinnerungsspeicher"' nach der passenden Information sucht und ihn mir f"ur die richtige Antwort bereitstellt. Dies l"asst sich gut "ubertragen auf das chinesische Zimmer. Die Zettel mit den Schriftzeichen sind die Informationen, das Handeln der Person ist der Prozess. 

Doch die Form nat"urlichsprachlicher Grammatiken ist meist so, dass eine unendlich gro"se Menge an S"atzen und damit auch Fragen formuliert werden kann. Da der Nachschlageprozess nicht in der Lage ist, eigene Antworten zu entwickeln, m"usste die Menge der Zettel in dem Zimmer entsprechend auch unendlich gro"s sein.\footnote{Kontextsensitive Fragen wie "`Was war die vorherige Frage?"' w"aren in diesem Konzept sowieso nicht vorgesehen, was das von Searle unterstellte Bestehen des Turing-Tests schon fragw"urdig macht. Dieses Problem lie"se sich jedoch l"osen indem man alle m"oglichen Fragenketten in eigene Stapel unterteilt, wobei wir bei einer "uberabz"ahlbar unendlichen Menge an Zetteln w"aren.} "Ahnlich k"onnte einem Taschenrechner f"ur jede m"ogliche Eingabe das passende Ergebnis eingespeichert werden, doch dann w"aren schon f"ur Taschenrechner, die nur Addition und Multiplikation k"onnen sollen, extrem gro"se Speicher n"otig. Auf solche Weise abgespeichert w"urde man sie \emph{explizite} Informationen nennen.

Aus diesem Grund werden in digitalen Maschinen viele Informationen in die Prozesse verlagert, sodass aus der Verarbeitung der Informationen durch die Prozesse neue, sogenannte \emph{implizite} Informationen generiert werden k"onnen. So k"onnte zum Beispiel ein "`Additionsprozess"' existieren, der viele eigene Informationen zum verarbeiten von Daten hat und f"ur jegliche zwei eingegebene Zahlen ein richtiges Ergebnis ausgibt. An diesem Beispiel erkennt man, dass das Ersetzen von Informationen durch komplexere Prozesse keinen Informationsverlust zur Folge hat und trotzdem \emph{das Verhalten nach au"sen identisch ist}. Dies kann als eine Art der "`Improvisationsf"ahigkeit"' angesehen werden, durch die nicht nur weniger Informationen gespeichert werden m"ussen, sondern durch die auch eine Flexibilit"at gegen"uber unvorhersehbaren Eingaben entsteht. 

\vspace{5mm}
\noindent\textbf{$(vi)$ Wissen und Denken}

\noindent Ich unterstelle nun, dass dies auch mit dem menschlichen Denken vergleichbar ist, da man zum Beispiel nicht schon jeden m"oglichen Satz im Kopf gespeichert hat, sondern die Regeln der Grammatik sowie die Vokabeln der Sprache kennt. Diese "`Metainformationen"' k"onnen nun von dazugeh"origen Denkprozessen benutzt werden zum Verstehen von neuen, bisher ungeh"orten S"atzen.

Das chinesische Zimmer hat, wie wir wissen, alle m"oglichen Antworten in den Zetteln gespeichert. Wenn nun also der Person in dem chinesischen Zimmer bestimmte F"ahigkeiten beigebracht werden, durch die sie zum Beispiel bestimmte markierte Zettel kombinieren kann, wodurch f"ur die gleiche Menge an Antworten eine geringere Anzahl von diesen ben"otigt wird, h"atten wir schon bestimmte Informationen in den Prozess verlagert. Dadurch sind also nicht mehr alle Informationen \emph{explizit} vorhanden, sondern ein paar auch \emph{implizit}, n"amlich durch das Zusammenspiel von Prozessen und dazugeh"origen, "`abstrakten"' Informationen. Dies l"asst sich immer weiter f"uhren bis zur absoluten Optimierung der Zettelverwaltung, ohne dass die Person auch nur ein chinesisches Schriftzeichen "`versteht"'.\footnote{An Programmiersprachen wie \emph{Lisp} k"onnte man weiterf"uhrend zeigen, dass tats"achlich kein Unterschied zwischen expliziten Informationen (Daten) und impliziten Informationen (Prozesse, \emph{Prozeduren}) besteht.} Und nach au"sen hat sich das Verhalten des Zimmers auch nicht ge"andert. 

Wissen kann also eine spezielle Form des Denkens, und auch Denken als eine spezielle Form des Wissens angesehen werden. Es ist nach au"sen hin f"ur eine Maschine egal ob ihr Wissen gro"steilig explizit oder implizit ist, der einzige Unterschied l"age in der n"otigen Speicherkapazit"at und der Rechenleistung. Searle hat durch die Explizitheit der Informationen (Zettel) verschleiert, dass denkende Subjekte eben auf unterschiedliche Arten  Informationen "`abrufen"' und  "`speichern"' k"onnen. Das "`Verstehen"' liegt meiner Meinung nach aber genau in diesen komplexen Denkprozessen, die implizite Informationen entdecken, neu verkn"upfen und sonst wie verarbeiten k"onnen.

\vspace{5mm}
\noindent\textbf{$(vii)$ Konklusion}

\noindent Ich bezweifle, dass die Materialwahl f"ur Searle wirklich das Problem ist. Da er mit "`Verst"andnis"' etwas meint, das au"serhalb von Wissen und Denken liegt und weil er die Systemtheorie f"ur unsinnig h"alt, dann ist es nicht anders m"oglich, als dass er damit etwas meint, das \emph{"uber den K"orper hinausgeht}. Da "`Verst"andnis"' f"ur ihn nicht auf Denkebene anzutreffen sein kann, scheint dieses "`etwas"' nichts weiter zu sein als etwas die physische Materie transzendierendes, oder auch "`Seele"'.

Abschliessend l"asst sich also feststellen, dass man genau zwei Meinungen nur vertreten kann: Entweder versteht das chinesische Zimmer die chinesische Sprache, oder es wird es nie k"onnen, weil es keine "`Seele"' hat. 




%Tractatus 4.002

%Is mentality like milk or like a song/proof? minds I s95
 
 
%It is all very well to insist that you can conceive of Ralph Nurn existing with all his clever behavior but entirely lacking in consciousness (Searle makes such a claim in selection 22, "Minds, Brains, and grams.") Indeed you can always view a robot that way if you want. concentrate on images of little bits of internal hardware and re yourself that they are vehicles of information only by virtue of cleverly designed interrelationships between events in the sensed environment robotic actions, and the rest. But equally, you can view a human being way if you are really intent on it. Just concentrate on images of little of brain tissue-neurons and synapses and the like-and remind your that they are vehicles of information only by virtue of wonderfully signed interrelationships between sensed events in the environment bodily actions, and the rest. What you would leave out if you insisted viewing another person that way would be that person's point of view, we say.

\noindent 
%Lady Lovelace ( 1842). In it she states, "The Analytical Engine has no pretensions to originate anything. It can do whatever we know how to order it to perform" (her italics). 

%Problematisch: Turing schreibt "`It is natural that we should wish to permit every kind of engineering technique to be used in our machines. We also wish to allow the possibility than an engineer or team of engineers may construct a machine which works, but whose manner of operation cannot be satisfactorily described by its constructors because they have applied a method which is largely experimental. Finally, we wish to exclude from the machines men born in the usual manner. [...] we only permit digital computers to take part in our game"', der Mensch in dem Zimmer ist nach Turing schon gar nicht zul"assig.

%\begin{itemize}
%  \item seitenangabe bei paper in journal relativ oder exakt? journal
%  \item englisches zitat innerhlab deutschem satz?
%  \item Genitiv bei englischen begriffen: dann wird plural draus
%\end{itemize}

%\begin{itemize}
%  \item turing
%  \item zimmer
%  \item kernbegriffe: 
%  \item intentionalität
%  \item verständnis
%\end{itemize}

\vspace{3mm}

%\noindent\textbf{$(v)$ Gegen"uberstellung}



\end{onehalfspace}
%\nocite{*}
\bibliography{schreibwerkstatt-essay}

\end{document}
