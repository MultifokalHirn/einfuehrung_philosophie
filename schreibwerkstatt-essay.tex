\documentclass[a4paper, emulatestandardclasses, 12pt]{scrartcl}
\usepackage{graphicx}
\usepackage{fullpage}
%\usepackage{parskip}
\usepackage{color}
\usepackage[ngerman]{babel}
\usepackage{hyperref}
\usepackage{calc} 
\usepackage{enumitem}
\usepackage{titlesec}
%\pagestyle{headings}
\usepackage{setspace} %halbzeilig
\usepackage[authoryear,round]{natbib}
\bibliographystyle{natdin}

\date{\vspace{-3ex}}
\begin{document}

\title{\vspace{5ex}
	\includegraphics*[width=0.72\textwidth]{images/hu_logo.png}\\
	\vspace{30pt}
	\scshape\LARGE{Warum Searles chinesisches\\Zimmer Chinesisch kann}}
	
	\subtitle{\vspace{20pt}"Ubung Schreiben und Argumentieren\\
          %\vspace{6pt}
          %Essay\\
          }

\author{\vspace{-4pt}Lennard Wolf\\
        \small{\href{mailto:lennard.wolf@student.hu-berlin.de}{lennard.wolf@student.hu-berlin.de}}}      
%Abstract?

\maketitle

\vspace{\fill}

\begin{minipage}[b]{\textwidth}
    \centering
    \onehalfspacing
    \large   
    %30. November 2016\\
    Wintersemester 2016/2017

    \vspace{-20mm} 
\end{minipage}%
\thispagestyle{empty}
\newpage
\clearpage
\setcounter{page}{1}

\begin{onehalfspace} 

%\noindent\textbf{Fragestellung:}\\
%\indent Kann man mit Hilfe von Russells Theorie der Kennzeichnungen Freges R"atsel l"osen, ohne (wie Frege) die Existenz von Sinnen anzunehmen?\\\indent Wenn ja: Wie? Wenn nein: Warum nicht?
%
%\begin{center}
%\vspace{-9pt}\line(1,0){450}
%\end{center}


\noindent In dieser Arbeit m"ochte ich Gr"unde daf"ur anf"uhren, weshalb Searles Gedankenexperiment des chinesischen Raums die M"oglichkeit von maschinellem, beziehungsweise Software-getriebenem Bewusstsein nicht widerlegt.

Dazu werde ich wie folgt vorgehen. In $(i)$ umrei"se ich den diskursiven Kontext (?), in welchem das Gedankenexperiment entstanden ist und in $(ii)$ beschreibe ich es. %Es folgt in $(v)$ eine Gegen"uberstellung der beiden L"osungen, in der gezeigt wird, warum Russells Modell auf Freges Sinne verzichten kann.% In $(vi)$ schliesse ich den Kurzessay ab. 
\vspace{5mm}


\noindent\textbf{$(i)$ Der "`Turing Test"'}

\noindent In seinem einflussreichen Essay "`Computing machinery and intelligence"' \citep{turing1950computing} stellte Alan Turing das erste Mal seine Variation des \emph{Imitation Game} vor, heute besser bekannt als "`Turing Test"'. Dieser Test, der zugleich als Gedankenexperiment, wie auch als tats"achlich ausf"uhrbar anzusehen ist, geht der Frage nach, ob \emph{Maschinen denken k"onnen}. Es geht dabei um ein Verh"ohr, in dem eine Person zwei Individuen und Fragen stellt um herauszufinden, welche von beiden ein \emph{Mensch} und welche eine \emph{Maschine} ist. Zentral ist, dass die Kommunikation nur auf \emph{schriftlichem} Wege stattzufinden hat, denn Turing zufolge sei es nicht zielf"uhrend, das menschliche Denken auf "Au"serlichkeiten zu reduzieren.\footnote{Dies l"asst sich zum Beispiel an den folgenden Textstellen erkennen: "`[...] we should feel there was little point in trying to make a `thinking machine' more human by dressing it up in [...] artificial flesh. [...] We do not wish to penalise the machine for its inability to shine in beauty competitions, [...]"' \citep[S. 434]{turing1950computing}} Turing betont also die Trennung des Denkens von seinem physikalischen Ursprung, oder auch "`K"orper"'. Eine Maschine hat genau dann den Turing Test bestanden, wenn die verh"ohrende Person es nicht schafft herauszufinden, welche der befragten Individuen der "`echte Mensch"' ist. Solch eine Maschine w"are dann also in der Lage so zu denken, wie es Menschen tun. Mit der Existenz solch einer Maschine w"are die urspr"ungliche Frage, ob Maschinen denken k"onnen, beantwortet.

Es sollte angemerkt werden, dass Turing, nach der Darstellung des \emph{Imitation Game}, eine Reihe von m"oglichen Einw"anden gegen die Aussagekraft des Experiments vorstellt und jeweils anf"uhrt, warum sie seiner Meinung nach nicht gelten. %Ich m"ochte hier schon erw"ahnt wissen, dass der nun folgende Einwand von Searle mir nur wie eine Kombination aus den von Turing schon besprochenen Einw"anden ist.

\vspace{5mm}
\noindent\textbf{$(ii)$ Das chinesische Zimmer}

\noindent Das Gedankenexperiment des chinesischen Zimmers stellte John Searle in seinem Essay "`Minds, brains, and programs"' vor. In diesem wird ein Turing Test auf chinesisch vollzogen, und die "`Maschine"' besteht aus einem Zimmer, in dem sich ein Mensch ohne jedwede Kenntnisse der chinesischen Sprache befindet, sowie eine Menge Zetteln auf denen eine gro"se Menge an Paaren chinesischer Schriftzeichenketten stehen. Wenn die (chinesisch sprechende) verh"ohrende Person einen Zettel mit einer Schriftzeichenkette hineinreicht, sucht die Person in dem Zimmer nach dieser in seinem Zettelstapel, schreibt die damit gepaarte Schriftzeichenkette (die "`Antwort"' auf die "`Frage"') auf einen Zettel und reicht sie wieder nach drau"sen. 

Auf die fragende Person w"urde es nun so wirken, dass die Person in dem Zimmer chinesisch versteht, was ja aber bekannterweise nicht der Fall ist. Damit m"ochte Searle zeigen, dass eine Maschine, die den Turing Test besteht, die Verh"ohrfragen nicht zwangsl"aufig \emph{verstehen} muss, und genauso wenig die Ausgaben, die sie produziert. Vielmehr wurde nur die F"ahigkeit pr"asentiert, syntaktische Anweisungen zu befolgen, und kein semantisches Verst"andnis, was jedoch ein Kernmerkmal des menschlichen Denkens sei. Daher, so Searle, k"onne man aus dem Turing Test weder auf Intelligenz noch Bewusstsein schlie"sen. Einer Turing Test bestehenden Maschine zu unterstellen, sie bes"a"se ein Denkverm"ogen, das mit dem der Menschen vergleichbar ist, so folgert Searle, w"are also nicht g"ultig.\footnote{vgl. \cite[S.????]{searle1980minds}.}

\vspace{5mm}
\noindent\textbf{$(iii)$ "`Verst"andnis"'}

\noindent Da Searle davon spricht, dass eine Maschine kein "`Verst"andis"' (\emph{understanding}) von ihren "`Ein- und Ausgaben"' vergleichbar zu einem Menschen habe, ist es wichtig seine Deutung des Begriffs noch einmal genauer zu betrachten.

Searle f"uhrt als Beispiel an, dass es einen Unterschied gibt zwischen seinem Verst"andnis von der Englischen Sprache und dem Verst"andnis einer automatischen T"ur von ihren Instruktionen. Dieses "`Verst"andnis"' das die T"ur von den Signalen ihres Sensors hat ist nach Searle gleichzusetzen mit dem eines Taschenrechners von der Mathematik, und dem eines Autos vom Fahren: N"amlich in dem Sinne, dass es sich bei all diesen \emph{gar nicht um "`Verst"andnis"' handelt}, sondern nur um das Ausf"uhren von maschinellen Instruktionen.\footnote{vgl. \cite[S.????]{searle1980minds}.}

Im Gegensatz dazu h"atten Menschen eine \emph{Intentionalit"at}. Dieser Begriff ist wesentlicher Bestandteil des kontempor"aren Diskurses der Philosophie des Geistes und ist zu verstehen als eine Art "`Gerichtetheit"' eines geistigen Wesens gegen"uber einem Bezugsobjekt.\footnote{vgl. \cite{sep-intentionality}.} Menschen seien also in der Lage, ihren Geist gezielt auf etwas zu richten, Maschinen nicht. Daraus scheint f"ur Searle zu folgen, dass es eben dieser "`Geist"' ist, der die Menschen von den Maschinen unterscheidet, die n"amlich nur \emph{Symbolmanipulation}, also Datenauswertung, betreiben. 

Doch woher genau dieser "`Geist"' kommt, beantwortet Searle nur insoweit als dass er sagt, dass aus einem ihm unbekannten Grund der menschliche Organismus eine bio-chemische Struktur aufweist, die unter bestimmten Umst"anden f"ahig ist zu Perzeption, Verst"andnis, Lernen, Intentionalit"at und dergleichen.\footnote{vgl. \cite[S.????]{searle1980minds}.} Daraus folgert er, dass es theoretisch denkbar w"are, einen k"unstlichen Organismus gleich einem Menschen zu kreieren, dieser k"onne dann potenziell auch Intentionalit"at, und damit ein  aufweisen. Doch eine Maschine, die nur aus elektronischen und mechanischen Einzelteilen besteht, w"are dazu nicht in der Lage. 

Dies unterstreicht er mit dem Argument, dass in dem chinesischen Zimmer weder der Mensch, noch die Zettel, also keines der Bestandteile des Zimmers Chinesisch versteht. So verhalte es sich auch mit allen anderen Maschinen mit denen man den Turing Test durchf"uhren w"urde. Und wenn keines der Bestandteile ein Verst"andnis davon hat, was es gerade tut, dann habe auch die gesamte Maschine kein Verst"andnis davon.

\vspace{5mm}
\noindent\textbf{$(iv)$ Die systemtheoretische Antwort}

\noindent Eine klassische Entgegnung, die Searle auch selbst in seinem Paper bespricht, ist die systemtheoretische Antwort.\footnote{Original: \emph{systems reply} --  "Ubersetzung des Autors.} Ihr liegen Gedanken aus der Komplexit"atsforschung zugrunde, denen zufolge sich das Verhalten komplexer Systeme nicht sinnvoll durch Betrachtung der Einzelteile verstehen l"asst. Dies liege daran, dass dieses Verhalten \emph{emergent} sei, das hei"st aus der Komplexit"at heraus "`entstehend"', und damit das Verhalten der Subsysteme transzendiere. 

Ein klassisches Beispiel f"ur emergentes Verhalten ist der Vogelschwarm: Die einzelnen V"ogel haben ein relativ simples Verhalten, dem zufolge sie ihre Flugrichtung und Tempo an den anliegenden V"ogeln orientieren. Dies allein betrachtet weist wenig Komplexit"at auf, doch wenn man den gesamten Schwarm betrachtet entstehen sich stets wandelnde, schwer fassbare Muster. In der Systemtheorie w"urde man nun sagen, dass ein \emph{dem Schwarm eigenes Verhalten} emergiert.

Entsprechend w"are die systemtheoretische Antwort zu Searles Gedanken, dass das chinesische Zimmer als Gesamtsystem betrachtet eben doch Chinesisch verst"unde, denn aus dem Zusammenspiel aus Zetteln und Mensch entsteht eine neue Entit"at, die "uber die Summe ihrer Einzelteile hinausgeht. Und diese habe, wie an ihrem Verhalten erkennbar ist, ein "`Verst"andnis"' von der chinesischen Sprache.

"Ubertragen auf den Turing Test und die Frage, ob Maschinen denken k"onnen, hie"se das, dass es m"oglich sein k"onnte, menschliches Denken aus einer komplexen Kombination von bestimmten einzelnen Subsystemen als emergentes Verhalten zu erzeugen. Gleichzeitig kann man diesen Ansatz auch als Erkl"arung unseres Bewusstseins betrachten: Die einzelnen Neuronen in unserem Gehirn haben wahrscheinlich f"ur sich betrachtet weder Intentionalit"at noch Verst"andnis, doch aus ihrer schieren Menge, der komplizierten Vernetzung und den darauf basierenden vielschichtigen Interaktionen k"onnte das menschliche Bewusstsein emergieren.

\vspace{5mm}
\noindent\textbf{$(iv)$ Konklusion}

%Tractatus 4.002

%Is mentality like milk or like a song/proof? minds I s95
 
 
%It is all very well to insist that you can conceive of Ralph Nurn existing with all his clever behavior but entirely lacking in consciousness (Searle makes such a claim in selection 22, "Minds, Brains, and grams.") Indeed you can always view a robot that way if you want. concentrate on images of little bits of internal hardware and re yourself that they are vehicles of information only by virtue of cleverly designed interrelationships between events in the sensed environment robotic actions, and the rest. But equally, you can view a human being way if you are really intent on it. Just concentrate on images of little of brain tissue-neurons and synapses and the like-and remind your that they are vehicles of information only by virtue of wonderfully signed interrelationships between sensed events in the environment bodily actions, and the rest. What you would leave out if you insisted viewing another person that way would be that person's point of view, we say.

\noindent 
%Lady Lovelace ( 1842). In it she states, "The Analytical Engine has no pretensions to originate anything. It can do whatever we know how to order it to perform" (her italics). 

%Problematisch: Turing schreibt "`It is natural that we should wish to permit every kind of engineering technique to be used in our machines. We also wish to allow the possibility than an engineer or team of engineers may construct a machine which works, but whose manner of operation cannot be satisfactorily described by its constructors because they have applied a method which is largely experimental. Finally, we wish to exclude from the machines men born in the usual manner. [...] we only permit digital computers to take part in our game"', der Mensch in dem Zimmer ist nach Turing schon gar nicht zul"assig.

%\begin{itemize}
%  \item seitenangabe bei paper in journal relativ oder exakt? journal
%  \item englisches zitat innerhlab deutschem satz?
%  \item Genitiv bei englischen begriffen: dann wird plural draus
%\end{itemize}

%\begin{itemize}
%  \item turing
%  \item zimmer
%  \item kernbegriffe: 
%  \item intentionalität
%  \item verständnis
%\end{itemize}

\vspace{3mm}

%\noindent\textbf{$(v)$ Gegen"uberstellung}



\end{onehalfspace}
%\nocite{*}
\bibliography{schreibwerkstatt-essay}

\end{document}
