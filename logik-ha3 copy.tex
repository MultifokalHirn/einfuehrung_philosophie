\documentclass[a4paper]{article}

\usepackage{rotating}
\usepackage{qtree}
%\usepackage{KMcalc} %Lennard



%KM-Kalkül, geordnetes paar und gather-Umgebung für Formeln:
\newcommand{\UB}{$\und$B}
\newcommand{\UE}{$\und$E}
\newcommand{\OrE}{$\oder$E}
\newcommand{\OrB}{$\oder$B}
\newcommand{\EB}{$\ex$B}
\newcommand{\EE}{$\ex$E}
\newcommand{\paar}[1]{\left \langle #1 \right \rangle}
\newcommand{\gth}[2]{\begin{gather*} #1 \label{#2} \end{gather*}}
\newcommand{\gthd}[2]{\begin{gather} \begin{gathered}#1 \label{#2}\end{gathered}\end{gather}}

%objektsprachliche Symbole:
\newcommand{\und}{\wedge}
\newcommand{\oder}{\vee}
\newcommand{\then}{\rightarrow}
\newcommand{\eq}{\leftrightarrow}
\newcommand{\uu}{\cup}
\newcommand{\C}{\cap}
\newcommand{\TM}{\subseteq}
\newcommand{\all}{\forall}
\newcommand{\ex}{\exists}
% \neg gibt es schon

%Zahlbereichsymbole, Potenzmenge und Sprachnamen
\newcommand{\NN}{\mathbb{N}}
\newcommand{\QQ}{\mathbb{Q}}
\newcommand{\RR}{\mathbb{R}}
\newcommand{\Pp}{\mathcal{P}}
\newcommand{\Ll}{\ensuremath{\mathcal{L}}}
\newcommand{\pair}[1]{\langle #1 \rangle}
\newcommand{\quine}[1]{\ulcorner #1 \urcorner}
\newcommand{\mq}[1]{\mlq #1 \mrq} % steht für "math quotes"
\newcommand{\sse}{\subseteq}
\newcommand{\gesch}{\cap} % steht für "geschnitten"
\newcommand{\verin}{\cup} % Habe hier mit Absicht einen Buchstaben weggelassen, ähnlich wie \infty
\newcommand{\set}[1]{\{#1\}}
\newcommand{\LPL}{\ensuremath{\mathcal{L}_{PL}}}
\newcommand{\LAL}{{\ensuremath{\mathcal{L}_{AL}}}} % zusätzliche „{}“, um es in Subskripts nutzen zu können
\newcommand{\FmLAL}{\ensuremath{\mathcal{F}m_\LAL}} % zur besseren Lesbarkeit der AL-Definitionen
\newcommand{\SKLAL}{\ensuremath{\mathcal{SK}_\LAL}} % zur besseren Lesbarkeit der AL-Definitionen
\newcommand{\MIb}{{\pair{M, I}, \beta}} % zur Benutzung im Math-Mode


%metasprachliche Symbole:
\newcommand{\Land}{
    \raisebox{-0.12em}{ \begin{turn}{90} $\eqslantgtr$ \end{turn} }
}
\newcommand{\Lor}{
     \raisebox{-0.12em}{ \begin{turn}{90} $\eqslantless$ \end{turn} }
}
\newcommand{\Then}{
    \Rightarrow
}
\newcommand{\Gdw}{
    \Leftrightarrow
}
\newcommand{\Neg}{
    \neg\hspace*{-0.5em}\neg
}
\newcommand{\Forall}{
    \raisebox{0.18em}{\scriptsize{\textbackslash}}\hspace*{-0.175em}\forall
}
\newcommand{\Exists}{
    \exists\hspace*{-0.45em}\exists
}

\newcommand{\nehT}{\Leftarrow}

\DeclareMathSymbol{\mlq}{\mathord}{operators}{``}
\DeclareMathSymbol{\mrq}{\mathord}{operators}{`'}
\newcommand{\concat}{\raisebox{0.45em}{\scalebox{0.7}{$\smallfrown$}}}

%models gibt es schon
%\

\newcommand{\deduces}{\vdash}
\newcommand{\sidew}[1]{\begin{sideways} #1 \end{sideways}}
%Anführungszeichen

\newcommand{\qleft}{\ulcorner}
\newcommand{\qright}{\urcorner}
\newcommand{\qc}[1]{\qleft #1 \qright}
\newcommand{\anf}[1]{`#1'}
\newcommand{\manf}[1]{\text{`}#1\text{}}
\newcommand{\tmanf}[1]{\text{`#1'}}
%\renewcommand{\models}{\vDash}
\newcommand{\Anf}[1]{„#1“}
\newcommand{\cn}{\/^{\smallfrown}}
%Kopf


\newcommand{\Ex}{\Exists}
\newcommand{\All}{\Forall}
\newcommand{\Und}{\Land}
\newcommand{\Oder}{\Lor}
\newcommand{\Eq}{\Gdw}
\usepackage{stmaryrd}
\usepackage{graphicx}
\usepackage{fullpage}
%\usepackage{parskip}
\usepackage{color}
\usepackage[ngerman]{babel}
\usepackage{hyperref}
\usepackage{calc} 
\usepackage{enumitem}
\usepackage{titlesec}
\usepackage{bussproofs}
\usepackage[export]{adjustbox}
%\pagestyle{headings}
\usepackage{amssymb} % fuer logik

\titleformat{name=\section,numberless}
  {\normalfont\Large\bfseries}
  {}
  {0pt}
  {}
\date{\vspace{-3ex}}
\begin{document}

\title{
    \vspace{-30pt}
	\includegraphics*[width=0.1\textwidth,right]{ErstesSem/images/hu_logo2.png}\\
	\vspace{-10pt}
	Einf"uhrung in die Logik -- Aufgabenblatt 4}
\author{Lennard Wolf\\
        \small{\href{mailto:lennard.wolf@student.hu-berlin.de}{lennard.wolf@student.hu-berlin.de}}}
\maketitle
\vspace{-4pt}

\section*{Aufgabe 1}
\large

\textbf{a) }

\begin{description}[leftmargin=!,labelwidth=\widthof{\bfseries (4))}]
  \item[(1)] Jede Zahl ist mit sich selbst identisch.
  \item[(2)] F"ur jedes Objekt gilt: Wenn $\underline{\textrm{es}}$ eine Zahl ist, dann ist $\underline{\textrm{es}}$ mit sich selbst identisch.
  \item[(3)] F"ur jedes Objekt $x$ gilt: Wenn $x$ eine Zahl ist, dann ist $x$ identisch mit $x$.
  \item[(4)] $\Forall x \ \big(\varphi(x) \Then x = x \big)$
\end{description}
\vspace{10pt}

\noindent \textbf{b) }

\begin{description}[leftmargin=!,labelwidth=\widthof{\bfseries (4))}]
  \item[(1)] Alle geraden und alle nat"urlichen Zahlen sind nat"urliche Zahlen.
  \item[(2)] F"ur jedes Objekt gilt: Wenn $\underline{\textrm{es}}$ eine gerade Zahl ist, dann ist $\underline{\textrm{es}}$ eine nat"urliche Zahl und wenn $\underline{\textrm{es}}$ eine nat"urliche Zahl ist, dann ist $\underline{\textrm{es}}$ eine nat"urliche Zahl. 
  \item[(3)] F"ur jedes Objekt $x$ gilt: Wenn $x$ eine gerade Zahl ist, dann ist $x$ eine nat"urliche Zahl und wenn $x$ eine nat"urliche Zahl ist, dann ist $x$ eine nat"urliche Zahl.
  \item[(4)] $\Forall x \ \big((\varphi(x) \Then \psi(x)) \Land (\psi(x) \Then \psi(x)) \big)$
\end{description}
\vspace{10pt}

\noindent \textbf{c) }

\begin{description}[leftmargin=!,labelwidth=\widthof{\bfseries (4))}]
  \item[(1)] Es gibt Zahlen\footnote{Das steht hier im Plural, aber ich gehe davon aus, dass  das einfach im Sinne des Existenzquantors gemeint ist. Alternativ w"are folgendes m"oglich: $\Forall x \ ( \Neg ( x$ ist eine Zahl $ \Then \Neg (x$ ist ungerade$)))$, jedoch w"are dies nicht \emph{direkt} der Form des Satzes zu entnehmen.}, die nicht gerade sind.
  \item[(2)] Es gibt ein Objekt, f"ur das gilt: $\underline{\textrm{Es}}$ ist eine Zahl und $\underline{\textrm{es}}$ ist nicht gerade.
  \item[(3)] Es gibt ein Objekt $x$, f"ur das gilt: $x$ ist eine Zahl und $x$ ist nicht gerade.
  \item[(4)] $\Ex x \ \big(\varphi(x) \Land  \psi(x) \big)$
\end{description}
\vspace{10pt}

\section*{Aufgabe 2}

\textbf{a) }

\begin{description}[leftmargin=!,labelwidth=\widthof{\bfseries (4))}]
  \item[(1)] $\Ex x \ \big( \varphi(x) \Lor \psi(x)\big)$
  \item[(2)] $\Ex x \ x$ ist ein lustiger Film oder $x$ ist ein trauriger Film.
  \item[(3)] Es gibt einen Film der lustig oder traurig ist.
\end{description}
\vspace{10pt}

\noindent \textbf{b) }

\begin{description}[leftmargin=!,labelwidth=\widthof{\bfseries (4))}]
  \item[(1)] $\Forall x \ \big(\varphi(x) \Land  \psi(x) \Then \psi(x)\big)$
  \item[(2)] $\Forall x$ wenn $x$ ein Mensch und ein Tier ist, dann ist $x$ ein Tier.
  \item[(3)] Alle, die Menschen und Tiere sind, sind Tiere.
\end{description}
\vspace{10pt}

\noindent \textbf{c) }

\begin{description}[leftmargin=!,labelwidth=\widthof{\bfseries (4))}]
  \item[(1)] $\Forall x \ \big(\varphi(x) \Then \Ex y \ \big( \varphi(y) \Land \Neg \ y = x\big)\big)$
  \item[(2)] $\Forall x$ wenn $x$ ein Mensch ist, dann $\Ex y \ y$ ist ein Mensch, und $x$ und $y$ sind verschieden.
  \item[(3)] F"ur jeden Menschen gibt es einen anderen Menschen.
\end{description}
\vspace{10pt}


\section*{Aufgabe 3}


\textbf{a) }
\vspace{4pt}

Zu zeigen:

\vspace{2pt}
($\circ$) \hspace*{1em} $\Ex x \ (x \textrm{ ist rot} \Land x\textrm{ ist blau} ) \Then (\Ex x \ x \textrm{ ist rot}  \Land \Ex x \ x \textrm{ ist blau} )$

\vspace{2pt}
Angenommen: 

\vspace{2pt}
(1) \hspace*{1em} $\Ex x \ (x \textrm{ ist rot} \Land x\textrm{ ist blau} )$

\vspace{2pt}
Zu zeigen:

\vspace{2pt}
($\dagger$) \hspace*{1em} $\Ex x \ x \textrm{ ist rot}  \Land \Ex x \ x \textrm{ ist blau}$

\vspace{2pt}
Und-Beseitigung aus (1): 

\vspace{2pt}
(2) \hspace*{1em}  $\Ex x \ x \textrm{ ist rot}$
\vspace{2pt}

Und-Beseitigung aus (1):

\vspace{2pt}
(3) \hspace*{1em}  $\Ex x \ x \textrm{ ist blau}$

\vspace{2pt}
Und-Einf"uhrung aus (2) und (3):

\vspace{2pt}
(4) \hspace*{1em} $\Ex x \ x \textrm{ ist rot} \Land \Ex x \ x \textrm{ ist blau}$

\vspace{10pt}
(4) $\Leftrightarrow$ ($\dagger$). 

Somit ist gezeigt, dass ($\dagger$) aus (1) folgt, womit ($\circ$) bewiesen ist.

\newpage


\noindent \textbf{b) }
\vspace{4pt}

Zu zeigen:

\vspace{2pt}
($\circ$) \hspace*{1em} $\Forall x \ \big(\varphi(x) \Then \psi(x)\big) \Then \big(\Ex x \ \varphi(x) \Then \Ex x \ \psi(x)\big)$

\vspace{2pt}
Angenommen:

\vspace{2pt}
(1) \hspace*{1em}  $\Forall x \ \big(\varphi(x) \Then \psi(x)\big)$

\vspace{2pt}
Zu zeigen: 

\vspace{2pt}
($\dagger$) \hspace*{1em}  $\Ex x \ \varphi(x) \Then \Ex x \ \psi(x)$

\vspace{2pt}
Angenommen:

\vspace{2pt}
(2) \hspace*{1em}  $\Ex x \ \varphi(x)$

\vspace{2pt}
Zu zeigen:

\vspace{2pt}
($\ddagger$) \hspace*{1em} $\Ex x \ \psi(x)$

\vspace{2pt}
Sei $t$ beliebig. Zu zeigen:

\vspace{2pt}
($\ast$) \hspace*{1em}  $\psi(t)$

\vspace{2pt}
Aus (1) durch Spezialisierung auf $t$:

\vspace{2pt}
(3) \hspace*{1em} $\varphi(t) \Then \psi(t)$

\vspace{2pt}
Aus (2) durch Existenz-Beseitigung auf $t$:

\vspace{2pt}
(4) \hspace*{1em} $\varphi(t)$

\vspace{2pt}
Aus (3) und (4) per \emph{modus ponens}:

\vspace{2pt}
(4) \hspace*{1em} $\psi(t)$


\vspace{10pt}
(4) $\Leftrightarrow$ ($\ast$). 

Da ($\ast$) gezeigt ist, ist auch ($\ddagger$) gezeigt, wodurch ($\dagger$) und damit ($\circ$) gezeigt sind.

\vspace{14pt}


\noindent \textbf{c) }
\vspace{4pt}

Zu zeigen:

\vspace{2pt}
($\circ$) \hspace*{1em} $\Forall x \ \big( \Neg \ x = x \Then x \textrm{ ist ein Drache}\big)$

\vspace{2pt}
Angenommen f"ur die \emph{reductio} von ($\circ$):

\vspace{2pt}
(1) \hspace*{1em} $\Ex x \ \Neg\big(\Neg \ x = x \Then x \textrm{ ist ein Drache}\big)$

\vspace{2pt}
Sei $t$ beliebig. Existenz-Beseitigung auf $t$: 

\vspace{2pt}
(2) \hspace*{1em}  $\Neg\big(\Neg \ t = t \Then t \textrm{ ist ein Drache}\big)$
\vspace{2pt}

Umformung von (2) basierend auf $(A \Then B) \Leftrightarrow (A \Land \Neg B)$:

\vspace{2pt}
(3) \hspace*{1em}  $\Neg \ t = t \Land \Neg~t \textrm{ ist ein Drache}$

\vspace{2pt}
Und-Beseitigung aus (3):

\vspace{2pt}
(4) \hspace*{1em} $\Neg \ t = t$

\vspace{10pt}
(4) steht im Widerspruch mit dem Gesetz der Selbstidentit"at. Daher ist (1) logisch falsch und da es die Kontradiktion von ($\circ$) ist, ist ($\circ$) damit bewiesen.

\newpage



\noindent \textbf{d) }
\vspace{4pt}

Zu zeigen:

\vspace{2pt}
($\circ$) \hspace*{1em} $\Forall x \ (a=x \Then Px) \Then Pa$

\vspace{2pt}
Angenommen:

\vspace{2pt}
(1) \hspace*{1em} $\Forall x \ (a=x \Then Px)$

\vspace{2pt}
Zu zeigen:  

\vspace{2pt}
($\dagger$) \hspace*{1em}  $Pa$

\vspace{2pt}
Sei $t$ beliebig. Angenommen:

\vspace{2pt}
(2) \hspace*{1em} $a=t$

\vspace{2pt}
Aus (1) und (2) per \emph{modus ponens}: 

\vspace{2pt}
(3) \hspace*{1em}  $Pt$
\vspace{2pt}

Aus (3) mit der Regel der Substitutivit"at: 

\vspace{2pt}
(4) \hspace*{1em}  $Pa$


\vspace{10pt}
(4) $\Leftrightarrow$ ($\dagger$). 

Damit ist ($\circ$) gezeigt.




%
%\noindent \textbf{d) }
%\vspace{4pt}
%
%Zu zeigen:
%
%\vspace{2pt}
%($\circ$) \hspace*{1em} $\Forall x \ (a=x \Then Px) \Then Pa$
%
%\vspace{2pt}
%Angenommen f"ur die \emph{reductio} von ($\circ$):
%
%\vspace{2pt}
%(1) \hspace*{1em} $\Ex x \ \Neg \big( (a=x \Then Px) \Then Pa\big)$
%
%\vspace{2pt}
%Sei $t$ beliebig. Existenz-Beseitigung auf $t$. Zu zeigen:  
%
%\vspace{2pt}
%($\ast$) \hspace*{1em}  $\Neg \big( (a=t \Then Pt) \Then Pa\big)$
%
%\vspace{2pt}
%Angenommen:
%
%\vspace{2pt}
%(1) \hspace*{1em} $\Forall x \ (a=x \Then Px)$
%
%\vspace{2pt}
%Zu zeigen: 
%
%\vspace{2pt}
%($\dagger$) \hspace*{1em}  $Pa$
%\vspace{2pt}
%

\end{document}
