\documentclass[a4paper]{article}
\usepackage{graphicx}
\usepackage{fullpage}
%\usepackage{parskip}
\usepackage{color}
\usepackage[ngerman]{babel}
\usepackage{hyperref}
\usepackage{calc} 
\usepackage{enumitem}
\usepackage{titlesec}
\usepackage[export]{adjustbox}
%\pagestyle{headings}

\titleformat{name=\section,numberless}
  {\normalfont\Large\bfseries}
  {}
  {0pt}
  {}
\date{\vspace{-3ex}}
\begin{document}

\title{
    \vspace{-30pt}
	\includegraphics*[width=0.1\textwidth,left]{images/hu_logo2.png}\\
	\vspace{-10pt}
	Einf"uhrung in die Philosophie}
\author{Lennard Wolf\\
        \small{\href{mailto:lennard.wolf@student.hu-berlin.de}{lennard.wolf@student.hu-berlin.de}}}
\maketitle
\vspace{0pt}

\section*{AB 5: Fragen stellen}
\large

\vspace{10pt}
\noindent\textbf{(i)}\\
"`\emph{Was sind Ihre Gedanken zu einer vollkommen skeptizistischen Haltung gegen"uber allem Wissen?}"'


\vspace{8pt}
\noindent\textbf{(2)}\\
Ich pers"onlich w"urde f"ur diese Frage in zwei Richtungen denken: Zum einen stellt sich n"amlich die Frage, was genau die \emph{Gr"unde} daf"ur sein k"onnen, eine solche skeptizistische Haltung einzunehmen, und zum anderen w"urde ich den Gedanken verfolgen, was die \emph{Konsequenzen} aus so einer Haltung w"aren.

Der erste Abschnitt liesse sich zum einen abhandeln "uber Gedanken zu der Eingeschr"anktheit unserer Sinne und unseres Ged"achtnisses, im Kontrast zu (anscheinend) verl"asslichen Aufzeichnungen von elektronischen Maschinen. Wichtig w"are f"ur mich zu er"ortern, dass "`echtes Wissen"' m"oglicherweise nur aus von uns selbst aufgestellten Axiomen entstehen k"onnte, also \emph{a priori} sein \emph{m"usste}. Wissen \emph{a posteriori} liesse sich dann vielleicht allgemein besser als "`Erfahrung"' (lies: unzuverl"assig) bezeichnen, besonders weil die Menge des m"oglichen Erfahrungsschatzes vermutlich stets nicht das gesamte All erfassen kann. 

Doch bisher konnte die Menschheit mit dieser heuristischen "`Erfahrung"' in den Wissenschaften und im technischen wie sozialen Fortschritt anscheinend sehr gute Ergebnisse erzielen. Daraus liesse sich sicher argumentieren, dass es relativ schwer sein k"onnte, das Weltbild einer Skeptizistin zu widerlegen, jedoch die daraus entstehende Konsequenz eines absoluten Unwissens eine schlechtere weil n"amlich unbrauchbarere Haltung f"ur das allt"agliche Leben wie auch f"ur die Wissenschaften sein k"onnte.

\end{document}
