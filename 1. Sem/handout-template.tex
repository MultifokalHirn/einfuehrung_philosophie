% Vorlage fuer Handouts
% zum Seminar "Diskrete Geometrie und Kombinatorik -- ein topologischer Zugang"
% im WS 2008/09
%
%%%%%%%%%%%%%%%%%%%%%%%%%%%%%%%%%%%%%%%%%%%%%%%%%%%%%%%%%%%%%%%%%%%%%%%%%%%%
% Allgemeine Hinweise
% - Halten Sie den LaTeX-Code so uebersichtlich wie moeglich;
%   (La)TeX-Fehlermeldungen sind oft kryptisch -- in einem ordentlich 
%   strukturierten Quellcode lassen sich Fehler leichter finden und 
%   beseitigen
%
%
%%%%%%%%%%%%%%%%%%%%%%%%%%%%%%%%%%%%%%%%%%%%%%%%%%%%%%%%%%%%%%%%%%%%%%%%%%%%
% Jedes LaTeX-Dokument muss eine \documentclass-Deklaration enthalten
%   Diese sorgt fuer das allgemeine Seiten-Layout, das Aussehen der 
%   Ueberschriften etc.
\documentclass[a4paper,oneside,DIV8,10pt]{scrartcl}
  
  %%%%%%%%%%%%%%%%%%%%%%%%%%%%%%%%%%%%%%%%%%%%%%%%%%%%%%%%%%%%%%%%%%%%%%%%%%
  % Einbinden weiterer Pakete
  \usepackage{german}    % fuer die deutschen Trennmuster
  % \usepackage{ngerman} % entsprechend fuer die neue Rechtschreibung
  \usepackage[latin1]{inputenc} % falls Sie Umlaute in den Quellen verwenden wollen
  \usepackage{amsmath}   % enthaelt nuetzliche Makros fuer Mathematik
  \usepackage{amsthm}    % fuer Saetze, Definitionen, Beweise, etc.
  \usepackage{relsize}   % fuer \smaller 

  %%%%%%%%%%%%%%%%%%%%%%%%%%%%%%%%%%%%%%%%%%%%%%%%%%%%%%%%%%%%%%%%%%%%%%%%%%
  % Deklaration eigener Mathematik-Makros
  \newcommand{\N}{\ensuremath{\mathbf{N}}}   % natuerliche Zahlen
  \newcommand{\Z}{\ensuremath{\mathbf{Z}}}   % ganze Zahlen
  \newcommand{\Q}{\ensuremath{\mathbf{Q}}}   % rationale Zahlen
  \newcommand{\R}{\ensuremath{\mathbf{R}}}   % reelle Zahlen

  %%%%%%%%%%%%%%%%%%%%%%%%%%%%%%%%%%%%%%%%%%%%%%%%%%%%%%%%%%%%%%%%%%%%%%%%%%
  % Deklaration eigener Satz-/Definitions-/Beweisumgebungen mit amsthm
  \newtheorem{satz}{Satz}[section]
  \newtheorem{lemma}[satz]{Lemma}
  \newtheorem{korollar}[satz]{Korollar}
  \theoremstyle{definition}
  \newtheorem{definition}[satz]{Definition}
  \newtheorem{bemerkung}[satz]{Bemerkung}
  \newtheorem{aufgabe}[satz]{Aufgabe}
  \newenvironment{beweis}%
    {\begin{proof}[Beweis]}
    {\end{proof}}
  \newtheorem{beispiel}[satz]{Beispiel}

  %%%%%%%%%%%%%%%%%%%%%%%%%%%%%%%%%%%%%%%%%%%%%%%%%%%%%%%%%%%%%%%%%%%%%%%%%%
  % Deklaration weiterer Makros
  \renewcommand{\labelitemi}{--}             % aendert die Symbole bei unnumerierten Aufzaehlungen
  \makeatletter                              % Fussnote ohne Symbol
    \def\blfootnote{\xdef\@thefnmark{}\@footnotetext}
  % Titel des Handouts
  %   #1 Name des Vortragenden
  %   #2 email-Adresse 
  %   #3 Datum des Vortrags
  %   #4 Titel des Vortrags
  \newcommand{\handouttitle}[4]
   {\begin{center}
      \Large #4
    \end{center}

    \bigskip

    \noindent
    #1 (\textsf{#2})
    \hfill
    #3%
    \blfootnote{Seminar \glqq Diskrete Geometrie und Kombinatorik\grqq, 
      WS~2008/2009, {\smaller WWU}~M\"unster}
  
    \noindent
    \rule{\linewidth}{.5pt}

    \bigskip

    \@afterindentfalse\@afterheading
   }
  \makeatother
  \renewcommand{\sectfont}{\normalfont}       % aendert den Font fuer Ueberschriften

%%%%%%%%%%%%%%%%%%%%%%%%%%%%%%%%%%%%%%%%%%%%%%%%%%%%%%%%%%%%%%%%%%%%%%%%%%%%
% Anfang des eigentlichen Dokuments
\begin{document}

  % Titel fuer das Handout -- Sie koennen natuerlich auch selbst etwas entwerfen!
  \handouttitle{V.~Ortragender}
               {mail@turbospam.org}
               {30.~Februar~2009}
               {Gruppenoperationen}

  Gruppenoperationen werden in den meisten Krankenh\"ausern mittlerweile
  nicht mehr empfohlen. Satz~\ref{hauptsatz} zeigt jedoch, da\ss\
  es immer noch zahlreiche Gruppenoperationen gibt.
  
  \section{Der Hauptsatz \"uber Gruppenoperationen}

  \begin{satz}[Hauptsatz \"uber Gruppenoperationen]\label{hauptsatz}
    Zu jeder Menge~$X$ und jeder Gruppe~$G$ gibt es eine
    Grupenoperation von~$G$ auf~$X$.
  \end{satz}
  \begin{beweis}
    Sei $X$ eine Menge und $G$ eine Gruppe. Dann ist
    \begin{align*}
      G \times X & \longrightarrow X \\
      (g,x)      & \longmapsto     x
    \end{align*}
    eine Operation von~$G$ auf~$X$.
  \end{beweis}

  Auf dieselbe Art und Weise lassen sich nat\"urlich auch Lemmata und
  Korollare etc.\ mit \LaTeX\ darstellen.

  Bei Fragen zu \LaTeX\ ist der \emph{\LaTeX\ Companion}~\cite{companion} 
  eine gro\ss e Hilfe; Sie k\"onnen sich aber auch gerne an Clara
  L\"oh (\textsf{clara.loeh@uni-muenster.de}) wenden.

  \section{Beispiele}

  \begin{beispiel}
    \hfil
    \begin{itemize}
      \item Hier ein Beispiel 
      \item \dots und noch eins
      \item \dots und noch eins
    \end{itemize}
  \end{beispiel}

  \begin{aufgabe}
    Vergessen Sie nicht, ein paar Aufgaben einzustreuen, an denen die
    Teilnehmer nochmal ihre Kenntnisse \"uberpr\"ufen k\"onnen.
  \end{aufgabe}

  \begin{beispiel}
    \hfil
    \begin{enumerate}
      \item Es gibt auch Beispiele, \dots
      \item \dots die numeriert sind.
    \end{enumerate}
  \end{beispiel}



%%%%%%%%%%%%%%%%%%%%%%%%%%%%%%%%%%%%%%%%%%%%%%%%%%%%%%%%%%%%%%%%%%%%%%%%%%%%
% Literaturverzeichnis
% - Der Einfachheit halber sind hier bereits alle Quellen eingetragen, 
%   die im Seminarprogramm auftreten
% - Bitte entfernen Sie alle Quellen, die Sie nicht in Ihrem Handout 
%   zitieren
% - Umgekehrt muessen Sie natuerlich, wenn Sie weitere Literatur
%   zitieren wollen, die entsprechenden Quellen hier einfuegen; 
%   hierbei kann www.ams.org/mathscinet helfen, die noetigen 
%   Informationen zu den Quellen zu sammeln
  \begin{thebibliography}{99}
    \bibitem{book} M.~Aigner, G.M.~Ziegler, \emph{Proofs from The
        Book}, dritte Auflage, Springer, 2004.

    \bibitem{armstrong} M.A.~Armstrong, \emph{Basic Topology},
      korrigierter Nachdruck des Originals von~1979, 
      Undergraduate Texts in Mathematics, Springer, 1983.

    \bibitem{diestel} R.~Diestel. \emph{Graph theory}, dritte Auflage, 
      Graduate Texts in Mathematics, Band~173, Springer, 2005.

    \bibitem{hatcher} A.~Hatcher. \emph{Algebraic Topology}, 
      Cambridge University Press, 2002. 
      Online verf\"ugbar unter
      \textsf{http://www.math.cornell.edu/\~{}hatcher/}.

    \bibitem{jaenich} K.~J\"anich. \emph{Topologie}, achte Auflage,
      Springer, 2005.

    \bibitem{borsuk} J.~Matou\v sek. \emph{Using the Borsuk-Ulam
        Theorem}, Lectures on topological methods in combinatorics and
      geometry. Written in cooperation with Anders Bj\"orner and G\"unter
      M.~Ziegler, \emph{Universitext}, Springer, 2003.

    \bibitem{companion} F.~Mittelbach, M.~Goossens, J.~Braams,
      D.~Carlisle, C.~Rowley. \emph{The \LaTeX\ Companion}, zweite
      Auflage, Addison-Wesley, 2004.

    \bibitem{rudin} W.~Rudin. \emph{Real and complex analysis}, dritte
      Auflage, McGraw-Hill Book Co., 1987.

  \end{thebibliography}

%%%%%%%%%%%%%%%%%%%%%%%%%%%%%%%%%%%%%%%%%%%%%%%%%%%%%%%%%%%%%%%%%%%%%%%%%%%%
% Ende des Dokuments -- alles, was nach dieser Zeile steht, wird 
% von LaTeX ignoriert!
\end{document}