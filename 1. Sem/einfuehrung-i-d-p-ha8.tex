\documentclass[a4paper]{article}
\usepackage{graphicx}
\usepackage{fullpage}
%\usepackage{parskip}
\usepackage{color}
\usepackage[ngerman]{babel}
\usepackage{hyperref}
\usepackage{calc} 
\usepackage{enumitem}
\usepackage{titlesec}
%\usepackage{natbib}
\usepackage[export]{adjustbox}
\bibliographystyle{alphadin}
%\pagestyle{headings}

\titleformat{name=\section,numberless}
  {\normalfont\Large\bfseries}
  {}
  {0pt}
  {}
\date{\vspace{-3ex}}
\begin{document}

\title{
    \vspace{-30pt}
	\includegraphics*[width=0.1\textwidth,left]{images/hu_logo2.png}\\
	\vspace{-10pt}
	Einf"uhrung in die Philosophie}
\author{Lennard Wolf\\
        \small{\href{mailto:lennard.wolf@student.hu-berlin.de}{lennard.wolf@student.hu-berlin.de}}}
\maketitle
\vspace{0pt}

\section*{AB 8: Essayschreiben (II) –- Abstract}
\large

%\vspace{10pt}
\noindent\textbf{Ist das Ganze in seine Teile }

\noindent In Buch 2, Abschnitt 19, des \emph{Grundri"s der pyrrhonischen Skepsis}, argumentiert Sextus Empiricus gegen die M"oglichkeit der Einteilung des Ganzen in seine Teile anhand des Beispiels der Zahl Zehn. (vgl. \cite[S. 209f.]{empiricus2013grundriss}) 

Nach einer genaueren Beschreibung der Argumentation von Empiricus werde ich zeigen, dass sie sich auch auf anderes, im besonderen den Menschen und den Satz, anwenden l"asst. Die daraus gezogenen Schl"usse verwende ich als Begr"undung f"ur das neue Gebiet der Komplexit"atstheorie. Basierend auf \cite{mitchell2009complexity} zeige ich, dass aus der Nichtreduzierbarkeit von Dingen wie Ameisenkolonien oder dem Internet eine neue Denkweise n"otig ist, um sie als Ganzes betrachten zu k"onnen. 


\bibliography{einfuehrung-i-d-p-ha8}

\end{document}
