\documentclass[a4paper]{article}
\usepackage{graphicx}
\usepackage{fullpage}
%\usepackage{parskip}
\usepackage{color}
\usepackage[ngerman]{babel}
\usepackage{hyperref}
\usepackage{calc} 
\usepackage{enumitem}
\usepackage{titlesec}
\usepackage[export]{adjustbox}
%\pagestyle{headings}

\titleformat{name=\section,numberless}
  {\normalfont\Large\bfseries}
  {}
  {0pt}
  {}
\date{\vspace{-3ex}}
\begin{document}

\title{
    \vspace{-30pt}
	\includegraphics*[width=0.1\textwidth,left]{images/hu_logo2.png}\\
	\vspace{-10pt}
	Einf"uhrung in die Philosophie}
\author{Lennard Wolf\\
        \small{\href{mailto:lennard.wolf@student.hu-berlin.de}{lennard.wolf@student.hu-berlin.de}}}
\maketitle
\vspace{0pt}

\section*{AB 3a: Argumente kritisieren}
\large
\vspace{2pt}
\textbf{Sche­ma­ti­sie­rung der Rekonstruktion auf dem Arbeitsblatt}
\begin{description}[leftmargin=!,labelwidth=\widthof{\bfseries (P21)}]
  \item[PAP] Jones ist f"ur sein Handeln nur dann verantwortlich, wenn er anders h"atte handeln k"onnen.
  \item[P1] Aufgrund der Anwesenheit des Mechanismus kann Jones nicht anders handeln (und beabsichtigen), als er tats"achlich handelt (und beabsichtigt).
  \item[P2] Dies untergr"abt nicht seine moralische Verantwortung.
  \item[K] PAP ist falsch. (aus P1 und P2)
\end{description}
\vspace{10pt}
\textbf{(3b)}
\vspace{8pt}
\\
Im Folgenden m"ochte ich zeigen, dass die von Frankfurt angef"uhrte Argumentation gegen das \emph{Principle of Alternate Possibilities} (PAP) nur dann gilt, wenn der Determinismus unwahr ist. Darauf aufbauend werde ich die Konsequenzen der von mir aufgezeigten Einschr"ankung besprechen.

Gesetzt den Fall, dass der Determinismus unwahr ist, und wir voraussetzen, dass der von Frankfurt beschriebene Mechanismus in einer nichtdeterminierten Welt verl"asslich funktioniert, w"urde er die Handlungsfreiheit des Jones wie beschrieben einschr"anken. Da Jones sich in dem Szenario aber f"ur die Handlung entscheidet, braucht der Mechanismus nicht zu greifen, und die moralische Verantwortung f"ur die Handlung l"age ganz bei Jones, trotz der Abwesenheit von Handlungsalternativen. Entsprechend zieht Frankfurt den folgerichtigen Schluss, dass PAP in solch einer Welt falsch w"are.

Wenn nun aber der Determinismus wahr ist, so h"atte Jones von vornherein nicht anders handeln k"onnen, da die Naturgesetze den gesamten Verlauf des Universums, inklusive der hier besprochenen Handlung von Jones, schon vorbestimmt h"atten. Die Pr"amisse P1 benennt hier mit dem Mechanismus also eine falsche Kausalit"at f"ur das Fehlen von Handlungsalternativen. Traditionell ist Kontrolle eine notwendige Bedingung f"ur moralische Verantwortung, und in einer determinierten Welt w"are die Verantwortung aufgrund der fehlenden Kontrolle vonseiten Jones untergraben. P1 und im Besonderen P2 w"aren hier also falsch und die Folgerung zu K daher nicht mehr zul"assig.

Das in der Vorlesung angef"uhrte Argument f"ur den Inkompatibilismus, welches PAP zur ersten Pr"amisse hat, setzt in seiner zweiten Pr"amisse voraus, dass der Determinismus wahr ist. Innerhalb des Kontexts dieses Arguments f"ande Frankfurts Entgegnung zu PAP wie oben gezeigt also keine g"ultige Verwendung. Es stellt sich daher die Frage, welche Relevanz sie "uberhaupt im Diskurs zur Willensfreiheit hat und zu welchem Zweck man PAP mit der Voraussetzung einer nichtdeterminierten Welt widerlegt.



\end{document}
