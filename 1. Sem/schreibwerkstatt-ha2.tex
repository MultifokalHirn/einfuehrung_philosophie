\documentclass[a4paper]{article}
\usepackage{graphicx}
\usepackage{fullpage}
%\usepackage{parskip}
\usepackage{color}
\usepackage[ngerman]{babel}
\usepackage{hyperref}
\usepackage{calc} 
\usepackage{enumitem}
\usepackage{titlesec}
%\pagestyle{headings}

\titleformat{name=\section,numberless}
  {\normalfont\Large\bfseries}
  {}
  {0pt}
  {}

\begin{document}

\title{
	\includegraphics*[width=0.6\textwidth]{images/hu_logo.png}\\
	\vspace{18pt}
	Philosophische Schreibwerkstatt}
%\subtitle{Vorlesung WS 16/17\\
%          Prof. Template\\
%          Philosophisches Institut I \\ 
%          Humboldt Universit"at zu Berlin}
\author{Lennard Wolf\\
        \href{mailto:lennard.wolf@student.hu-berlin.de}{lennard.wolf@student.hu-berlin.de}}
\maketitle


\section*{Argumentrekonstruktion}
\large

\emph{The Cult of Moral Grayness} -- Ayn Rand

\begin{description}[leftmargin=!,labelwidth=\widthof{\bfseries P2}]
  \item[P1] Eine moralische Grauzone bedingt eine Definition von `wei\ss' und `schwarz' (Gut und B"ose).
  \item[P2] Wenn Gut und B"ose definiert sind, kann bei Vorhandensein aller n"otigen Informationen durch rationales Denken immer ein klares, d.h. bin"ares, moralisches Urteil getroffen werden. 
  \item[K1] Es gibt keine moralische Grauzone.
  \item[P3] Menschen treffen aus vielerlei Gr"unden (z.B. Faulheit, Angst) moralisch schlechte Entscheidungen. 
  \item[P4] Menschen wollen sich nicht eingestehen, dass ihr Handeln `schwarz' sein kann.
  \item[K2] Sie erschaffen sich die moralische Grauzone, um ihr Handeln stattdessen als `grau' bezeichnen zu d"urfen.
    \item[K3] Moralische Grauzonen sind immer vehement abzulehnen.
\end{description}
\vspace{9pt}


P1 ist problematisch, da eine moralische Grauzone auch dadurch entstehen kann, dass es gerade mehrere Definitionen von Gut und B"ose gibt. Entsprechend muss es sich nicht nur um Schattierungen auf einer eindimensionalen Moralskala handeln, sondern kann durch eine unbestimmte Zahl moralischer Systeme entstehen. Ich w"urde P2 prinzipiell zustimmen, jedoch ist fraglich, ob jedes moralische System die Werkzeuge an die Hand gibt, "uber jedes erdenkliche Handeln ein moralisches Urteil f"allen zu k"onnen. 

K2 in Verbindung mit P3 und P4 ist nur \emph{eine} m"ogliche Erkl"arung f"ur die Existenz moralischer Grauzonen, aber andere Gr"unde k"onnten mindestens genauso berechtigt sein. Ein m"oglicher anderer Grund w"are, dass die Geschichte gezeigt hat, dass fundamentalistisch durchgesetzte Moralsysteme h"aufig zu Krieg und anderen Problemen f"uhren.

Da die von Rand f"ur K1 gegebene Definition der moralischen Grauzone nur sehr eingeschr"ankt ist und der in K2 dargereichte Existenzgrund nicht als alleinstehend angesehen werden kann, ist die Schlu\ss folgerung K3 nicht hinreichend begr"undet. 

\end{document}
