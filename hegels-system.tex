\documentclass[emulatestandardclasses]{scrartcl}
\usepackage{graphicx}
\usepackage{color}
\usepackage[ngerman]{babel}
\usepackage{hyperref}
\usepackage[utf8]{inputenc}
\usepackage{fullpage}
\usepackage{calc} 
\usepackage{enumitem}
\usepackage{titlesec}
\newcommand{\todo}[1]{\textcolor{red}{TODO: #1}\PackageWarning{TODO:}{#1!}}
\date{\vspace{-3ex}}
\begin{document}

\title{
	\includegraphics*[width=0.75\textwidth]{ErstesSem/images/hu_logo.png}\\
	\vspace{24pt}
	Hegels System}
\subtitle{Proseminar WS 17/18\\
          Prof. Dr. Andreas Arndt\\
          Theologische Fakult"at \\ 
          Humboldt Universit"at zu Berlin}
\author{Lennard Wolf\\
        \small{\href{mailto:lennard.wolf@student.hu-berlin.de}{lennard.wolf@student.hu-berlin.de}}}
\maketitle
\begin{abstract}
Hegels System liegt nicht in einem fertigen Zustand vor, sondern nur als \emph{work in progress}, in Grundrissen bzw. Grundlinien und Ausarbeitungen einzelner Teile, zum Teil - wie in den Vorlesungen - in immer neuen Anläufen. Entsprechend geht es weniger darum, die einzelnen Teile des Hegelschen Systementwurfs (Logik, Naturphilosophie, Geistesphilosophie) im Einzelnen zu referieren, als vielmehr um die Frage, wie sich diese Teile zueinander verhalten und was ihr Gravitationszentrum ist.

\end{abstract}
\newpage

\tableofcontents
\listoffigures
\newpage


\section{Einf"uhrungssitzung\\(17.10.17)}

\subsection{Organisatorisches}

\begin{itemize}
  \item Moodle-PW: 
\end{itemize}

\subsection{Einf"uhrung - Worum wird es gehen?}

\subsubsection{Fragen}

\begin{itemize}
  \item Verhältnis Objektiver Geist/Absoluter Geist?
  \item Was ist mit Ende der Philosophie/Weltgeschichte gemeint?
  \item Was macht ein System aus und was ist das Systematische bei Hegel?
\end{itemize}

\subsubsection{Welches "`System"'?}

\begin{itemize}
  \item \emph{Enzyklopädie der philosophischen Wissenschaften}: Kompendium der gelehrten Thesen
  \item System bleibt in Grundrissen/Versuchen $\rightarrow$ Nicht fertig/abgeschlossen!
  \item Verein der Freunde des Verewigten: "`Letztgültige Herausgabe"' von Hegels Schriften (Zusammenarbeit Schüler und Witwe) - Mit Zusätzen, die möglicherweise nicht im Geiste Hegels waren
  \item Grundlinien: Grundrisse der Grundrisse (Rechtsphilosophie: Objektiver Geist)
  \item Doch der Grundriss ist nicht die Ausführung und der Inhalt des Systems
\end{itemize}

\subsubsection{Grenzen}

\begin{itemize}
  \item Es gibt keine Philosophie des subjektiven Geistes $\rightarrow$ hat sich seit der Enzyklopädie natürlich weiter entwickelt, doch dies wurde nie offiziell schriftlich festgehalten in der Form der Enzyklopädie
  \item Hegel wird Position des "`Absoluten Idealismus"' nachgesagt, doch woher stammt dies?
  \item Suhrkamp Ausgabe wissenschaftlich ärgerlich $\rightarrow$ Authentizität unklar! 
  \item 
\end{itemize}


\newpage
\section{"Uber den Professor}
Prof. Mustermann ist..


%\begin{figure}[h]
%	\centering
%	\includegraphics[width=0.5\textwidth]{images/template.png}
%	\caption{Template Bild}
%	\label{fig:template}
%\end{figure}

\end{document}
