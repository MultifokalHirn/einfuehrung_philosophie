\documentclass[emulatestandardclasses]{scrartcl}
\usepackage{graphicx}
\usepackage{color}
\usepackage[ngerman]{babel}
\usepackage{hyperref}
\usepackage[utf8]{inputenc}
\usepackage{fullpage}
\usepackage{calc} 
\usepackage{enumitem}
\usepackage{titlesec}
\newcommand{\todo}[1]{\textcolor{red}{TODO: #1}\PackageWarning{TODO:}{#1!}}
\date{\vspace{-3ex}}
\begin{document}

\title{
	\includegraphics*[width=0.75\textwidth]{ErstesSem/images/hu_logo.png}\\
	\vspace{24pt}
	Hegels System}
\subtitle{Proseminar WS 17/18\\
          Prof. Dr. Andreas Arndt\\
          Theologische Fakult"at \\ 
          Humboldt Universit"at zu Berlin}
\author{Lennard Wolf\\
        \small{\href{mailto:lennard.wolf@student.hu-berlin.de}{lennard.wolf@student.hu-berlin.de}}}
\maketitle
\begin{abstract}
Hegels System liegt nicht in einem fertigen Zustand vor, sondern nur als \emph{work in progress}, in Grundrissen bzw. Grundlinien und Ausarbeitungen einzelner Teile, zum Teil - wie in den Vorlesungen - in immer neuen Anläufen. Entsprechend geht es weniger darum, die einzelnen Teile des Hegelschen Systementwurfs (Logik, Naturphilosophie, Geistesphilosophie) im Einzelnen zu referieren, als vielmehr um die Frage, wie sich diese Teile zueinander verhalten und was ihr Gravitationszentrum ist.

\end{abstract}
\newpage

\tableofcontents
\listoffigures
\newpage


\section{Einf"uhrungssitzung\\(17.10.17)}

\subsection{Organisatorisches}

\begin{itemize}
  \item Moodle-PW: 
\end{itemize}

\subsection{Einf"uhrung: Was wird besprochen}

\subsubsection{Fragen}

\begin{itemize}
  \item Verhältnis Objektiver Geist /Absoluter Geist
  \item Ende der Philosophie/Weltgeschichte
  \item 
\end{itemize}

\subsubsection{Welches "`System"'}

\begin{itemize}
  \item \emph{Enzyklopädie der philosophischen Wissenschaften}: Kompendium der gelehrten Thesen
  \item 
\end{itemize}


\newpage
\section{"Uber den Professor}
Prof. Mustermann ist..


%\begin{figure}[h]
%	\centering
%	\includegraphics[width=0.5\textwidth]{images/template.png}
%	\caption{Template Bild}
%	\label{fig:template}
%\end{figure}

\end{document}
