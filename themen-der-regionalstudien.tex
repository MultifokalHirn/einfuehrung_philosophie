\documentclass[emulatestandardclasses]{scrartcl}
\usepackage{graphicx}
\usepackage{CJKutf8} % japanese
\usepackage{color}
\usepackage[ngerman]{babel}
\usepackage{hyperref}
\usepackage{fullpage}
\usepackage[utf8]{inputenc}
\usepackage{calc} 
\usepackage{enumitem}
\usepackage{titlesec}
\date{\vspace{-3ex}}
\begin{document}

\title{
	\includegraphics*[bb=0 0 720 200, width=0.72\textwidth]{ErstesSem/images/hu_logo.png}\\
	\vspace{25pt}
	Regionalwissenschaftliche Debatten}
\subtitle{\vspace{10pt}
			Prof. Dr. Vincent Houben\\
			Proseminar WS 17/18\\
          Institut für Asien- und Afrikawissenschaften\\ 
          Humboldt Universit"at zu Berlin}
\author{Lennard Wolf\\
        \small{\href{mailto:lennard.wolf@hu-berlin.de}{lennard.wolf@hu-berlin.de}}}
\maketitle
\begin{abstract}
In diesem Seminar erarbeiten sich die Studierenden eine Reihe von zentralen Diskussionsthemen, die für das Studium des BA Regionalstudien Asien/Afrika entscheidend sind.

\end{abstract}
\newpage

%\tableofcontents
%\listoffigures
\newpage



\section{Regional- und Sozialwissenschaften nach dem Aufstieg des globalen Südens (08.11.17)}

\begin{itemize}
  \item Regionalstudien am Anfang: Ethnologie, sodass Menschen für lange Zeit zu fremden Völkern gereist sind, mit denen gelebt haben und dann Bücher darüber geschrieben haben
  \item Später, Kalter Krieg: Kampf zwischen Kapitalismus und Sozialismus $\rightarrow$ Fokus auf Sinologie, Indologie etc., sodass der Kapitalismus dort verbreitet werden kann
  \item Aufstieg des Südens: Multizentrische Welt wirft Probleme für die Regionalwissenschaften auf $\rightarrow$ die Regionalwissenschaften müssen die Methoden der anderen Geisteswissenschaften übernehmen (und weg von "`mein Dorf"') und die Sozialwissenschaften müssen ihren Eurozentrismus aufgeben
  \item Autonome Theorie: ?
  \item Die Klassische Unterteilung der Fächer macht nicht mehr Sinn, da sie in der Realität sowieso schon alle nicht mehr so klar zu trennen sind
  \item Interdisziplinarität:
  \item Transdisziplinarität: Übertragung der Aspekte aus einer Wissenschaft in eine Andere (Verschmelzung der Ansätze)
  \item Multidisziplinarität: 
  \item 
\end{itemize}



\newpage


\end{document}
