\documentclass[emulatestandardclasses]{scrartcl}
\usepackage{graphicx}
\usepackage{CJKutf8} % japanese
\usepackage{color}
\usepackage[ngerman]{babel}
\usepackage{hyperref}
\usepackage{fullpage}
\usepackage[utf8]{inputenc}
\usepackage{calc} 
\usepackage{enumitem}
\usepackage{titlesec}
\date{\vspace{-3ex}}
\begin{document}

\title{
	\includegraphics*[bb=0 0 720 200, width=0.72\textwidth]{ErstesSem/images/hu_logo.png}\\
	\vspace{25pt}
	Regionalwissenschaftliche Debatten\\\&\\
	Themen der Regionalstudien}
\subtitle{\vspace{10pt}
			Prof. Dr. Vincent Houben\\
			Proseminar WS 17/18\\
          Institut für Asien- und Afrikawissenschaften\\ 
          Humboldt Universit"at zu Berlin}
\author{Lennard Wolf\\
        \small{\href{mailto:lennard.wolf@hu-berlin.de}{lennard.wolf@hu-berlin.de}}}
\maketitle
\begin{abstract}
In diesem Seminar erarbeiten sich die Studierenden eine Reihe von zentralen Diskussionsthemen, die für das Studium des BA Regionalstudien Asien/Afrika entscheidend sind.

\end{abstract}
\newpage

%\tableofcontents
%\listoffigures
\newpage

\section{Allgemeines}

\subsection{Klausur}

\begin{itemize}
  \item "`Wieso ist dieses Thema für die Regionalwissenschaften relevant?"'
\end{itemize}


\section{Regional- und Sozialwissenschaften nach dem Aufstieg des globalen Südens (08.11.17)}

\begin{itemize}
  \item Regionalstudien am Anfang: Ethnologie, sodass Menschen für lange Zeit zu fremden Völkern gereist sind, mit denen gelebt haben und dann Bücher darüber geschrieben haben
  \item Später, Kalter Krieg: Kampf zwischen Kapitalismus und Sozialismus $\rightarrow$ Fokus auf Sinologie, Indologie etc., sodass der Kapitalismus dort verbreitet werden kann
  \item Aufstieg des Südens: Multizentrische Welt wirft Probleme für die Regionalwissenschaften auf $\rightarrow$ die Regionalwissenschaften müssen die Methoden der anderen Geisteswissenschaften übernehmen (und weg von "`mein Dorf"') und die Sozialwissenschaften müssen ihren Eurozentrismus aufgeben
  \item Autonome Theorie: ?
  \item Die Klassische Unterteilung der Fächer macht nicht mehr Sinn, da sie in der Realität sowieso schon alle nicht mehr so klar zu trennen sind
  \item Interdisziplinarität:
  \item Transdisziplinarität: Übertragung der Aspekte aus einer Wissenschaft in eine Andere (Verschmelzung der Ansätze)
  \item Multidisziplinarität: ?
\end{itemize}

\subsection{Vorlesung - Die Konstruktion des Anderen: Orientalismus}


\textbf{Struktur}

\begin{itemize}
  \item Die Begriffe Orient und Okzident
  \item Wissen über den Orient
  \item Beginn der Orientalistik als wissenschaftliche Disziplin
  \item Orientalismus - eine Spielart des Ethnozentrismus?
  \item Orientalis im Verständnis von E. Said
  \item Beispiel
\end{itemize}

\textbf{Die Begriffe Orient und Okzident}

Was ist der Orient?

\begin{itemize}
  \item Geografie: Nicht genau festgelegt; "`Da wo die Sonne aufgeht"' / Morgenland | Vom Westen ausgehender Begriff, Blick gen Osten
  \item Alter Orient: Mesopotaniens, Iran, Anatolien, Ägypten (Islamische Welt) | aber auch: China, Indien $\rightarrow$ Das Wort war nie klar besetzt (Für China wäre Amerika der Orient)
  \item Kulturell: Orient war mal wild, unzivilisiert, rückständig, oder interessant, exotisch, eine Goldgrube etc etc
\end{itemize}

Was ist der Okzident?

\begin{itemize}
  \item Geografie: "`Da wo die Sonne untergeht"' / Abendland
  \item Kulturell: Die eigene Kultur (Aufklärung, Humanismus, Rationalismus, Wissenschaften)
  \item Okzident war immer am Orient interessiert, anders herum eher selten
  \item Islamischer Orient: Osmanisches Reich nahm Konstantinopel ein und stand irgendwann vor Wien: Der Orient dringt in den Okzident
\end{itemize}

\textbf{Wissen über den Orient}

\begin{itemize}
  \item Paul Valery (1938): "`\emph{Der Orient könne nur dann seine ganze Wirkung auf die Einbildungskraft des Menschen ausstrahlen, wenn dieser noch nie in jener kaum definierten Gegend gewesen sei.}"'
  \item Orient wird zum Traumbild, ein Zugang zu dem, was verloren geht im technischen Fortschritt; Wird übernommen im Okzident, z.B. die orientalischen Cafes in Paris
\end{itemize}

\textbf{Beginn der Orientalistik als wissenschaftliche Disziplin}

\begin{itemize}
  \item Orient als \emph{alte} Kultur mit großer Weisheit
  \item Zivilisierung des Orient durch die Orientalisten (Ganz normal im Sinne der Naturwissenschaften)
  \item F. Bacon: "`Man muss die Natur auf die Folter spannen, um ihr alle Geheimnisse zu entlocken."'
\end{itemize}

\section{Region II}

\subsection{Geographies of Knowing, Geographies of Ignorance: Jumping Scale in Southeast Asia}

\textbf{Lektürenotizen}

\begin{itemize}
  \item 
\end{itemize}


\newpage


\end{document}
