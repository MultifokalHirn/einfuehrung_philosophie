\documentclass[emulatestandardclasses]{scrartcl}
\usepackage{graphicx}
\usepackage{CJKutf8} % japanese
\usepackage{color}
\usepackage[ngerman]{babel}
\usepackage{hyperref}
\usepackage{fullpage}
\usepackage[utf8]{inputenc}
\usepackage{calc} 
\usepackage{enumitem}
\usepackage{titlesec}
\date{\vspace{-3ex}}
\begin{document}

\title{
	\includegraphics*[bb=0 0 720 200, width=0.72\textwidth]{ErstesSem/images/hu_logo.png}\\
	\vspace{25pt}
	Regionalwissenschaftliche Debatten\\\&\\
	Themen der Regionalstudien}
\subtitle{\vspace{10pt}
			Prof. Dr. Vincent Houben\\
			Proseminar WS 17/18\\
          Institut für Asien- und Afrikawissenschaften\\ 
          Humboldt Universit"at zu Berlin}
\author{Lennard Wolf\\
        \small{\href{mailto:lennard.wolf@hu-berlin.de}{lennard.wolf@hu-berlin.de}}}
\maketitle
\begin{abstract}
In diesem Seminar erarbeiten sich die Studierenden eine Reihe von zentralen Diskussionsthemen, die für das Studium des BA Regionalstudien Asien/Afrika entscheidend sind.

\end{abstract}
\newpage

%\tableofcontents
%\listoffigures
\newpage

\section{Allgemeines}

\subsection{Klausur}

\begin{itemize}
  \item "`Wieso ist dieses Thema für die Regionalwissenschaften relevant?"'
  \item nochmal szanton lesen
\end{itemize}


\section{Regional- und Sozialwissenschaften nach dem Aufstieg des globalen Südens (08.11.17)}

\begin{itemize}
  \item Regionalstudien am Anfang: Ethnologie, sodass Menschen für lange Zeit zu fremden Völkern gereist sind, mit denen gelebt haben und dann Bücher darüber geschrieben haben
  \item Später, Kalter Krieg: Kampf zwischen Kapitalismus und Sozialismus $\rightarrow$ Fokus auf Sinologie, Indologie etc., sodass der Kapitalismus dort verbreitet werden kann
  \item Aufstieg des Südens: Multizentrische Welt wirft Probleme für die Regionalwissenschaften auf $\rightarrow$ die Regionalwissenschaften müssen die Methoden der anderen Geisteswissenschaften übernehmen (und weg von "`mein Dorf"') und die Sozialwissenschaften müssen ihren Eurozentrismus aufgeben
  \item Autonome Theorie: ?
  \item Die Klassische Unterteilung der Fächer macht nicht mehr Sinn, da sie in der Realität sowieso schon alle nicht mehr so klar zu trennen sind
  \item Interdisziplinarität:
  \item Transdisziplinarität: Übertragung der Aspekte aus einer Wissenschaft in eine Andere (Verschmelzung der Ansätze)
  \item Multidisziplinarität: ?
\end{itemize}

\subsection{Vorlesung - Die Konstruktion des Anderen: Orientalismus}


\textbf{Struktur}

\begin{itemize}
  \item Die Begriffe Orient und Okzident
  \item Wissen über den Orient
  \item Beginn der Orientalistik als wissenschaftliche Disziplin
  \item Orientalismus - eine Spielart des Ethnozentrismus?
  \item Orientalis im Verständnis von E. Said
  \item Beispiel
\end{itemize}

\textbf{Die Begriffe Orient und Okzident}

Was ist der Orient?

\begin{itemize}
  \item Geografie: Nicht genau festgelegt; "`Da wo die Sonne aufgeht"' / Morgenland | Vom Westen ausgehender Begriff, Blick gen Osten
  \item Alter Orient: Mesopotaniens, Iran, Anatolien, Ägypten (Islamische Welt) | aber auch: China, Indien $\rightarrow$ Das Wort war nie klar besetzt (Für China wäre Amerika der Orient)
  \item Kulturell: Orient war mal wild, unzivilisiert, rückständig, oder interessant, exotisch, eine Goldgrube etc etc
\end{itemize}

Was ist der Okzident?

\begin{itemize}
  \item Geografie: "`Da wo die Sonne untergeht"' / Abendland
  \item Kulturell: Die eigene Kultur (Aufklärung, Humanismus, Rationalismus, Wissenschaften)
  \item Okzident war immer am Orient interessiert, anders herum eher selten
  \item Islamischer Orient: Osmanisches Reich nahm Konstantinopel ein und stand irgendwann vor Wien: Der Orient dringt in den Okzident
\end{itemize}

\textbf{Wissen über den Orient}

\begin{itemize}
  \item Paul Valery (1938): "`\emph{Der Orient könne nur dann seine ganze Wirkung auf die Einbildungskraft des Menschen ausstrahlen, wenn dieser noch nie in jener kaum definierten Gegend gewesen sei.}"'
  \item Orient wird zum Traumbild, ein Zugang zu dem, was verloren geht im technischen Fortschritt; Wird übernommen im Okzident, z.B. die orientalischen Cafes in Paris
\end{itemize}

\textbf{Beginn der Orientalistik als wissenschaftliche Disziplin}

\begin{itemize}
  \item Orient als \emph{alte} Kultur mit großer Weisheit
  \item Zivilisierung des Orient durch die Orientalisten (Ganz normal im Sinne der Naturwissenschaften)
  \item F. Bacon: "`Man muss die Natur auf die Folter spannen, um ihr alle Geheimnisse zu entlocken."'
\end{itemize}

\section{Region II\\22.11.2017}

\subsection{Geographies of Knowing, Geographies of Ignorance: Jumping Scale in Southeast Asia}

\textbf{Sitzung}

\begin{itemize}
  \item James Scott: The art of not being governed - An anarchist history of Southeast Asia
  \item Gegengeschichte über die Grenzregionen in Zomia "`state evasion"'
  \item Zomiaexperte hier am Institut: Tony Huber (?)
  \item Text von Szanton war eine Bestandsaufnahme der area studies, van Schendel will neue Wege gehen $\rightarrow$ van Schendels Ideen haben sich zT auch durchgesetzt
  \item Die Ausführungen des Textes lassen sich im Grunde auf alle Regionen anwenden
  \item Mutilinguistischer, multikultureller Kontext wirft Frage auf: "`Wo bin ich eigentlich?"' - Südostasien? Vietnam?
  \item Entwicklung der \emph{area studies} nach WWII: \textbf{scramble} (Wettrennen) um die Regionen $\rightarrow$ \textbf{conceptual empires} (zB southeastasian studies)
  \item Lokale Regionalstudien
  \item Die Regionalstudien sind zu weit gegangen, indem sie "`Themenghettos"' erschaffen haben
  \item Zomia: gehört nicht zu den Südostasienwissenschaften, nicht zu Südchinawissenschaften etc.; Aber hat soziale, kulturelle, sprachliche Ähnlichkeiten $\rightarrow$ könnte also eigentlich als eine Region anerkannt werden $\rightarrow$ Leitfrage: Warum ist Zomia nicht zu einer Region geworden? $\rightarrow$ Regionen sind wirklich: "`\emph{expressions of a particular geography of power, they were instruments to naturalise the geopolitical arrangements of the day. As expressions of certain academic interests and disciplines, they were instruments in institutional strategies with regard to funds, students, jobs, and prestige. And they contributed to a certain ghettoisation of critical insights as area studies tended toward the guild model.}"' S. 287
  \item Zomia verschwindet heute wieder: Die Leute können sich der Staatsbildung langsam nicht mehr entziehen
  \item In China etc. wurde in den area studies die regionale Einteilung aus dem Westen übernommen
  \item \textbf{Prozessgeografie}: Kartierung der Welt durch Flüsse und Prozesse, Interessiert an Bewegung im Raum; Neue Vorstellung von Raum, \textbf{Geografische Skalierung} als andere alternative Raumvorstellung
  \item Kritik: Die Prozessgeografie reicht mit ihrem sozialwirtschaftlichen Blickwinkel noch nicht aus, es muss noch an die area studies angepasst werden: Muss um Kultur erweitert werden; Scales: urban, rural, lokal, national, area, global | scales sind noch sehr westlich 
  \item \textbf{Andere Perspektiven}: Distanz (dehnbare und relativ), Interregionale Verbindungen (Grenzen zwischen areas aufbrechen), Fokus auf \textbf{Grenzregionen} (\emph{inside out}: was am Rand war ist jetzt im Zentrum), Flüsse von Kultur, Geld und Menschen
  \item Konklusion: Neue räumliche Formen (Hohlkreise, Flickteppiche)
  \item Houben: Das neue regionalwissenschaftliche Wissen ist nicht zentriert auf ein Kernthema, sondern multizentrisch, daher hier im BA erhält man breiten Blick auf viele verschiedene
\end{itemize}

\section{Debatten um Orientalism\\22.11.2017}

\begin{itemize}
  \item Moodle PW: RVL Modul 1
  \item Said: Anglikanischer Palästinenser (Professor of Terror da er sich Palästina eingesetzt hat, gleichzeitig sind Bücher von ihm in Palästina verboten, da er ) "`Speaking truth to power"' als Motto
  \item Deutsche Morgenländische Gesellschaft, ZmO
  \item Said: Mix aus Gramschi (?) (ich hab die Wahrheit erkannt, ihr habt falsches Bewusstsein) und Foucault (Diskurse sind Mengen von sagbaren Dingen, und diese stellen die Wahrheit her), die eigentlich schwer zu verbinden sind
\end{itemize}

\subsection{Bernard Lewis: Mangelnde Fachkenntnis \& Verzerrungen seitens Saids}

\begin{itemize}
  \item Bernard Lewis "`The Question of Orientalism"'
  \item Said hat kein wissenschaftliches Ziel, sondern Polemik
  \item Zur Veröffentlichung von Saids Buch gab es eigentlich schon keine Orientialisten mehr die sich als solche bezeichneten
  \item Said sei nicht originell
  \item Selektiv: um seine These, Orientalismus = Kolonialismus zu stützen, werden dem zustimmende Beispiel rausgesucht und anderes unterschlagen
  \item Said mache sachliche Fehler, hat falsches Wissenschaftsverständnis und unangemessene Begrifflichkeit (Freud war zu der Zeit wieder hip, daher: sexuelle Begriffsnutzung, Penetration des Orients)
  \item Mangelnde intellektuelle Redlichkeit: Literaturwissenschaftliche Überinterpretation der historischen Quellen
  \item Said verachtet den Orient: schlimmer als jeder arroganter europäischer Imperialist
  \item Saids Verständnis von Wissen und Macht: knowledge/power, Verzerrung der Fachgeschichte
  \item Der Orient ist ein Diskurs, erst von Europa erschaffen
\end{itemize}


\subsection{Aijaz Ahmad \& James Clifford: Kritik an theoretischer Inkonsistenz von Said}

\begin{itemize}
  \item Ende 80er, Anfang 90er; sind eigentlich wohlgesonnen!
  \item James Clifford: wichtiger Postkolonialer Theoretiker
  \item Aijaz Ahmad: Bekanntester, "`scharfzüngigster"' Intelektueller aus dem globalen Süden
  \item Saids Theoriegebäude sei widersprüchlich!
  \item Geht es um Repräsentationen oder den "`realen Orient"'? 
  \item Kritisiert er nun, dass es einen eigentlichen Orient gibt und der Westen missrepräsentiert diesen, oder gibt es den 
  \item Saids Eklektizismus: Benutzt verschiedenste Argumente, je nachdem wie es ihm gerade passt: Geschichte, Literaturwissenschaften
  \item Psychologisierende Logik: dass Der Westen das Andere braucht um sich selbst zu verstehen
  \item Saids Nutzen: Der Orientalismus kann für alles Schlimme, das orientalische Länder erlebt haben und weiterhin erleben, beschuldigt werden und damit hat man ein super 
  \item Qui bono? Wem nützt es? 
\end{itemize}


\subsection{Arif Dirlik: Auto-Orientalismus}

\begin{itemize}
  \item Mitte/Ende 90er
  \item Wie haben die Regionen selber an den Verzerrung 
  \item Hat Said Wesen und Stellung des Orientalismus korrekt dargestellt? Ist der Orient nicht vllt. in einer Zusammenarbeit des Westens mit den Eliten Asiens entstanden?("`Is orientalism a think or a relationship?"')
  \item Ohne die Orientalen keinen Orient.
  \item Selbst-Orientalisierung: Das asiatische Selbstbild und das westliche Bild von Asien ist in einem konstanten Austausch, beide bedingen sich
  \item Folgen des Auto-Orientalismus: Indem das orientalistische Bild aufgenommen wird, wird der Westen geothered und interne Differenzen werden unterdrückt um ein allverbindendes Selbstbild zu erzeugen
\end{itemize}

\section{Edward Said: Orientalism\\29.11.2017 (Debatten)}

\subsection{Edward Said: Hintergrund}

\begin{itemize}
  \item Edward Said war christlicher Palästinenser: Prof. für Englisch und Vergleichende Literaturwissebschaften
  \item Orientalism war sehr einflussreich für viele Wissenschaften: Ethnologie, Kulturwissenschaften etc. / Mussten ihre Grundlagen überdenken
  \item Fundamentale Kritik der westlichen Sichtweise auf den Orient (Er war sehr auf Mittleren Osten fokussiert, doch es wird heute auf alles nicht-westliche angewandt)
  \item Youtube-Clip: "`Said: On Orientalism"'
  \item Besonders Philologen nennen sich auch heute noch in Teilen Orientalisten, seid Said etwas schwierig, da einiges (für sie ungewolltes) mitschwingt
\end{itemize}

\subsection{Orientalism}

\begin{itemize}
  \item Essentialistische Kategorien: Westen und Nicht-Westen
  \item Orientalistisches Denkmuster in allem stark vertreten
  \item Orientalismus: Kein Abbild sondern eine Repräsentation, Art des Sprechens über den anderen
  \item Orientalismus, 3 Ebenen: Akademischer Betrieb, Denkstil mit ontologischer und epistemologischer Färbung, Diskurs zur Domination
  \item Präzisierungen: Will nicht die Beziehung zwischen Orient und Orientalismus erforschen, will hegemoniale (Gramsci: Ideologien haben mehrere machtstrukturelle Dimensionen) Machtstrukturen herausbilden
  \item Es gibt keine neutrale Wissensproduktion
  \item Methode: literaturwissenschaftliche Arbeit, wie geht man mit orientalistischen Texten um? wer hat Texte über den Orient wie und warum geschrieben? $\rightarrow$ "`Quellenkritik"'. Intertextualität: Geflecht der Verknüpfungen. | "`\textbf{Externalität}"': Analysieren was außerhalb des Textes steht: "`\emph{Wie wird der Orient repräsentiert?}"'
  \item Teil des orientalistischen Denkstils: Der Orient könne nicht über sich sprechen.
  \item Orientwissenschaften im 19. Jhd.: Aufbauend auf den kolonialistischen Bildern der bösen Sultane, der sensuellen Frauen, der Rückständgikeit etc.
  \item $\rightarrow$ Das vermeintliche Wissen muss verworden werden $\rightarrow$ Theoretische Antwort auf Orientalismus: \textbf{Postkoloniale Theorie}
  \item Manifester vs. Latenter Orientalismus
\end{itemize}

\subsection{Folgen für uns}

\begin{itemize}
  \item Man muss in seinen Texten immer darüber reflektieren, wenn man ethnozentristische Weltbilder unkritisch reproduziert
  \item  Beispiel: myth of lazy native 
\end{itemize}


\subsection{Kritikpunkte}

\begin{itemize}
  \item 1.: Orientalismus sei totalitärer Riesenkomplex, der "`here to stay"' ist
  \item 2.: ?
\end{itemize}


\subsection{Fragen}

\begin{itemize}
  \item Es wird doch auch innerhalb des Okzidents geothered?
  \item Unterschied kulturelle Hegemonie und Ideologie?
  \item Wie steht es um die Analyse historischer Völker? Suche nach den großen Staaten: um nachzuweisen, dass diese Zivilisation ($\rightarrow$ burruburru? 10. Jahrhundert; Khmer; Handelsbeziehungen zwischen Java und China schon 1000 v.C.)
\end{itemize}

\section{Kolonialismus I\\29.11.2017 (Vorlesung)}

\subsection{Disclaimer zu Sklavenhaltung}

\begin{itemize}
  \item Disclaimer: Sklavenhaltung als Phänomen ist ein \emph{vorkoloniales Phänomen}!
  \item Die Europäer fingen die Schwarzen nicht einfach ein in Afrika sondern kauften sie Afrikanischen Sklavenhändlern ab, wo sie herkamen interessierte sie nicht! 
  \item $\rightarrow$Hier zeichnet sich schon die kooperativen Strukturen der Afrikaner und der Europäer in der Kolonialzeit ab!
  \item Stichwort: Heiße Debatte: Sklaverei vor der Kolonialzeit
\end{itemize}

\subsection{Einführung}

\begin{itemize}
  \item 1898: Spanisch-Amerikanischer Krieg
  \item Informal Empire: um 1900 große Teile Südamerikas waren Teil des britischen informal empires (gestützt auf der wirtschaftlichen Kraft des Zuckerhandels etc.)
  \item \textbf{Definition 1}: "`Eine \textbf{Kolonie} ist ein durch Invasion (Eroberung und/oder Siedlungskolonisation), \textbf{in Anknüpfung an vorkoloniale Zustände} neu geschaffenes politisches Gebilde, dessen landfremde Herrschaftsträger in dauerhaften Abhängigkeitsbeziehungen zu einem räumlich entfernten 'Mutterland' oder imperialen Zentrum stehen, welches exklusive 'Besitz'-ansprüche auf die Kolonie erhebt."' (Quelle: Osterhammel: \emph{Kolonialismus})
  \item \emph{princely states}: In vielen Regionen haben Rhadjas in British India weiter regiert! (Großbritannien nahm natürlich noch bedingten Einfluss: Steuereinnahmen, Witwenverbrennung musste eingestellt werden)
  \item \emph{Kondominium}: zB Seychellen
  \item \textbf{Definition 2}: "`\textbf{Kolonialismus} ist eine Herrschaftsbeziehung zwischen Kollektiven, bei welcher die fundamentalen Entscheidungen über die Lebensführung der Kolonisierten durch eine kulturell andersartige und kaum anpassungswillige Minderheit von Kolonialherren unter vorrangiger Berücksichtigung externer Interessen getroffen und tatsächlich durchgesetzt werden. Damit verbinden sich in der Neuzeit in der Regel sendungsideologische Rechtfertigungsdoktrinen, die auf der Überzeugung der Kolonialherren von ihrer eigenen kulturellen Höherwertigkeit beruhen."' (Quelle: Osterhammel: \emph{Kolonialismus})
  \item Neokolonialismus: "`Regierungen und Unternehmen der reichen Industriestaaten – vor allem der USA, der EU und in den letzten Jahren verstärkt auch China – sich die Kontrolle über die Ressourcen, Finanz- und Warenmärkte der ärmeren Länder zu sichern versuchen."' (wikipedia)
  \item Ist das, was zB die Briten in Irland gemacht haben schon Kolonialismus? Manche sagen auch hier "`ja"'.
  \item \textbf{Koloniales Dilemma}: Zum einen betonen die Kolonialherren den kulturellen Unterschied (und verbieten den einheimischen zB westliche Lebensweisen), zum anderen gab es die Zivilisierungsmaßnahmen.
  \item \emph{White Mans Burden} (Kipling): Zivilisatorische Pflicht, vom Schicksal auferlegt, die Welt zu zivilisieren. (Unterton: ist ein Scheißjob, aber einer muss es ja machen, wir sind nunmal überlegen, und die moralische Pflicht zwingt uns)
  \item \textbf{Definition 3}: "``\textbf{Imperialismus}' ist der Begriff, unter dem alle Kräfte und Aktivitäten zusammengefasst werden, die zum Aufbau und zur Erhaltung [...] transkolonialer Imperien beitrugen. Zum Imperialismus gehört auch der Wille und das Vermögen eines imperialen Zentrums, die eigenen nationalstaatlichen Interessen als imperiale zu definieren und in der Anarchie des internationalen Systems weltweit geltend zu machen. Imperialismus impliziert also nicht bloß Kolonialpolitik, sondern `Weltpolitik', für welche Kolonien nicht alleine Zwecke in sich selbst, sondern auch Pfänder in globalen Machtspielen sind."' (Quelle: Osterhammel: \emph{Kolonialismus}) 
  \item Imperialismus = Kolonialpolitik + Weltpolitik
  \item zB die Holländer hatten in ihrer Kolonialpolitik keine Weltpolitik machen wollen
  \item \textbf{Definition 4}: "`Informal Empire"': ``Big Brother' hat sich als Ergebnis von punktuell ausgeübtem Druck (`Kanonenbootdiplomatie') in `ungleichen Verträgen' Privilegien verbriefen lassen. Deren Inhalt ist zumeist der Schutz ausländischer Staatsangehöriger vor der Geltung einheimischer Gesetze durch Konsularjurisdiktion und Extraterritorialität, die Festlegung eines Freihandelsregimes (niedrige Importzölle bei fehlender Zollhoheit), das Recht zur Stationierung fremder Truppen auf den Hoheitsgewässern und an vereinbarten Landpunkten. `Big Brother' ist durch Konsuln, Diplomaten oder `Residenten' vertreten, die `beratend' in die einheimische Politik, besonders auch in Nachfolgekämpfe, eingreifen und ihrem `Rat' notfalls durch Androhung militärischer Intervention Nachdruck verleihen."' (Quelle: Osterhammel: \emph{Kolonialismus})
  \item Ungleiche Verträge: gemeint ist China 
\end{itemize}

\subsection{Fragen}

\begin{itemize}
  \item Wo wurden Sklaven überall eingekauft?
\end{itemize}

\section{Kolonialismus II\\29.11.2017 (Vorlesung)}

\newpage


\end{document}
