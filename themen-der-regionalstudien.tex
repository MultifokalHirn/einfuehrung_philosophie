\documentclass[emulatestandardclasses]{scrartcl}
\usepackage{graphicx}
\usepackage{CJKutf8} % japanese
\usepackage{color}
\usepackage[ngerman]{babel}
\usepackage{hyperref}
\usepackage{fullpage}
\usepackage[utf8]{inputenc}
\usepackage{calc} 
\usepackage{enumitem}
\usepackage{titlesec}
\date{\vspace{-3ex}}
\begin{document}

\title{
	\includegraphics*[bb=0 0 720 200, width=0.72\textwidth]{ErstesSem/images/hu_logo.png}\\
	\vspace{25pt}
	Regionalwissenschaftliche Debatten\\\&\\
	Themen der Regionalstudien}
\subtitle{\vspace{10pt}
			Prof. Dr. Vincent Houben\\
			Proseminar WS 17/18\\
          Institut für Asien- und Afrikawissenschaften\\ 
          Humboldt Universit"at zu Berlin}
\author{Lennard Wolf\\
        \small{\href{mailto:lennard.wolf@hu-berlin.de}{lennard.wolf@hu-berlin.de}}}
\maketitle
\begin{abstract}
In diesem Seminar erarbeiten sich die Studierenden eine Reihe von zentralen Diskussionsthemen, die für das Studium des BA Regionalstudien Asien/Afrika entscheidend sind.

\end{abstract}
\newpage

%\tableofcontents
%\listoffigures
\newpage

\section{Allgemeines}

\subsection{Klausur}

\begin{itemize}
  \item "`Wieso ist dieses Thema für die Regionalwissenschaften relevant?"'
\end{itemize}


\section{Regional- und Sozialwissenschaften nach dem Aufstieg des globalen Südens (08.11.17)}

\begin{itemize}
  \item Regionalstudien am Anfang: Ethnologie, sodass Menschen für lange Zeit zu fremden Völkern gereist sind, mit denen gelebt haben und dann Bücher darüber geschrieben haben
  \item Später, Kalter Krieg: Kampf zwischen Kapitalismus und Sozialismus $\rightarrow$ Fokus auf Sinologie, Indologie etc., sodass der Kapitalismus dort verbreitet werden kann
  \item Aufstieg des Südens: Multizentrische Welt wirft Probleme für die Regionalwissenschaften auf $\rightarrow$ die Regionalwissenschaften müssen die Methoden der anderen Geisteswissenschaften übernehmen (und weg von "`mein Dorf"') und die Sozialwissenschaften müssen ihren Eurozentrismus aufgeben
  \item Autonome Theorie: ?
  \item Die Klassische Unterteilung der Fächer macht nicht mehr Sinn, da sie in der Realität sowieso schon alle nicht mehr so klar zu trennen sind
  \item Interdisziplinarität:
  \item Transdisziplinarität: Übertragung der Aspekte aus einer Wissenschaft in eine Andere (Verschmelzung der Ansätze)
  \item Multidisziplinarität: ?
\end{itemize}

\subsection{Vorlesung - Die Konstruktion des Anderen: Orientalismus}


\textbf{Struktur}

\begin{itemize}
  \item Die Begriffe Orient und Okzident
  \item Wissen über den Orient
  \item Beginn der Orientalistik als wissenschaftliche Disziplin
  \item Orientalismus - eine Spielart des Ethnozentrismus?
  \item Orientalis im Verständnis von E. Said
  \item Beispiel
\end{itemize}

\textbf{Die Begriffe Orient und Okzident}

Was ist der Orient?

\begin{itemize}
  \item Geografie: Nicht genau festgelegt; "`Da wo die Sonne aufgeht"' / Morgenland | Vom Westen ausgehender Begriff, Blick gen Osten
  \item Alter Orient: Mesopotaniens, Iran, Anatolien, Ägypten (Islamische Welt) | aber auch: China, Indien $\rightarrow$ Das Wort war nie klar besetzt (Für China wäre Amerika der Orient)
  \item Kulturell: Orient war mal wild, unzivilisiert, rückständig, oder interessant, exotisch, eine Goldgrube etc etc
\end{itemize}

Was ist der Okzident?

\begin{itemize}
  \item Geografie: "`Da wo die Sonne untergeht"' / Abendland
  \item Kulturell: Die eigene Kultur (Aufklärung, Humanismus, Rationalismus, Wissenschaften)
  \item Okzident war immer am Orient interessiert, anders herum eher selten
  \item Islamischer Orient: Osmanisches Reich nahm Konstantinopel ein und stand irgendwann vor Wien: Der Orient dringt in den Okzident
\end{itemize}

\textbf{Wissen über den Orient}

\begin{itemize}
  \item Paul Valery (1938): "`\emph{Der Orient könne nur dann seine ganze Wirkung auf die Einbildungskraft des Menschen ausstrahlen, wenn dieser noch nie in jener kaum definierten Gegend gewesen sei.}"'
  \item Orient wird zum Traumbild, ein Zugang zu dem, was verloren geht im technischen Fortschritt; Wird übernommen im Okzident, z.B. die orientalischen Cafes in Paris
\end{itemize}

\textbf{Beginn der Orientalistik als wissenschaftliche Disziplin}

\begin{itemize}
  \item Orient als \emph{alte} Kultur mit großer Weisheit
  \item Zivilisierung des Orient durch die Orientalisten (Ganz normal im Sinne der Naturwissenschaften)
  \item F. Bacon: "`Man muss die Natur auf die Folter spannen, um ihr alle Geheimnisse zu entlocken."'
\end{itemize}

\section{Region II\\22.11.2017}

\subsection{Geographies of Knowing, Geographies of Ignorance: Jumping Scale in Southeast Asia}

\textbf{Sitzung}

\begin{itemize}
  \item James Scott: The art of not being governed - An anarchist history of Southeast Asia
  \item Gegengeschichte über die Grenzregionen in Zomia "`state evasion"'
  \item Zomiaexperte hier am Institut: Tony Huber (?)
  \item Text von Szanton war eine Bestandsaufnahme der area studies, van Schendel will neue Wege gehen $\rightarrow$ van Schendels Ideen haben sich zT auch durchgesetzt
  \item Die Ausführungen des Textes lassen sich im Grunde auf alle Regionen anwenden
  \item Mutilinguistischer, multikultureller Kontext wirft Frage auf: "`Wo bin ich eigentlich?"' - Südostasien? Vietnam?
  \item Entwicklung der \emph{area studies} nach WWII: \textbf{scramble} (Wettrennen) um die Regionen $\rightarrow$ \textbf{conceptual empires} (zB southeastasian studies)
  \item Lokale Regionalstudien
  \item Die Regionalstudien sind zu weit gegangen, indem sie "`Themenghettos"' erschaffen haben
  \item Zomia: gehört nicht zu den Südostasienwissenschaften, nicht zu Südchinawissenschaften etc.; Aber hat soziale, kulturelle, sprachliche Ähnlichkeiten $\rightarrow$ könnte also eigentlich als eine Region anerkannt werden $\rightarrow$ Leitfrage: Warum ist Zomia nicht zu einer Region geworden? $\rightarrow$ Regionen sind wirklich: "`\emph{expressions of a particular geography of power, they were instruments to naturalise the geopolitical arrangements of the day. As expressions of certain academic interests and disciplines, they were instruments in institutional strategies with regard to funds, students, jobs, and prestige. And they contributed to a certain ghettoisation of critical insights as area studies tended toward the guild model.}"' S. 287
  \item Zomia verschwindet heute wieder: Die Leute können sich der Staatsbildung langsam nicht mehr entziehen
  \item In China etc. wurde in den area studies die regionale Einteilung aus dem Westen übernommen
  \item \textbf{Prozessgeografie}: Kartierung der Welt durch Flüsse und Prozesse, Interessiert an Bewegung im Raum; Neue Vorstellung von Raum, \textbf{Geografische Skalierung} als andere alternative Raumvorstellung
  \item Kritik: Die Prozessgeografie reicht mit ihrem sozialwirtschaftlichen Blickwinkel noch nicht aus, es muss noch an die area studies angepasst werden: Muss um Kultur erweitert werden; Scales: urban, rural, lokal, national, area, global | scales sind noch sehr westlich 
  \item \textbf{Andere Perspektiven}: Distanz (dehnbare und relativ), Interregionale Verbindungen (Grenzen zwischen areas aufbrechen), Fokus auf \textbf{Grenzregionen} (\emph{inside out}: was am Rand war ist jetzt im Zentrum), Flüsse von Kultur, Geld und Menschen
  \item Konklusion: Neue räumliche Formen (Hohlkreise, Flickteppiche)
  \item Houben: Das neue regionalwissenschaftliche Wissen ist nicht zentriert auf ein Kernthema, sondern multizentrisch, daher hier im BA erhält man breiten Blick auf viele verschiedene
\end{itemize}

\section{Debatten um Orientalism\\22.11.2017}

\begin{itemize}
  \item Moodle PW: RVL Modul 1
  \item Said: Anglikanischer Palästinenser (Professor of Terror da er sich Palästina eingesetzt hat, gleichzeitig sind Bücher von ihm in Palästina verboten, da er ) "`Speaking truth to power"' als Motto
  \item Deutsche Morgenländische Gesellschaft, ZmO
  \item Said: Mix aus Gramschi (?) (ich hab die Wahrheit erkannt, ihr habt falsches Bewusstsein) und Foucault (Diskurse sind Mengen von sagbaren Dingen, und diese stellen die Wahrheit her), die eigentlich schwer zu verbinden sind
\end{itemize}

\subsection{Bernard Lewis: Mangelnde Fachkenntnis \& Verzerrungen seitens Saids}

\begin{itemize}
  \item Bernard Lewis "`The Question of Orientalism"'
  \item Said hat kein wissenschaftliches Ziel, sondern Polemik
  \item Zur Veröffentlichung von Saids Buch gab es eigentlich schon keine Orientialisten mehr die sich als solche bezeichneten
  \item Said sei nicht originell
  \item Selektiv: um seine These, Orientalismus = Kolonialismus zu stützen, werden dem zustimmende Beispiel rausgesucht und anderes unterschlagen
  \item Said mache sachliche Fehler, hat falsches Wissenschaftsverständnis und unangemessene Begrifflichkeit (Freud war zu der Zeit wieder hip, daher: sexuelle Begriffsnutzung, Penetration des Orients)
  \item Mangelnde intellektuelle Redlichkeit: Literaturwissenschaftliche Überinterpretation der historischen Quellen
  \item Said verachtet den Orient: schlimmer als jeder arroganter europäischer Imperialist
  \item Saids Verständnis von Wissen und Macht: knowledge/power, Verzerrung der Fachgeschichte
  \item Der Orient ist ein Diskurs, erst von Europa erschaffen
\end{itemize}


\subsection{Aijaz Ahmad \& James Clifford: Kritik an theoretischer Inkonsistenz von Said}

\begin{itemize}
  \item Ende 80er, Anfang 90er; sind eigentlich wohlgesonnen!
  \item James Clifford: wichtiger Postkolonialer Theoretiker
  \item Aijaz Ahmad: Bekanntester, "`scharfzüngigster"' Intelektueller aus dem globalen Süden
  \item Saids Theoriegebäude sei widersprüchlich!
  \item Geht es um Repräsentationen oder den "`realen Orient"'? 
  \item Kritisiert er nun, dass es einen eigentlichen Orient gibt und der Westen missrepräsentiert diesen, oder gibt es den 
  \item Saids Eklektizismus: Benutzt verschiedenste Argumente, je nachdem wie es ihm gerade passt: Geschichte, Literaturwissenschaften
  \item Psychologisierende Logik: dass Der Westen das Andere braucht um sich selbst zu verstehen
  \item Saids Nutzen: Der Orientalismus kann für alles Schlimme, das orientalische Länder erlebt haben und weiterhin erleben, beschuldigt werden und damit hat man ein super 
  \item Qui bono? Wem nützt es? 
\end{itemize}


\subsection{Arif Dirlik: Auto-Orientalismus}

\begin{itemize}
  \item Mitte/Ende 90er
  \item Wie haben die Regionen selber an den Verzerrung 
  \item Hat Said Wesen und Stellung des Orientalismus korrekt dargestellt? Ist der Orient nicht vllt. in einer Zusammenarbeit des Westens mit den Eliten Asiens entstanden?("`Is orientalism a think or a relationship?"')
  \item Ohne die Orientalen keinen Orient.
  \item Selbst-Orientalisierung: Das asiatische Selbstbild und das westliche Bild von Asien ist in einem konstanten Austausch, beide bedingen sich
  \item Folgen des Auto-Orientalismus: Indem das orientalistische Bild aufgenommen wird, wird der Westen geothered und interne Differenzen werden unterdrückt um ein allverbindendes Selbstbild zu erzeugen
\end{itemize}



\newpage


\end{document}
