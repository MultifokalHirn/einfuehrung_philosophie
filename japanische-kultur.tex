\documentclass[emulatestandardclasses]{scrartcl}
\usepackage{graphicx}
\usepackage{CJKutf8} % japanese
\usepackage{color}
\usepackage[ngerman]{babel}
\usepackage{hyperref}
\usepackage{fullpage}
\usepackage[utf8]{inputenc}
\usepackage{calc} 
\usepackage{enumitem}
\usepackage{titlesec}
\date{\vspace{-3ex}}
\begin{document}

\title{
	\includegraphics*[bb=0 0 720 200, width=0.72\textwidth]{ErstesSem/images/hu_logo.png}\\
	\vspace{25pt}
	Einführung in die japanische\\Kultur und Gesellschaft}
\subtitle{\vspace{10pt}
			Prof. Dr. Gerhard Leinss\\
			Proseminar WS 17/18\\
          Institut für Asien- und Afrikawissenschaften\\ 
          Humboldt Universit"at zu Berlin}
\author{Lennard Wolf\\
        \small{\href{mailto:lennard.wolf@student.hu-berlin.de}{lennard.wolf@student.hu-berlin.de}}}
\maketitle
\begin{abstract}
Der Kurs bietet einen Überblick über historische Entwicklungen und gegenwärtige Aspekte der japanischen Kultur und Gesellschaft. Es werden dabei in erster Linie Sachinhalte vermittelt und Fragestellungen diskutiert, aber auch spezifische Hilfsmittel und Vorgehensweisen vorgestellt, die für die wissenschaftliche Beschäftigung mit dem Land im Verlauf des weiteren Studiums nützlich sein können.

\end{abstract}
\newpage

\tableofcontents
%\listoffigures
\newpage


\section{Early Japan\\(30.10.17)}

\subsection{Lektürenotizen}

\begin{itemize}
  \item First agricultural societies in Yayoi period (300BC)
\end{itemize}


\section{Nara Periode (710-794)\\(06.11.17)}

\begin{itemize}
  \item へいじょうきょう Hauptstadt Heijou
  \item $\rightarrow$ buddha Figur
  \item dazaifu und taga-jo waren die zwei Enden der Straßensysteme
  \item Buddhastatuen sind eingeweiht wenn die Augen ausgemalt sind
  \item 784 nach wurde Hauptstadt Nagaoka (südlich von Kyoto) verlegt
  \item Möglicherweise erster Hofstaat der Geschichte
  \item Ende der Zeit: Umzug des Kaisers
  \item Kyoto war bis 8xx Haupstadt
\end{itemize}



\newpage


\end{document}
