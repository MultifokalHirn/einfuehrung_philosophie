\documentclass[emulatestandardclasses]{scrartcl}
\usepackage{graphicx}
\usepackage{CJKutf8} % japanese
\usepackage{color}
\usepackage[ngerman]{babel}
\usepackage{hyperref}
\usepackage{fullpage}
\usepackage[utf8]{inputenc}
\usepackage{calc} 
\usepackage{enumitem}
\usepackage{titlesec}
\date{\vspace{-3ex}}
\begin{document}

\title{
	\includegraphics*[bb=0 0 720 200, width=0.72\textwidth]{ErstesSem/images/hu_logo.png}\\
	\vspace{25pt}
	Einführung in die japanische\\Kultur und Gesellschaft}
\subtitle{\vspace{10pt}
			Prof. Dr. Gerhard Leinss\\
			Proseminar WS 17/18\\
          Institut für Asien- und Afrikawissenschaften\\ 
          Humboldt Universit"at zu Berlin}
\author{Lennard Wolf\\
        \small{\href{mailto:lennard.wolf@student.hu-berlin.de}{lennard.wolf@student.hu-berlin.de}}}
\maketitle
\begin{abstract}
Der Kurs bietet einen Überblick über historische Entwicklungen und gegenwärtige Aspekte der japanischen Kultur und Gesellschaft. Es werden dabei in erster Linie Sachinhalte vermittelt und Fragestellungen diskutiert, aber auch spezifische Hilfsmittel und Vorgehensweisen vorgestellt, die für die wissenschaftliche Beschäftigung mit dem Land im Verlauf des weiteren Studiums nützlich sein können.

\end{abstract}
\newpage

\tableofcontents
%\listoffigures
\newpage


\section{Early Japan\\(30.10.17)}

\subsection{Lektürenotizen}

\begin{itemize}
  \item First agricultural societies in Yayoi period (300BC)
\end{itemize}


\section{Nara Periode (710-794) / Chronologie Japans\\(06.11.17)}

\begin{itemize}
  \item へいじょうきょう Hauptstadt Heijou
  \item $\rightarrow$ buddha Figur
  \item dazaifu und taga-jo waren die zwei Enden der Straßensysteme
  \item Buddhastatuen sind eingeweiht wenn die Augen ausgemalt sind
  \item 784 nach wurde Hauptstadt Nagaoka (südlich von Kyoto) verlegt
  \item Möglicherweise erster Hofstaat der Geschichte
  \item Ende der Zeit: Umzug des Kaisers
  \item Kyoto war bis 8xx Haupstadt
\end{itemize}

\subsection{Zusammenfassung Nara}

\begin{itemize}
  \item \textbf{Anfang/Ende der Periode 710-794: Neue Haupstadt "`Fujiwara"'}
  \item | Wanderung zu neuer Hauptstadt
  \item Was kennzeichnet diesen Zentralstaat nach chinesischem Vorbild? 
  \item | Grundlage des Staates: Das Land \emph{gehört} dem Kaiser ($\rightarrow$ Land-Kaiser; \emph{himmlischer Erhabener} (「天皇」(tennou))), die Bauern bewirtschafteten es nur und gaben Steuern ab; Bürokratiesystem organisierte das Ganze - Japan mehr Fokus auf Herkunft; China war Leistungsgesellschaft (Prüfungen)
  \item Wie weit reichte der Einflußbereich des Zentralstaates geographisch am Ender der Periode?
  \item | bis ca. 2/3 des Nordens der Hauptinsel
  \item Welche Schriften sind aus dieser Periode überliefert?
  \item | Nihonshoki ; Kojiki (alte Begebenheiten/Mythen); Manjoushuu (Gedichte)
  \item Weshalb gab es nach 770 keine weiblichen \emph{Tennos} (?) mehr?
  \item | Eine tennou wurde von einem Mönch verführt, und so wurde Frauen nicht mehr getraut
  \item Welche zwei Attraktionen beherbergt der Toodaiji-Tempel?
  \item | Buddha? nochmal nachschauen
\end{itemize}

\subsection{Zusammenfassung Chronologie Japans}

\begin{itemize}
  \item \textbf{Die großzügigste Einteilung der japanischen Geschichte erfolgt in Perioden, die nach dem geographischen Sitz der aktuellen Machthaber benannt sind bzw. nach deren Familiennamen.}
  \item Die am wenig gebräuchlichste Einteilung ist die nach der Abfolge der Tenno: in der offiziellen Chronologie ist der aktuelle Tenno der 125. Tenno der offiziellen Linie.
  \item Jahre in Daten werden seit 702 in erster Linie mit Ära-Namen (oder Jahresdevisen) benannt. Das sind gewisse Mottos, die für eine begrenzte Anzahl von Jahren gelten. Die Zählung von Jahren in einer Ära beginnt mit dem Jahr eins und endet in dem Jahr, in dem eine neue Ära benannt wird. Seit 1868 sind diese Ära-Namen an die Herrschaft des jeweiligen Tenno gekoppelt: die aktuelle Ära "`Friedliches Gelingen"' (Heisei 平成) ist die 247. Ära der japanischen Geschichte. Sie wurde im Januar 1989 mit der Inthronisation ......
  \item In der japanischen Chronologie
  \item Herkunft \emph{Dutzend}: 4x3 Abschnitte der Hände
  \item Früher: 10er Zählsystem, Später 12er, um an die Tierkreiszeichen anzupassen
\end{itemize}

\section{Heian Periode (794-1185) / Mittelalter\\(13.11.17)}

\subsection{Heian Periode}

\begin{itemize}
  \item Kyouto: Historischer Name Heian-Kyou 平安京, "`Hauptstadt Frieden und Ruhe"' (Wirklich großteilig Frieden) - 4.5kmx5km (ca. 5 Mio Einwohner; )
  \item Nach den Frauen: Kammu Tenno - Mönche wurden immer einflussreicher
  \item Meritokratie wurde immer weiter aufgeweicht
  \item Kopfkissenbuch
  \item Reichtum der Aristokratie basierte auf "`Ausbeutung von Land und Arbeit"'
  \item Fujiwara haben immer männliche Nachkommen gezeugt
  \item "`\textbf{shouen}"': beurkundete Ländereien, die nicht der zentralen Administration unterstanden, bei denen die Besitzer abwesend waren, und lokale Administratoren Teil des Landertrags für sich behalten konnten.
  \item Es gab keinen Harem
  \item Taoismusforscherwitz: Das einzige was Japan nicht aus China importiert hat: Eunuchen, Fußbinden und Taoismus.
\end{itemize}

\subsection{Mittelalter}

\begin{itemize}
  \item Periode der Dezentralisierung (Aufsplittung)
  \item Kamakura Zeit (1185-1136): bakufu 幕府 (Zeltregierung/Militär-|Kriegerregierung)
  \item -> Er meinte zwei Eckdaten sind hier wichtig
\end{itemize}

\newpage


\end{document}
