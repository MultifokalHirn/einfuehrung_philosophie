\documentclass[twoside,twocolumn]{book}
\usepackage{graphicx}
\usepackage{color}
\usepackage{hyperref}
\usepackage{german}
\usepackage{calc} 
\usepackage{enumitem}

\usepackage[utf8]{inputenc}
\usepackage{dici}
%\pagestyle{headings}

\newcommand{\todo}[1]{\textcolor{red}{TODO: #1}\PackageWarning{TODO:}{#1!}}

\begin{document}

\title{
	\includegraphics*[width=0.75\textwidth]{images/hu_logo.png}\\
	\vspace{24pt}
	Philosophisches W"orterbuch}
\author{Lennard Wolf\\
        \href{mailto:lennard.wolf@student.hu-berlin.de}{lennard.wolf@student.hu-berlin.de}}
\maketitle

\textbf{Anmerkung des Autors:} 

Definitionen können zum Teil oder komplett aus Internetquellen (zumeist der Wikipedia) direkt oder editiert übernommen worden sein. Daher erhebe ich in diesem Werk \emph{keinen} Anspruch auf originär auktorale Arbeit. Es handelt sich ausschließlich um eine Begleitarbeit zum Studium. Der Übersicht halber werden Quellenangaben auch unterlassen. 

Sollte sich die Person, von der ein sich in diesem Text befindlichen Textstück ursprünglich verfasst wurde, daran stören, bitte ich diese Person mich über die auf der ersten Seite angegebene e-mail Adresse zu kontaktieren.

\begin{dictionary}
\bigletter{A}
\term{Arbeit (philosophische Kategorie)}{Erfasst alle Prozesse der bewussten schöpferischen Auseinandersetzung des Menschen mit der Natur und der Gesellschaft. Sinngeber dieser Prozesse sind die selbstbestimmt und eigenverantwortlich handelnden Menschen mit ihren individuellen Bedürfnissen, Fähigkeiten und Anschauungen im Rahmen der aktuellen Naturgegebenheiten und gesellschaftlichen Arbeitsbedingungen.}

\bigletter{B}

\bigletter{C}
\term{Cartesianismus}{Die Philosophie nach Ren\'{e} Descartes (1596-1650), welche rationalistisches Denken progapagiert}
\term{Cartesianischer Dualismus}{Nach Ren\'{e} Descartes (1596-1650); Lehrt die Existenz zweier miteinander wechselwirkender, voneinander verschiedener \emph{Substanzen} – Geist und Materie. Im Gegensatz dazu siehe \emph{Monismus} und \emph{Newtons dualistische Naturphilosophie}.}

\bigletter{D}
\term{Dilemma}{Lemma mit zwei H"ornern. Siehe \emph{Lemma}}
\term{Dualismus}{Position, dass sich alle Phänomene der Welt auf zwei einander ausschließende Grundprinzipien/Entitäten/Substanzen zurückführen lassen. Beispiele sind \emph{Newtons dualistische Naturphilosophie}, sowie \emph{Cartesianischer Dualismus}. Im Gegensatz dazu siehe \emph{Monismus} und \emph{Pluralismus}.}

\bigletter{E}
\term{Empirismus}{}
\term{Entit"at}{}
\term{Extension}{}


\bigletter{G}
\term{Genealogie}{???}

\bigletter{H}
\term{Hamlet}{???}

\bigletter{I}
\term{Intension}{???}
\term{Induktionsproblem}{Aus einer endlichen Menge an Beobachtungen kann man auf keine Allaussagen schlie"sen}

\bigletter{K}
\term{Kontingenz}{???}
\term{Kontextprinzip}{???}

\bigletter{L}
\term{Lemma}{...}

\bigletter{M}
\term{Monismus}{Position, dass sich alle Phänomene der Welt auf ein Grundprinzip zurückführen lassen (steht dem Pantheismus nah). Im Gegensatz dazu siehe \emph{Dualismus} und \emph{Pluralismus}}
\term{Metaphysik}{???}

\bigletter{N}
\term{Natural Kind Term}{(`Nat"urliche Art Ausdr"ucke') Ausdr"ucke f"ur Substanzen oder Arten von Gegenst"anden, wie sie in unserer nat"urlichen Umwelt vorkommen und von den Naturwissenschaften kategorisiert werden.}
\term{Newtons dualistische Naturphilosophie}{Nach Sir Isaac Newton (1643-1727); Lehrt die Existenz der Wechselwirkung aktiver immaterieller \emph{Kräfte der Natur} mit der absolut passiven Materie. Im Gegensatz dazu siehe \emph{Cartesianischer Dualismus} und \emph{Monismus}.}

\bigletter{O}
\term{Ontologie}{???}
\term{Ontologische Verpflichtung (Quine)}{???}

\bigletter{P}
\term{Pluralismus}{Position, dass sich alle Phänomene der Welt auf eine Vielzahl einander ausschließender Grundprinzipien/Entitäten/Substanzen zurückführen lassen. Es existieren viele verschiedene Formen des Pluralismus Im Gegensatz dazu siehe \emph{Monismus} und \emph{Pluralismus}.}

\term{Prop"adeutik}{}

\term{Positivismus}{}

\bigletter{T}
\term{Tatsache}{}

\bigletter{U}
\term{Urteil}{}

\bigletter{V}
\term{Verfremdung (Hegel)}{}

\bigletter{W}
\term{Wissenschaftstheorie}{[auch: \emph{Wissenschaftsphilosophie}]}

\bigletter{X}
\term{}{}

\bigletter{Y}
\term{}{}

\bigletter{Z}
\term{}{}

\end{dictionary}
\end{document}
