\documentclass[twoside,twocolumn]{book}
\usepackage{graphicx}
\usepackage{color}
\usepackage{hyperref}
\usepackage{german}
\usepackage{calc} 
\usepackage{enumitem}

\usepackage[utf8]{inputenc}
\usepackage{dici}
%\pagestyle{headings}

\newcommand{\todo}[1]{\textcolor{red}{TODO: #1}\PackageWarning{TODO:}{#1!}}

\begin{document}

\title{
	\includegraphics*[width=0.75\textwidth]{images/hu_logo.png}\\
	\vspace{24pt}
	Kompendium gelernter Begriffe\\aus den akademischen Studien}
\author{Lennard Wolf\\
        \href{mailto:lennard.wolf@student.hu-berlin.de}{lennard.wolf@student.hu-berlin.de}}
\maketitle

\begin{dictionary}
\bigletter{A}


\bigletter{B}

\bigletter{C}
\term{Cartesianismus}{Die Philosophie nach Ren\'{e} Descartes (1596-1650), welche rationalistisches Denken progapagiert}
\term{Cartesianischer Dualismus}{Nach Ren\'{e} Descartes (1596-1650); Lehrt die Existenz zweier miteinander wechselwirkender, voneinander verschiedener \emph{Substanzen} – Geist und Materie. Im Gegensatz dazu siehe \emph{Monismus} und \emph{Newtons dualistische Naturphilosophie}.}

\bigletter{D}
\term{Dualismus}{Position, dass sich alle Phänomene der Welt auf zwei einander ausschließende Grundprinzipien/Entitäten/Substanzen zurückführen lassen. Beispiele sind \emph{Newtons dualistische Naturphilosophie}, sowie \emph{Cartesianischer Dualismus}. Im Gegensatz dazu siehe \emph{Monismus} und \emph{Pluralismus}.}

\bigletter{K}
\term{Kontingenz}{???}

\bigletter{M}
\term{Monismus}{Position, dass sich alle Phänomene der Welt auf ein Grundprinzip zurückführen lassen (steht dem Pantheismus nah). Im Gegensatz dazu siehe \emph{Dualismus} und \emph{Pluralismus}}

\bigletter{N}
\term{Newtons dualistische Naturphilosophie}{Nach Sir Isaac Newton (1643-1727); Lehrt die Existenz der Wechselwirkung aktiver immaterieller \emph{Kräfte der Natur} mit der absolut passiven Materie. Im Gegensatz dazu siehe \emph{Cartesianischer Dualismus} und \emph{Monismus}.}

\bigletter{O}
\term{Ontologie}{???}

\bigletter{P}
\term{Pluralismus}{Position, dass sich alle Phänomene der Welt auf eine Vielzahl einander ausschließender Grundprinzipien/Entitäten/Substanzen zurückführen lassen. Es existieren viele verschiedene Formen des Pluralismus Im Gegensatz dazu siehe \emph{Monismus} und \emph{Pluralismus}.}

\bigletter{V}
\term{Verfremdung (Hegel)}{}

\end{dictionary}
\end{document}
